
%
% Chapter 2 of my master thesis:
% Still quiete a lot introductory
%

\chapter{Deformation quantization}

The starting point for every theory of quantization is surely mechanics. Here, 
mechanics means both, the classical and the quantum theory. Since we want to 
explain how one can link them together, we will give a short overview of both 
theories and show their similarities and their differences in the first section 
of this chapter. Then we will come to different aspects of quantization and 
state the most important ideas in the next part. In section three, we will 
focus on Deformation quantization and give an overview of this rather young 
theory.



\section{Mechanics: The Classical and the Quantum World}
\label{sec:chap2_Mechanics}

\subsection{Classical Mechanical Systems}
\label{subsec:chap2_Classical}
We want to recall briefly the notions of classical mechanics. There are many 
good books on this subject and also the notation is mostly uniform everywhere, 
but nevertheless want to refer to the books of Marsden and Ratiu 
\cite{marsden.ratiu:1999a} and Arnold \cite{arnold:1989a}, which give very good 
introductions and overviews of the whole theory.
Imagine the simplest model for a mechanical system, which is not trivial: a 
single particle with mass $m$ in $\mathbb{R}^3$ in a scalar potential $V$. We 
will denote his position by $q = (q^1, q^2, q^3) \in Q = \mathbb{R}^3$ and call 
the set of all possible positions $Q$ the configuration space. Since we also 
have a time coordinate $t \in \mathbb{R}$, we can describe the path on which 
the particle moves by $q(t)$. The state of the particle is completely described 
by its position and its velocity $(q(t), \dot q(t))$ and the velocity should be 
understood as a tangent vector $\dot q(t) \in T_{q(t)}Q = \mathbb{R}^3$. 
Therefore, the tangent bundle $TQ$, which is in this case $\mathbb{R}^6$, is 
sometimes called the state space. In classical mechanics, we can describe the 
movement of the particle with the Euler-Lagrange equations, which are derived 
from the Lagrange function by a variational principle. The Lagrange function of 
this system reads
\begin{equation*}
	\mathcal{L}(q, \dot q)
	=
	T(q, \dot q) - V(q, \dot q)
\end{equation*}
and $T(q, \dot q) = \frac{1}{2} m  \sum_{i=1}^3 \dot{q^i}^2$ is the kinetic 
energy. The action 
is defined as the integral
\begin{equation*}
	\mathcal{S}(q, \dot q)
	=
	\int\limits_{t_0}^{t_1}
	\mathcal{L}(q, \dot q)
	dt.
\end{equation*}
It is a physical observation, similar to a mathematical axiom, that this action 
functional is minimized along the trajectories of the particle. So fixing a 
starting point $q_0$ and an end point $q_1$ of a trajectory, we find
\begin{equation*}
	\delta \mathcal{S}
	\at{q(t_0) = q_0, q(t_1) = q_1}
	=
	0.
\end{equation*}
From this, one finds the Euler-Lagrange equations for $i = 1, 2, 3$
\begin{equation*}
	\frac{d}{dt} 
	\frac{\partial \mathcal{L}}{\partial \dot q^i}
	-
	\frac{\partial \mathcal{L}}{\partial q^i}
	=
	0.
\end{equation*}
In our case this means
\begin{equation*}
	\frac{d}{dt} 
	\frac{\partial T}{\partial \dot q^i}
	-
	\frac{\partial V}{\partial q^i}
	=
	m \ddot q
	-
	\frac{\partial V}{\partial q^i}
	=
	0
\end{equation*}
and we can integrate the equations to get the trajectory. So far, we described 
the system in the state space and for obvious reasons, this formalism is called 
the Lagrange formalism. It is not the only formalism, which describes to 
behaviour of the system: one can go to the so called Hamilton formalism, which 
is based on the conjugate momenta $p = (p_1, p_2, p_3) \in T_{q(t)}^*Q$
\begin{equation*}
	p_i(t)
	=
	\frac{\partial \mathcal{L}}{\partial \dot q^i}
\end{equation*}
and linked to the Lagrange formalism by a Legendre transformation. Now we can 
define the Hamilton function
\begin{equation*}
	H(q, p)
	=
	p_i q^i
	-
	\mathcal{L}(q, \dot q),
\end{equation*} 
which is the crucial quantity in this setting. It represents the energy of the 
system. On finds the equations
\begin{equation*}
	\frac{\partial H}{\partial p_i}
	=
	\frac{d q^i}{dt}
	\quad \text{ and } \quad
	- \frac{\partial H}{\partial q^i}
	=
	\frac{d p_i}{dt}.
\end{equation*}
They remind strongly of a symplectic structure and indeed this is the case: 
using the standard sympectic matrix
\begin{equation*}
	\omega
	=
	\begin{pmatrix}
		0 & \Unit
		\\
		- \Unit & 0
	\end{pmatrix}
\end{equation*}
one finds
\begin{equation*}
	\frac{d}{dt} \vector{q^i}{p_i}
	=
	\omega
	\vector{\frac{\partial H}{\partial q^i}}
	{\frac{\partial H}{\partial p_i}}.
\end{equation*}
More generally, one can define the Poisson bracket $\{f, g\}$ for two functions 
$\Cinfty(T^*Q)$ by
\begin{equation*}
	\{f, g\}
	=
	\sum\limits_{i=1}^3
	\left(
		\frac{\partial f}{\partial q^i}
		\frac{\partial g}{\partial p_i}
		-
		\frac{\partial f}{\partial p_i}
		\frac{\partial g}{\partial q^i}
	\right)
\end{equation*}
and finds the time evolution of a function $f$ given by
\begin{equation*}
	\frac{d}{dt} f
	=
	\{ f, H \}.
\end{equation*}
All of those objects have, of course, a geometrical interpretation which remain 
true for and allow generalizations from this very easy example. For a more 
complex system than one particle in a scalar potential, the configuration space 
$Q$ will be a smooth manifold, the state space $TQ$ is described by the tangent 
bundle and the phase space $T^*Q$ by the cotangent bundle. The transition from 
the Lagrange to the Hamilton formalism is done by a fiber derivative, the 
kinetic energy has a mathematical interpretation as a Riemannian metric and the 
Poisson bracket is the one which is due to the canonical symplectic form on the 
cotangent bundle. So in some sense, classical mechanics can be described by 
symplectic geometry.


The last conclusion, however, was a bit too fast. There are mechanical systems 
like the rigid body, which can not be described in the symplectic formalism, 
or which have certain symmetries and therefore allow a \emph{reduction}. A 
symmetry can mathematically be described by a Lie group $G$, which acts on 
$T^*Q$ and a reduction by a momentum map $J \colon T^*Q \longrightarrow 
\lie{g}^*$, where $\lie{g}$ is the Lie algebra of the Lie group and $\lie{g}^*$ 
its dual. The reduced phase space is a quotient of the cotangent bundle and 
does not need to be symplectic any more. Anyway, in both cases, the result will 
be something more general than a symplectic manifold: it will be a Poisson 
manifold. 
\begin{definition}[Poisson Manifold]
	A Poisson manifold is a tuple $(M, \{ \cdot, \cdot \})$ of a smooth 
	manifold $M$ and a Poisson bracket $\{ \cdot, \cdot\}$. The bracket is a 
	biderivation
	\begin{equation*}
		\{ \cdot, \cdot \}
		\colon
		\Cinfty
		\times
		\Cinfty(M)
		\longrightarrow
		\Cinfty(M),
	\end{equation*}
	which is anti-symmetric and fulfils the Jacobi identity
	\begin{equation*}
		\{ \{f, g\}, h \}
		+
		\{ \{g, h\}, f \}
		+
		\{ \{h, f\}, g \}
		=
		0.
	\end{equation*}
\end{definition}
A classical mechanical system will be described as a Poisson manifold together 
with a Hamilton function $H$, such that the time derivation of every function 
$f \in \Cinfty(M)$ is given by the relation
\begin{equation*}
	\frac{d}{dt} f(t)
	=
	\{f, H\}.
\end{equation*}
The observable algebra of this classical system is then modelled as the algebra 
of smooth functions $\Cinfty(M)$ on this Poisson manifold.



\subsection{Quantum Mechanics}
\label{subsec:chap2_Quantum}

\subsection{The Correspondence Principle}
\label{subsec:chap2_Correspondence}



\section{Making Things Noncommutative: Quantization}
\label{sec:chap2_Quantization}

\subsection{Noncommutative Geometry}
\label{subsec:chap2_NoncommGeometry}

\subsection{Deformation Quantization}
\label{subsec:chap2_Deformation}



\section{Formal vs. Strict Deformation Quantization}
\label{sec:chap2_FormalStrict}

\subsection{A Mathematical Theory}
\label{subsec:chap2_MathTheory}
Example how Physics gives rise to new Math,
Including a short history,
Things got stuck somewhere


\subsection{From Formal to Strict}
\label{subsec:chap2_Formal2Strict}
Three steps: Formal -> Strict -> Representations
Concepts of strict DQ (Rieffel vs. Waldmann,
meaning C* vs. Locally convex)
So far everything is Math, Physics would start after the last step, Maybe one day...
