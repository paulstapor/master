
%
% Chapter 2 of my master thesis:
% Still quiete a lot introductory
%

\chapter{Deformation quantization}

The starting point for every theory of quantization is without any doubt the 
theory of mechanics. Here, 
mechanics means both, the classical and the quantum theory. Since we want to 
explain how one can link those two together, we will give a short overview of both 
theories and show their similarities and their differences in the first section 
of this chapter. Then we will explain what a quantization should actually be and 
what kind of approaches to it exist in the next part. In section three, we will 
focus on Deformation quantization and give an overview of this rather young 
theory.



\section{Mechanics: The Classical and the Quantum World}
\label{sec:chap2_Mechanics}

\subsection{Classical Mechanical Systems}
\label{subsec:chap2_Classical}
We want to recall briefly the notions of classical mechanics. There are many 
good books on this subject and also the notation for the basic concepts is more or 
less uniform everywhere, but nevertheless want to refer to the books of Marsden 
and Ratiu \cite{marsden.ratiu:1999a} and Arnold \cite{arnold:1989a}, which give 
very good introductions and overviews of the theory.
Imagine the simplest model for a mechanical system, which is not trivial: a 
single particle with mass $m$ moving in $\mathbb{R}^3$ in a scalar potential $V$. 
We will denote its position by $q = (q^1, q^2, q^3) \in Q = \mathbb{R}^3$ and call 
the set $Q$ of all possible positions the \emph{configuration space}. Since we 
also have a time coordinate $t \in \mathbb{R}$, we can describe the path on which 
the particle moves by $q(t)$. The state of the particle is completely described 
by its position and its velocity $(q(t), \dot q(t))$ and the velocity should be 
understood as a tangent vector $\dot q(t) \in T_{q(t)}Q = \mathbb{R}^3$. 
Therefore, the tangent bundle $TQ$, which is in this case $\mathbb{R}^6$, is 
sometimes called the state space. In classical mechanics, we can describe the 
movement of the particle using the Euler-Lagrange equations, which are derived 
from the Lagrange function by a variational principle. The Lagrange function of 
this system reads
\begin{equation*}
	\mathcal{L}(q, \dot q)
	=
	T(q, \dot q) - V(q, \dot q)
\end{equation*}
and $T(q, \dot q) = \frac{m}{2}  \sum_{i=1}^3 \left( \dot{q}^i \right)^2$ is the 
kinetic energy. The action along a path is defined as the integral
\begin{equation*}
	\mathcal{S}(q, \dot q)
	=
	\int\limits_{t_0}^{t_1}
	\mathcal{L}(q, \dot q)
	dt.
\end{equation*}
It is a physical observation, similar to a mathematical axiom, that this action 
functional is minimized along the trajectories of the particle. So by fixing a 
starting point $q_0$ and an end point $q_1$ of a trajectory, we find
\begin{equation*}
	\delta \mathcal{S}
	\at{q(t_0) = q_0, q(t_1) = q_1}
	=
	0.
\end{equation*}
From this, one finds the Euler-Lagrange equations for $i = 1, 2, 3$
\begin{equation*}
	\frac{d}{dt} 
	\frac{\partial \mathcal{L}}{\partial \dot q^i}
	-
	\frac{\partial \mathcal{L}}{\partial q^i}
	=
	0.
\end{equation*}
In our case this means
\begin{equation*}
	\frac{d}{dt} 
	\frac{\partial T}{\partial \dot q^i}
	-
	\frac{\partial V}{\partial q^i}
	=
	m \ddot q
	-
	\frac{\partial V}{\partial q^i}
	=
	0
\end{equation*}
and we can integrate the equations to get the trajectory. So far, we described 
the system in the state space. This description is called 
the Lagrange formalism. It is not the only picture, which describes to 
behaviour of the system: one can go to the so called Hamilton formalism, which 
is based on the conjugate momenta $p = (p_1, p_2, p_3) \in T_{q(t)}^*Q$
\begin{equation*}
	p_i(t)
	=
	\frac{\partial \mathcal{L}}{\partial \dot q^i}
\end{equation*}
and linked to the Lagrange formalism by a Legendre transformation. The set $T^*Q$ 
also describes all possible states of the system and is called the phase space. 
Now we can define the Hamilton function
\begin{equation*}
	H(q, p)
	=
	p_i q^i
	-
	\mathcal{L}(q, \dot q),
\end{equation*} 
which is the crucial quantity in this setting. It represents the energy of the 
system. One finds the equations
\begin{equation*}
	\frac{\partial H}{\partial p_i}
	=
	\frac{d q^i}{dt}
	\quad \text{ and } \quad
	- \frac{\partial H}{\partial q^i}
	=
	\frac{d p_i}{dt}.
\end{equation*}
They remind strongly of a symplectic structure and indeed this is the case: 
using the standard sympectic matrix
\begin{equation*}
	\omega
	=
	\begin{pmatrix}
		0 & \Unit
		\\
		- \Unit & 0
	\end{pmatrix}
\end{equation*}
one finds
\begin{equation*}
	\frac{d}{dt} \vector{q^i}{p_i}
	=
	\omega
	\vector{\frac{\partial H}{\partial q^i}}
	{\frac{\partial H}{\partial p_i}}.
\end{equation*}
More generally, one can define the Poisson bracket $\{f, g\}$ for two functions 
$\Cinfty(T^*Q)$ by
\begin{equation*}
	\{f, g\}
	=
	\sum\limits_{i=1}^3
	\left(
		\frac{\partial f}{\partial q^i}
		\frac{\partial g}{\partial p_i}
		-
		\frac{\partial f}{\partial p_i}
		\frac{\partial g}{\partial q^i}
	\right)
\end{equation*}
and finds the time evolution of a function $f$ given by
\begin{equation*}
	\frac{d}{dt} f
	=
	\{ f, H \}.
\end{equation*}
All of those objects have, of course, geometrical interpretations which 
allow generalizations from this very easy example. For a more 
complex system than one particle in a scalar potential, the configuration space 
$Q$ will be a smooth manifold, the state space $TQ$ is described by the tangent 
bundle and the phase space $T^*Q$ by the cotangent bundle. Points in the phase 
space, which can also be understood as Dirac measures in $T^*Q$, describe the 
possible states of the system. This interpretation allows to speak of positive 
Borel measures on $T^*Q$ as mixed states, which describe a probabilistic 
distribution of the state of the system. The transition from 
the Lagrange to the Hamilton formalism is done by a fiber derivative, the 
kinetic energy has a mathematical interpretation as a Riemannian metric and the 
Poisson bracket is the one which is due to the canonical symplectic form on the 
cotangent bundle. So in some sense, classical mechanics can be described by 
symplectic geometry.


The last conclusion, however, was a bit too fast. There are mechanical systems 
like the rigid body, which can not be described in the symplectic formalism, 
or which have certain symmetries and therefore allow a \emph{reduction}. A 
symmetry can mathematically be described by a Lie group $G$, which acts on 
$T^*Q$ and a reduction by a momentum map $J \colon T^*Q \longrightarrow 
\lie{g}^*$, where $\lie{g}$ is the Lie algebra of the Lie group and $\lie{g}^*$ 
its dual. The reduced phase space is a quotient of the cotangent bundle and 
does not need to be symplectic any more. In both cases, the result will 
be something more general than a symplectic manifold, because the quotient 
procedure may not leave closed and nondegenerate $2$-form on the reduced phase 
space any more. The Poisson bracket however will still be there. So the objects we 
actually want to use to describe classical mechanics are Poisson manifolds. 
\begin{definition}[Poisson Manifold]
	A Poisson manifold is a tuple $(M, \{ \cdot, \cdot \})$ of a smooth 
	manifold $M$ and a Poisson bracket $\{ \cdot, \cdot\}$. The bracket is a 
	biderivation
	\begin{equation*}
		\{ \cdot, \cdot \}
		\colon
		\Cinfty
		\times
		\Cinfty(M)
		\longrightarrow
		\Cinfty(M),
	\end{equation*}
	which is anti-symmetric and fulfils the Jacobi identity
	\begin{equation*}
		\{ \{f, g\}, h \}
		+
		\{ \{g, h\}, f \}
		+
		\{ \{h, f\}, g \}
		=
		0.
	\end{equation*}
\end{definition}
The objects we can observe in this setting are the smooth functions on the Poisson 
manifold. Together with the Poisson bracket, they form the prototype of a 
\emph{Poisson algebra}: an algebra with a bracket, which is an antisymmetric 
biderivation and which fulfils the Jacobi identity. The range, sometimes also 
called the spectrum, of those smooth functions should be seen as the measurable 
values of the observable. So in full generality, we could say that a classical 
mechanical system is a Poisson manifold together with a Hamilton function $H$, and 
the time evolution of every observable $f \in \Cinfty(M)$ is given by the relation
\begin{equation*}
	\frac{d}{dt} f(t)
	=
	\{f, H\}.
\end{equation*}
We will soon see that this formalism is as close as we can get to the one of 
quantum mechanics.



\subsection{Quantum Mechanics}
\label{subsec:chap2_Quantum}

In quantum theory, the model of a physical system looks completely different on 
the first sight. There is indeed no obvious way to describe how this formalism was 
derived out of the one describing classical mechanics. It was a hard 
and non-straightforward way, taken in small steps of which some were 
more or less obvious and some were ingenious guesses. Built up on first ideas 
by Max Planck, who described the radiation of a black body, Albert Einstein, 
who explained the photo-electrical effect and Niels Bohr, who solved the 
problem of atomic spectra, a hand full of physicists including Max Born, 
Werner Heisenberg, Erwin Schr\"odinger, Paul Dirac, Wolfgang Pauli and Pascual 
Jordan developed a complex, counter-intuitive but extremely well working theory, 
which was able to give precise answers to very most of the open questions at that 
time. However, this theory, mainly risen between 1925 and 1928, did not have a 
satisfying mathematical fundament. It took around ten more years until 
mathematicians, mainly John von Neumann, worked out a mathematical theory for 
quantum mechanics and gave also a physical interpretation to their mathematical 
formulation \cite{vonneumann:1996a}. Today, quantum mechanics is usually 
introduced as a mathematical theory based on axioms and most of the textbooks 
(at least most of the more mathematical ones) use this axiomatical approach to 
explain the theory. Two very nice mathematical introductions are given by 
Bongaarts \cite{bongaarts:2015a} and Hall \cite{hall:2013a}.


The state of a physical system is described by a vector $\psi$ on a 
Hilbert space $\hilbert H$ and the observables are self-adjoint (and 
usually unbounded) operators on it. The spectrum of these operators describes 
the measurable values. The theory is probabilistic, but has a deterministic 
time evolution: the pair $(A, \psi)$ gives a probabilistic distribution, 
from which we can calculate the probability to measure $A$ with the value $a 
\in \spec (A)$ in the state $\psi$ using the spectral resolution of $A$. We 
want to look again at the example of a particle in $\mathbb{R}^3$ 
moving in a potential. Let $x = (x_1, x_2, x_3) \in \mathbb{R}^3$ be a 
coordinate vector, then we have the so called Schr\"odinger representation 
given by the Hilbert space $\hilbert{H} = \Lzwei(\mathbb{R}^3)$ of square 
integrable functions with the Lebesgue measure. The state $\psi$ of a particle 
is a time-dependent element of this Hilbert space which fulfils the 
Schr\"odinger equation
\begin{equation}
	\label{QM:Schroedinger}
	i \hbar \frac{d}{dt} \psi(x, t)
	=
	\hat{H}
	\psi(x, t)
\end{equation}
where $\hat{H}$ denotes the Hamilton operator which is given by
\begin{equation}
	\hat{H}
	=
	- \frac{\hbar^2}{2 m}
	\Laplace + V(x).
\end{equation}
Similar to the Hamilton function in classical mechanics, the Hamilton operator 
(or more precisely: its spectrum) describes the energy of this system, $m$ is the 
mass of the particle and $V$ the potential. The operators which correspond to 
the coordinate $x_i$ and the momentum $p_j$ of the particle are given by
\begin{align*}
	\left( \hat{q}_i
	\psi \right) (x, t)
	&=
	x_i
	\psi(x, t)
	\\
	\left( \hat{p}_j
	\psi \right) (x, t)
	&=
	- i \hbar 
	\frac{\partial}{\partial x_j}
	\psi(x, t).
\end{align*}
This turns Equation \eqref{QM:Schroedinger} into
\begin{equation*}
	i \hbar \frac{d}{dt} \psi(x, t)
	=
	\left( \frac{\hat{p}^2}{2m} + V(x) \right)
	\psi(x, t)
\end{equation*}
and looks therefore similar to the classical time evolution.
Since the  state is a function of the coordinate $x$, one calls this the 
representation in \emph{position space}. Sometimes it is more convenient to 
describe a state in terms of its momentum and one changes to \emph{momentum 
space} via Fourier transformation. Both representations are equivalent, very 
similar to the variables $q$ and $p$ in the Hamilton form of classical 
mechanics, from which this picture is very much inspired. Unlike the classical 
position and momentum observables being functions, the quantum mechanical 
observables are \emph{operators} and do not commute any longer. One can check 
these by calculating commutators and finds the so called Heisenberg or 
\emph{classical commutations relations}
\begin{align}
	[p_i, p_j]
	& =
	[q_i, q_j]
	=
	0,
	\\
	[q_i, p_j]
	& =
	i \hbar \delta_{ij}.
\end{align}
These relations are the starting point for all quantization theories. The main 
goal is always to find an algebra which fulfils these (or maybe equivalent) 
relations and this is what makes a quantum algebra so different from a 
classical one. Physically spoken, this noncommutativity means that a 
measurement of the position of a particle influences its momentum and vice 
versa, such that it becomes impossible to measure both, position and momentum, 
at the same time with arbitrarily high precision. This still very vague 
statement can be made precise using the Hilbert norm and the Cauchy-Schwarz 
inequality and yields the famous Heisenberg uncertainty relation
\begin{equation*}
	\Laplace A_\psi
	\Laplace B_\psi
	\geq
	\frac{1}{2}
	\left|
		\SP{\psi, [A, B] \psi}
	\right|.
\end{equation*}
Here $A, B$ are observables and $\Laplace A_\psi, \Laplace B_\psi$ denote their 
standard deviations in the state $\psi$. For the position and momentum of a 
particle, this gives
\begin{equation*}
	\Laplace \hat{p}
	\Laplace \hat{q}
	\geq
	\frac{\hbar}{2}.
\end{equation*}
However, the Schr\"odinger picture is not the only possibility to describe the 
behaviour of a quantum system. Heisenberg proposed a description in which not 
the particles, but the observables depend on the time. Since quantum mechanics,
time evolution is described by a one-parameter group of unitary operators, we have
\begin{equation*}
	\psi(x,t)
	=
	U(t - t_0) \psi(x, t_0),
\end{equation*}
at least in systems where the interactions are not explicitly time dependent. 
This makes it possible to apply unitary transformations to the states 
as well as to the operators 
\begin{equation*}
	\psi(x)
	=
	\psi(x, t_0)
	=
	U(t_0 - t) \psi(x, t)
	\quad \text{ and } \quad
	A(t)
	=
	U(t_0 - t)
	A
	U(t - t_0).
\end{equation*}
One gets the so called \emph{Heisenberg picture}, in which the 
time-dependence is not expressed in the states but in the operators. Both 
pictures are equivalent and it is just a question of convenience which one is 
used. In our example, the unitary operators are given by
\begin{equation*}
	U(t - t_0)
	=
	\E^{- \frac{i}{\hbar} \hat{H} t},
\end{equation*}
since they describe the time evolution of solutions of the Schr\"odinger 
equation. If one differentiates an operator with respect to $t$ in the 
Heisenberg picture, one finds the \emph{Heisenberg equation}
\begin{equation*}
	\frac{d}{dt}
	A(t)
	=
	\frac{1}{i \hbar}
	[A(t), H].
\end{equation*}
This reminds us of course of the time evolution in terms of Poisson bracekts in 
classical mechanics. The idea that both objects, the quantum mechanical commutator 
and the Poisson bracket, should somehow be seen as counterparts to each other, is 
called the \emph{correspondence principle}. It is the second staring point for 
nearly all theories of quantization.



\section{Making Things Noncommutative: Quantization}
\label{sec:chap2_Quantization}

\subsection{The Task}
\label{subsec:chap2_Task}

We have had a brief overview over the mathematical formulation of classical and 
quantum mechanics. On one hand, if we compare for example the way states of a 
physical system are described (a point in the cotangent bundle of a manifold and a 
vector in a Hilbert space) we will see that they are very different. On the other 
hand, we have seen that both theories have certain things in common: the time 
evolution can be described in a similar way. We want to give a short list of the 
main concepts of both theories and compare them (a much more detailed list and 
discussion can be found in chapter 5 of \cite{waldmann:2007a}):
\bgroup
\renewcommand{\arraystretch}{1.6}
\begin{center}
	\begin{tabular}
	{lll}
		~ 
		&
		\textbf{Classical} 
		&
		\textbf{Quantum}
		\\
		\parbox{4cm}
		{
			Observables
		}
		&
		\parbox{5cm}{
			Poisson-*-algebra $\algebra{A}_{Cl}$
			of \\
			smooth functions $\Cinfty(M)$ \\
			on a Poisson manifold
		}
		&
		\parbox{5cm}{
			*-algebra $\algebra{A}_{QM}$
			of self-adjoint operators
			on a Hilbert space $\hilbert{H}$
		}
		\\
		\parbox{4cm}
		{
			Measurable 
			Values
		}
		&
		$\spec (f) \subseteq \mathbb{R}$
		&
		$\spec (A) \subseteq \mathbb{R}$
		\\
		States
		&
		\parbox{5cm}{
			Points in the
			phase space
		}
		&
		\parbox{5cm}{
			Vectors in a
			Hilbert space
		}
		\\
		Time evolution
		&
		Hamilton function $H$
		&
		Hamilton operator $\hat{H}$
		\\
		\parbox{4cm}{
			Infinitesimal\\
			time evolution
		}
		&
		$
		\frac{d}{dt} f(t)
		=
		\{f(t), H\}
		$
		&
		$\frac{d}{dt} A(t)
		=
		\frac{1}{i \hbar}
		[A(t), H]
		$
	\end{tabular} 
\end{center}
\egroup
When we look at this table, we see that is is probably a good guideline to 
understand the observable algebras as the main concepts of mechanics, rather than 
the states. We know that the classical theory emerges as a limit of many quantum 
systems, such that the physical constant $\hbar$ becomes small in enough (in 
comparison to the characteristic energy scale of the system) to be neglected. How 
to construct a correspondence the other way round is the question which the theory 
of quantization addresses. Physicists used to solve this by "making a hat on the 
variables $p$ and $q$ and saying that they the do not commute any more". 
Surprisingly enough, this approach, which is called \emph{canonical quantization} 
in physics, works for most of the simple examples. In particular, this is the 
case when we do not have high powers of $\hat p$ and $\hat q$ and no 
mixed terms. That this idea is however not canonical in a \emph{mathematical 
sense} is almost needless to say and of course it causes severe problems when 
higher and mixed terms in position and momentum appear. Briefly stated, the 
problem is that in the classical theory, the expressions $q^2p, qpq, pq^2$ 
describe the same polynomial, but in the quantum theory, they do not. So we must 
think about the question, to which operators we want to map mixed polynomials, and 
canonical quantization does not provide an answer. To create a mathematical theory 
of quantization, one needs to impose an ordering on the $\hat p$ and $\hat q$. 
Already Hermann Weyl knew about this problem and looked for ways how to solve it 
by proposing a totally symmetrized expression \cite{weyl:1931a} for such terms. 
This ordering is now known as the Weyl ordering and his formula became the 
starting point for the theory of deformation quantization many years later.


\subsubsection{A Definition}
First, we want to take a step back and try to formalize what we actually mean by a 
quantization. See also section 3.2 in \cite{esposito:2015a} or section 5.1.2 in
\cite{waldmann:2007a} for a good introduction to this topic. For us, a 
quantization should be a correspondence
\begin{equation*}
	\mathcal{Q}
	\colon
	\algebra{A}_{\mathrm{Classical}}
	\longrightarrow
	\algebra{A}_{\mathrm{Quantum}}
\end{equation*}
between a commutative algebra $\algebra{A}_{\mathrm{Classical}}$ and a 
noncommutative algebra $\algebra{A}_{\mathrm{Quantum}}$, which is a 
''bijection'' in some sense: all classical observables appear as classical 
limits from quantum observables, hence we should expect $\mathcal{Q}$ to be 
''injective''. On the other hand, if the Quantum algebra was bigger than the 
classical algebra, the whole concept of quantization would be pointless, since 
we could never recover all observables and hence we want $\mathcal{Q}$ to be 
''surjective'', too. The correspondence should also keep somehow track of the 
physical meaning of our observables, or at least tell us, what the observable 
$\mathcal{Q}(f)$ should be for some $f \in \algebra{A}_{\mathrm{Classical}}$. 
Moreover, $\mathcal{Q}^{-1}$ should behave like the classical limit, since this 
is the rather well-understood part of both. Finally, we want the correspondence to
keep associative structures: the classical algebra is associative and we want 
the product in our noncommutative algebra to correspond to the concatenation of 
operators on a Hilbert space which is associative, too. One possibility of 
formulating this would be the following: in the very prototype of a classical 
observable algebra, a symplectic vector space of dimension $2n$, $n \in 
\mathbb{N}$, we want that the following three axioms are fulfilled:
\begin{enumerate}
	\item[(Q1)]
	We want a ''map'', which allows a representation in the standard 
	picture of quantum mechanics:
	$\mathcal{Q}(1) = \Unit$, 
	$\mathcal{Q}(q^i) = \hat{q}^i$, 
	$\mathcal{Q}(p_j) = - i \hbar \frac{\partial}{\partial q^j}$.
	
	\item[(Q2)]
	The correspondence principle should be fulfilled:
	$[\mathcal{Q}(f), \mathcal{Q}(g)] = i \hbar \mathcal{Q}(\{f, g\})$, 
	for all $f, g \in \algebra{A}_{\mathrm{Classical}}$.
	
	\item[(Q3)]
	For mathematical simplicity, $\mathcal{Q}$ should be linear.
\end{enumerate}
We finally split up the whole process: in a first step. we \emph{quantize} the 
system and in a second step, we look for \emph{representations} on a Hilbert 
space.
\begin{center}
    \begin{tikzpicture}
    	\tikzstyle{myarrowq} = 
    	[line width = 3.4mm,
    	draw = {rgb:black,1;white,2},
    	-triangle 45,
    	shorten >= -3mm,
    	postaction = {
    		draw, 
    		line width = 5.3mm, 
    		shorten >=4.5mm,
    		-},
    	postaction = {
    		decorate,
    		decoration = {
    			text along path, 
    			text = {|\large \color{white}| Quantization}, 
    			text align = {align = center},
    			raise = -0.7ex
    			}
    		},
    	]
    	\tikzstyle{myarrowg} = 
    	[line width = 3.2mm,
    	draw = {rgb:black,1;white,2},
    	-triangle 45,
    	postaction = {
    		draw, 
    		line width = 5.5mm, 
    		shorten >=6.5mm,
    		-},
    	postaction = {
    		decorate,
    		decoration = {
    			text along path, 
    			text = {|\large \color{white}| Goal }, 
    			text align = {align = center},
    			raise = -0.7ex
    			}
    		}
    	]
    	\tikzstyle{myarrowr} = 
    	[line width = 3.2mm,
    	draw = {rgb:black,1;white,2},
   		shorten >=-5mm,
   		shorten <=-4mm,
    	-triangle 45,
    	postaction = {
    		draw, 
    		line width = 5.3mm, 
    		shorten >=1.5mm,
    		-},
    	postaction = {
    		decorate,
    		decoration = {
    			text along path, 
    			text = {|\large \color{white}|Representations}, 
    			text align = {align = left},
    			raise = -0.6ex
    			}
    		}
    	]
    		    	
        \matrix (m)[
        matrix of nodes,
        row sep=3em,
        column sep=0pt,
		ampersand replacement=\&
        ]
        {
        	\parbox{4.5cm}
        	{
        	  \begin{center}
        		{\large
        			\textbf{Classical}
        		}
        		\\
        		{\normalsize
        			Smooth functions on
        			a Poisson manifold
        		}
        	  \end{center}
        	}
        	\&
        	\&
        	\parbox{4.5cm}
        	{
        	  \begin{center}
        		{\large
        			\textbf{Quantum}
        		}
        		\\
        		{\normalsize
        			Self-adjoint operators
        			on a Hilbert space
        		}
        	  \end{center}
        	}
        	\\
        	\&
        	\parbox{4.5cm}
        	{
        	  \begin{center}
        		{\large
        			\textbf{Intermediate}
        		}
        		\\
        		{\normalsize
        			Abstract algebra
        		}
        	  \end{center}
        	}
        	\&
        	\\
        };
        \draw
        [-stealth]
        (m-1-1) edge[myarrowg]					(m-1-3) 
        		edge[myarrowq, bend right=30]	(m-2-2)
        (m-2-2) edge[myarrowr, bend right=30]	(m-1-3)
		;
    \end{tikzpicture}
\end{center}


\subsubsection{Different approaches}
There exist various ideas about how to build up such a scheme. For example, in 
axiomatic quantum field theory one usually wants the quantum algebra to be a 
$C^*$-algebra (see for example \cite{haag:1993a} or 
\cite{baer.ginoux.pfaeffle:2007a}). The reason is that the bounded operators 
$\Bounded(\hilbert{H})$ naturally form such an algebra (and actually even more 
than that). Unfortunately, most of the operators in quantum mechanics are 
unbounded. This problem is cured by looking at the exponentiated operators
\begin{equation*}
	A
	\longmapsto
	\E^{i t A},
	\quad t \in \mathbb{R}.
\end{equation*}
This way, one gets a one parameter group for $t \in \mathbb{R}$ and unitary 
operators are clearly bounded. Moreover, there is a very nice correspondence 
between $C^*$-algebras and locally compact Hausdorff spaces, which is known as 
the Gelfand-Naimark theorem. Roughly stated, this means that there is an 
equivalence of categories between compact Hausdorff spaces and unital commutative 
$C^*$-algebras: every such space gives rise to a commutative algebra 
of complex-valued functions, which form a $C^*$-algebra. On the other hand, from 
every unital commutative $C^*$-algebra $\algebra{A}$ one can construct a 
compact Hausdorff space $X$ such that $\algebra{A} \cong \Stetig(X)$ . This 
correspondence can be extended to open and closed subsets of the space, to 
homeomorphism, locally compact spaces, their compactification, vector bundles and 
even more. There is a strong link between commutative algebra and topology or 
geometry. One possibility to think of a quantized system is to think of continuous 
or smooth functions on a noncommutative space, which then should correspond to a 
noncommutative $C^*$-algebra. This idea leads to noncommutative geometry, which 
is mostly due to Alain Connes. A very detailed, but not necessarily easy to read 
book is \cite{connes:1994a} by Connes, another and rather brief introduction is 
for example \cite{varilly:2006a}.


A different approach, called geometric quantization, tries to fulfil all 
of the three axioms (Q1) - (Q3) from the previous part. Unfortunately, this 
causes problems: Already for a symplectic vector space, it is impossible to have a 
$1$ to $1$ correspondence of the Poisson bracket and the quantum mechanical 
commutator. This is known as the Groenewold-van Hove theorem, which was found 
around 1950 \cite{vanhove:1951a, groenewold:1946a}. More precisely spoken: no 
representation of the Lie algebra, which is generated by the $q^i$ and $p_j$ and 
which is defined by the classical commutation relations, can be extended 
irreducibly and faithfully to the commutator Lie algebra which comes from the 
associative unital algebra which is generated by the $q^i$ and $p_j$. Thus 
geometric quantization restricts to smaller observable algebras, which are not 
problematic. This ideas are mostly due to Souriau \cite{souriau:1970a}, Kostant 
and Segal.


There are also different approaches. Berezin proposed a quantization scheme for 
particular K\"ahler manifolds, \cite{berezin:1975a, berezin:1975b, berezin:1975c}. 
Still, new ideas keep coming up, but in the following, we want to concentrate on a 
particular type of quantization, which is called \emph{deformation quantization}.



\section{Deformation Quantization}
\label{sec:chap2_DQ}

\subsection{The Concept}
\label{subsec:chap2_Concept}

Deformation quantization tries to realize the three points (Q1) and (Q2) from the 
previous section, but weakens the third. If it is not possible to have such a 
correspondence exactly, we will at least want to have it 
\emph{asymptotically}. The motivating example is the Weyl quantization which we 
already talked of. There are actually two such formulas, that can be given: the 
first maps a function $f$ in the variables $q, p \in \mathbb{R}^n$ to a 
differential operator on $\mathbb{R}^n$, which is actually a formal power series 
in the parameter $\hbar$. For polynomial functions, this is series is a sum and 
well defined, for general functions this will really be a formal power series. 
The second is given by an integral formula and holds for another class of 
functions (Schwartz functions), but one gets the first formula out of the second 
as an asymptotic expansion for $\hbar \longrightarrow 0$. With these 
quantizations, one can also define a product of two functions $f, g$, which will 
necessarily take those two functions to a formal power series in $\hbar$. Moyal 
showed that the commutator of this product can be understood as a series which 
approximates the quantum mechanical commutator \cite{moyal:1949a}. The reason why 
seemingly all of a sudden power series appear is the following: if one wants the 
correspondence principle to be asymptotically fulfilled, i.e.
\begin{equation*}
	[\mathcal{Q}(f), \mathcal{g}]
	=
	i \hbar \mathcal{Q}(\{ f, g \})
	+ \mathcal{O}(\hbar^2)
\end{equation*}
and the multiplication of these quantized functions to be associative 
(as needed for representations on the Hilbert space), one will necessarily get 
higher and higher orders in $\hbar$. This iteration can never be stopped without 
loosing associativity. Motivated by this observation, a group of mathematicians, 
the so called Dijon-school, started working out this idea of products as formal 
power series. They understood quantization as a \emph{deformation} of the 
commutative product by a formal parameter (mostly called $\hbar$, $\lambda$ or 
$\nu$, in this work we will call it $z$ from now on), which controls the 
noncommutativity of the theory. These deformed products should moreover fulfil 
some compatibility conditions with the classical theory. This was the 
hour of birth of deformation quantization. The main characters of this group 
were Flato, Lichnerowicz, Bayen, Fr{\o}nsdal and Sternheimer, who published 
their ideas in the late 70's \cite{bayen.et.al:1977a, bayen.et.al:1978a} and 
gave a first definition of a star product. These two articles became the starting 
point for what has now become a rich and fruitful theory.
\begin{definition}[Star Product]
	\label{Def:StarProduct}
	Let $(M, \{\cdot, \cdot\})$ be a Poisson manifold over a field 
	$\mathbb{K}$ ($\mathbb{K} = \mathbb{R}$ or $\mathbb{C}$).
.	A star product on $M$ is a bilinear map
	\begin{equation*}
	    \star_z \colon 
    	\Cinfty(M) 
    	\times 
	    \Cinfty(M) 
	    \longrightarrow
	    \Cinfty(M) \llbracket z \rrbracket
	    , \
	    (f,g) 
	    \longmapsto 
	    f \star_z g 
	    =
	    \sum\limits_{n = 0}^\infty 
	    z^n C_n(f,g)
	\end{equation*}
	such that its $\mathbb{K}\llbracket z \rrbracket$-linear extension to 
	$\Cinfty(M) \llbracket z \rrbracket$ fulfils the following properties:
	\begin{definitionlist}
		\item
		$\star_z$ is associative.
		
		\item
		$C_0(f, g) = f \cdot g,\ \forall_{f,g \in \Cinfty(M)}$ (Classical limit).
		
		\item
		$C_1(f, g) - C_1(g, f)= z \{f, g\},\ \forall_{f,g \in \Cinfty(M)}$
		(Semi-classical limit).
		
		\item
		$1 \star_z f = f \star_z 1 = f,\ \forall_{f \in \Cinfty(M)}$.
	\end{definitionlist}
	If the $C_n$ are bidifferential operators, the star product is said to be 
	differential and if the order of differentiation of the $C_n$ does not 
	exceed $n$ in both arguments, a differential star product is said to be 
	natural. Moreover, we will say that a star product $\star_z$ is of Weyl-type,
	if $\cc{f \star_z g} = \cc{g} \star_z \cc{f}$ for all $f,g \in \Cinfty(M)$
	and $\cc{\cdot}$ denotes the complex conjugation.
\end{definition}
They also defined a notion of equivalence of star products. The idea behind is 
that two equivalent star products should give rise to the same physics.
\begin{definition}[Equivalence of Star Products]
	Two star products $\star_z$ and $\widehat{\star}_z$ for a Poisson manifold 
	$(M, \{\cdot, \cdot\})$ are said to be equivalent, if there is a formal 
	power series
	\begin{equation*}
		T
		=
		\id +
		\sum\limits_{n=0}^{\infty}
		z^n T_n
	\end{equation*}
	of linear maps $T_n \colon \Cinfty(M) \longrightarrow \Cinfty(M)$, which 
	extends $\mathbb{K}\llbracket z \rrbracket$-linearly to 
	$\Cinfty(M) \llbracket z \rrbracket$, such that the following statements hold:
	\begin{equation*}
		f \star_z g
		=
		T^{-1}
		\left(
			T(f) \widehat{\star}_z T(g)
		\right)
		, \ 
		\forall_{f,g \in Cinfty(M) \llbracket z \rrbracket}
		\quad \text{ and } \quad
		T(1) 
		= 
		1.
	\end{equation*}
	For differential or natural star products, we accordingly speak of 
	differential or 	natural equivalences.
\end{definition}
Note that these definitions are purely algebraic, since we do not ask for the
convergence of those power series. Hence the theory which was developed from 
this in the following years is also a mostly algebraic theory. Like for the Weyl 
product, there were also integral formulas around for other types of star products 
for which one can also make sense of convergence. But as already pointed out, one 
has to strongly restrict the algebra of functions, for example to the Schwartz 
space.



\subsection{A Mathematical Theory}
\label{subsec:chap2_MathTheory}
Deformation quantization is a good example for a mathematical theory, which is 
motivated by a physical idea. It is not really talking about a physical problem, 
since the world does not need to be quantized -- it already is. It is talking 
about a mathematical problem: how to recover the quantized (and very 
counter-intuitive) mathematics, which describe the world on very small scales out 
of the classical (and much more intuitive) mathematics, which describe the world 
on our scale? Deformation quantization tries to give an answer to that, but is 
unfortunately (at least at its present state) still far from doing so completely. 
Like many such theories, it started of from a more or less precise physical 
background and developed into something very different: a mostly algebraic, purely 
mathematical theory. That is also due to the fact that the questions, which had to 
be answered in the beginning, were very hard and of mathematical nature. It took a 
lot of time to find answers and meanwhile, the mathematicians working on them
were interested other aspects of the theory. We want to give a short overview of
those questions and their answers very briefly and summarize a bit the history of 
deformation quantization next. A more detailed summary can be found in section 6.1 
of \cite{waldmann:2007a}.


The prototype is, as already mentioned, a symplectic vector space, for which 
Weyl proposed a star product with a certain (symmetric) ordering (although the 
definition of a star product did not exist at his time). However, other 
orderings are possible: one can have a standard or an anti-standard ordering, 
where the $\hat p$'s are all ordered to the right or to the left, respectively, or 
something in  between. One of the first questions was, if one could also construct 
star products on symplectic manifolds and if these products will be standard or 
Weyl ordered, if they will be differential or natural and so on. Locally, the 
answer was yes, but it took some time and many small steps, until DeWilde and 
Lecomte could show that every cotangent bundle of a smooth manifold has star 
products \cite{dewilde.lecomte:1983a} and then extended this result to arbitrary 
smooth manifolds \cite{dewilde.lecomte:1983b}. Another proof was given 
independently from that by Omori, Maeda and Yoshioka 
\cite{omori.maeda.yoshioka:1991a} and then by Fedosov, who presented a simple 
and very geometric construction \cite{fedosov:1994a}, which always give rise to 
natural star products, as shown in \cite{bordemann.waldmann:1997a} 
or more generally in \cite{gutt.rawnsley:2003a}. Moreover, every star product on a 
symplectic manifold is equivalent to a Fedosov star product 
\cite{bertelson.cahen.gutt:1997a}. A lot of 
results were found for K\"ahler manifolds and also the already mentioned 
procedure, which is due to Berezin, gives rise to star products. There are 
moreover standard, anti-standard ordered and many other types of star products 
on every cotangent bundle. The next question was the one concerning the 
equivalence classes of star products in the symplectic case. One can show that 
locally, two star products on a symplectic manifold are always equivalent. Hence 
a classification result should depend on global phenomena. Indeed, this is the 
case and it can be shown that star products on a symplectic manifold are 
classified by its second deRham cohomology $\HdR^2(M)$. This result is due to 
Deligne \cite{deligne:1995a}, a different proof was given by Nest and Tsygan 
\cite{nest.tsygan:1995a, nest.tsygan:1995a}, another by Bertelson, Cahen and Gutt 
\cite{bertelson.cahen.gutt:1997a}. More exactly: the choice of an equivalence 
class of closed, nondenerate 2-forms $\omega \in \Formen^2(M)$ determines a 
Fedosov star product and from every Fedosov 
star product one can calculate such an equivalence class. This already 
determines all star products on symplectic manifolds, since every star product 
on such a manifold is equivalent to a Fedosov star product. The case of Poisson 
manifolds took longer and was much harder to solve, since associativity turned 
out to be a complicated condition to fulfil. There were some examples of star 
products known for particular Poisson structure, like the Gutt star product 
\cite{gutt:1983a}, which was also found by Drinfel'd \cite{drinfeld:1983a} 
idependently, but the general existence (and also the classification) result was 
proven by Kontsevich \cite{kontsevich:1997:pre, kontsevich:2003a} many years 
later. His classification result is known as the formality theorem and needs the 
notion of $L_{\infty}$-algebras, which are fairly involved objects. He gave an 
explicit construction, how star products can be built out of Poisson brackets on 
$\mathbb{R}^d$. This construction was extended by Cattaneo, Felder and Tomassini 
to Poisson manifolds \cite{cattaneo.felder.tomassini:2002b} and indepedently from 
that by Dolgushev \cite{dolgushev:2005a}. Another and easier formulation of the 
Kontsevich construction on $\mathbb{R}^d$ in terms of operads was later given by 
Tamrakin \cite{tamarkin:2003a}.



\subsection{From Formal to Strict}
\label{subsec:chap2_Formal2Strict}

So far, one could say that the big cornerstones of the theory are already there 
and that it is somehow ''finished''. For two reasons, this is not the case. 
First, a mathematical theory is never actually ''finished'', since there are 
always a lot of new things which can be found. There are still many different 
types of star products to classify, like invariant or equivariant star products 
in the case that one has Lie group or Lie algebra actions. A very recent result 
concerning the classification of equivariant star products on symplectic 
manifolds is, for example, due to Reichert and Waldmann 
\cite{reichert.waldmann:2015a:pre}. Second, the theory of deformation 
quantization still has a different aspect: all we talked about so far was purely 
algebraic and there is no notions of convergence of these formal power series. 
If some day, this theory shall have a real drawback on physics, it will be 
necessary to talk about the convergence properties of these star products, since 
in physics $\hbar$ is \emph{not} a formal parameter but a constant with a 
dimension and a fixed value and therefore the question of convergence matters. 
When we dace those problems and speak about continuous star products, we leave the 
field of \emph{formal} deformation quantization and come to \emph{strict} 
deformation quantization.


Although it is closer to physics, strict deformation quantization is still an 
mathematical theory. There are two different approaches to it: we 
already mentioned integral formulas, which allow to speak of continuous star 
products. The second approach uses the formal power series instead and wants to 
construct a topology on the polynomial algebra, such that the star product becomes 
continuous. Then one completes the tensor algebra over the vector space to a 
subalgebra of the smooth functions, on which the star product will still be 
continuous.


The first approach is mostly due to Rieffel, who developed these ideas in some of 
his papers \cite{rieffel:1989a, rieffel:1990c, rieffel:1993a}. He wants to 
realize a strict deformation quantization by actions of an abelian Lie group on a 
$C^*$-algebra of classical observables. Later Rieffel formulated a list of open 
questions, which strict deformation quantization should try to answer 
\cite{rieffel:1998a} in the next years. His approach was carried on by Bieliavsky 
and Gayral \cite{bieliavsky:2002a, bieliavsky.gayral:2015a}, who extended these 
concepts to much more general Lie groups and different manifolds. To get 
reasonably big observable algebras, they used oscillatory integrals and pushed 
this theory forward. A similar idea was realized by Natsume, Nest and 
Peter \cite{natsume.nest.peter:2003a}, who could show that under certain 
topological conditions, symplectic manifolds always admit strict deformations.


The second approach is due to Beiser and Waldmann 
\cite{beiser:2011a, beiser.waldmann:2014a, waldmann:2014a}. They restrict to the 
local situation, that means to Poisson structures on vector spaces. Then, they 
look at the polynomial functions on this vector spaces and try to find 
continuity estimates for them by constructing an explicit locally convex 
topology on the symmetric tensor algebra (which is isomorphic to the polynomial 
algebra). The aim is to make the topology as coarse as possible, to get then a 
large completion and hence a big quantized algebra of 
observables. There are two big advantages of this approach: the first one is 
that it can be applied to infinite dimensional vector spaces, what is necessary 
for quantum field theory, which deals with infinitely many degrees of freedom. 
The second is that we can really speak about all observables, also those 
which will correspond to unbounded operators, without having to exponentiate. In 
this sense, this idea is somehow more fundamental. The disadvantage is, however, 
that it is just a local theory at the moment. The idea is worked out just for 
one type of star products by now, which are star products of exponential type like 
the Weyl product. This means until now one can only control star products on 
symplectic vector spaces which come from a constant Poisson tensor 
\cite{waldmann:2014a}. This theory was carried on in the master thesis of Matthias 
Sch\"otz \cite{schoetz:2014a}, who rephrased it using semi-inner product, which 
are a somehow more physical notion, since one can interpret spaces with such a 
topology as projective limits of pre-Hilbert spaces. This also allows a slightly 
coarser topology and hence a larger completion of the symmetric tensor algebra.


In this work, we will follow the second approach and apply it to 
another type of star product, the Gutt star product, which comes from a linear 
Poisson tensor on a vector space. Of course, this is the next logical step after 
constant Poisson tensors. However, these are also the first \emph{non-symplectic}
Poisson structures, which will be strictly quantized this way. Thus this 
master thesis really contributes something new to the theory of strict 
deformation quantization: a second example, in which Waldmann's locally convex 
topology on the tensor algebra leads to a continuous star product, when 
considered as a power series and not as an integral. Note that this also has a 
certain effect on Lie theory: the result can be seen as a functorial 
construction for a locally convex topology on the universal enveloping algebra 
$\algebra{U}(\lie{g})$ of a (possibly infitely-dimensional) Lie algebra 
$\lie{g}$. Therefore they may have applications to, for example, the 
representation theory of $\algebra{U}(\lie{g})$.
