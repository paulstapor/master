
%
% Chapter 2 of my master thesis:
% Still quiete a lot introductory
%

\chapter{Deformation quantization}

The starting point for every theory of quantization is surely mechanics. Here, 
mechanics means both, the classical and the quantum theory. Since we want to 
explain how one can link them together, we will give a short overview of both 
theories and show their similarities and their differences in the first section 
of this chapter. Then we will come to different aspects of quantization and 
state the most important ideas in the next part. In section three, we will 
focus on Deformation quantization and give an overview of this rather young 
theory.



\section{Mechanics: The Classical and the Quantum World}
\label{sec:chap2_Mechanics}

\subsection{Classical Mechanical Systems}
\label{subsec:chap2_Classical}
We want to recall briefly the notions of classical mechanics. There are many 
good books on this subject and also the notation is mostly uniform everywhere, 
but nevertheless want to refer to the books of Marsden and Ratiu 
\cite{marsden.ratiu:1999a} and Arnold \cite{arnold:1989a}, which give very good 
introductions and overviews of the whole theory.
Imagine the simplest model for a mechanical system, which is not trivial: a 
single particle with mass $m$ in $\mathbb{R}^3$ in a scalar potential $V$. We 
will denote his position by $q = (q^1, q^2, q^3) \in Q = \mathbb{R}^3$ and call 
the set of all possible positions $Q$ the configuration space. Since we also 
have a time coordinate $t \in \mathbb{R}$, we can describe the path on which 
the particle moves by $q(t)$. The state of the particle is completely described 
by its position and its velocity $(q(t), \dot q(t))$ and the velocity should be 
understood as a tangent vector $\dot q(t) \in T_{q(t)}Q = \mathbb{R}^3$. 
Therefore, the tangent bundle $TQ$, which is in this case $\mathbb{R}^6$, is 
sometimes called the state space. In classical mechanics, we can describe the 
movement of the particle with the Euler-Lagrange equations, which are derived 
from the Lagrange function by a variational principle. The Lagrange function of 
this system reads
\begin{equation*}
	\mathcal{L}(q, \dot q)
	=
	T(q, \dot q) - V(q, \dot q)
\end{equation*}
and $T(q, \dot q) = \frac{1}{2} m  \sum_{i=1}^3 \dot{q^i}^2$ is the kinetic 
energy. The action 
is defined as the integral
\begin{equation*}
	\mathcal{S}(q, \dot q)
	=
	\int\limits_{t_0}^{t_1}
	\mathcal{L}(q, \dot q)
	dt.
\end{equation*}
It is a physical observation, similar to a mathematical axiom, that this action 
functional is minimized along the trajectories of the particle. So fixing a 
starting point $q_0$ and an end point $q_1$ of a trajectory, we find
\begin{equation*}
	\delta \mathcal{S}
	\at{q(t_0) = q_0, q(t_1) = q_1}
	=
	0.
\end{equation*}
From this, one finds the Euler-Lagrange equations for $i = 1, 2, 3$
\begin{equation*}
	\frac{d}{dt} 
	\frac{\partial \mathcal{L}}{\partial \dot q^i}
	-
	\frac{\partial \mathcal{L}}{\partial q^i}
	=
	0.
\end{equation*}
In our case this means
\begin{equation*}
	\frac{d}{dt} 
	\frac{\partial T}{\partial \dot q^i}
	-
	\frac{\partial V}{\partial q^i}
	=
	m \ddot q
	-
	\frac{\partial V}{\partial q^i}
	=
	0
\end{equation*}
and we can integrate the equations to get the trajectory. So far, we described 
the system in the state space and for obvious reasons, this formalism is called 
the Lagrange formalism. It is not the only formalism, which describes to 
behaviour of the system: one can go to the so called Hamilton formalism, which 
is based on the conjugate momenta $p = (p_1, p_2, p_3) \in T_{q(t)}^*Q$
\begin{equation*}
	p_i(t)
	=
	\frac{\partial \mathcal{L}}{\partial \dot q^i}
\end{equation*}
and linked to the Lagrange formalism by a Legendre transformation. Now we can 
define the Hamilton function
\begin{equation*}
	H(q, p)
	=
	p_i q^i
	-
	\mathcal{L}(q, \dot q),
\end{equation*} 
which is the crucial quantity in this setting. It represents the energy of the 
system. On finds the equations
\begin{equation*}
	\frac{\partial H}{\partial p_i}
	=
	\frac{d q^i}{dt}
	\quad \text{ and } \quad
	- \frac{\partial H}{\partial q^i}
	=
	\frac{d p_i}{dt}.
\end{equation*}
They remind strongly of a symplectic structure and indeed this is the case: 
using the standard sympectic matrix
\begin{equation*}
	\omega
	=
	\begin{pmatrix}
		0 & \Unit
		\\
		- \Unit & 0
	\end{pmatrix}
\end{equation*}
one finds
\begin{equation*}
	\frac{d}{dt} \vector{q^i}{p_i}
	=
	\omega
	\vector{\frac{\partial H}{\partial q^i}}
	{\frac{\partial H}{\partial p_i}}.
\end{equation*}
More generally, one can define the Poisson bracket $\{f, g\}$ for two functions 
$\Cinfty(T^*Q)$ by
\begin{equation*}
	\{f, g\}
	=
	\sum\limits_{i=1}^3
	\left(
		\frac{\partial f}{\partial q^i}
		\frac{\partial g}{\partial p_i}
		-
		\frac{\partial f}{\partial p_i}
		\frac{\partial g}{\partial q^i}
	\right)
\end{equation*}
and finds the time evolution of a function $f$ given by
\begin{equation*}
	\frac{d}{dt} f
	=
	\{ f, H \}.
\end{equation*}
All of those objects have, of course, a geometrical interpretation which remain 
true for and allow generalizations from this very easy example. For a more 
complex system than one particle in a scalar potential, the configuration space 
$Q$ will be a smooth manifold, the state space $TQ$ is described by the tangent 
bundle and the phase space $T^*Q$ by the cotangent bundle. The transition from 
the Lagrange to the Hamilton formalism is done by a fiber derivative, the 
kinetic energy has a mathematical interpretation as a Riemannian metric and the 
Poisson bracket is the one which is due to the canonical symplectic form on the 
cotangent bundle. So in some sense, classical mechanics can be described by 
symplectic geometry.


The last conclusion, however, was a bit too fast. There are mechanical systems 
like the rigid body, which can not be described in the symplectic formalism, 
or which have certain symmetries and therefore allow a \emph{reduction}. A 
symmetry can mathematically be described by a Lie group $G$, which acts on 
$T^*Q$ and a reduction by a momentum map $J \colon T^*Q \longrightarrow 
\lie{g}^*$, where $\lie{g}$ is the Lie algebra of the Lie group and $\lie{g}^*$ 
its dual. The reduced phase space is a quotient of the cotangent bundle and 
does not need to be symplectic any more. Anyway, in both cases, the result will 
be something more general than a symplectic manifold: it will be a Poisson 
manifold. 
\begin{definition}[Poisson Manifold]
	A Poisson manifold is a tuple $(M, \{ \cdot, \cdot \})$ of a smooth 
	manifold $M$ and a Poisson bracket $\{ \cdot, \cdot\}$. The bracket is a 
	biderivation
	\begin{equation*}
		\{ \cdot, \cdot \}
		\colon
		\Cinfty
		\times
		\Cinfty(M)
		\longrightarrow
		\Cinfty(M),
	\end{equation*}
	which is anti-symmetric and fulfils the Jacobi identity
	\begin{equation*}
		\{ \{f, g\}, h \}
		+
		\{ \{g, h\}, f \}
		+
		\{ \{h, f\}, g \}
		=
		0.
	\end{equation*}
\end{definition}
A classical mechanical system will be described as a Poisson manifold together 
with a Hamilton function $H$, such that the time derivation of every function 
$f \in \Cinfty(M)$ is given by the relation
\begin{equation*}
	\frac{d}{dt} f(t)
	=
	\{f, H\}.
\end{equation*}
The observable algebra of this classical system is then modelled as the algebra 
of smooth functions $\Cinfty(M)$ on this Poisson manifold.



\subsection{Quantum Mechanics}
\label{subsec:chap2_Quantum}

In quantum theory, the modelling of a physical system looks completely 
different on the first sight. There is indeed no obvious way to describe how 
the transition from classical to quantum mechanics happened, but it was a hard 
and non-straightforward way which was taken in small steps of which some were 
more or less obvious and some were ingenious guesses. Built up on first ideas 
by Max Planck, who described the radiation of a black body, Albert Einstein, 
who explained the photo-electrical effect and Niels Bohr, who solved the 
problem of atomic spectra, a hand full of physicists including Max Born, 
Werner Heisenberg, Erwin Schr\"odinger, Paul Dirac, Wolfgang Pauli and Pascual 
Jordan developed a non-intuitive and complex but extremely well working theory, 
which was able to give precise answers to very most of the open questions of 
physics at that time. However, this theory, mainly risen between 1925 and 1928, 
did not have a satisfying mathematical fundament. It took around ten more 
years, until mathematicians, mainly John von Neumann, worked out a mathematical 
theory for this and gave also a physical interpretation to their mathematical 
formulation \cite{vonneumann:1996a}. Today, quantum mechanics is usually 
introduced as a mathematical theory based on axioms and most of the textbooks 
(at least most of the more mathematical ones) use this axiomatical approach to 
explain the theory. Two very nice mathematical introductions are given by 
Bongaarts \cite{bongaarts:2015a} and Hall \cite{hall:2013a}.


The \emph{state} of a physical system is described by a vector $\psi$ on a 
Hilbert space $\hilbert H$ and the \emph{observables} are self-adjoint (and 
usually unbounded) operators on it. The spectrum of these operators describes 
the \emph{measurable values}. The theory is probabilistic with a deterministic 
time development: the pair ($A$, $\phi$) gives a probabilistic distribution, 
from which we can calculate the probability to measure $A$ with the value $a 
\in \spec (A)$ in the state $\phi$ using the spectral resolution of $A$. We 
want to illustrate this again using the example of a particle in $\mathbb{R}^3$ 
moving in a potential. Let $x = (x_1, x_2, x_3) \in \mathbb{R}^3$ be a 
coordinate vector, then we have the so called Schr\"odinger representation 
given by the Hilbert space $\hilbert{H} = \Lzwei(\mathbb{R}^3)$ of square 
integrable functions with the Lebesgue measure. The state $\phi$ of a particle 
is a time-variant element of this Hilbert space which fulfils the 
Schr\"odinger equation
\begin{equation}
	\label{QM:Schroedinger}
	i \hbar \frac{d}{dt} \psi(x, t)
	=
	\hat{H}
	\psi(x, t)
\end{equation}
where $\hat{H}$ denotes the Hamilton operator which is given by
\begin{equation}
	\hat{H}
	=
	- \frac{\hbar^2}{2 m}
	\Laplace + V(x).
\end{equation}
Similar to the Hamilton function in classical mechanics, the Hamilton operator 
(or more precise: its spectrum) gives the energy of this system, $m$ is the 
mass of the particle and $V$ the potential. The operators which correspond to 
the coordinate $x_i$ and the momentum $p_j$ of the particle are given by
\begin{align*}
	\left( \hat{q}_i
	\psi \right) (x, t)
	&=
	x_i
	\psi(x, t)
	\\
	\left( \hat{p}_j
	\psi \right) (x, t)
	&=
	- i \hbar 
	\frac{\partial}{\partial x_j}
	\psi(x, t).
\end{align*}
This turns Equation \eqref{QM:Schroedinger} into
\begin{equation*}
	i \hbar \frac{d}{dt} \psi(x, t)
	=
	\left( \frac{\hat{p}^2}{2m} + V(x) \right)
	\psi(x, t)
\end{equation*}
and makes it therefore similar to the classical time evolution.
Since the  state is a function of the coordinate $x$, one calls this the 
representation in \emph{position space}. Sometimes it is more convenient to 
describe the state in terms of its momentum and one changes to \emph{momentum 
space} via a Fourier transformation. Both representations are equivalent, very 
similar to the variables $q$ and $p$ in the Hamilton form of classical 
mechanics, from which this picture is very much inspired. Unlike the classical 
position and momentum observables being functions, the quantum mechanical 
observables are \emph{operators} and do not commute any longer. One can check 
these by calculating commutators and finds the so called Heisenberg or 
\emph{classical commutations relations}
\begin{align}
	[p_i, p_j]
	& =
	[q_i, q_j]
	=
	0
	\\
	[q_i, p_j]
	& =
	i \hbar \delta_{ij}.
\end{align}
These relations are the starting point for all quantization theories. The main 
goal is always to find an algebra which fulfils these (and maybe other) 
relations and this is what makes a quantum algebra so different from a 
classical one. Physically spoken, this noncommutativity means that a 
measurement of the position of a particle influences its momentum and vice 
versa, such that it becomes impossible to measure both, position and momentum, 
at the same time with arbitrarily high precision. This still very vague 
statement can be made precise using the Hilbert norm and the Cauchy-Schwarz 
inequality and yields the famous Heisenberg uncertainty relation
\begin{equation*}
	\Laplace a_\psi
	\Laplace b_\psi
	\geq
	\frac{1}{2}
	\left|
		\sp{\psi, [a,b] \psi}
	\right|.
\end{equation*}
Here $a,b$ are observables and $\Laplace a_\psi, \Laplace b_\psi$ denotes the 
standard deviation in the state $\psi$. For the position and momentum of a 
particle, this gives
\begin{equation*}
	\Laplace p
	\Laplace q
	\geq
	\frac{\hbar}{2}.
\end{equation*}
However, the Schr\"odinger picture is not the only possibility to describe the 
behaviour of a quantum system. Heisenberg proposed a description in which not 
the particles, but the observables are time variant. Since time evolution is 
described by a one-parameter group of unitary operators we have
\begin{equation*}
	\psi(x,t)
	=
	U(t - t_0) \psi(x, t_0),
\end{equation*}
at least in systems where the interactions are not explicitly time dependent. 
This makes it of course possible to apply unitary transformations to the states 
as well as to the operators 
\begin{equation*}
	\psi(x)
	=
	\psi(x, t_0)
	=
	U(t_0 - t) \psi(x, t)
	\quad \text{ and } \quad
	A(t)
	=
	U(t_0 - t)
	A
	U(t - t_0).
\end{equation*}
this gives the so called \emph{Heisenberg picture}, in which the 
time-dependence is not expressed ion the states but in the operators. Both 
pictures are equivalent and it is just a question of convenience which one is 
used. The unitary operators can be given more explicitly as
\begin{equation*}
	U(t - t_0)
	=
	\E^{- \frac{i}{\hbar} \hat{H} t},
\end{equation*}
since they describe the time evolution of solutions of the Schr\"odinger 
equation. If one differentiates an operator with respect to $t$ in the 
Heisenberg picture, one finds the \emph{Heisenberg equation}
\begin{equation*}
	\frac{d}{dt}
	A(t)
	=
	\frac{1}{i \hbar}
	[A(t), H].
\end{equation*}
This reminds us of course of the time evolution in terms of Poisson bracekts in 
classical mechanics., which is the second staring point for a quantization 
theories.



\subsection{The Correspondence Principle}
\label{subsec:chap2_Correspondence}

We have had a brief overview over the mathematical formulation of classical and quantum mechanics. On one hand, if we compare the description of the state of a physical system (a point in the cotangent bundle of a manifold and a vector in a Hilbert space) we will see that they are very different. On the other hand, we have seen that both theories have certain things in common: the time evolution can be described in a similar way and we have position and momentum operators in the quantum theory which play a similar role to the position and momentum functions in the classical theory, except for their noncommutativity. Let us give a short list of the main concepts of both theories and compare them (a much more detailed list and discussion can be found in \cite{waldmann:2007a}):
\bgroup
\renewcommand{\arraystretch}{1.6}
\begin{center}
	\begin{tabular}
	{lll}
		~ 
		&
		\textbf{Classical} 
		&
		\textbf{Quantum}
		\\
		\parbox{4cm}
		{
			Observables
		}
		&
		\parbox{5cm}{
			Poisson-*-algebra $\algebra{A}_{Cl}$
			of \\
			smooth functions $\Cinfty(M)$ \\
			on a Poisson manifold
		}
		&
		\parbox{5cm}{
			*-algebra $\algebra{A}_{QM}$
			of self-adjoint operators
			on a Hilbert space $\hilbert{H}$
		}
		\\
		\parbox{4cm}
		{
			Measurable 
			Values
		}
		&
		$\spec (f) \subseteq \mathbb{R}$
		&
		$\spec (A) \subseteq \mathbb{R}$
		\\
		States
		&
		\parbox{5cm}{
			Points in the
			phase space
		}
		&
		\parbox{5cm}{
			Vectors in a
			Hilbert space
		}
		\\
		Time evolution
		&
		Hamilton function $H$
		&
		Hamilton operator $\hat{H}$
		\\
		\parbox{4cm}{
			Infinitesimal\\
			time evolution
		}
		&
		$
		\frac{d}{dt} f(t)
		=
		\{f(t), H\}
		$
		&
		$\frac{d}{dt} A(t)
		=
		\frac{1}{i \hbar}
		[A(t), H]
		$
	\end{tabular} 
\end{center}
\egroup

This little table gives us a good hint: if one really wants to compare both theories, then we should understand the observable algebras as their main concepts rather than the states. We know that the classical theory emerges as a limit of many quantum system, such that the physical constant $\hbar$ becomes small enough to be neglected. How to construct correspondence the other way round is the question to which the theory of quantization addresses. Physicists used to solve this by "making a hat on the variables $p$ and $q$ and saying they the do not commute any more". Surprisingly, this approach, which is called \emph{canonical quantization} in physics works for most of the relevant cases, which is namely when we do not have high powers of $\hat p$ and $\hat q$ and not mixed terms. That this idea is however not canonical in a \emph{mathematical sense} is almost needless to say and of course it causes severe problems when higher and mixed terms in position and momentum appear. Those problems can be made precise and is known as the Groenewold-van Hove theorem (\cite{vanhove:1951a, groenewold:1946a}). To create a mathematical theory of quantization, other ways are needed and one needs to impose an ordering on the $\hat p$ and $\hat q$. Already Hermann Weyl knew about this problem and looked for ways how to solve it by proposing a symmetrization \cite{weyl:1931a} which leads to an integral formula for such terms. This ordering is now known as the Weyl ordering and his integral formula became the starting point for the theory of deformation quantization any years later.



\section{Making Things Noncommutative: Quantization}
\label{sec:chap2_Quantization}

\subsection{The Task}
\label{subsec:chap2_Task}

First, we want to take a step back and formalize, what we actually mean by a quantization. A quantization should be a correspondence
\begin{equation*}
	\mathcal{Q}
	\colon
	\algebra{A}_{\mathrm{Classical}}
	\longrightarrow
	\algebra{A}_{\mathrm{Quantum}}
\end{equation*}
between a commutative algebra $\algebra{A}_{\mathrm{Classical}}$ and a noncommutative algebra $\algebra{A}_{\mathrm{Quantum}}$, which is a ''bijection'' in some sense: all classical observables appear as classical limits from quantum observables, hence we should expect $\mathcal{Q}$ to be ''injective''. On the other hand, if the Quantum algebra was bigger than the classical algebra, the whole concept of quantization would be pointless, since we could never recover all observables and thus we want $\mathcal{Q}$ to be ''surjective''. Our correspondence should also keep somehow the physical meaning of our observables, or at least tell us, what the observable $\mathcal{Q}(f)$ is for some $f \in \algebra{A}_{\mathrm{Classical}}$. Moreover, $\mathcal{Q}^{-1}$ should behave like the classical limit, since this is the rather well-known direction of both. Finally, the correspondence should keep associative structures: the classical algebra is commutative and we want the product in our noncommutative algebra to correspond to the concatenation of operators on a Hilbert space which is associative, too.

Finally, we split up the whole procedure: in a first step, we \emph{quantize} the system and in a second step, we look for \emph{representations} on a Hilbert space.


\subsection{Different approaches and Deformation Quantization}
\label{subsec:chap2_Deformation}



\section{Formal vs. Strict Deformation Quantization}
\label{sec:chap2_FormalStrict}

\subsection{A Mathematical Theory}
\label{subsec:chap2_MathTheory}
Example how Physics gives rise to new Math,
Including a short history,
Things got stuck somewhere


\subsection{From Formal to Strict}
\label{subsec:chap2_Formal2Strict}
Three steps: Formal -> Strict -> Representations
Concepts of strict DQ (Rieffel vs. Waldmann,
meaning C* vs. Locally convex)
So far everything is Math, Physics would start after the last step, Maybe one day...
