
%
% the Introduction of my master thesis
%

\chapter{Introduction}
\label{sec:Intro}

Throughout history, the fields of mathematics and physics have been 
closely linked to each other. The great physicists of the past have always been 
great mathematicians and vice versa: Carl Friedrich Gauss, for some persons the 
most brilliant mathematician, did not only find plenty of mathematical 
relations, prove a lot of theorems and develop many new ideas, which 
should become rich and fruitful new fields in later mathematics, he also has a 
large number of credits in physics: the recovery of the dwarf planet Ceres in 
astronomy, new results in electromagnetism (like the Gauss's law or a 
representation for the unit of magnetism, which was named after him) and the 
Gaussian lens formula in geometric optics are just some of his best known 
merits. Isaac Newton on the other hand may have been rather a physicist, but 
it was his so called second law of motion that introduced for the first time 
something like differential calculus and therefore opened up the door for new 
branches of mathematics. Even if one want claims that differential calculus 
was actually invented by Gottfried Wilhelm Leibniz, this does not change 
the basic observation, since Leibniz wrote a lot of essays on physics and can 
be considered as the inventor of the concept of kinetic energy (or, as he 
called it, the vis viva) and its conservation in certain mechanical 
systems. Of course one has to name Joseph-Louis Lagrange, who was an ingenious 
mathematician with rich contributions to number theory and algebra, but also to 
the fields of analytical mechanics and astronomy. Still today, every physics 
student has to learn his Lagrangian formalism in the lecture on theoretical 
mechanics and how one can derive the laws of motion for various mechanical 
systems from it. A last name we want to mention here is Paul 
Dirac, who is certainly one of the founding fathers of quantum mechanics. He 
provided the ideas for a lot of structures and relations in differential 
geometry, functional analysis and distribution theory. Many of the concept he 
introduced using his physicist's intuition were later proven to be correct or 
used as starting points for new theories by mathematicians.


A lot of developments in mathematics can be seen as triggered by physics: they 
were necessary to describe the physical behaviour of our world and therefore 
pushed forward by scientists. We already mentioned differential calculus, 
without whom modern analysis, the theory of ordinary or partial differential 
equations or differential geometry would not be possible. Besides the also 
named field of functional analysis, also Lie theory and many parts of geometry 
provide examples for mathematics which was inspired by physics. Of course, this 
correspondence is not a one way street since the understanding of nature made 
great progress due to a better knowledge of the mathematical laws that describe 
it. A good example therefor is Lebesgue's theory of integration and its 
application to quantum mechanics: the space $L^2(\mathbb{R}^{3n})$ is the state 
space of standard $n$-particle system in quantum mechanics.


There are good reasons to say that this extremely tight binding of mathematics 
and physics persisted until the 20th century. Without any doubt, those two 
areas are still closely linked, but one could say that at a certain point in 
history they started walking away from each other. Of course, there have always 
been mathematicians who did not take their motivation from physics and 
physicists who did not use elaborate mathematics or even find new theorems to 
describe aspects of the world around them, but for a long time, the vast 
majority of both groups showed at least an interest for the other domain. This 
definitely changed during the 20th century. The main reason for this can 
surely be found in the extremely fast development which both of the domains 
experienced in this time. It is already impossible for one person to overview 
the whole field of mathematics or the one of physics, since there are too many 
new things coming up every day and one has to specialize for being able to do 
research. Another reason is surely the fact that modern mathematics is 
strongly influenced by the desire to formulate things as clean as possible, 
without using handwaving ''physical arguments''. This is a principle which 
surely allowed many new and fruitful evolutions in the last decades and which 
is mostly due to the Bourbaki movement in the middle of the last century, but 
it also forgets about the fact that physical intuition was often a powerful 
tool for new ideas or also for heuristics which led to proofs of important 
theorems. Another reason, which is more situated in the domain of physics, is 
certainly the incredibly fast development of the knowledge about 
semiconductors. This became possible due to quantum mechanics which forms the 
foundation of this theory, but for very most of modern applications a basic 
understanding of the quantum theory behind is enough or one can even get new 
results with so called semi-classical approaches. Here, a lot of new results 
can be established without going deep into mathematics and hence without giving 
a new stimulus to it. In this sense, it is enough for many modern physicists to 
acquire a certain amount of mathematical knowledge and then they never have to 
care about mathematical theories again.


Certainly, this situation is due to a natural development does not present a 
problem, although it is a bit of a pity. However, it would be too much to say 
that those fields are falling apart: there are still a lot of intersections of 
the two sciences and these contact areas provide rich domains of research. The 
range of topics, where either mathematics takes its motivation from physics, 
or where theoretical physics needs very elaborate mathematical methods, is 
usually grouped under the name \emph{mathematical physics}. One of its younger 
fields is the theory of quantization and therein the theory deformation 
quantization can be settled. It belongs to the area of 
pure mathematics, but takes its inspiration from physics and is therefore a 
part of mathematical physics. The idea is, roughly spoken, to find a 
correspondence between the quantum and the classical world in physics. The 
mathematical description of their laws are different but yet there are a lot of 
similarities. It is more or less clear, how the classical world is created out 
of a huge number of quantum objects and the mathematics of classical mechanics 
can be understood as a limit case the behaviour of $n$ quantum objects where 
$n \longrightarrow \infty$. The other way round, it is not clear how one can 
create the mathematical description of a quantum system out of the one of a 
classical system. This reversed process is usually called quantization and its 
understanding is a mathematical task, not a physical one. Deformation 
quantization tries to ''deform'' the idealized algebra of classical physical 
observables by making it noncommutative and wants to get an idealized algebra 
of quantum mechanical observables this way. This is done by replacing the 
pointwise product of functions (since the classical algebra of obervables is 
usually modelled as the smooth function on a Poisson manifold) with a 
noncommutative product, which takes into account certain derivatives of the 
functions and plugs in a formal parameter which is called $\hbar$ and 
corresponds to Planck's constant in physics. This new product is called 
a star product and becomes a formal power series in $\hbar$. The zeroth order 
in the formal parameter represents classical mechanics and the first order 
quantum mechanics. Different mechanical systems allow different star products 
and their classification has been one of the main tasks of the theory for a 
long time. Besides the purely algebraic aspects of this theory, one also wants 
that this deformation is continuous or smooth in a certain sense and that a 
suitable subalgebra can be found, for which the formal power series is also 
convergent, since $\hbar$ is not a formal parameter in physics, but a nature 
constant with a specific value.


This work focusses on a particular star product, the so called Gutt star 
product, which can be established on a certain class of Poisson manifolds. Its 
goal is to find a large subalgebra of the smooth functions and a locally 
convex topology on them, such that the Gutt star product is convergent and that 
the commutative classical algebra can be deformed smoothly into the 
noncommutative quantum algebra. Of course, one has to give a proper definition 
how this smoothness is meant. Moreover, we will try to find as many convenient 
properties of this construction as possible and relate the topic to other 
fields of mathematics, such as Lie theory. For example, we will see that the 
locally convex observable algebra is closely linked to a universal enveloping 
algebra and therefore a Hopf algebra.


This master thesis is organized as follows: In the next part, chapter 2, we 
introduce the most important concepts of classical and quantum mechanics, 
explain their relations and give an overview over the field of deformation 
quantization, its history and its current state. We will also class this thesis 
into the theory. We will also give an outlook on the next steps, which can be 
done using the results of this work.
In the third chapter, we will explain in more detail the kind of Poisson 
systems this thesis deals with and see that they are in fact Lie algebras. We 
will construct the Gutt star product, which is characteristic for those 
systems, in different ways and show that these constructions are equal. We will 
explain the link to Lie theory, the Poincar\'e-Birkhoff-Witt and the 
Baker-Campbell-Hausdorff theorem and will introduce those results on the 
Baker-Campbell-Hausdorff series, which we will need for our later work.
Chapter four is devoted to finding explicit formulas for the Gutt star product 
and explaining them using two examples, as well as to some easy conclusions one 
can draw from those formulas.\\
Chapter five is the core part of this thesis. First, we introduce briefly the 
concept of locally convex topologies and explain why they are the convenient 
setting for our task. We show in detail how the locally convex topology for the 
Gutt star product is constructed, what their properties are and what kind of 
topology our Poisson system must have had that the construction of our 
topology is possible. At this point, we will introduce the concept of 
\emph{asymptotic estimate algebras}, 
which can be seen as a concept between locally multiplicatively convex algebras 
an general locally convex ones. Then we will show that the Gutt star product is 
indeed continuous with respect to our topology, that the deformation is 
analytic (even entire, if the underlying field is $\mathbb{C}$), that the 
construction is functorial and we will analyse the completion of this algebra. 
We will also show that this topology is optimal in a certain way.
The sixth chapter is devoted to particular systems, namely to nilpotent Lie 
algebras. We will show how the results we found previously can be increased, 
but we will also see the limits of our construction. We will establish 
the link to a previous work of Stefan Waldmann's and show that we come to the 
same conclusion by taking a different way using the Gutt star product. Finally, 
we will see that those stronger results are not bounded to the very case of 
strictly nilpotent Lie algebras, but that there are (in infinite dimensions) 
weaker notions of nilpotency which lead to the same result, when the 
construction of the topology is adapted a bit.
In the end, chapter seven treats the Hopf algebraic part which is very short 
due to the algebraic properties of the deformation. We will see that 
the co-structure and the antipode remain undeformed and continuous with respect 
to our topology, too.
In the appendix, we will give some more informations about asymptotic estimate 
algebras and try to settle them as precisely as possible among the locally 
convex algebras. We will also treat some examples and think about the possible 
future developments of this theory.


I want to thank my advisor, Stefan Waldmann, for the time he invested in this 
work and for his intense supervision. I am also very grateful to Chiara 
Esposito, who also spent a lot of time in discussion with me and was so kind to 
offer me a place in her office, so that I had a place at the chair where I 
could work. It is mainly due to both of them, that this work progressed so 
quickly and finally became a preprint \cite{esposito.stapor.waldmann:2015a:pre} 
which will hopefully become a publication in the next time. I am also grateful 
to Matthias Sch\"otz and Thorsten Reichert for many fruitful discussions 
and their patience with me bothering them with questions. Moreover, I am 
grateful for the support of the Studienstiftung and the Max-Weber Programm, who 
financed my studies and without whom my life would have looked very 
differently during the past six years. Finally, I want to 
say thank you to my parents who supported me all the time of school and 
studies, who encouraged me and who awakened my interest for the understanding 
of nature by their way of educating me. A last thank you is dedicated to my 
girlfriend for encouraging and supporting me when I was unmotivated and for not 
being angry when I spent a lot of time working on this thesis. :-)
