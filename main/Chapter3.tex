
%
% Chapter 3 of my master thesis:
% The first real chapter
%

\chapter{Algebraic Preliminaries}


\section{Linear Poisson structures}
\label{sec:chap3_LinearPoisson}

As we have seen before, there has already been done some work on how to 
strictly quantize local Poisson structures. The Weyl-Moyal-product on locally 
convex vector spaces was topologized by Stefan Waldmann in 
\cite{Wladmann:2004:Nuclear} and then investigated more closely by Matthias 
Schötz in \cite{Schoetz:Master}. It is thus clear that in the next step linear 
Poisson structures on locally convex vector spaces should be done. This will 
give a new big class of Poisson structures, which will be deformable in a 
strict way. Before we do so in the rest of this master thesis, we recall 
briefly some basics on linear Poisson structures.
\begin{remark}[The axiom of choice]
	Our final goal is to  do some locally convex functional analysis. Since in 
	this game it is mandatory for us to sell our souls to the devil for the 
	sake of the axiom of choice (e.g. in form of the Hahn-Banach theorem and 
	the projective tensor product), there is no point in not doing it right 
	from the beginning.
\end{remark}
First of all, linear Poisson structures are actually something familiar: 
We will always take a vector space $V$ and look at Poisson structures on its 
dual space $V^*$. But a linear Poisson structure on $V^*$ is nothing but the 
structure of a Lie algebra on $V$ itself, at least for finite-dimensional 
vector spaces.
\begin{proposition}
	\label{Prop:Alg:LinPoissonIsLieAlg}
	Let $V$ be a vector-space of dimension $n \in \mathbb{N}$ and $\pi \in 
	\Secinfty(\Anti^2(TV^*))$. Then the two following things are 
	equivalent:
	\begin{propositionlist}
		\item
		$\pi$ is a Poisson tensor.
		
		\item
		$V$ has a uniquely determined Lie algebra structure.
	\end{propositionlist}
\end{proposition}
\begin{proof}
	First, we want to think of what it means, that $\pi$ is a Poisson tensor. 
	It is already a antisymmetric bi-vectorfield. The only thing which it must 
	satisfy is the Jacobi-identity.
\end{proof}


We are interested in the polynomial algebra on the dual of the original vector 
space, since the construction is inspired by the formal deformation 
quantization on the cotangent bundle $T^*G$ of a Lie group $G$, which was 
investigated in \cite{BNW:1998:TStarQ1}. Now we know, that this original 
vector space really is a Lie algebra and for this reason we will call it in 
the following $\lie{g}$. 


The polynomial algebra $\Pol(\lie{g}^*)$ has many appearances and it is not an 
easy question how the find a good generalization for infinite-dimensional 
vector spaces. One could of course think of a good definition of the 
polynomial functions on the dual of an infinite-dimensional Lie algebra $g$. 
In order to keep the very most of the physically interesting cases in there, 
we will assume it to be locally convex. But we know, that even the topological 
dual $\lie{g}'$ will be a huge vector space and whatever the polynomials there 
should be, there will be a lot of them. We will furthermore not be able to 
find the nice correspondence of the Lie algebra structure on $\lie{g}$ and the 
linear polynomials on $\lie{g}'$, since the procedure we used involved the 
double-dual of $\lie{g}$. In general, this will be really bigger than 
$\lie{g}$ and starting from a linear Poisson structure on $\lie{g}'$, we will 
find a Lie algebra structure on $\lie{g}''$. Of course, we could just use the 
canonical embedding $\lie{g} \subseteq \lie{g}''$, but it could (and, in 
general, it will) happen, that the Lie bracket of $x,y \in \lie{g}$ will not 
be in $\lie{g}$ any more, but just in its double-dual. In most of the physical 
cases, we are actually not interested in the double-dual, but in the original 
vector space.

The little reflection shows, that we will have to think of a different 
generalization of linear Poisson structure for the infinite-dimensional case.
Luckily, in the finite-dimensional case, there's a different way of seeing the 
polynomials on $\lie{g}^* = \lie{g}'$, which allows a much easier 
generalization to infinite dimensions: It is the symmetric tensor algebra over 
$\lie{g}$ itself.
\begin{proposition}
	\label{Prop:Alg:PolIsSym}
	Let $\lie{g}$ be a vector space of dimension $n \in \mathbb{N}$. Then 
	the algebras $\Sym^{\bullet}(\lie{g})$ and $\Pol(\lie{g}^*)$ are 
	canonically isomorphic.
\end{proposition}
\begin{proof}
	Again via basis and double-dual...
\end{proof}
Here again, we used the double dual, so one could ask why this situation 
should differ from the foregoing one. But there is a difference: instead of 
looking at $\Sym^{\bullet}(\lie{g}^{**})$ we can directly restrict to 
$\Sym^{\bullet}(\lie{g})$. This is a setting in which closedness of the 
Poisson bracket is automatically fulfilled, since we don't to to characterize 
the object in $\Sym^{\bullet}(\lie{g}) \subseteq \Sym^{\bullet}
(\lie{g}^{**})$. We can take the interesting linear Poisson structure on 
$\Pol(\lie{g}^*)$ directly to be bilinear maps
\begin{equation*}
	\Sym^{\bullet}(\lie{g})
	\times
	\Sym^{\bullet}(\lie{g})
	\longrightarrow
	\Sym^{\bullet}(\lie{g})
\end{equation*}
which satisfy certain conditions on the degree. This is the way we want to go. 
We generalize the finite-dimensional situation, where we have $\Pol(\lie{g}^*) 
\cong \Sym^{\bullet}(\lie{g})$ in this sense, that we look directly at 
$\Sym^{\bullet}(\lie{g})$ for infinite-dimensional Lie algebras, since those 
(and their tensor products) are much better known and much easier to control.



\section{The universal enveloping algebra}
\label{sec:chap3_UniversalEnveloping}


\section{The Baker-Campbell-Hausdorff formula}
\label{sec:chap3_BCH}
