
%
% Chapter 3 of my master thesis:
% The first real chapter
%

\chapter{Algebraic Preliminaries}


\section{Linear Poisson structures}
\label{sec:chap3_LinearPoisson}

As we have seen before, there has already been done some work on how to strictly 
quantize local Poisson structures. The Weyl-Moyal-product on locally convex vector 
spaces was topologized by Stefan Waldmann in \cite{w_nuc} and then investigated more 
closely by Matthias Schötz in \cite{Schoetz}. It is thus clear that in the next step 
linear poisson structures on locally convex vector spaces must be done. Before we do 
so in the rest of this master thesis, we recall briefly some basics on linear Poisson 
structures.


\begin{remark}[The axiom of choice]
	Our final goal is to  do some locally convex functional analysis. Since in this 
	game it is mandatory for us to sell our souls to the devil for the sake of the 
	axiom of choice (e.g. in form of the Hahn-Banach theorem and the projective tensor 
	product), there is no point in not doing it right from the beginning.
\end{remark}
First of all, linear Poisson structures are actually something familiar: They are nothing but Lie algebras: Let $\{e^i\}_{i \in I}$ be a basis of a vector space $V*$ equipped with a Poisson bracket $\{\argument , \argument \}$.



\section{The universal enveloping algebra}
\label{sec:chap3_UniversalEnveloping}


\section{The Baker-Campbell-Hausdorff formula}
\label{sec:chap3_BCH}
