
%
% Chapter 3 of my master thesis:
% The first real chapter
%

\chapter{Algebraic Preliminaries}


\section{Linear Poisson structures in infinite dimensions}
\label{sec:chap3_LinearPoisson}

As we have seen before, there has already been done some work on how to 
strictly quantize Poisson structures on vector spaces. Star products of
exponential type on locally convex vector spaces were topologized by Stefan 
Waldmann in \cite{waldmann:2014a} and then investigated more closely by 
Matthias Schötz in \cite{schoetz:2014a}. Hence, as a the next step, we try
to attack linear Poisson structures on locally convex vector spaces. This 
will give a new big class of Poisson structures, which will be deformable in a 
strict way. Before we do so in the rest of this master thesis, we recall 
briefly some basics on linear Poisson structures.


We will always take a vector space $V$ and look at Poisson structures on the 
coordinates which are elements of the dual space $V^*$. In order to cover most 
of the physically interesting examples by our reflections, we will assume that 
$V$ is a locally convex vector space. Every finite-dimensional vector space is 
normable and complete and hence locally convex, so it fits in this framework. 
It is clear what $V^*$ should be and there is just one interesting topology on 
it. For infinite-dimensional spaces, the situation is more delicate:
we have to think about what coordinates should be and how a Poisson structure 
on them could look like. A priori, it is not clear which dual we should 
consider: the algebraic dual $V^*$ of all linear forms on $V$, or the 
topological dual $V'$ which contains just the continuous linear forms? Here, 
one could argue that only $V'$ is of real interest, since otherwise we would 
encounter the very strange effect of having discontinuous polynomials, and the 
aim of constructing a continuous star product on them seems somehow pointless. 
But even if we stick to $V'$, the question of the topology still remains: 
do we want to consider the weak or the strong topology there and why one of 
them should be more interesting. In any case, we have to choose a topology on 
this space. Once this is done, we have to think about a good notion of Poisson 
tensors in this context. However, we encounter quite a number of question,
which have no trivial answer. For this reason, it is worth looking at some 
equivalent formulations of $Pol(V^*)$ in the finite-dimensional case, since 
they may allow better generalizations.


Let $V$ be a finite dimensional vector space. Now, there is now question 
about the dual or its topology, since $V^* = V'$ is finite-dimensional, too, 
and we deal with polynomials on it. A linear Poisson structure on $V^*$ is 
something very familiar: it is equivalent to a Lie algebra structure on $V$.
\begin{proposition}
	\label{Alg:Prop:LinPoissonIsLieAlg}
	Let $V$ be a vector-space of dimension $n \in \mathbb{N}$ and $\pi \in 
	\Secinfty(\Anti^2(TV^*))$. Then the two following things are 
	equivalent:
	\begin{propositionlist}
		\item
		$\pi$ is a linear Poisson tensor.
		
		\item
		$V$ has a uniquely determined Lie algebra structure.
	\end{propositionlist}
\end{proposition}
\begin{proof}
	We choose a basis $e_1, \ldots, e_n \in V$ and denote its dual basis $e^1, 
	\ldots, e^n \in V^*$. Then we call the linear coordinates in these bases 
	$x_1, \ldots, x_n \in \Cinfty(V^*)$ and $\xi^1, \ldots, \xi^n \in 
	\Cinfty(V)$, such that for all $\xi \in V, x \in V^*$
	\begin{equation*}
		\xi
		=
		\xi^i(\xi) e_i
		\quad \text{ and } \quad
		x
		=
		x_i (x) e^i.
	\end{equation*}
	In these coordinates, the Poisson tensor reads
	\begin{equation*}
		\pi
		=
		\frac{1}{2}
		\pi_{ij}
		\frac{\partial}{\partial x_i}
		\wedge
		\frac{\partial}{\partial x_j},
	\end{equation*}
	where $\pi$ is linear in the coordinates and we have
	\begin{equation*}
		\pi_{ij}(x)
		=
		c_{ij}^k x_k.
	\end{equation*}
	This equivalent to a tensor
	\begin{equation*}
		c
		=
		\frac{1}{2}
		c_{ij}^k 
		e_k \tensor e^i \wedge e^j
	\end{equation*}
	which gives for $f,g \in \Cinfty(V^*)$
	\begin{equation}
		\label{Alg:KksInCoordinates}
		\{f, g\} (x)
		=
		\pi(df, dg)(x)
		=
		x_k c_{ij}^k 
		\frac{\partial f}{\partial x_i}
		\frac{\partial g}{\partial x_j},
	\end{equation}
	using the identification $T^*V^* \cong V^{**} \cong V$.
	But now, the statement is obvious, since antisymmetry of $\pi$ means
	antisymmetry of the $c_{ij}^k$ in the indices $i$ and $j$ and the Jacobi
	identity for the Poisson tensor gives
	\begin{equation}
		\label{Alg:JacobiInStructureConst}
		c_{ij}^\ell c_{\ell k}^m
		+
		c_{j k}^\ell c_{\ell i}^m
		+
		c_{ki}^\ell c_{\ell j}^m
		=
		0
	\end{equation}
	for all $i, j, k, m$, since it must be fulfilled for all smooth functions.
	Vicely versa, \eqref{Alg:JacobiInStructureConst} ensures the Jacobi 
	identity of $\pi$ in \eqref{Alg:KksInCoordinates}. Hence the map
	\begin{equation}
		\label{Alg:LieBracketOfKks}
		[ \cdot, \cdot ]
		\colon
		V
		\times
		V
		\longrightarrow
		V
		\quad
		(e_i, e_j)
		\longmapsto
		c_{ij}^k e_k
	\end{equation}
	defines a Lie bracket, since the $c_{ij}^k$ are antisymmetric and fulfil 
	the Jacobi identity and are therefore structures constants. Conversely, 
	the structure constants of a Lie algebra on $V$ define a Poisson tensor on 
	$V^*$ via \eqref{Alg:KksInCoordinates}.
\end{proof}


Since we know now, that $V$ is actually a Lie algebra, we will call it 
$\lie{g}$ from now on. Since they carry additional structure, these Poisson 
systems have a special name.
\begin{definition}[Kirillov-Kostant-Souriau bracket]
	\label{Def:KKS}
	Let $\lie{g}$ be a finite-dimensional Lie algebra. Then the Poisson 
	bracket $\{ \cdot , \cdot \}_{KKS}$, which is given by 
	Proposition~\ref{Alg:Prop:LinPoissonIsLieAlg} on $\lie{g}^*$ is called the 
	Kirillov-Kostant-Souriau bracket.
\end{definition}



The correspondence from Proposittion~\ref{Alg:Prop:LinPoissonIsLieAlg} is a 
first hint how we could extend our ideas to infinite dimensionsal systems. 
Unfortunately, we will not be able to find this nice 
correspondence of a Lie algebra structure on $\lie{g}$ and the linear 
polynomials on $\lie{g}'$, since the procedure we used involves the 
double-dual of $\lie{g}$. In general, this will be really bigger than 
$\lie{g}$ itself, and starting from some analogon of a linear Poisson 
structure on $\lie{g}'$, we will find a Lie algebra structure on $\lie{g}''$. 
Of course, we could just use the canonical embedding $\lie{g} \subseteq 
\lie{g}''$, but it could (and, in general, it will) happen, that the Lie 
bracket of $x,y \in \lie{g}$ will not be in $\lie{g}$ any more, but just in 
its double-dual. In most of the physical cases, we are actually not interested 
in the double-dual, but in the original vector space. Therefore, it seems to 
be a good choice to translate ''linear Poisson structure on $\lie{g}^*$'' as 
''$\lie{g}$ is a Lie algebra'' in infinite dimensions. Remark however that 
this is a choice and not a necessity, and other choices would have been 
possible.


The next task are the polynomials on $\lie{g}'$. As already mentioned, it is 
not easy to find a good generalization for them, since for a locally convex 
Lie algebra $\lie{g}$, even $\lie{g}'$ will be a rather huge vector space. 
Again, it is helpful to go back to the finite-dimensional case, where we have 
the following result:
\begin{proposition}
	\label{Alg:Prop:PolIsSym}
	Let $\lie{g}$ be a vector space of dimension $n \in \mathbb{N}$. Then 
	the algebras $\Sym^{\bullet}(\lie{g})$ and $\Pol^{\bullet}(\lie{g}^*)$ 
	are canonically isomorphic.
\end{proposition}
\begin{proof}
	since this is a very well-known result, we just want to sketch the proof 
	briefly: Take a basis $e_1, \ldots, e_n$ of $\lie{g}$ and its linear 
	coordinates $x_1, \ldots, x_n \in \Cinfty(\lie{g}^*)$ with $x_i(\xi) = 
	e_i(\xi)$ for $\xi \in \lie{g}^*$. On homogeneous symmetric tensors this 
	yields the map
	\begin{equation*}
		\mathcal{J}
		\colon
		\Sym^{\bullet}(\lie{g})
		\longrightarrow
		\Pol^{\bullet}(\lie{g}^*),
		\quad
		e_1^{\mu_1} \ldots e_n^{\mu_n}
		\longmapsto
		\xi_1^{\mu_1} \ldots \xi_n^{\mu_n}.
	\end{equation*}
	From the construction, we see that this is an isomorphism, but note, that 
	we have used the identification $\lie{g}^{**} \cong \lie{g}$ via
	\begin{equation*}
		e_i(\xi)
		=
		\langle \xi, e_i \rangle.
	\end{equation*}
\end{proof}
This gives a new idea for generalizing $\Pol(\lie{g}^*)$. One could argue that 
this isomorphism also uses the double dual which we wanted to avoid, since we 
could ''drop out'' of our original algebra and end up in $\Sym^{\bullet}
(\lie{g}^{**})$. Luckily, things are different now, and this is not possible. 
We can extend the Poisson bracket on $\lie{g}$ as a bi-derivation to 
$\Sym^{\bullet}(\lie{g})$ and get a closed Poisson algebra now. This way, we 
never even need to talk about $\Sym^{\bullet}(\lie{g}^{**})$. Of course, 
$\Sym^{\bullet}(\lie{g})$ only injects in $\Pol(\lie{g}')$ and we don't have 
an isomorphism any more, but we have good reason to think that this is enough: 
we get a closed and reasonably big subalgebra of of the polynomials. Moreover, 
the symmetric tensor algebra is defined on infinite-dimensional spaces 
exactly in the syme way as on finite-dimensional ones, there is no question 
about how to generalize, the construction is canonical.


Finally, we found a suitable way of speaking about object of interest: We 
replace linear Poisson structures on $\Pol^{\bullet}(\lie{g}^*)$ by 
$\Sym^{\bullet}(\lie{g})$. In finite dimension, this will not make a 
difference, but for the infinite-dimensional case, this is a choice and it is 
not mandatory to do it like this. Anyway, a decision had to be made and we 
have good reasons to believe that we are not completely misguided.



\section{The Gutt star product}
\label{sec:chap3_GuttStar}

Our final aim is to endow the symmetric algebra, and hence the polynomial 
algebra, with a new, noncommutative product. This is possible in a very 
natural way, due to the Poincar\'e-Birkhoff-Wit theorem. It links the 
symmetric tensor algebra $\Sym^{\bullet}(\lie{g})$ of a Lie algebra $\lie{g}$ 
to its universal enveloping algebra $\algebra{U}(\lie{g})$.


\subsection{The universal enveloping algebra}
\label{subsec:chap3_UniversalEnvelopingAlgebra}

If $\algebra{A}$ is an associative algebra, one can construct a Lie algebra 
out of it by using the commutator
\begin{equation*}
	[a,b]
	=
	a \cdot b - b \cdot a
	, \quad
	a,b \in \algebra{A}.
\end{equation*}
This construction is in fact functorial, since it doesn't only map associative 
algebras to Lie algebras, but also morphisms of the former to those of the 
latter. While getting a Lie algebra out of an associative algebra is easy, the 
reversed process is more complicated, but also possible. It is a well-known 
fact that every Lie algebra $\lie{g}$ can be embedded into an associative 
algebra, known as the universal enveloping algebra $\algebra{U}(\lie{g})$. It 
is uniquely determined (up to isomorphism) by the universal property: for 
every unital associative algebra $\algebra{A}$ and every homomorphism for Lie 
algebras $\phi\colon \lie{g} \longrightarrow \algebra{A}$ using the 
commutator on $\algebra{A}$, one gets a unital homomorphism of associative 
algebras $\Phi \colon \algebra{U}(\lie{g}) \longrightarrow \algebra{A}$ such 
that the following diagram is commutative:
\begin{center}
    \begin{tikzpicture}
        \matrix (m)[
        matrix of math nodes,
        row sep=2.5em,
        column sep=8em
        ]
        {
          \mathcal{U}(\lie{g}) & \\
           & \algebra{A} \\
          \lie{g} &  \\
        };
        \draw
        [-stealth]
        (m-1-1) 	edge node 
        			[above] 
        			{$\Phi$}
        					(m-2-2)
        	(m-3-1)	edge node
        			[below]
        			{$\phi$}
        					(m-2-2)
        					
        			edge node
        			[left]
        			{$\iota$}
        					(m-1-1)
        	;
    \end{tikzpicture}
\end{center}
The proof of existence and uniqueness of the universal enveloping algebra can 
be found in every standard textbook on Lie theory like 
\cite{hilgert.neeb:2012a} or \cite{varadarajan:1971a}, and we won't do it here 
in detail. Just recall that existence is proven by an explicit construction: 
one takes the tensor algebra $\Tensor^{\bullet}(\lie{g})$ and considers the 
two-sided ideal
\begin{equation*}
	\mathrm{I}
	=
	< \xi \tensor \eta - \eta \tensor \xi - [\xi, \eta] >
	, \quad
	\forall_{\xi, \eta \in \lie{g}}
\end{equation*}
inside of it. Then one gets the universal enveloping algebra by the quotient
\begin{equation}
	\label{Alg:UnivEnvAlg}
	\algebra{U}
	=
	\frac{\Tensor^{\bullet}(\lie{g})}{\mathrm{I}}.
\end{equation}
It follows from this construction, that $\algebra{U}(\lie{g})$ is a filtered 
algebra
\begin{equation*}
	\algebra{U}(\lie{g})
	=
	\bigcup_{k \in \mathbb{N}}
	\algebra{U}^k(\lie{g})
	, \quad
	\algebra{U}^k(\lie{g})
	= 
	\Big\lbrace
		x 
		= 
		\sum_i
		\xi_1^i \cdot \ldots \cdot \xi_n^i
	\ \Big| \ 
		\xi_j^i \in \lie{g}
		, \
		1 \leq j \leq n,
		i \in \mathbb{N}
	\Big\rbrace.
\end{equation*}
We just get a filtration, not a graded structure, however, since the ideal 
$\mathrm{I}$ is not homogeneous. Moreover, $\algebra{U}(\lie{g})$ is 
commutative (and graded) if and only if $\lie{g}$ was commutative. But 
$\algebra{U}(\lie{g})$ is much more than an associative algebra: it is also a 
Hopf algebra, since one can define a coassociative, cocommutative coproduct on 
it
\begin{equation*}
	\Delta \colon
	\algebra{U}(\lie{g})
	\longrightarrow
	\algebra{U}(\lie{g})
	\tensor
	\algebra{U}(\lie{g})
	, \quad
	\xi
	\longmapsto
	\xi \tensor \Unit
	+
	\Unit \tensor \xi
	, \quad
	\forall_{\xi \in \lie{g}}
\end{equation*} 
which extends to $\algebra{U}(\lie{g})$ via algebra homomorphism, as well as an antipode
\begin{equation*}
	S \colon
	\algebra{U}(\lie{g})
	\longrightarrow
	\algebra{U}(\lie{g})
	, \quad
	\xi
	\longmapsto
	- \xi
	, \quad
	\forall_{\xi \in \lie{g}}
\end{equation*}
which extends to $\algebra{U}(\lie{g})$ via algebra antihomomorphism.

\subsection{The Poincar\'e-Birkhoff-Witt theorem}
\label{subsec:chap3_PoincareBirkhoffWitt}

It is a well-known fact that one has a basis in $\algebra{U}(\lie{g})$. This 
result is due to the already mentioned theorem of Poincar\'e, Birkhoff and 
Witt:
\begin{theorem}[Poincar\'e-Birkhoff-Witt theorem]
	\label{Thm:Alg:PBW}
	Let $\lie{g}$ be a Lie algebra with a basis $\mathcal{B}_{\lie{g}} = \{ 
	\beta_i \}_{i \in I}$. Then the set
	\begin{equation*}
		\mathcal{B}_{\algebra{U}(\lie{g})}
		=
		\big\{
			\beta_{i_1}^{\mu_{i_1}}
			\cdot \ldots \cdot
			\beta_{i_n}^{\mu_{i_n}}
		\ \big| \
			n \in \mathbb{N}, \ 
			i_k \in I
			\text{ with } i_1 \earlier \ldots \earlier i_n 
			\text{ and } \beta_{i_k} \in \mathcal{B}_{\lie{g}}, \
			\mu_{i_1}, \ldots, \mu_{i_n} \in \mathbb{N}
		\big\}
	\end{equation*}
	defines a basis of $\algebra{U}(\lie{g})$.
\end{theorem}
there are different proofs for this statement. While a geometrical proof is 
very convenient in the finite-dimensional case, a combinatorial argument must 
be used in infinite dimensions. Most textbooks give the latter one, but 
restrict to finite-dimensional Lie algebras in order to avoid speaking of 
basis in infinite dimensions (except, of course, \cite{bourbaki:}). Once one 
has accepted working with the Lemma of Zorn, the proof will work the same way 
for any Lie algebra, since the idea relies on ordered index sets which can be 
defined in any dimension.
The PBW theorem allows us to set up an isomorphism between $\Sym^{\bullet}
(\lie{g})$ and $\algebra{U}(\lie{g})$ immediately, since a basis of the former 
can be given by almost the same expression
\begin{equation*}
	\mathcal{B}_{\Sym^{\bullet}(\lie{g})}
	=
	\big\{
		\beta_{i_1}^{\mu_{i_1}}
		\ldots
		\beta_{i_n}^{\mu_{i_n}}
	\ \big| \
		n \in \mathbb{N}, \ 
		i_k \in I, 1 \leq k \leq n,
		i_1 \earlier \ldots \earlier i_n 
		\text{ and } \beta_{i_k} \in \mathcal{B}_{\lie{g}}, \
		\mu_{i_1}, \ldots, \mu_{i_n} \in \mathbb{N}
	\big\}
\end{equation*}
where we just have replaced the noncommutative product in $\algebra{U}
(\lie{g})$ by the symmetric tensor product $\vee$ (which we will usually 
denote without a symbol, if possible) in $\Sym^{\bullet}(\lie{g})$. This 
allows us to write down an isomorphsim between the symmetric tensor algebra 
and the universal enveloping algebra, just by mapping the basis vectors to 
each other in a naive way. Of course, this can never be an isomorphism in the 
sense of algebras, but only of (filtered) vector spaces, because one of the 
algebras is commutative and the other isn't. Moreover, the symmetric algebra 
has a grading in the sense that
\begin{equation*}
	\Sym^{\bullet}(\lie{g})
	=
	\bigoplus\limits_{n = 0}^{\infty}
	\Sym^n(\lie{g})
	, \quad
	\Sym^n(\lie{g})
	=
	\underbrace{
		\lie{g} \vee \ldots \vee \lie{g}
	}_{
		n \text{ times}
	},
\end{equation*}
which induces a filtration by $\Sym^{(k)}(\lie{g}) = \sum_{j=0}^k 
\Sym^j(\lie{g})$. Our simple isomorphism will respect the filtration, but it 
isn't the only isomorphism which one can write down. In \cite{berezin:1971a}, 
Berezin proposed another isomorphism which is more helpful to use:
\begin{equation}
	\label{Alg:BerezinQuantization}
	\mathfrak{q}_n
	\colon
	\Sym^n(\lie{g})
	\longrightarrow
	\algebra{U}^n(\lie{g})
	, \quad
	\beta_{i_1} \ldots \beta_{i_n}
	\longmapsto
	\frac{1}{n!}
	\sum\limits_{\sigma \in S_n}
	\beta_{i_{\sigma(1)}} 
	\cdot \ldots \cdot
	\beta_{i_{\sigma(n)}}
	, \quad
	\mathfrak{q}
	=
	\sum\limits_{n = 0}^{\infty}
	\mathfrak{q}_n.
\end{equation}
We will refer to it as the quantization map, for reasons that will soon become 
clear. It also respects the filtration and transfers the symmetric product to 
another symmetric expression. In this sense, we can now switch between both 
algebras and use the setting, which is more convenient in the current 
situation: the graded structure of $\Sym^{\bullet}(\lie{g})$, or the Hopf 
algebra structure of $\algebra{U}(\lie{g})$.


\subsection{The Gutt star product}
Since we know, that the universal enveloping and the symmetric tensor algebra 
are isomorphic as vector spaces, we have a good tool at hand to endow the 
symmetric tensor algebra, and hence the polynomials, with a noncommutative 
product. This is exactly what Gutt did in \cite{gutt:1983a}. She constructed a 
star product on $\Pol^{\bullet}(\lie{g}^*)$ by pulling back the one from 
$\algebra{U}(\lie{g})$ while encoding the noncommutativity in a formal 
parameter $z \in \mathbb{C}$ in a convenient way. 
\begin{definition}[Gutt star product]
	\label{Def:GuttStar}
	Let $\lie{g}$ be a Lie algebra, $z \in \mathbb{C}$, and 
	$f, g \in \Sym^{\bullet}(\lie{g})$ of degree $k$ and $\ell$ respectively. 
	Then we define the Gutt star product:
	\begin{equation}
		\star_z
		\colon
		\Sym^{\bullet}(\lie{g})
		\times
		\Sym^{\bullet}(\lie{g})
		\longrightarrow
		\Sym^{\bullet}(\lie{g})
		, \quad
		(f,g)
		\longmapsto
		\sum\limits_{n = 0}^{e + \ell - 1}
		z^n
		\pi_{k + \ell - n}
		\left(
			\mathfrak{q}^{-1} \left(
				\mathfrak{q}(f) \cdot \mathfrak{q}(g)
			\right)
		\right),
	\end{equation}
	where the $\pi_n \colon \Sym^{\bullet}(\lie{g}) \longrightarrow 
	\Sym^n(\lie{g})$ are the projections on the homogeneous components of 
	degree $n$. 
\end{definition}
This is the original way in which Gutt defined her star product in 
\cite{gutt:1983a}, but there are two more ways to do it. Define
\begin{equation*}
	\mathrm{I}_z
	=
	< \xi \tensor \eta - \eta \tensor \xi - z [\xi, \eta] >
\end{equation*}
for $z \in \mathbb{C}$. Then we set 
\begin{equation}
	\label{Alg:DeformedUnivEnvAlg}
	\algebra{U}(\lie{g}_z)
	=
	\frac{\Tensor^{\bullet}(\lie{g})}{\mathrm{I}_z},
\end{equation}
and get the map
\begin{equation}
	\label{Alg:DeformedQuantization}
	\mathfrak{q}_{z,n}
	\colon
	\Sym^n(\lie{g})
	\longrightarrow
	\algebra{U}^n(\lie{g}_z)
	, \quad
	\beta_{i_1} \ldots \beta_{i_n}
	\longmapsto
	\frac{1}{n!}
	\sum\limits_{\sigma \in S_n}
	\beta_{i_{\sigma(1)}} 
	\cdot \ldots \cdot
	\beta_{i_{\sigma(n)}}
	, \quad
	\mathfrak{q}_z
	=
	\sum\limits_{n = 0}^{\infty}
	\mathfrak{q}_{z, n}.
\end{equation}
This way, we also get a star product:
\begin{equation}
	\label{Alg:DeformedStar}
	\widehat{\star}_z
	\colon
	\Sym^{\bullet}(\lie{g})
	\times
	\Sym^{\bullet}(\lie{g})
	\longrightarrow
	\Sym^{\bullet}(\lie{g})
	, \quad
	(f,g)
	\longmapsto
	\mathfrak{q}_z^{-1} 
	\left(
		\mathfrak{q}_z(f) \cdot \mathfrak{q}_z(g)
	\right).
\end{equation}
In \cite{drinfeld:1981a}, Drinfeld also constructed a star product using the 
Baker-Campbell-Hausdorff series: take $\xi, \eta \in \lie{g}$ and set
\begin{equation}
	\label{Alg:DrinfeldStar}
	\exp(\xi) \ast_z \exp(\eta)
	=
	\exp \left(
		\frac{1}{z}
		\bch{z \xi}{z \eta}
	\right),
\end{equation}
where the exponential series is understood a formal power series in $\xi$ and 
$\eta$. By differentiating, on gets the star product on all polynomials.


Of course, our aim is to show that these three maps are in fact identical and 
that they define a star product. Since this is a long way to go, we postpone 
the proof to the end of this chapter. It will be useful to learn something 
about the Baker-Campbell-Hausdorff series and the Bernoulli number first.



\section{The Baker-Campbell-Hausdorff series}
\label{sec:chap3_BCH}

Since we have a formula for $\star_z$ which involves the
Baker-Campbell-Hausdorff series, we want to give a short overview
about it and introduce some results, that will be helpful later on.
Note however, that there is not \emph{the} BCH formula, since one can always 
rearrange terms using antisymmetry, Jacobi and higher identities, but for 
$\xi, \eta \in \lie{g}$ for some Lie algebra $\lie{g}$, we can always write it 
as
\begin{equation}
	\label{Alg:NamesOfBCH}
	\bch{\xi}{\eta}
	=
	\sum\limits_{n = 1}^{\infty}
	\bchpart{n}{\xi}{\eta}
	=
	\sum\limits_{a,b = 0}^{\infty}
	\bchparts{a}{b}{\xi}{\eta},
\end{equation}
where $\bchpart{n}{\xi}{\eta}$ denotes all expressions having $n$ letters and 
$\bchparts{a}{b}{\xi}{\eta}$ denotes all expressions with $a$ $\xi$'s and $b$ 
$\eta$'s. Clearly we have
\begin{equation*}
	\bchpart{n}{\xi}{\eta}
	=
	\sum\limits_{b + b = n}
	\bchparts{a}{b}{\xi}{\eta}.
\end{equation*}
This only postpones the problem of non-uniqueness to the partial expressions, 
but it will be helpful, since we don't have to choose a special form for 
writing BCH every time.



\subsection{Some general and historical remarks}
Let a noncommutative algebra $\algebra{A}$ be given. For $\xi, \eta \in 
\algebra{A}$ we look at the identity
\begin{equation*}
	\exp(\chi)
	=
	\exp(\xi)
	\exp(\eta).
\end{equation*}
Then, algebraically speaking, the BCH series is the formal power series for 
$\chi$ as a commutator expression in $\xi$ and $\eta$. We can take the formal 
power series for the exponentials
\begin{equation*}
	\exp(\xi)\exp(\eta) 
	= 
	\sum\limits_{n,m=0}^{\infty} 
	\frac{\xi^n \eta^m}{n! m!}
\end{equation*}
and use the formal power series for the logarithm
\begin{equation*}
	\log(\chi) 
	= 
	\sum\limits_{k = 1}^{\infty} 
	\frac{ (-1)^{k-1} }{ k } (\chi - 1)^k.
\end{equation*}
This yields
\begin{equation}
	\label{Alg:BCHinRaw}
	\chi
	= 
	\log \left( \exp(\xi) \exp(\eta) \right) 
	= 
	\sum\limits_{k = 1}^{\infty} 
	\frac{ (-1)^{k - 1} }{k} 
	\sum\limits_{ \substack{ 
		i \in \{1, \ldots, k\} \\ 
		n_i, m_i \geq 0 \\ n_i + m_i \geq 1
	}}
	\frac{
		\xi^{n_1} \eta^{m_1} 
		\ldots 
		\xi^{n_k} \eta^{m_k}
	}
	{n_1! m_1! \ldots n_k! m_k!}.
\end{equation}
It is obviously far from trivial, if and how this can be expressed using 
commutators, and the first one to find a general way for this was Dynkin in 
the 1950's.


Now let $\lie{g}$ be a finite-dimensional Lie algebra with a corresponding Lie 
group $G$. From the geometric point of view, the BCH formula is the 
infinitesimal counterpart of the multiplication law in $G$. One can express it 
as a series of Lie brackets, which converges just in a small neighbourhood 
around the unit element $e \in G$. Since the exponential function  $\exp 
\colon \lie{g} \longrightarrow G$ is locally diffeomorphic there, one gets for 
$\xi, \eta \in \lie{g}$
\begin{align*}
	\bch{\xi}{\eta}
	& =
	\log \left( \exp(X) \exp(Y) \right)
	\\
	& =
	\xi + \eta
	+ \frac{1}{2} [\xi, \eta]
	+ \frac{1}{12} ( [\xi, [\xi, \eta]] + [\eta, [\eta, \xi]] )
	+ \ldots
\end{align*}
The formula was recovered by Campbell \cite{campb1, campb2}, Baker 
\cite{baker} and Hausdorff \cite{hausd} independently, founding on a work by 
Schur \cite{schur}. A lot of work has been done on it since, and one can find 
a nice summary about it in the introductory part of \cite{casmur}.


\subsection{Forms of the BCH}
As already mentioned, there are different forms of stating the BCH formula and 
depending on the problem one wants to solve, not each one is well suited. One 
can classify them roughly into four groups.
\begin{enumerate}
	\item
	There are recursive formulas, which calculate each term from the previous 
	one. The first expression due to Baker, Campbell and Hausdorff were of 
	this kind. Though the idea is old, this approach is still much used and 
	allows powerful applications: Casas and Murua found an efficient 
	algorithms \cite{casas.murua:yeara} for calculating the BCH series up to 
	high orders, based on a recursive formula given by Varadarajan in his 
	textbook \cite{varadarajan:yeara}.
	
	\item
	Most textbooks prove an integral form of the series, like 
	\cite{hall:yeara} and \cite{hilgert.neeb:2012a}. since we will use it, 
	too, we want to introduce it here briefly. Take the meromorphic function
	\begin{equation}
		\label{Alg:DefinitionLogBernoullis}
		g \colon 
		\mathbb{C}
		\longrightarrow
		\mathbb{C},
		\quad
		z 
		\longmapsto
		\frac{ z \log(z) }{z - 1}
	\end{equation}
	and denote for $\xi \in \lie{g}$ by
	\begin{equation*}
		\ad_{\xi}
		\colon
		\lie{g}
		\longrightarrow
		\lie{g},
		\quad
		\eta
		\longmapsto
		[\xi, \eta]
	\end{equation*}
	the usual ad-operator. Then one has for $\xi, \eta \in \lie{g}$
	\begin{equation}
		\label{Alg:BCHinIntegral}
		\bch{\xi}{\eta}
		=
		\xi + 
		\int\limits_0^1
		g \left( 
			\exp \left(
				\ad_{\xi}
			\right)
			\exp \left(
				t \ad_{\eta}
			\right)
		\right)
		(\eta) 
		dt.
	\end{equation}
	
	\item
	As already mentioned, Dynkin found a closed form for \eqref{Alg:BCHinRaw},
	which is the only one of this kind known so far. One can its proof for 
	example in \cite{jacobson:yeara}. It reads
	\begin{equation}
		\label{Alg:BCHinDynkin}
		\bch{\xi}{\eta}
		=
		\sum\limits_{k = 1}^{\infty} 
		\frac{ (-1)^{k - 1} }{k}
		\sum\limits_{\substack{
			i \in \{1, \ldots, k\} \\
			n_i, m_i \geq 0 \\ 
			n_i + m_i \geq 1
		}}
		\frac{1}{ \sum_{i = 1}^k(n_i + m_i) }
		\frac{ \left[
			\xi^{n_1} \eta^{m_1} \ldots \xi^{n_k} \eta^{m_k}
		\right]}
		{ n_1! m_1! \ldots n_k! m_k! },
	\end{equation}
	where the expression $[ \ldots ]$ denotes Lie brackets nested to the left:
	\begin{equation*}
		[\xi \eta \eta \xi]
		=.
		[[[\xi, \eta], \eta], \xi]
	\end{equation*}
	Unfortunately, the combinatorics get extremely complicated for higher 
	degrees, since it contains a lot of redundancies.
	
	\item
	Goldberg gave a form of the series which is based on words in two letters:
	\begin{equation}
		\bch{\xi}{\eta}
		=
		\sum\limits_{n = 1}^{\infty}
		\sum\limits_{|w| = n}
		g_w w.
	\end{equation}
	The $g_w$ are coefficients, which can be calculated using the recursively
	defined Goldberg polynomials (see \cite{goldberg:1956a}). It was put into 
	commutator form by Thompson in \cite{thompson:1982a}:
	\begin{equation}
		\label{Alg:BCHinGoldbergThompson}
		\bch{\xi}{\eta}
		=
		\sum\limits_{n = 1}^{\infty}
		\sum\limits_{|w| = n}
		\frac{g_w}{n} [w].
	\end{equation}
	Again, the $[w]$ are Lie brackets nested to the left. Of course, this 
	formula will also have redundancies, but its combinatorial aspect is much 
	easier than the one of \eqref{Alg:BCHinDynkin}. Since there are estimates 
	for the coefficients $g_w$, we will use this form for our Main Theorem.
\end{enumerate}


\subsection{The Goldberg-Thompson formula and some results}

\subsubsection{Goldberg's theorems}
We now introduce the results of Goldberg: he noted a word in the letters $\xi$ 
and $\eta$ as
\begin{equation*}
	w_{\xi}(s_1, s_2, \ldots, s_m)
	=
	\xi^{s_1} \eta^{s_2} \ldots (\xi \vee \eta)^{s_m},
\end{equation*}
with $m \in \mathbb{N}$ and the last letter will be $\xi$ if $m$ is odd and 
$\eta$ if $m$ is even. The index $\xi$ of $w_{\xi}$ means, that the word 
starts with a $\xi$. Now we can assign to each word $w_{\xi \vee \eta}(s_1, 
\ldots, s_m)$ a coefficient $c_{\xi \vee \eta}(s_1, \ldots, s_m)$. This is 
done by the following formula:
\begin{equation}
	\label{Alg:GoldbergCoeff1}
	c_{\xi}(s_1, \ldots, s_m) 
	= 
	\int\limits_0^1 
	t^{m'} (t - 1)^{m''} 
	G_{s_1}(t) \ldots G_{s_m}(t) dt,
\end{equation}
where we have $m' = \lfloor \frac{m}{2} \rfloor$, $m'' = \lfloor \frac{m-1}{2} 
\rfloor$ with $\lfloor \cdot \rfloor$ denoting the entire part of a real 
number and we have $n = \sum_{i = 1}^m s_i$. For $c_{\eta}$ we have
\begin{equation}
	\label{Alg:GoldbergCoeff2}
	c_{\eta}\left( s_1, \ldots, s_m \right)
	= 
	\begin{cases}
		c_{\xi}\left( s_1, \ldots, s_m \right)
		& 
		\text{ if } m \text{ is odd} 
		\\
		(-1)^{n-1} c_{\xi} \left( s_1, \ldots, s_m \right)
		& 
		\text{ if } m \text{ is even.}
	\end{cases}
\end{equation}
The $G_s$ are the recursively defined Goldberg polynomials
\begin{equation}
	\label{Alg:GoldbergPolynomials}
	G_s(t) 
	= 
	\frac{1}{s} 
	\frac{d}{dt} 
	t(t-1) G_{s-1}(t),
\end{equation}
for $s > 1$ and $G_1(t) = 1$. From \eqref{Alg:GoldbergCoeff2} we get 
immediately
\begin{equation*}
	c_{\xi}\left( s_1, \ldots, s_m \right)
	= 
	c_{\eta}\left( s_1, \ldots, s_m \right) 
	= 
	0 
	\quad 
	\text{ if $m$ is odd and $n$ is even.}
\end{equation*}
Of course, Goldberg found interesting identities which are fulfilled by the 
coefficients. A very remarkable one is that for all permutations $\sigma \in 
S_m$ one has
\begin{equation*}
	c_{\xi}\left( s_1, \ldots, s_m \right)
	=
	c_{\xi}\left( s_{\sigma(1)}, \ldots, s_{\sigma(m)} \right),
\end{equation*}
since \eqref{Alg:GoldbergCoeff1} obviously doesn't see the ordering of the 
$s_i$ and $m'$, $m''$ and $n$ are not affected by reordering. For words with 
$m = 2$, an easier formula can be found:
\begin{equation*}
	c_{\xi}(s_1, s_2) 
	= 
	\frac{ (-1)^{s_1} }{s_1! s_2!} 
	\sum\limits_{n = 1}^{s_2} 
	\binom{s_2}{n} B_{s_1 + s_2 - n},
\end{equation*}
where the $B_s$ denote the Bernoulli numbers, which will be explained more 
precisely in the next paragraph. First, we note that the only case which 
matters to us is of course $s_1 = 1$, since for $s_1, s_2 > 1$ we will find 
something like $[[\xi, \xi], \ldots ] = 0$. For simplicity, let's set 
$s_2 = 1$ and to permute $s_1 \leftrightarrow s_2$:
\begin{equation}
	\label{Alg:GoldbergCoeffBernoulli}
	c_{\xi}(1, n-1)
	=
	\frac{ (-1)^{n-1} }{s!}
	B_s.
\end{equation}


\subsubsection{Bernoulli numbers}
We have seen the Bernoulli numbers $B_n$ showing up and we will encounter them 
very often in the following. Hence it is useful to learn a few important 
things about them. They are defined by the series expansion of
\begin{equation}
	\label{Alg:DefBernoulli}
	g \colon
	\mathbb{C}
	\longrightarrow
	\mathbb{C}
	, \quad
	z \longmapsto
	\frac{z}{\E^z - 1}
	=
	\sum\limits_{n=0}^{\infty}
	\frac{B_n}{n!}z^n.
\end{equation}
Clearly, $g$ has poles at $z = 2 k \pi i$, $k \in \mathbb{Z}\backslash \{0\}$.
Moreover, one can easily show that all odd Bernoulli numbers are zero, except 
for $B_1 = - \frac{1}{2}$ and since in some applications one needs $B_1$ to be 
positive, there is a different convention for naming them: one often 
encounters $B_n^* = (-1)^n B_n$ (which only differs for $n = 1$). The nonzero 
Bernoulli number alternate in sign and for there absolute value, one can show 
the asymptotic behaviour (see \cite{luschny.oeis})
\begin{equation*}
	B_{2n} 
	\sim 
	(-1)^{n-1} 
	4\sqrt{\pi n} 
	\left( \frac{n}{\pi e} \right)^{2n}.
\end{equation*}
This is not surprising, since we now that the generating function $g$ had 
poles at $\pm 2\pi i$. The Bernoulli numbers can also be calculated by the 
recursion formula
\begin{equation}
	B_n
	=
	- \frac{1}{n + 1}
	\sum\limits_{k = 0}^{n- 1}
	\binom{n + 1}{k}
	B_k,
\end{equation}
which is well-known in the literature. Since we will deal a lot with them, we 
want to give the first numbers of this series here.
\begin{center}
	\begin{tabular}
	{c||c|c|c|c|c|c|c|c|c|c|c|c|c|c|c|c|c}
		$n$ & $0$ & $1$ & 
		$2$ & $3$ & $4$ & 
		$5$ & $6$ & $7$ & 
		$8$ & $9$ & $10$ & 
		$11$ & $12$ & $13$ & 
		$14$ & $15$ & $16$
		\\
		\hline 
		$B_n$ & $1$ & $-\frac{1}{2}$ & 
		$\frac{1}{6}$ & $0$ & $-\frac{1}{30}$ & 
		$0$ & $\frac{1}{42}$ & $0$ & 
		$-\frac{1}{30}$ & $0$ & $\frac{5}{66}$ & 
		$0$ & $\frac{691}{2730}$ & $0$ & 
		$\frac{7}{6}$ & $0$ & $\frac{3617}{510}$
	\end{tabular} 
\end{center}


\subsubsection{BCH up to first order}

\begin{proposition}
	Let $\lie{g}$ be a Lie algebra and the Bernoulli numbers as defined 
	before. Then we have for $\xi, \eta \in \lie{g}$
	\begin{align}
		\label{Alg:BCHFirstOrderXi}
		\bch{\xi}{\eta}
		& =
		\sum\limits_{n = 0}^{\infty}
		\frac{B_n^*}{n!}
		\left( \ad_{\xi} \right)^n (\eta)
		+ \mathcal{O} \left( \eta^2 \right)
		\\
		\label{Alg:BCHFirstOrderEta}
		& =
		\sum\limits_{n = 0}^{\infty}
		\frac{B_n}{n!}
		\left( \ad_{\xi} \right)^n (\eta)
		+ \mathcal{O} \left( \xi^2 \right)
	\end{align}
\end{proposition}
\begin{proof}
	This is something we can perfectly calculate using the Goldberg 
	coefficients. Remind that we will put words to Lie brackets, and for 
	computing the coefficients we will need the words $\eta \xi^n$ \emph{and} 
	$\xi \eta \xi^{n-1}$ because of antisymmetry. Now let $n \in \mathbb{N}$. 
	We have
	\begin{equation*}
		c{\eta}(1,n) 
		= 
		(-1)^n c_{\xi}(1,n) 
		= 
		(-1)^n 
		\frac{(-1)^n}{n!} 
		B_n 
		= 
		\frac{B_n}{n!}.
	\end{equation*}
	By $n-fold$ skew-symmetry and \eqref{Alg:BCHinGoldbergThompson}, we get 
	the contribution
	\begin{equation*}
		\frac{(-1)^n}{(n+1)!} 
		B_n \left( \ad_{\xi} \right)^n (\eta)
		=
		\frac{1}{(n+1)!} 
		B_n^* \left( \ad_{\xi} \right)^n (\eta)		
	\end{equation*}
	Now we need $c_{\xi}(1, 1, n-1)$: let $n > 1$, then
	\begin{align*}
		c_{\xi}(1, 1, n - 1) 
		& = 
		\int\limits_0^1 
		t(t-1) G_{n-1}(t) 
		dt 
		\\
		& = 
		- \int\limits_0^1 
		t \frac{d}{dt} 
		\left( t(t-1) G_{n-1}(t) \right) 
		dt 
		\\
		& =
		- \int\limits_0^1 
		n t G_n(t) 
		dt 
		\\
		& = 
		-n 
		c_{\xi}(1,n) 
		\\
		& = 
		-n 
		\frac{(-1)^n}{n!} 
		B_n 
		\\
		& =  
		(-1)^{n + 1} 
		\frac{1}{(n-1)!} B_n
	\end{align*}
	So by using $n-1$ times the skew-symmetry of the Lie bracket, we get
	\begin{equation*}
		\frac{1}{n + 1} 
		\cdot (-1)^{n + 1} 
		\frac{1}{ (n - 1)! }
		B_n 
		[\ldots [\xi, \eta], \xi] \ldots ], \xi] 
		= 
		\frac{n}{(n+1)!} 
		B_n 
		\left( \ad_{\xi} \right)^n (\eta)
	\end{equation*}
	For $n > 1$, we add up those two and use the fact that $B_n = B_n^*$ and 
	find the result we want. For $n = 1$, there is just the first contribution 
	and $c_{\xi}(1,1) = - B_1$, which gives
	\begin{equation*}
		B_1^* \ad_{\xi}(\eta)
	\end{equation*}
	in total. For $n = 0$ , we get $c_{\xi}(1) = c_{\eta}(1) = 1$ and finally 
	get \eqref{Alg:BCHFirstOrderXi}. For \eqref{Alg:BCHFirstOrderEta}, note 
	that we need $c_{\xi}(1, n)$ and $c_{\eta}(1, 1, n-1)$. We have 
	$c_{\xi}(1,n) = (-1)^n c_{\eta}(1,n)$ and 
	$c_{\eta}(1, 1, n-1) = (-1)^n c_{\xi}(1,1,n-1$.
	this gives a global $(-1)^n$ and makes hence $B_n$ out of $B_n^*$.
\end{proof}
\begin{remark}
	Note that we could also have used the integral formula 
	\eqref{Alg:BCHinIntegral} to prove this. We have two exponential series, 
	but the second one can be cut after the constant term, since we are 
	looking for contributions which are linear in $\eta$. The function in the 
	integral is 	nothing but $(g \circ \log)(z)$. Since we insert 
	$\exp(\ad_{\xi})$, we get
	\begin{equation*}
		\bch{\xi}{\eta}
		=
		\xi + 
		\int_0^1
		g \left( \ad_{\xi} \right)
		(\eta)
		dt
		+ \mathcal{O} \left( \eta^2 \right)
		=
		\xi +
		\sum\limits_{n = 1}^{\infty}
		\frac{B_n^*}{n!}
		\left( \ad_{\xi} \right)^n (\eta)
		+ \mathcal{O} \left( \eta^2 \right)
	\end{equation*}
	since there is no dependence on $t$ any more.
\end{remark}



\section{The Equality of the Star Products}
\label{sec:chap3_StarProductProof}

The first step is associativity, since it will simplify the proof later.
\begin{proposition}
	\label{Alg:Lemma:Associtivity}
	The three maps $\star_z$, $\widehat{\star}_z$ and $\ast_z$ define
	associative multiplications.
\end{proposition}
\begin{proof}
  \mbox{}
  \begin{enumerate}
	\item 
	For $\widehat{\star}_z$, associativity is clear from the very 
	construction, since the multiplication in $\algebra{U}(\lie{g}_z)$
	is associative. Bilinearity is also immediate.
	
	\item
	For $\star_z$, we have to move around sums and shift projections.
	Recall that $\pi_n(f \star_z g) = 0$, if $n > \deg(f) + \deg(g)$.
	Take homogeneous tensors $f, g, h \in \Sym^{\bullet}(\lie{g})$
	of degree $k, \ell, m \in \mathbb{N}$ respectively. The we have
	\begin{align*}
		\left(
			f \star_z g
		\right)
		\star_z
		h
		& =
		\sum\limits_{j=0}^{k + \ell -1}
		\sum\limits_{i = 0}^{k + \ell + m -j - 1}
		z^i
		\left( 
			\pi_{k + \ell + m - j - i} 
			\circ 
			\mathfrak{q}^{-1}
		\right)
		\left(
			\mathfrak{q}
			\left(
				z^j
				\left(
					\pi_{k + \ell - j} 
					\circ 
					\mathfrak{q}^{-1}
				\right)
				\left(
					\mathfrak{q} (a)
					\mathfrak{q} (b)
				\right)
			\right)
			\mathfrak{q}(h)
		\right)
		\\
		& =
		\sum\limits_{j=0}^{k + \ell -1}
		\sum\limits_{i = 0}^{k + \ell + m - 1}
		z^{i-j}
		\left( 
			\pi_{k + \ell + m - i} 
			\circ 
			\mathfrak{q}^{-1}
		\right)
		\left(
			\mathfrak{q}
			\left(
				z^j
				\left(
					\pi_{k + \ell - j} 
					\circ 
					\mathfrak{q}^{-1}
				\right)
				\left(
					\mathfrak{q} (a)
					\mathfrak{q} (b)
				\right)
			\right)
			\mathfrak{q}(h)
		\right)
		\\
		& =
		\sum\limits_{i = 0}^{k + \ell + m - 1}
		z^i
		\left( 
			\pi_{k + \ell + m - i} 
			\circ 
			\mathfrak{q}^{-1}
		\right)
		\left(
			\mathfrak{q}
			\left(
				\sum\limits_{j=0}^{k + \ell -1}
				z^{-j}
				z^j
				\left(
					\pi_{k + \ell - j} 
					\circ 
					\mathfrak{q}^{-1}
				\right)
				\left(
					\mathfrak{q} (a)
					\mathfrak{q} (b)
				\right)
			\right)
			\mathfrak{q}(h)
		\right)
		\\
		& =
		\sum\limits_{i = 0}^{k + \ell + m - 1}
		z^i
		\left( 
			\pi_{k + \ell + m - i} 
			\circ 
			\mathfrak{q}^{-1}
		\right)
		\left(
			\mathfrak{q} a)
			\mathfrak{q}(b)
			\mathfrak{q}(h)
		\right),
	\end{align*}
	and we just need to reverse this procedure on the right hand side
	in order to get the wanted result. All we did was using the linearity 
	of the involved maps. Bilinearity is also clear for $\star_z$, since
	all the maps involved are (bi-)linear.
		
	\item
	Also for $\ast_z$, associativity is not complicated, since one can 
	write the star product, using the formal power series of the 
	exponential and the logarithm. Since in this setting, we have
	\begin{equation*}
		\exp(\log(z))
		=
		z
		, \quad
		\forall_{z \in \lie{g}},
	\end{equation*}
	one has
	\begin{align*}
		\left( 
			\exp(\xi) \ast_z \exp(\eta) 
		\right) 
		\ast_z \exp(\chi)
		&=
		\exp
		\left(
			\frac{1}{z}
			\bch{
				\left(
					\frac{1}{z}
					\bch{z \xi}{z \eta}
				\right)
			}
			{z \chi}
		\right)
		\\
		&=
		\exp
		\left(
			\frac{1}{z}
			\bch{z \xi}
			{
				\left(
					\frac{1}{z}
					\bch{z \eta}{z \chi}
				\right)
			}
		\right)
		\\
		&=
		\exp(\xi)
		\ast_z
		\left( 
			\exp(\eta) \ast_z \exp(\chi) 
		\right), 
	\end{align*}
	since
	\begin{align*}
	\bch{
		\left(
			\frac{1}{z}
			\bch{z \xi}{z \eta}
		\right)
	}
	{z \chi}
	& =
	\log\left(
		\exp\left(
			\log\left(
				\frac{1}{z}
				\exp\left(
					z \xi
				\right)
				\exp\left(
					z \eta
				\right)
			\right)
		\right)
		\exp\left(
			z \chi
		\right)
	\right)
	\\
	& =
	\log\left(
		\frac{1}{z}
		\exp\left(
			z \xi
		\right)
		\exp\left(
			z \eta
		\right)
		\exp\left(
			z \chi
		\right)
	\right)
	\\
	& =
	\log\left(
		\exp\left(
			z \xi
		\right)
		\log\left(
			\left(
				\frac{1}{z}
				\exp\left(
					z \eta
				\right)
				\exp\left(
					z \chi
				\right)
			\right)
		\right)
	\right)
	\\
	& =
	\bch{z \xi}
	{
		\left(
			\frac{1}{z}
			\bch{z \eta}{z \chi}
		\right)
	}.
	\end{align*}
	Bilinearity follows from differentiating the formula and is 
	a simple computation.
  \end{enumerate}	 
\end{proof}
Note that for a star product, the maps must fulfil the classical and the 
semi-classical limit. This will be done in Corollary 
\ref{Formulas:Cor:LimitCases}. It is left to show, that the star product are 
in fact equal. It is enough to show that they coincide for terms of the form 
$\xi^k \star \eta$ with $\xi, \eta \in \lie{g}$ and $k \in \mathbb{N}$, since 
we get them on homogeneous polynomials of the form $\xi_1 \ldots \xi_k \star 
\eta$ by polarization and for two polynomials like $\xi_1 \ldots \xi_k \star 
\eta_1 \ldots \eta_{\ell}$ by iteration, which is possible due to 
associativity. The next lemma will be a first big step:
\begin{lemma}
	Let $\lie{g}$ be a Lie algebra, then we have
	\begin{equation}
		\label{Alg:GuttStarFormula}
		\xi^k \widehat{\star}_z \eta
		=
	 	\sum\limits_{n=0}^k 
	 	z^n \binom{k}{n} B_n^*
	 	\xi^{k-n} 
	 	\left( \ad_{\xi} \right)^n (\eta),
	\end{equation}
	where the $B_n^*$ are the Bernoulli numbers defined as above.
\end{lemma}
\begin{proof}
	This proof is divided into the two following lemmata:
	\begin{lemma}
		\label{Alg:Lemma:BadSublemma1}
		Let $\xi, \eta \in \mathfrak{g}$ and $k\in \mathbbm{N}$. Then we have
		\begin{equation*}
			{q}_z \left(\sum\limits_{n=0}^k z^n
			\binom{k}{n} B_n^* \xi^{k-n}
			\left(\ad_{\xi}\right)^n(\eta)\right)
			=
			\sum\limits_{s=0}^k\mathcal{K}(k,s)
			\xi^{k-s} \cdot \eta \cdot \xi^s
		\end{equation*}
		with
		\begin{equation*}
			 \mathcal{K}(k,s)
			 =
			 \frac{1}{k + 1} \sum\limits_{n=0}^k
			 \binom{k+1}{n} B_n^*
			 \sum\limits_{j=0}^n
			 (-1)^j \binom{n}{j}
			 \sum\limits_{\ell=0}^{k-n} \delta_{s, \ell+j}
		\end{equation*}
	\end{lemma}
	\begin{subproof}
		Since the map $\mathfrak{q}_z$ is linear, we can pull out the
		constants and get
		\begin{equation*}
			\mathfrak{q}_z \left( \sum\limits_{n=0}^k
			z^n \binom{k}{n} B_n^* \xi^{k-n} 
			\left(\ad_{\xi}\right)^k(\eta)\right)
			=
			\sum\limits_{n=0}^k
			\binom{k}{n} B_n^*
			\mathfrak{q}_z\left(z^n \xi^{k-n}
			\left(\ad_{\xi}\right)^k(\eta)\right)
		\end{equation*}
		Now we need the two equalities
		\begin{align*}
			\mathfrak{q}_z\left( \xi^k \eta \right)
			& = 
			\frac{1}{k+1}\sum\limits_{j=0}^k 
			\xi^{k-l} \cdot \eta \cdot \xi^k
		\intertext{and}
			\mathfrak{q}_z\left(
			\left(z^n \ad_{\xi}\right)^k (\eta) \right)
			& = 
			\sum\limits_{j=0}^k
			(-1)^j \binom{k}{j}
			\xi^{k-j} \cdot \eta \cdot \xi^j
		\end{align*}
		which can easily be shown by induction and plug them in:
		\begin{align*}
			\sum\limits_{n=0}^k
			\binom{k}{n} B_n^* \mathfrak{q}_z(\ldots)
			& = 
			\sum\limits_{n=0}^k
			\binom{k}{n} \frac{B_n^*}{k-n+1}
			\sum\limits_{\ell=0}^{k-n}
			\xi^{k-n-\ell} \cdot \left(
			\sum\limits_{j=0}^n
			(-1)^j \binom{n}{j} \xi^{n-j} \cdot \eta \cdot \xi^j
			\right) \cdot \xi^{\ell}
			\\
			& =
			\sum\limits_{n=0}^k
			\binom{k}{n} \frac{B_n^*}{k-n+1}
			\sum\limits_{\ell=0}^{k-n}
			\sum\limits_{j=0}^n
			(-1)^j \binom{n}{j} 
			\xi^{k-\ell-j} \cdot \eta \cdot \xi^{\ell+j} 
			\\	
			& = 
			\frac{1}{k + 1} \sum\limits_{n=0}^k
			\binom{k+1}{n} B_n^*
			\sum\limits_{j=0}^n (-1)^j \binom{n}{j}
			\sum\limits_{\ell=0}^{k-n}
			\xi^{k-\ell-j} \cdot \eta \cdot \xi^{\ell+j}
		\end{align*}
		We just need to collect those terms for which we have $\ell + j = s$
		for all	$s = 0, \ldots, k$. If we do this with a Kronecker-delta, we 
		will get exactly the $\mathcal{K}(k, s)$.
	\end{subproof}
	For the second lemma, we need some statement on Bernoulli numbers and 
	binomial coefficients:
	Let $k,m,n \in \mathbbm{N}$. Then we have the following identities for 
	Bernoulli numbers and binomial coefficients:
	\begin{align}
		\label{Alg:Lemmastuff:1_B0}
		\sum\limits_{j=0}^k \binom{k+1}{n} B_j
		& = 0 \\
		\label{Alg:Lemmastuff:1_B1}
		\sum\limits_{j=0}^k
		\binom{k+1}{n} B_j^* 
		& = k+1 \\
		\label{Alg:Lemmastuff:1_Carl}
		(-1)^k \sum\limits_{j=0}^k
		\binom{k}{j} B_{m+j} 
		& = 
		(-1)^m \sum\limits_{i=0}^m 
		\binom{m}{i} B_{k+i} \\
		\label{Alg:Lemmastuff:1_Bin1} 
		\sum\limits_{j=0}^m
		(-1)^j \binom{n}{j} 
		& = 
		(-1)^m \binom{n-1}{m} \\
		\label{Alg:Lemmastuff:1_Bin2}
		\binom{n}{m}\binom{m}{k}
		& =
		\binom{n}{k}\binom{n-k}{m-k}.
	\end{align}
	The identity \eqref{Alg:Lemmastuff:1_B1} is the standard recursion for 
	the $B_n^*$ (see e.g. \cite{aik}), \eqref{Alg:Lemmastuff:1_Bin1} and 
	\eqref{Alg:Lemmastuff:1_Bin2} can be found in good textbooks on 
	combinatorics such as \cite{aig_enum}. Equation 
	\eqref{Alg:Lemmastuff:1_Carl} is a theorem due to Carlitz, proven in 
	\cite{carl}. With them, we can show the next lemma which will finish this 
	subproof.
	\begin{lemma}
		\label{Alg:Lemma:BadSublemma2}
		Let $\mathcal{K}(k,s)$ be defined as in Lemma 
		\ref{Alg:Lemma:Badsublemma1}, then we have for all $k\in \mathbbm{N}$
		\begin{equation*}
			\mathcal{K}(k, s)
			=
			\begin{cases}
				1 & s=0 \\
				0 & \text{else}.		
			\end{cases}
		\end{equation*}
	\end{lemma}
	\begin{subproof}
		\mbox{}
		\begin{enumerate}[(i)]
		  \item $s=0$:
			The Kronecker-delta will always be zero unless $l=j=0$. So we get
			\begin{equation*}
				\mathcal{K}(k,0)
				=
				\frac{1}{k+1} \sum\limits_{n=0}^k
				\binom{k+1}{n} B_n^*
				=
				\frac{k+1}{k+1}
				=
				1
			\end{equation*}
			where we have used (\ref{1_B1}).

			\item $s=1$:
			To get a contribution, we must have $(j, \ell) = (1,0)$ or
			 $(0,1)$. Except for $n=0$ and $n=k$, both cases are possible. We 
			 split them off:
			\begin{align*}
				\mathcal{K}(k,1)
				& = 
				\underbrace{\frac{1}{k+1}}_{n=0}
				- \underbrace{k B_k^*}_{n=k}
				+ \frac{1}{k+1}
				\sum\limits_{n=1}^{k-1}
				\binom{k+1}{n}B_n^*
				\left(1 + (-1) \binom{n}{1} \right)\\
				& = 
				\frac{1}{k+1}
				+ \frac{1}{k+1} \sum\limits_{n=1}^{k-1}
				\binom{k+1}{n}B_n^* 
				- \frac{1}{k+1}	\sum\limits_{n=1}^{k-1}
				\binom{k+1}{n}n B_n^* 
				- k B_k^* \\
				& = 
				\underbrace{\frac{1}{k+1}
				\sum\limits_{n=0}^{k-1}}_{ = 1 - B_k^*} 
				- \frac{1}{k+1} \sum\limits_{n=0}^k
				\binom{k+1}{n}n B_n^*\\
				& =
				1 - B_k^* 
				- \underbrace{\frac{k+1}{k+1}
				\sum\limits_{n=0}^k
				\binom{k+1}{n} B_n^*}_{ = k+1}
				+ \sum\limits_{n=0}^k 
				\underbrace{\frac{k+1-n}{k+1}
				\binom{k+1}{n}}_{\binom{k}{n}} B_n^*\\
				& = 
				1 - B_k^* - k - 1 
				+ \sum\limits_{n=0}^{k-1}
				\binom{k}{n} B_n^* - B_k^*\\
				& =
				- k + 
				\sum\limits_{n}^{k-1} 
				\binom{k}{n} B_n^*\\
				& = 
				0
			\end{align*}
	
		  \item $s \mapsto s+1$:
			It is sufficient to prove 
			$\mathcal{K}(k,s+1) - \mathcal{K}(k,s) = 0$.
			In order to do that, we look for a better (i.e. well-suited, not
			necessarily nicely looking) form of $\mathcal{K}(k,s)$.
			\begin{align*}
				\mathcal{K}(k,s)
				& = 
				\frac{1}{k + 1}	\sum\limits_{n=0}^k
				\binom{k+1}{n} B_n^* \sum\limits_{j=0}^n
				(-1)^j \binom{n}{j} \sum\limits_{\ell=0}^{k-n}
				\delta_{s, \ell+j}\\
				& = 
				\frac{1}{k + 1} \sum\limits_{n=0}^s
				\binom{k+1}{n} B_n^* \sum\limits_{j=0}^n
				(-1)^j \binom{n}{j} \sum\limits_{\ell=0}^{k-n}
				\delta_{s, \ell+j} \\
				& \quad + \frac{1}{k + 1} \sum\limits_{n=s+1}^k
				\binom{k+1}{n} B_n^* \sum\limits_{j=0}^s
				(-1)^j \binom{n}{j} \sum\limits_{\ell=0}^{k-n}
				\delta_{s, \ell+j} \\
				& = \frac{1}{k + 1} \sum\limits_{n=0}^s
				\binom{k+1}{n} B_n^* 
				\sum\limits_{j=\max\{0, s+n-k\}}^n
				(-1)^j \binom{n}{j} \\
				& \quad + \frac{1}{k + 1}
				\sum\limits_{n=s+1}^k
				\binom{k+1}{n} B_n^*
				\sum\limits_{j=\max\{0, s+n-k\}}^s
				(-1)^j \binom{n}{j}
			\end{align*}
			As long as $\max\{0, s+n-k\} = 0$, the second sum in the first 
			summand will be zero as it is just the binomial expansion of 
			$(1-1)^n$. We may not forget that for $n=0$ we cannot plug this 
			in. Hence we get a special case and a shorter first sum. In the 
			sums over $j$ we use again the binomial expansion of $(1-1)^n$ and 
			get 
			\begin{align*}
				\mathcal{K}(k,s)
				& = 
				\frac{1}{k+1} \left[ 
				1 + \sum\limits_{k+1-s}^s 
				\binom{k+1}{n} B_n^* \left(
				- \sum\limits_{j=0}^{s+n-k-1} 
				(-1)^j \binom{n}{j}\right) \right. \\
				& \quad + \left. 
				\sum\limits_{n=s+1}^k 
				\binom{k+1}{n} B_n^* \left( 
				- \sum\limits_{j=0}^{s+n-k-1} 
				(-1)^j\binom{n}{j} - 
				\sum\limits_{j=s+1}^n 
				(-1)^j \binom{n}{j} \right) \right]
			\end{align*}
			Now it is helpful to use (\ref{1_Bin1}) and 
			$\binom{k}{n-k} = \binom{k}{n}$. 
			We also get some $-1$-terms which we can put together with the 
			$B_n^*$:
			\begin{align*}
				\mathcal{K}(k,s)
				& = 
				\frac{1}{k+1} \left[
				1+ \sum\limits_{n=k+1-s}^s
				\binom{k+1}{n} B_n (-1)^{k-s}
				\binom{n-1}{k-s} \right. \\
				& \quad + \left.
				\sum\limits_{n=s+1}^k
				\binom{k+1}{n}B_n \left(
				(-1)^{k-s}\binom{n-1}{k-s} + (-1)^{n+s}\binom{n-1}{s}
				\right) \right]
			\end{align*}
			We finally found the "better" form. We hence must compute
			$\mathcal{K}(k,s+1) - \mathcal{K}(k,s)$.
			Since we want to show that it is $0$, we can multiply it with
			$k+1$ in order to get rid of the factor in front:
			\begin{align*}
				& \quad
				(k+1) \left( \mathcal{K}(k,s+1) - \mathcal{K}(k,s) \right)\\
				& =
				\sum\limits_{n=k-s}^{s+1}
				\binom{k+1}{n} B_n (-1)^{k-s-1}
				\binom{n-1}{k-s-1} - \sum\limits_{n=k+1-s}^s
				\binom{k+1}{n} B_n (-1)^{k-s} \binom{n-1}{k-s} \\
				& \quad
				+ \sum\limits_{n=s+2}^k
				\binom{k+1}{n} B_n \left(
				(-1)^{k-s-1}\binom{n-1}{k-s-1}
				+ (-1)^{n+s+1}\binom{n-1}{s+1}
				\right) \\
				& \quad
				- \sum\limits_{n=s+1}^k
				\binom{k+1}{n} B_n \left(
				(-1)^{k-s}\binom{n-1}{k-s} + (-1)^{n+s}\binom{n-1}{s}
				\right) \\
				& =
				-\sum\limits_{n=k-s}^k
				\binom{k+1}{n} B_n (-1)^{k-s}
				\binom{n-1}{k-s-1} -\sum\limits_{n=k-s+1}^k
				\binom{k+1}{n}B_n (-1)^{k-s} \binom{n-1}{k-s} \\
				& \quad -
				\sum\limits_{n=s+2}^k
				\binom{k+1}{n}B_n (-1)^{n+s}
				\binom{n-1}{s+1} - \sum\limits_{n=s+1}^k
				\binom{k+1}{n}B_n (-1)^{n+s} \binom{n-1}{s} \\
				& =
				-\sum\limits_{n=k-s}^k
				\binom{k+1}{n}B_n (-1)^{k-s} \left(
				\binom{n-1}{k-s-1} + \binom{n-1}{k-s}
				\right) \\
				& \quad -
				\sum\limits_{n=s+1}^k
				\binom{k+1}{n} B_n (-1)^{n+s}
				\left( \binom{n-1}{s+1} + \binom{n-1}{s} \right)
			\end{align*}
			We have just rearranged the sums, added some $0$-terms and put 
			them together to get it nicer. Now we will use the recursion 
			formula for the binomial coefficients
			\begin{equation*}
				\binom{n-1}{k-1} + \binom{n-1}{k}
				=
				\binom{n}{k}
			\end{equation*}
			and our binomial multiplication equality (\ref{1_Bin2}):
			\begin{align*}
				& =
				- \sum\limits_{n=k-s}^k
				\binom{k+1}{n} B_n (-1)^{k-s} \binom{n}{k-s}
				- \sum\limits_{n=s+1}^k
				\binom{k+1}{n} B_n (-1)^{n+s} \binom{n}{s+1} \\
				& = 
				- \sum\limits_{n=k-s}^k 
				\binom{k+1}{s+1} \binom{s+1}{n+s-k} B_n (-1)^{k-s}
				- \sum\limits_{n=s+1}^k
				\binom{k+1}{s+1} \binom{k-s}{n-s-1} B_n (-1)^{n+s}
			\end{align*}
			Since we want to show that this is $0$, we can divide by
			$\binom{k+1}{s+1}$, because this will not be zero as long as 
			$s\leq k$. After doing so, can use the fact that $n>1$ in the 
			second sum of this induction (since $s\geq 1$) and hence only even 
			$n$ will come up, as for odd $n$ the Bernoulli numbers are zero. 
			For this reason we get $(-1)^n = 1$. Then we rewrite these sums by 
			shifting the indices and add two zeros:
			\begin{align*}
				& \quad
				- \sum\limits_{n=k-s}^k
				\binom{s+1}{n+s-k} B_n (-1)^{k-s}
				+ \sum\limits_{n=s+1}^k
				\binom{k-s}{n-s-1} B_n (-1)^{s+1} \\
				& =
				(-1)^{s+1} \sum\limits_{\ell=0}^{k-s-1}
				\binom{k-s}{\ell} B_{\ell+s+1}
				- (-1)^{k-s}\sum\limits_{\ell=0}^s 
				\binom{s+1}{\ell} B_{\ell + k - s} \\
				& =
				(-1)^{s+1} \sum\limits_{\ell=0}^{k-s}
				\binom{k-s}{\ell} B_{\ell+s+1}
				- (-1)^{k-s} \sum\limits_{\ell=0}^{s+1}
				\binom{s+1}{\ell} B_{\ell + k - s}\\
				& \quad
				- (-1)^{s+1} \binom{k-s}{k-s} B_{k+1} 
				+ (-1)^{k-s} \binom{s+1}{s+1} B_{k+1}
			\end{align*}
			The first two terms are exactly the Carlitz-identity 
			(\ref{1_Carl}), they vanish for this reason. So we are left with 
			the last two terms. We get
			\begin{equation*}
				- (-1)^{s+1} B_{k+1} + (-1)^{k-s} B_{k+1}
				=
				(-1)^s B_{k+1} \left(1 + (-1)^k\right) = 0
			\end{equation*}
			since the bracket will be zero if $k$ is odd and $B_{k+1}=0$ if 
			$k$ is even.
		\end{enumerate}
	\end{subproof}
\end{proof}
In Lemma \ref{Formulas:Lemma:LinearMonomial1}, we will see that also	$\ast_z$ 
fulfils this identity. Hence $\ast_z = \widehat{\star}_z$. We only need to 
show $\widehat{\star}_z = \star_z$. For $z = 1$, the two 	maps are clearly 
identical and therefore we find
\begin{equation*}
	\xi^k \star_r \eta
	=
	\sum\limits_{n=0}^k
	\binom{k}{n} B_n^* \xi^{k-n}
	\left( \ad_{\xi} \right)^k(\eta).
\end{equation*}
But knowing this and inserting it into the definition of $\star_z$, we get 
$\widehat{\star}_z = \star_z$. So with the proof in chapter 4, we have finally 
proven the following theorem:
\begin{theorem}
	\label{Alg:Thm:ThreeStarsAreOne}
	The three maps $\star_z$, $\widehat{\star}_z$ and $\ast_z$ 
	coincide on $\Sym^{\bullet}(\lie{g})$ and define star products.
\end{theorem}

