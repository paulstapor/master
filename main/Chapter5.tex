
%
% Chapter 5 of my master thesis:
% The analysis part begins
%

\chapter{A locally convex topology for the Gutt star product}

We have finished the algebraic part of this work, except for some little 
lemmas concerning the Hopf theoretic chapter. Our next goal is setting up a 
locally convex topology on the symmetric tensor algebra, in which the Gutt 
star product will be continuous. At the beginning of this chapter, we will 
first give a motivation why the setting of locally convex algebras is 
convenient and necessary. In the second part, we will briefly recall the most 
important things on locally convex algebras and introduce the topology which 
we will work with. In the third section, the core of this chapter, the 
continuity of the star product and the dependence on the formal parameter are 
proven. Part four treats the case when the formal parameter $z = 1$ and hence 
talks about a locally convex topology on the universal enveloping algebra of a 
Lie algebra. We will also show, that our topology is ''optimal'' in a specific 
sense.



\section{Why locally convex?}
\label{sec:chap5_Prelim}

The first question one could ask is, why we want the observable algebra 
to be a \textit{locally convex} one. There are a lot of different choices 
and most of them would even make things simpler: we could think of 
locally multiplicatively convex algebras, Banach algebras, $C^*$- or even 
von Neumann algebras. All of them have much more structure than just 
locally convex algebras. We would have an entire holomorphic calculus 
within our algebra if we assumed it to be locally m-convex or even a 
continuous one if we wanted it to be $C^*$.


The reason is, that all these nice features are simply not there, in 
general. Quantum mechanics shows us, that the algebra made up by 
$\hat{q}$ and $\hat{p}$ can not be locally m-convex.
\begin{proposition}
	\label{Prop:LCAna:QMnotLMC}
	Let $\algebra{A}$ be a unital associative algebra which contains the 
	quantum mechanical observables $\hat{q}$ and $\hat{p}$ and in which 
	the canonical commutation relation
	\begin{equation*}
		[\hat{q}, \hat{p}]
		=
		i \hbar
	\end{equation*}
	is fulfilled. Then the only submultiplicative semi-norm on it is 
	$p = 0$.
\end{proposition}
\begin{proof}
	First, we need to show a little lemma:
	\begin{lemma}
		\label{Lemma:LCAna:NotLMCHelp}
		In the given algebra, we have for $n \in \mathbb{N}$
		\begin{equation}
			\label{LCAna:NotLMCHelp}
			\left( \ad_{\hat{q}} \right)^n (\hat{p}^n)
			=
			(i \hbar)^n n! \Unit.
		\end{equation}
	\end{lemma}
	\begin{subproof}
		To show it, we use the fact that for $a \in \algebra{A}$ the 
		operator $\ad_a$ is a derivation, which is always true for a Lie 
		algebra which comes from an associative algebra with the 
		commutator, since for $a, b, c \in \algebra{A}$ we have
		\begin{equation*}
			[a, bc]
			=
			a b c - b c a
			=
			a b c - b a c + b a c - b c a
			=
			[a, b] c + b [a, c].
		\end{equation*}
		Now for $n = 1$ Equation \eqref{Lemma:LCAna:NotLMCHelp} is 
		certainly true. So let's look at the step $n \rightarrow n+1$.
		We make use of the derivation property and have
		\begin{align*}
			\left( \ad_{\hat{q}} \right)^{n+1}
			\left( \hat{p}^{n+1} \right)
			& =
			\left( \ad_{\hat{q}} \right)^{n}
			\left(
				i \hbar \hat{p}^n
				+
				\hat{p} 
				\ad_{\hat{q}} \left( \hat{p}^n \right)
			\right)
			\\
			& =
			(i \hbar)^{n + 1} n!
			+
			\left( \ad_{\hat{q}} \right)^{n}
			\left(
				\hat{p}
				\ad_{\hat{q}} \left( \hat{p}^n \right)
			\right)
			\\
			& =
			(i \hbar)^{n + 1} n!
			+
			\left( \ad_{\hat{q}} \right)^{n-1}
			\left(
				[\hat{q}, \hat{p}]
				\ad_{\hat{q}} \left( \hat{p}^n \right)
				+
				\hat{p}
				\left( \ad_{\hat{q}} \right)^2
				\left( \hat{p}^n \right)
			\right)
			\\
			& =
			(i \hbar)^{n + 1} n!
			+
			i \hbar
			\left( \ad_{\hat{q}} \right)^{n}
			\left( \hat{p}^n \right)
			+
			\left( \ad_{\hat{q}} \right)^{n-1}
			\left(
				\hat{p}
				\left( \ad_{\hat{q}} \right)^2
				\left( \hat{p}^n \right)
			\right)
			\\
			& =
			2 (i \hbar)^{n+1} n!
			+
			\left( \ad_{\hat{q}} \right)^{n-1}
			\left(
				\hat{p}
				\left( \ad_{\hat{q}} \right)^2
				\left( \hat{p}^n \right)
			\right)
			\\
			& \ot{($*$)}{=} 
			\quad \vdots
			\\
			& =
			n (i \hbar)^{n+1} n!
			+
			\ad_{\hat{q}}
			\left(
				\hat{p}
				\left( \ad_{\hat{q}} \right)^n
				\left( \hat{p}^n \right)
			\right)
			\\
			& =
			n (i \hbar)^{n+1} n!
			+
			i \hbar (i \hbar)^n n!
			\\
			& =
			(i \hbar)^{n + 1} (n + 1)!.
		\end{align*}
		At ($*$) we actually used another statement which is to be 
		proven by induction over $k$ and says
		\begin{equation*}
			\left( \ad_{\hat{q}} \right)^{n + 1}
			\left( \hat{p}^{n + 1} \right)
			=
			k (i \hbar)^{n + 1} n!
			+
			\left( \ad_{\hat{q}} \right)^{n + 1 - k}
			\left(
				\hat{p}
				\left( \ad_{\hat{q}} \right)^k
				\left( \hat{p}^n \right)
			\right).
		\end{equation*}
		Since this proof is analogous to the first lines of the 
		computation before, we omit it here and the lemma is proven.
	\end{subproof}	
	Now we can go on with the actual proof. Let $\norm{\cdot}$ be a 
	submultiplicative semi-norm. Then we see from Equation 
	\eqref{Lemma:LCAna:NotLMCHelp} that
	\begin{equation*}
		\norm{
			\left( \ad_{\hat{q}} \right)^n
			(\hat{p}^n)
		}
		=
		|\hbar|^n n! \norm{ \Unit }.
	\end{equation*}
	On the other hand, we have
	\begin{align*}
		\norm{
			\left( \ad_{\hat{q}} \right)^n
			(\hat{p}^n)
		}
		& =
		\norm{
			\hat{q}
			\left( \ad_{\hat{q}} \right)^{n-1}
			(\hat{p}^n)
			-
			\left( \ad_{\hat{q}} \right)^{n-1}
			(\hat{p}^n)
			\hat{q}
		}
		\\
		& \leq
		2 \norm{\hat{q}}
		\norm{
			\left( \ad_{\hat{q}} \right)^{n-1}
			(\hat{p}^n)
		}
		\\
		& \leq
		\quad \vdots
		\\
		& \leq
		2^n \norm{\hat{q}}^n
		\norm{ \hat{p}^n }
		\\
		& \leq
		2^n 
		\norm{\hat{q}}^n
		\norm{\hat{p}}^n
	\end{align*}
	So in the end we get
	\begin{equation*}
		|\hbar|^n n! \norm{ \Unit }
		\leq
		c^n
	\end{equation*}
	for some $c \in \mathbb{R}$. This cannot be fulfilled for all 
	$n \in \mathbb{N}$ unless $\norm{ \Unit } = 0$. But then, by
	submultiplicativity, the semi-norm itself must be equal to $0$.
\end{proof}
\begin{remark}
	The so called Weyl algebra, which fulfils the properties of the 
	foregoing proposition, can be constructed from a Poisson algebra with 
	constant Poisson tensor. On one hand, it is a fair question, why this 
	restriction of not being locally m-convex should also be put on 
	linear Poisson systems. On the other hand, there is no reason to 
	expect that things become easier when we make the Poisson system more 
	complex. Moreover, the Weyl algebra is actually nothing but a 
	quotient of the Universal enveloping algebra of the so called 
	Heisenberg algebra, a particular Lie algebra. So there is also no 
	reason why the original algebra should have a ''better'' analytical 
	structure than the quotient, since the ideal, which is divided out by 
	this procedure, is a closed one.
\end{remark}
There is a second good reason why we should avoid our topology to be locally 
m-convex. The topology we set up on $\Sym^{\bullet}(\lie{g})$ for a Lie 
algebra $\lie{g}$ will also give a topology on $\mathcal{U}(\lie{g})$.
In Proposition \ref{Prop:LCAna:NoExpTopology} we will show, that, under weak 
(but for our purpose necessary) additional assumptions, there can be no 
topology on $\mathcal{U}(\lie{g})$ which allows an entire holomorphic 
calculus. This underlines the results from Proposition 
\ref{Prop:LCAna:QMnotLMC}, since locally m-convex algebras always have such a 
calculus.


In this sense, we have good reasons to think that 
$\Sym^{\bullet}(\lie{g})$ will not allow a better setting than the
one of a locally convex algebra if we want the Gutt star product to 
be continuous. Before we attack this task, we have to recall some
technology from locally convex analysis.



\section{Locally convex algebras}
\label{sec:chap5_LCAlg}

\subsection{Locally convex spaces and algebras}

Every locally convex algebra is of course also a locally convex space which 
is, of course, a topological vector space. To make clear, what we talk about, 
we first give a definition, which is taken from \cite{Rudin:Blue}.
\begin{definition}[Topological vector space]
	\label{Def:TVSpace}
	Let $V$ be a vector space endowed with a topology $\tau$. Then we call 
	$(V, \tau)$ (or for short just $V$, if there is no confusion about the 
	topology possible) a topological vector space, if the two following things 
	hold:
	\begin{definitionlist}
		\item
		for every point in $x \in V$ the set $\{x\} $ is a closed and
		
		\item
		the vector space operations (addition, scalar multiplication) are 
		continuous.
	\end{definitionlist}
\end{definition}
Not all books require axiom $(i)$ for a topological vector space. It is, 
however useful, since it assures that the topology in a topological vector 
space is Hausdorff -- a feature, which we will always want to have. The proof 
for this is not difficult, but since we don't want to go too much into detail 
here, we refer to \cite{Rudin:Blue} again, where it can be found as Theorem 
1.12.


The most important class of topological vector spaces are, at least, but not 
only, from a physical point of view, locally convex ones. Almost all 
interesting physical examples belong to this class: Finite-dimensional spaces, 
inner product (or pre-Hilbert) spaces, Banach spaces, Fr\'echet spaces, 
nuclear spaces and many more. Now, there are at least two equivalent 
definitions of what is a locally convex space. While the first is more 
geometrical, the second is better suited for our analytic purpose.
\begin{theorem}
	\label{Thm:LCAna:LCSpace}
	For a topological vector space $V$, the following things are equivalent.
	\begin{theoremlist}
		\item
		$V$ has a local base $\algebra{B}$ of the topology whose members are 
		convex.
		
		\item
		The topology on $V$ is generated by a separating family of semi-norms 
		$\algebra{P}$.
	\end{theoremlist}
\end{theorem}
\begin{proof}
	This theorem is a very well-known result and can be found in standard 
	literature, such as \cite{Rudin:Blue}, where it is divided into the 
	Theorems 1.36 and 1.37.
\end{proof}
\begin{definition}[Locally convex space]
	\label{Def:LCSpace}
	A locally convex space is a topological vector space in which one (and 
	thus all) of the properties from Theorem \ref{Thm:LCAna:LCSpace} are 
	fulfilled.
\end{definition}
The first property explains the term ''locally convex''. 
For our intention, the second property is more helpful, 
since in this setting proving continuity just means putting estimates on 
semi-norms. For this purpose, one often extends the set of semi-norms 
$\algebra{P}$ to the set of all continuous semi-norms $\mathcal{P}$, which 
contains all semi-norms that are compatible with the topology (e.g. sums, 
multiples and maxima of (finitely many) semi-norms from $\algebra{P}$).


The next step are locally convex algebras.
\begin{definition}[Locally convex algebra]
	\label{Def:LCAlgebra}
	A locally convex algebra is a locally convex vector space with an 
	additional algebra structure, which is continuous.
\end{definition}
More precisely, let $\algebra{A}$ be a locally convex algebra and 
$\mathcal{P}$ the set of all continuous semi-norms, then for all $p \in 
\mathcal{P}$ there exists a $q \in \mathcal{P}$ such that for all $x, y \in 
\algebra{A}$ one has
\begin{equation}
	\label{LCAna:ProductContinuity}
	p(a b)
	\leq
	q(a) q(b).
\end{equation}
Remind that we didn't require our algebras to be associative. The product in 
this equation could also be a Lie bracket. If we talk about associative 
algebras, we will always say it explicitly.



\subsection{A special class of locally convex algebras}

For our study of the Gutt star product, the usual continuity estimate 
\eqref{LCAna:ProductContinuity} will not be enough, since 
there will be an arbitrarily high number of nested brackets to control. 
We will need an estimate which does not depend on the number of Lie 
brackets implied. But Lie also algebras are just a special type of algebras  
and the property we need makes sense for other types of locally convex 
algebra, too. This motivates the following definition.
\begin{definition}[Asymptotic estimate algebra]
	\label{Def:AE}
	Let $\algebra{A}$ be a locally convex algebra (not necessarily 
	associative) with $\cdot$ denoting the multiplication 
	and $\algebra{P}$ the set of all continuous semi-norms. 
	For a given $p \in \algebra{P}$ we call $q \in \algebra{P}$ an asymptotic 
	estimate for $p$, if there exists an $m \in \mathbb{N}$ such that for all 
	$n \geq m$ $q$ satisfies the following condition: $\forall_{x_1, \ldots, 
	x_n \in \algebra{A}}$
	\begin{equation}
		\label{LCAna:AE}
		p(x_1 \cdot \ldots \cdot x_n)
		\leq
		q(x_1) \ldots q(x_n).
	\end{equation}
	We call a locally convex algebra an AE-algebra, if every continuous 
	semi-norm admits an asymptotic estimate.
\end{definition}
\begin{remark}
	\label{Rem:AE1}
	Without further restrictions, we can set $m = 1$ in the upper definition, 
	since this just means taking the maximum over a finite number of 
	continuous semi-norms. If $q$ satisfies the upper definition for some $m 
	\in \mathbb{N}$ and for all $i = 2, \ldots, m-1$ we have
	\begin{equation*}
		p(x_1 \cdot \ldots \cdot x_i)
		\leq
		q^{(i)}(x_1) \ldots q^{(i)}(x_i)
	\end{equation*}
	for all $x_1, \ldots, x_i \in \mathcal{A}$, then we just set
	\begin{equation*}
		q'
		=
		\max\{ 
			p, q^{(2)}, \ldots, q^{(m-1)}, q
		\}.
	\end{equation*}
	Clearly, $q'$ will again be a continuous semi-norm and an asymptotic 
	estimate for $p$.
\end{remark}
\begin{remark}
	\label{Rem:AE2}
	\mbox{}
	\begin{remarklist}
		\item
		The term asymptotic estimate has, to the best of our knowledge, 
		first been used by Czichowski at all in [REFERENCE]. They 
		defined asymptotic estimates in the same way we did, but their 
		notion of AE-algebra was different from ours: in their 
		definition of an AE algebra, not just one but a series of 
		asymptotic estimates has to exist which fulfils two more 
		properties. This is not the case in our definition, which is, in 
		general, weaker.
		
		\item
		In \cite{}, Neeb and Gl\"ockner used a property to which they 
		referred as $(*)$ for associative algebras. It was then used in 
		\cite{} by ... and ..., who called it the $GN$-property. It is 
		easy to see that it is equivalent to being AE. 
		
		\item
		There are, of course, a lot of example for AE (Lie) algebras. 
		All finite dimensional and Banach (Lie) algebras fulfil 
		\eqref{LCAna:AE}, just as locally m-convex (Lie) algebras do. 
		The same is true for nilpotent locally convex Lie algebras, 
		since here again one just has to take the maximum of a finite 
		number of semi-norms. It is, however, far from clear what is exactly 
		implied by \eqref{Intro:AE}. Are there examples for associative 
		algebras which are AE but not locally m-convex, for example? Are there 
		Lie algebras which are truly and not nilpotent?
	\end{remarklist}
\end{remark}
It is at least possible to make some easy observations: an associative 
AE-algebra $\algebra{A}$ will admit an entire holomorphic calculus: let 
$f \colon \mathbb{C} \longrightarrow \mathbb{C}$ be an entire function 
and $a_n$ such that $f(z) = \sum_n a_n z^n$. Then one has 
$\forall_{x \in \algebra{A}}$
\begin{equation*}
	p(f(x))
	=
	p \left(
		\sum\limits_{n=0}^{\infty}
		a_n x^n
	\right)
	\leq
	\sum\limits_{n=0}^{\infty}
	|a_n| 
	p \left( x^n \right)
	\leq
	\sum\limits_{n=0}^{\infty}
	|a_n| q(x)^n
	<
	\infty
\end{equation*}
where $p \in \algebra{P}$ and $q$ its asymptotic estimate. So in some sense, 
at least in the associative case, AE algebras are very close to locally 
m-convex ones. If the algebra is even commutative and Fr\'echet, then the two 
notions coincide: in \cite{MRZ:1962:EntireCalculus} Mitiagin, Rolewicz and 
Zelazko proved that a commutative Fr\'echet algebra admitting an entire 
calculus is in fact locally m-convex. For non-commutative algebras, the 
situation is different. It is a very interesting (and non-trivial) question, 
if a non locally m-convex but AE algebra exists at all and if yes, how it 
could look like.


\subsection{The projective tensor product}
 - Inequality of the symmetric tensor product



\section{A topology for the Gutt star product}
\label{sec:chap5_TopologyStar}

\subsection{Continuity of the product}
 - a counter-example
 
\subsection{Dependence on the formal parameter}

\subsection{Completion}

\subsection{Nuclearity}



\section{Alternative topologies and an optimal result}
\label{sec:chap5_AlternativeOptimal}
