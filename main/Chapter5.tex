
%
% Chapter 5 of my master thesis:
% The analysis part begins
%

\chapter{A locally convex topology for the Gutt star product}

We have finished the algebraic part of this work, except for some little 
lemmas concerning the Hopf theoretic chapter. Our next goal is setting up a 
locally convex topology on the symmetric tensor algebra, in which the Gutt 
star product will converge. At the beginning of this chapter, we will 
first give a motivation why the setting of locally convex algebras is 
convenient and necessary. In the second part, we will briefly recall the most 
important things on locally convex algebras and introduce the topology which 
we will work with. In the third section, the core of this chapter, the 
continuity of the star product and the dependence on the formal parameter are 
proven. Part four treats the case when the formal parameter $z = 1$ and hence 
talks about a locally convex topology on the universal enveloping algebra of a 
Lie algebra. We will also show, that our topology is ''optimal'' in a specific 
sense.



\section{Why locally convex?}
\label{sec:chap5_Prelim}

The first question one could ask is why we want the observable algebra 
to be a \textit{locally convex} one. There are a lot of different choices 
and most of them would even make things simpler: we could think of 
locally multiplicatively convex algebras, Banach algebras, $C^*$- or even 
von Neumann algebras. All of them have much more structure than just 
locally convex algebras. We would have an entire holomorphic calculus 
within our algebra if we assumed it to be locally m-convex, or even a 
continuous one if we wanted it to be $C^*$.


The reason is that, in general, all these nice features are simply not there. 
Quantum mechanics tells us that the algebra made up by the space and momentum 
operators $\hat{q}$ and $\hat{p}$ can not be locally m-convex.
\begin{proposition}
	\label{Prop:LCAna:QMnotLMC}
	Let $\mathcal{A}$ be a unital associative algebra which contains the 
	quantum mechanical observables $\hat{q}$ and $\hat{p}$ and in which 
	the canonical commutation relation
	\begin{equation*}
		[\hat{q}, \hat{p}]
		=
		i \hbar \Unit
	\end{equation*}
	is fulfilled. Then the only submultiplicative semi-norm on it is 
	$p = 0$.
\end{proposition}
\begin{proof}
	First, we need to show a little lemma:
	\begin{lemma}
		\label{Lemma:LCAna:NotLMCHelp}
		In the given algebra, we have for $n \in \mathbb{N}$
		\begin{equation}
			\label{LCAna:NotLMCHelp}
			\left( \ad_{\hat{q}} \right)^n (\hat{p}^n)
			=
			(i \hbar)^n n! \Unit.
		\end{equation}
	\end{lemma}
	\begin{subproof}
		To show it, we use the fact that for $a \in \mathcal{A}$ the 
		operator $\ad_a$ is a derivation, which is always true for a Lie 
		algebra which comes from an associative algebra with the 
		commutator, since for $a, b, c \in \mathcal{A}$ we have
		\begin{equation*}
			[a, bc]
			=
			a b c - b c a
			=
			a b c - b a c + b a c - b c a
			=
			[a, b] c + b [a, c].
		\end{equation*}
		For $n = 1$, Equation \eqref{LCAna:NotLMCHelp} is 
		certainly true. So let's look at the step $n \rightarrow n+1$.
		We make use of the derivation property and have
		\begin{align*}
			\left( \ad_{\hat{q}} \right)^{n+1}
			\left( \hat{p}^{n+1} \right)
			& =
			\left( \ad_{\hat{q}} \right)^{n}
			\left(
				i \hbar \hat{p}^n
				+
				\hat{p} 
				\ad_{\hat{q}} \left( \hat{p}^n \right)
			\right)
			\\
			& =
			(i \hbar)^{n + 1} n!
			+
			\left( \ad_{\hat{q}} \right)^{n}
			\left(
				\hat{p}
				\ad_{\hat{q}} \left( \hat{p}^n \right)
			\right)
			\\
			& =
			(i \hbar)^{n + 1} n!
			+
			\left( \ad_{\hat{q}} \right)^{n-1}
			\left(
				[\hat{q}, \hat{p}]
				\ad_{\hat{q}} \left( \hat{p}^n \right)
				+
				\hat{p}
				\left( \ad_{\hat{q}} \right)^2
				\left( \hat{p}^n \right)
			\right)
			\\
			& =
			(i \hbar)^{n + 1} n!
			+
			i \hbar
			\left( \ad_{\hat{q}} \right)^{n}
			\left( \hat{p}^n \right)
			+
			\left( \ad_{\hat{q}} \right)^{n-1}
			\left(
				\hat{p}
				\left( \ad_{\hat{q}} \right)^2
				\left( \hat{p}^n \right)
			\right)
			\\
			& =
			2 (i \hbar)^{n+1} n!
			+
			\left( \ad_{\hat{q}} \right)^{n-1}
			\left(
				\hat{p}
				\left( \ad_{\hat{q}} \right)^2
				\left( \hat{p}^n \right)
			\right)
			\\
			& \ot{($*$)}{=} 
			\quad \vdots
			\\
			& =
			n (i \hbar)^{n+1} n!
			+
			\ad_{\hat{q}}
			\left(
				\hat{p}
				\left( \ad_{\hat{q}} \right)^n
				\left( \hat{p}^n \right)
			\right)
			\\
			& =
			n (i \hbar)^{n+1} n!
			+
			i \hbar (i \hbar)^n n!
			\\
			& =
			(i \hbar)^{n + 1} (n + 1)!.
		\end{align*}
		At ($*$), we actually used another statement which is to be 
		proven by induction over $k$ and says
		\begin{equation*}
			\left( \ad_{\hat{q}} \right)^{n + 1}
			\left( \hat{p}^{n + 1} \right)
			=
			k (i \hbar)^{n + 1} n!
			+
			\left( \ad_{\hat{q}} \right)^{n + 1 - k}
			\left(
				\hat{p}
				\left( \ad_{\hat{q}} \right)^k
				\left( \hat{p}^n \right)
			\right).
		\end{equation*}
		Since this proof is analogous to the first lines of the 
		computation before, we omit it here and the lemma is proven.
	\end{subproof}	
	Now we can go on with the actual proof. Let $\norm{\cdot}$ be a 
	submultiplicative semi-norm. Then we see from Equation 
	\eqref{LCAna:NotLMCHelp} that
	\begin{equation*}
		\norm{
			\left( \ad_{\hat{q}} \right)^n
			(\hat{p}^n)
		}
		=
		|\hbar|^n n! \norm{ \Unit }.
	\end{equation*}
	On the other hand, we have
	\begin{align*}
		\norm{
			\left( \ad_{\hat{q}} \right)^n
			(\hat{p}^n)
		}
		& =
		\norm{
			\hat{q}
			\left( \ad_{\hat{q}} \right)^{n-1}
			(\hat{p}^n)
			-
			\left( \ad_{\hat{q}} \right)^{n-1}
			(\hat{p}^n)
			\hat{q}
		}
		\\
		& \leq
		2 \norm{\hat{q}}
		\norm{
			\left( \ad_{\hat{q}} \right)^{n-1}
			(\hat{p}^n)
		}
		\\
		& \leq
		\quad \vdots
		\\
		& \leq
		2^n \norm{\hat{q}}^n
		\norm{ \hat{p}^n }
		\\
		& \leq
		2^n 
		\norm{\hat{q}}^n
		\norm{\hat{p}}^n
	\end{align*}
	So in the end we get
	\begin{equation*}
		|\hbar|^n n! \norm{ \Unit }
		\leq
		c^n
	\end{equation*}
	for some $c \in \mathbb{R}$. This cannot be fulfilled for all 
	$n \in \mathbb{N}$ unless $\norm{ \Unit } = 0$. But then, by
	submultiplicativity, the semi-norm itself must be equal to $0$.
\end{proof}
\begin{remark}
	The so called Weyl algebra, which fulfils the properties of the 
	foregoing proposition, can be constructed from a Poisson algebra with 
	constant Poisson tensor. On one hand, it is a fair to ask the question, 
	why this restriction of not being locally m-convex should also be put on 
	linear Poisson systems. On the other hand, there is no reason to 
	expect that things become easier when we make the Poisson system more 
	complex. Moreover, the Weyl algebra is actually nothing but a 
	quotient of the universal enveloping algebra of the so called 
	Heisenberg algebra, which is a particular Lie algebra. There is 
	no reason why the original algebra should have a ''better'' analytical 
	structure than its quotient, since the ideal, which is divided out by 
	this procedure, is a closed one.
\end{remark}
There's a second good reason why we should avoid our topology to be locally 
m-convex. The topology we set up on $\Sym^{\bullet}(\lie{g})$ for a Lie 
algebra $\lie{g}$ will also give a topology on $\mathcal{U}(\lie{g})$.
In Proposition \ref{Prop:LCAna:NoBetterTopology} ,we will show that, under 
weak (but for our purpose necessary) additional assumptions, there can be no 
topology on $\mathcal{U}(\lie{g})$ which allows an entire holomorphic 
calculus. This underlines the results from Proposition 
\ref{Prop:LCAna:QMnotLMC}, since locally m-convex algebras always have such a 
calculus.


In this sense, we have good reasons to think that 
$\Sym^{\bullet}(\lie{g})$ will not allow a better setting than the
one of a locally convex algebra if we want the Gutt star product to 
be continuous. Before we attack this task, we have to recall some
technology from locally convex analysis.



\section{Locally convex algebras}
\label{sec:chap5_LCAlg}

\subsection{Locally convex spaces and algebras}

Every locally convex algebra is of course also a locally convex space which 
is, of course, a topological vector space. To make clear what we talk about, 
we first give a definition which is taken from \cite{Rudin:Blue}.
\begin{definition}[Topological vector space]
	\label{Def:TVSpace}
	Let $V$ be a vector space endowed with a topology $\tau$. Then we call 
	$(V, \tau)$ (or just $V$, if there is no confusion possible) 
	a topological vector space, if the two following things  hold:
	\begin{definitionlist}
		\item
		for every point in $x \in V$ the set $\{x\} $ is a closed and
		
		\item
		the vector space operations (addition, scalar multiplication) are 
		continuous.
	\end{definitionlist}
\end{definition}
Not all books require axiom $(i)$ for a topological vector space. It is, 
however, useful, since it assures that the topology in a topological vector 
space is Hausdorff -- a feature which we will always want to have. The proof 
for this is not difficult, but since we don't want to go too much into detail 
here, we refer to \cite{Rudin:Blue} again, where it can be found as Theorem 
1.12.


The most important class of topological vector spaces are, at least, but not 
only, from a physical point of view, locally convex ones. Almost all 
interesting physical examples belong to this class: Finite-dimensional spaces, 
inner product (or pre-Hilbert) spaces, Banach spaces, Fr\'echet spaces, 
nuclear spaces and many more. There are at least two equivalent 
definitions of what is a locally convex space. While the first is more 
geometrical, the second is better suited for our analytic purpose.
\begin{theorem}
	\label{Thm:LCAna:LCSpace}
	For a topological vector space $V$, the following things are equivalent.
	\begin{theoremlist}
		\item
		$V$ has a local base $\algebra{B}$ of the topology whose members are 
		convex.
		
		\item
		The topology on $V$ is generated by a separating family of semi-norms 
		$\mathcal{P}$.
	\end{theoremlist}
\end{theorem}
\begin{proof}
	This theorem is a very well-known result and can be found in standard 
	literature, such as \cite{Rudin:Blue} again, where it is divided into 
	two 	Theorems (namely 1.36 and 1.37).
\end{proof}
\begin{definition}[Locally convex space]
	\label{Def:LCSpace}
	A locally convex space is a topological vector space in which one (and 
	thus all) of the properties from Theorem \ref{Thm:LCAna:LCSpace} are 
	fulfilled.
\end{definition}
The first property explains the term ''locally convex''. 
For our intention, the second property is more helpful, 
since in this setting proving continuity just means putting estimates on 
semi-norms. For this purpose, one often extends the set of semi-norms 
$\mathcal{P}$ to the set of all continuous semi-norms $\algebra{P}$ which 
contains all semi-norms that are compatible with the topology (e.g. sums, 
multiples and maxima of (finitely many) semi-norms from $\mathcal{P}$).
From here, we can start looking at locally convex algebras.


\begin{definition}[Locally convex algebra]
	\label{Def:LCAlgebra}
	A locally convex algebra is a locally convex vector space with an 
	additional algebra structure which is continuous.
\end{definition}
More precisely, let $\mathcal{A}$ be a locally convex algebra and 
$\algebra{P}$ the set of all continuous semi-norms, then for all $p \in 
\algebra{P}$ there exists a $q \in \algebra{P}$ such that for all $x, y \in 
\mathcal{A}$ one has
\begin{equation}
	\label{LCAna:ProductContinuity}
	p(a b)
	\leq
	q(a) q(b).
\end{equation}
Remind that we didn't require our algebras to be associative. The product in 
this equation could also be a Lie bracket. If we talk about associative 
algebras, we will always say it explicitly.



\subsection{A special class of locally convex algebras}

For our study of the Gutt star product, the usual continuity estimate 
\eqref{LCAna:ProductContinuity} will not be enough, since 
there will be an arbitrarily high number of nested brackets to control. 
We will need an estimate which does not depend on the number of Lie 
brackets involved. Since Lie algebras are just one type of algebras, 
we can define the property we need also for other locally convex algebras. 
\begin{definition}[Asymptotic estimate algebra]
	\label{Def:AE}
	Let $\mathcal{A}$ be a locally convex algebra (not necessarily 
	associative) with the set of all continuous semi-norms $\algebra{P}$. 
	For a given $p \in \algebra{P}$ we call $q \in \algebra{P}$ an 
	asymptotic estimate for $p$, if there exists an $m \in \mathbb{N}$ 
	such that for all $n \geq m$ we have
	\begin{equation}
		\label{LCAna:AE}
		p(x_1 \cdot \ldots \cdot x_n)
		\leq
		q(x_1) \ldots q(x_n)
		\quad
		\forall_{x_1, \ldots, x_n \in \algebra{A}}.
	\end{equation}
	For non-associative algebras, we want this estimate to be fulfilled 
	for all ways of setting brackets on the left hand side.
	We call a locally convex algebra an AE algebra, if every $p \in 
	\algebra{P}$ has an asymptotic estimate.
\end{definition}
\begin{remark}
	\label{Rem:LCAna:AE1}
	Without further restrictions, we can set $m = 1$ in the upper definition, 
	since this just means taking the maximum over a finite number of 
	continuous semi-norms. If $q$ satisfies the upper definition for some $m 
	\in \mathbb{N}$ and for all $i = 2, \ldots, m-1$ we have
	\begin{equation*}
		p(x_1 \cdot \ldots \cdot x_i)
		\leq
		q^{(i)}(x_1) \ldots q^{(i)}(x_i)
	\end{equation*}
	for all $x_1, \ldots, x_i \in \mathcal{A}$, then we just set
	\begin{equation*}
		q'
		=
		\max\{ 
			p, q^{(2)}, \ldots, q^{(m-1)}, q
		\}.
	\end{equation*}
	Clearly, $q'$ will again be a continuous semi-norm and an asymptotic 
	estimate for $p$.
\end{remark}
\begin{remark}[The notion ''asymptotic estimate'']
	\label{Rem:LCAna:AE2}
	\mbox{}
	\begin{remarklist}
		\item
		The term asymptotic estimate has, to the best of our knowledge, 
		first been used by Boseck, Czichowski and Rudolph in 
		\cite{BCR:AsympEsti:1981}. They defined asymptotic estimates in 
		the same way we did, but their idea of an AE algebra was different 
		from ours: for them, in an AE algebra every continuous semi-norm 
		admits a series of asymptotic estimates. This series must fulfil two 
		additional properties, which actually make the algebra locally 
		m-convex. Clearly, our definition is weaker,	since it does not imply, 
		a priori, the existence of an topologically 
		equivalent set of submultiplicative semi-norms.
		
		\item
		In \cite{GN:CIARegGroup:2012}, Gl\"ockner and Neeb used a property to 
		which they referred as $(*)$ for associative algebras. It was then 
		used in 	\cite{} by ... and ..., who called it the $GN$-property. It 
		is easy to see that it is equivalent to our AE condition. 
	\end{remarklist}
\end{remark}

There are, of course, a lot of examples of AE (Lie) algebras. 
All finite dimensional and Banach (Lie) algebras fulfil 
\eqref{LCAna:AE}, just as locally m-convex (Lie) algebras do. 
The same is true for nilpotent locally convex Lie algebras, 
since here again one just has to take the maximum of a finite 
number of semi-norms, analogously to the procedure in Remark 
\ref{Rem:LCAna:AE1}.

		
It is far from clear what is exactly implied by the AE property. 
Are there examples for associative algebras which are AE but not locally 
m-convex, for example? Are there Lie algebras which are truly AE and not 
locally m-convex or nilpotent? We don't have an answer to this questions,
but we can make some simple observations, which allow us to give an answer
for special cases.
\begin{proposition}[Entire calculus]
	Let $\mathcal{A}$ be an associative AE algebra. Then it has an
	entire holomorphic calculus.
\end{proposition}
\begin{proof}
	The proof is the same as for locally m-convex algebras: let 
	$f \colon \mathbb{C} \longrightarrow \mathbb{C}$ be an entire 
	function with $f(z) = \sum_n a_n z^n$ and $p$
	a continuous semi-norm with an asymptotic estimate $q$.
	Then one has $\forall_{x \in \mathcal{A}}$
	\begin{equation*}
		p(f(x))
		=
		p \left(
			\sum\limits_{n=0}^{\infty}
			a_n x^n
		\right)
		\leq
		\sum\limits_{n=0}^{\infty}
		|a_n| 
		p \left( x^n \right)
		\leq
		\sum\limits_{n=0}^{\infty}
		|a_n| q(x)^n
		<
		\infty.
	\end{equation*}
\end{proof}
\begin{remark}[Entire Calculus, AE and LMC algebras]
	The fact that AE algebras have an entire calculus makes them very similar 
	to locally m-convex ones.
	Now there is something we can say about associative algebras which have an 
	entire calculus: if such an algebra is additionally commutative and 
	Fr\'echet, then must be even locally m-convex. This statement was proved 
	in \cite{MRZ:EntireCalculus:1962} by Mitiagin, Rolewicz and 
	Zelazko. Oudadess and El kinani extended this result to commutative, 
	associative algebras, in which the Baire category theorem holds.
	For non-commutative algebras, the situation is different. There are 
	associative ''Baire algebras'' having an entire calculus, which are not 
	locally m-convex. Zelazko gave an example for such an algebra in 
	\cite{Zelazko:EntButNotLMC:1998}. Unfortunately, his example is also not 
	AE. It seems to be is an interesting (and non-trivial) question, 
	if a non locally m-convex but AE algebra exists at all and if yes, 
	how an example could look like.
\end{remark}



\subsection{The projective tensor product}

We want to set up a topology on $\Sym^{\bullet}(\lie{g})$. Therefore, 
we will first construct a topology on the tensor algebra $\Tensor^{\bullet}
(\lie{g})$. As all the following constructions in this section don't use any 
algebra structure, we will do them on a locally convex vector space $V$ where 
$\algebra{P}$ is the set of continuous semi-norms. Then we can use the 
projective tensor product $\tensor[\pi]$ in order to get a locally convex 
topology on each tensor power $V^{\tensor[\pi] n}$. The precise construction 
can be found in standard textbooks on locally convex analysis like 
\cite{Jarchow:LCSpaces} or in the lecture notes \cite{Waldmann:LectureAlgDyn}. 
Recall that for $p_1, \ldots, p_n \in \algebra{P}$ we have a continuous 
semi-norm on $V^{\tensor[\pi] n}$ via
\begin{equation*}
	(p_1 \tensor[\pi] \ldots \tensor[\pi] p_n)(x)
	=
	\inf
	\left\{
	\left.
		\sum_i
		p_1 \left( x_i^{(1)} \right)
		\ldots
		p_n \left( x_i^{(n)} \right)
	\right|
		x
		=
		\sum_i
		x_i^{(1)}
		\tensor \ldots \tensor
		x_i^{(n)}
	\right\}.
\end{equation*}
On factorizing tensors, we moreover have the property
\begin{equation}
	\label{LCAna:FactorTensor}
	(p_1 \tensor[\pi] \ldots \tensor[\pi] p_n)
	(x_1 \tensor[\pi] \ldots \tensor[\pi] x_n)
	=
	p_1(x_1) \ldots p_n(x_n)
\end{equation}
which will be extremely useful in the following and which can be proven by
the Hahn-Banach theorem. We also have 
\begin{equation*}
	(p_1 \tensor \ldots \tensor p_n) 
	\tensor 
	(q_1 \tensor \ldots \tensor q_m) 
	= 
	p_1 \tensor \ldots \tensor p_n \tensor q_1 \tensor \ldots \tensor q_m.
\end{equation*}
For a given $p \in \algebra{P}$ we will denote $p^n = p^{\tensor[\pi] n}$ and 
$p^0$ is just the absolute value on the field $\mathbb{K}$. The $\pi$-topology 
on $V^{\tensor[\pi] n}$ is set up by all the projective tensor products of 
continuous semi-norms, or, equivalently, by all the $p^n$ for $p \in 
\algebra{P}$.

The projective tensor product has a very nice feature: if we want to show a 
(continuity) estimate on the tensor algebra, it is enough to do it on 
factorizing tensors. We will use this very often and just refer to it as the 
''infimum argument''.
\begin{lemma}[Infimum argument for the projective tensor product]
	\label{Lemma:LCAna:InfimumArgument}
	Let $V_1, \ldots, V_n, W$ be locally convex vector spaces and 
	\begin{equation*}
		\phi \colon
		V_1 \times \ldots \times V_n
		\longrightarrow
		W
	\end{equation*}
	a $n$-linear map, from which we get the linear map 
	$\Phi \colon V_1 \tensor[\pi] \ldots \tensor[\pi] V_n \longrightarrow W$.
	Then $\Phi$ is continuous if and only if this is true for $\phi$ and if 
	for $p, q \in \algebra{P}$ the estimate
	\begin{equation*}
		p \left(
			\Phi (x_1 \tensor \ldots \tensor x_n)
		\right)
		\leq
		q(v_1) \ldots q(v_n)
	\end{equation*}
	is fulfilled for all $x_i \in V_i, i=1, \ldots, n$, then we have
	\begin{equation*}
		p \left(
			\Phi (x)
		\right)
		\leq
		q(x)
	\end{equation*}
	for all $x \in V_1 \tensor \ldots \tensor V_n$.
\end{lemma}
\begin{proof}
	If $\Phi$ is continuous, the continuity of $\phi$ is clear. 
	The other implication is more interesting.
	Continuity for $\phi$ means, that for every continuous semi-norm $q$ on 
	$W$ we have continuous semi-norms $p_i$ on $V_i$ with $i = 1, \ldots, n$ 
	such that for all $x^{(i)} \in V_i$ the estimate
	\begin{equation}
		\label{LCAna:ContiFactorTensors}
		q \left(
			\phi\left( x^{(1)}, \ldots, x^{(n)} \right)
		\right)
		\leq
		p_1 \left( x^{(1)} \right) 
		\ldots 
		p_n \left( x^{(n)} \right)
	\end{equation}
	holds. Let $x \in V_1 \tensor[\pi] \ldots \tensor[\pi] V_n$, then it 
	has a representation in terms of factorizing tensors like
	\begin{equation*}
		x
		=
		\sum_j
		x_j^{(1)} 
		\tensor[\pi] \ldots \tensor[\pi] 
		x_j^{(n)}.
	\end{equation*}
	We thus have
	\begin{align*}
		q( \Phi(x) )
		& =
		q \left(
			\sum_j
			\Phi \left(
				x_j^{(1)} 
				\tensor[\pi] \ldots \tensor[\pi] 
				x_j^{(n)}
			\right)
		\right)
		\\
		& \leq
		\sum_j
		q \left(
			\phi \left(
				x_j^{(1)} 
				, \ldots ,
				x_j^{(n)}
			\right)
		\right)
		\\
		& \leq
		\sum_j
		p_1 \left( x_j^{(1)} \right) 
		\ldots 
		p_n \left( x_j^{(n)} \right).
	\end{align*}
	Now we take the infimum over all possibilities of writing $x$ as a sum of 
	factorizing tensors on both sides. While nothing will happen on the left 
	hand side, on the right hand side we will find $\left( p_1 \tensor[\pi] 
	\ldots \tensor[\pi] p_n \right)(x)$. This gives exactly the estimate we 
	wanted.
\end{proof}


Most of the time, we will deal with the symmetric tensor algebra. Therefore, 
we want to recall some basic facts about $\Sym^n(V)$, when it inherits the 
$\pi$-topology from the $V^{\tensor[\pi] n}$. We will call it 
$\Sym_{\pi}^n(V)$ when we endow it with this topology.
\begin{lemma}
	\label{Lemma:LCAna:ProjTensSymm}
	Let $V$ be a locally convex vector space, $p$ a continuous semi-norm 
	and $n,m \in \mathbb{N}$.
	\begin{lemmalist}
		\item
		The symmetrization map
		\begin{equation*}
			\Symmetrizer_n
			\colon
			V^{\tensor_{\pi} n}
			\longrightarrow
			V^{\tensor_{\pi} n}
			, \quad
			(x_1 \tensor \ldots \tensor x_n)
			\longmapsto
			\frac{1}{n!}
			\sum\limits_{\sigma \in S_n}
			x_{\sigma(1)}
			\tensor \ldots \tensor
			x_{\sigma(n)}
		\end{equation*}
		is continuous and we have for all $x \in V^{\tensor_{\pi} n}$ 
		the estimate
		\begin{equation}
			\label{LCAna:SymmProdCont}
			p^n(\Symmetrizer_n(x))
			\leq
			p^n(x).
		\end{equation}
		
		\item
		Each symmetric tensor power $\Sym_{\pi}^n(V) \subseteq 
		V^{\tensor_{\pi} n}$ is a closed subspace.
		
		\item
		For $x \in \Sym_{\pi}^n(V)$ and $y \in \Sym_{\pi}^m(V)$ we have
		\begin{equation*}
			p^{n + m}(xy)
			\leq
			p^n(x) p^m(y).
		\end{equation*}
	\end{lemmalist}
\end{lemma}
\begin{proof}
	The first part is very easy to see and uses most of the tools which are 
	typical for the projective tensor product. We have the estimate for 
	factorizing tensors $x_1 \tensor \ldots \tensor x_n$
	\begin{align*}
		p^n \left(
			\Symmetrizer \left(
				x_1 \tensor \ldots \tensor x_n
			\right)
		\right)
		& =
		p^n \left(
			\frac{1}{n!}
			\sum	\limits_{\sigma \in S_n}
			x_{\sigma(1)} 
			\tensor \ldots \tensor 
			x_{\sigma(n)}
		\right)
		\\
		& \leq
		\frac{1}{n!}
		\sum\limits_{\sigma \in S_n}
		p^n \left(
			x_{\sigma(1)} 
			\tensor \ldots \tensor 
			x_{\sigma(n)}
		\right)
		\\
		& =
		\frac{1}{n!}
		\sum\limits_{\sigma \in S_n}
		p \left( x_{\sigma(1)} \right)
		\ldots
		p \left( x_{\sigma(n)} \right)
		\\
		& =
		p \left( x_1 \right)
		\ldots
		p \left( x_n \right)
		\\
		& =
		p^n \left(
			x_1 \tensor \ldots \tensor x_n
		\right).
	\end{align*}
	Then we use the infimum argument from Lemma 
	\ref{Lemma:LCAna:InfimumArgument} and we are done.
	The second part is also easy since the kernel of a continuous map is
	always a closed subspace of the initial space and we have
	\begin{equation*}
		\Sym_{\pi}^n 
		= 
		\ker (\id - \Symmetrizer_n).
	\end{equation*}
	The third part is a consequence from the first and also immediate.
\end{proof}
One could maybe think that the inequality in the first part of this lemma is 
just an artefact which is due to the infimum argument and should actually be 
an equality, if one looked to it more closely. It is very interesting to see, 
that this is \textit{not} the case, since it may happen that this 
inequality is strict. The following example illustrates this.
\begin{example}
	We take $V = \mathbb{R}^2$ with the standard basis $e_1,	e_2$ 
	and $V$ is endowed with the maximum norm. Now look at 
	$e_1 \tensor e_2$, which has the norm
	\begin{equation*}
		\norm{
			e_1 \tensor e_2
		}
		= 
		\norm{e_1}
		\tensor
		\norm{e_2}
		=
		1
	\end{equation*}
	We now evaluate the symmetrization map on $V \tensor[\pi] V$:
	\begin{equation*}
		\Symmetrizer \left(
			e_1 \tensor e_2
		\right)
		=
		\frac{1}{2}
		\left(
			e_1 \tensor e_2
			+
			e_2 \tensor e_1
		\right).
	\end{equation*}
	Our aim is to show, that the projective tensor product of the norm of this 
	symmetrized vector is not $1$. Therefore we need to find another way of 
	writing it which has a norm of less than $1$. Observe that
	\begin{equation*}
		\frac{1}{2}
		(e_1 \tensor e_2 + e_2 \tensor e_1)
		=
		\frac{1}{4}
		(
			(e_1 + e_2) \tensor (e_1 + e_2)
			+
			(-e_1 + e_2) \tensor (e_1 - e_2)
		)
	\end{equation*}
	and we have
	\begin{align*}
		& 
		\frac{1}{4}
		\norm{
			(e_1 + e_2) \tensor (e_1 + e_2)
			+
			(-e_1 + e_2) \tensor (e_1 - e_2)
		}
		\\
		& \leq
		\frac{1}{4}
		(
			\norm{ (e_1 + e_2) \tensor (e_1 + e_2) }
			+
			\norm{ (-e_1 + e_2) \tensor (e_1 - e_2) }
		)
		\\
		& =
		\frac{1}{4}
		(1 + 1)
		\\
		& =
		\frac{1}{2}.
	\end{align*}
	So we have $\norm{\Symmetrizer(e_1 \tensor e_2)} \leq \frac{1}{2} < 1$.
\end{example}


\subsection{A topology for the Gutt star product}

The next step is to set up a topology on $\Tensor^{\bullet}(V)$ which 
has the $\pi$-topology on each component. A priori, there are a lot of such 
topologies and at least two natural ones: the direct sum topology which is 
very fine and has a very small closure, and the cartesian product topology 
which is very coarse and therefore has a very big closure. We need something 
in between, which we can adjust in a convenient way.
\begin{definition}[R-topology]
	Let $p$ be an continuous semi-norm on a locally convex vector space 
	$V$ and $R \in \mathbb{R}$. We define the semi-norm
	\begin{equation*}
		p_R 
		= 
		\sum\limits_{n=0}^{\infty}
		n!^R p^n
	\end{equation*}
	on the Tensor algebra $\Tensor^{\bullet}(V)$. We write for 
	the tensor or the symmetric algebra endowed with all such semi-norms  
	$\Tensor_R^{\bullet}(V)$ or $\Sym_R^{\bullet}(V)$ respectively.
\end{definition}
We now want to collect the most important results on the locally convex 
algebras $\left( \Tensor_R^{\bullet}(V), \tensor \right)$ and $\left( 
\Sym_R^{\bullet}(V), \vee \right)$. 
\begin{lemma}
    \label{Lemma:LCAna:Projections}%
    Let $R' \geq R \geq 0$ and $q, p$ are continuous semi-norms on $V$.
    \begin{lemmalist}
	  \item \label{Item:EstimateForSeminorms}
	    	If $q \geq p$ then $q_R \geq p_R$ and $p_{R'} \geq p_R$.
	  \item \label{Item:TensorProductContinuous}
	    	The tensor product is continuous and satisfies the following 
	    	inequality:
	    	 \begin{equation*}
	    		p_R(x \tensor y)
	    		\leq
	    		\left( 2^R p \right)_R(x)
	    		\left( 2^R p \right)_R(y)
	    	\end{equation*}
	  \item \label{Item:PitopologyOnComponents}
	    	For all $n \in \mathbb{N}$ the induced topology on 
	    	$\Tensor^n(V) \subset \Tensor_R^{\bullet}(V)$ and on 
	    	$\Sym^n(V) \subset \Sym_R^{\bullet}(V)$ is the $\pi$-
	    	topology.
	  \item \label{Item:ComponentProjectionsContinuous}
	    	For all $n \in \mathbb{N}$ the projection and the inclusion 
	    	maps
	        \begin{equation*}
	        	\begin{array}{ccccc}
		    	    \Tensor_R^{\bullet}(V)
		        	&
	    	   		\longrightarrow
	    	    		&
	    	    		V^{\tensor_{\pi} n}
	    	    		&
	    	    		\longrightarrow
	    		    &
	    		    \Tensor^{\bullet}(V)
	    		    \\
		        \Sym_R^{\bullet}(V)
		        &
	    	        \longrightarrow
	    	    		&
	    	    		\Sym_{\pi}^n(V)
	    	    		&
	    	    		\longrightarrow
	    		    &
	    		    \Sym_R^{\bullet}(V)
	        	\end{array}
	        \end{equation*}
	        are continuous.
	  \item \label{Item:CompletionExplicitly}
    		The completions $\widehat{\Tensor}_R^{\bullet}(V)$ of 
    		$\Tensor_R^{\bullet}(V)$ and $\widehat{\Sym}_R^{\bullet}(V)$ 
    		of $\Sym_R^{\bullet}(V)$ can be described explicitly as
    		\begin{equation*}
	    		\begin{array}{ccccc}
		    		\widehat{\Tensor}_R^{\bullet}(V)
		    		&
		    		=
		    		&
		    		\left\{
		    		\left.
		    			x
		    			=
		    			\sum\limits_{n=0}^{\infty}
		    			x_n
		    		\ \right| \ 
		    			p_R(x)
		    			<
		    			\infty
		    			, \text{ for all } p
		    		\right\}
		    		&
		    		\subseteq
		    		&
		    		\prod\limits_{n=0}^{\infty}
		    		V^{\hat{\tensor}_{\pi} n}
		    		\\
		    		\widehat{\Sym}_R^{\bullet}(V)
		    		&
		    		=
		    		&
		    		\left\{
		    		\left.
		    			x
		    			=
		    			\sum\limits_{n=0}^{\infty}
		    			x_n
		    		\ \right| \ 
		    			p_R(x)
		    			<
		    			\infty
		    			, \text{ for all } p
		    		\right\}
		    		&
		    		\subseteq
	    			&
	    			\prod\limits_{n=0}^{\infty}
	    			\Sym_{\hat{\tensor}_{\pi}}^n
	    		\end{array}
    		\end{equation*}
    		with $p$ running through all continuous semi-norms on $V$ and
    		the $p_R$ are extended to the Cartesian product allowing the 
	    	value $+ \infty$.
      \item \label{Item:StrictlyFinerForBiggerR}
    		If $R' > R$, then the topology on $\Tensor_{R'}^{\bullet}(V)$ 
    		is strictly finer than the one on $\Tensor_R^{\bullet}(V)$, 
    		the same holds for $\Sym_{R'}^{\bullet}(V)$ and 
    		$\Sym_R^{\bullet}(V)$. Therefore the completions get smaller 
    		for bigger $R$.
      \item \label{Item:ComponentInclusionsContinuous}
    		The inclusion maps $\widehat{\Tensor}_{R'}^{\bullet}(V) 
    		\longrightarrow \widehat{\Tensor}_R^{\bullet}(V)$ and 
    		$\widehat{\Sym}_{R'}^{\bullet}(\lie{g}) \longrightarrow 
    		\widehat{\Sym}_R^{\bullet}(\lie{g})$ are continuous.
	  \item \label{Item:LmcJustForZero}
    		The topology on $\Tensor_R^{\bullet}(V)$ with the tensor 
    		product and on $\Sym_R^{\bullet}(V)$ with the symmetric 
    		product is locally m-convex if and only if $R = 0$.
      \item \label{Item:FirstCountable}
    		The algebras $\Tensor_R^{\bullet}(V)$ and $\Sym_R^{\bullet}(V)$
    		are first countable if and only if this is true for $V$.
    \end{lemmalist}
\end{lemma}
\begin{proof}
	The first part is clear on factorizing tensors and extends to the whole 
	tensor algebra via the infimum argument. For part $(ii)$, take two 
	factorizing tensors
	\begin{equation*}
		x
		=
		x^{(1)} \tensor \ldots \tensor x^{(n)}
		\quad \text{ and }
		y
		=
		y^{(1)} \tensor \ldots \tensor y^{(m)}
	\end{equation*}
	and compute:
	\begin{align*}
		p_R \left(
			x \tensor y
		\right)
		& =
		(n + m)!^R
		p^{n + m} \left(
			x^{(1)} \tensor \ldots x^{(n)}
			\tensor
			y^{(1)} \tensor \ldots y^{(m)}
		\right)
		\\
		& =
		(n + m)!^R
		p^n \left(
			x^{(1)} \tensor \ldots x^{(n)}
		\right)
		p^m \left(
			y^{(1)} \tensor \ldots y^{(m)}
		\right)
		\\
		& =
		\binom{n + m}{n}^R
		n!^R m!^R
		p^n \left(
			x^{(1)} \tensor \ldots x^{(n)}
		\right)
		p^m \left(
			y^{(1)} \tensor \ldots y^{(m)}
		\right)
		\\
		& \leq
		2^{(n + m) R}
		p_R \left( x^{(1)} \tensor \ldots x^{(n)} \right)
		p_R \left( y^{(1)} \tensor \ldots y^{(m)} \right)
		\\
		& =
		\left( 2^R p\right)_R 
		\left( x^{(1)} \tensor \ldots x^{(n)} \right)
		\left( 2^R p \right)_R 
		\left( y^{(1)} \tensor \ldots y^{(m)} \right).
	\end{align*}
	The parts $(iii)$ and $(iv)$ are clear from the construction of the $R$-
	topology. In part $(v)$ we used the completion of the tensor product 
	$\hat{\tensor}$, the statement itself is clear and implies $(vi)$ 
	directly, since we have really more elements in the completion for $R < 
	R'$, like the series over $x^{n} \frac{1}{n!^t}$ for $t \in (R, R')$ and 
	$0 \neq x \in V$. Statement $(vii)$ follows from the first. For $(viii)$, 
	it is easy to see that $\Tensor_0^{\bullet}(V)$ and $\Sym_0^{\bullet}(V)$ 
	are locally m-convex. For every $R > 0$ we have
	\begin{equation*}
		p_R \left(x^n \right)
		=
		n!^R
		p(x)^n
	\end{equation*}
	for all $n \in \mathbb{N}$ and all $x \in V$. If we had a 
	submultiplicative semi-norm $\norm{\cdot}$ from an equivalent topology, 
	then we would have some $x \in V$, and a continuous semi-norm $p$ with 
	$p(x) \neq 0$ such that $p_R \leq \norm{\cdot}$, and hence
	\begin{equation*}
		n!^R p(x)^R
		\leq
		\norm{x^n}
		\leq
		\norm{x}^n.
	\end{equation*}
	Since this is valid for all $n \in \mathbb{N}$, we get a contradiction.
	For the last part, the tensor algebras cannot be first countable if $V$ 
	itself isn't. On the other hand, if $V$ has a finite base of the topology, 
	then $\Tensor_R^{\bullet}(V)$ and $\Sym_R^{\bullet}(V)$ are just a 
	countable multiple of $V$ and stay therefore first countable.
\end{proof}
The projective tensor product obviously keeps a lot of important and 
strong properties of the original vector space $V$. But Proposition 
\ref{Prop:LCAna:Projections} still leaves some important things. 
We will not make use of them in the following, but it is worth naming 
them for completeness. To do this in full generality, we 
need one more definition, which will be also very important in chapter 5.
\begin{definition}\label{ProjectiveLimit}
	For a locally convex vector space $V$ and $R \geq 0$ we set
	\begin{equation*}
		\Sym_{R^-}^{\bullet}(V)
		=
		\projlim\limits_{\epsilon \longrightarrow 0}
		\Sym_{1 - \epsilon}^{\bullet}(V)
	\end{equation*}
	and call its completion $\widehat{\Sym}_{R^-}^{\bullet}(V)$.
\end{definition}
Now we can state two more propositions. Since we won't use them, we omit the 
proofs here. They can be found in \cite{Waldmann:nuclear:2014}.
\begin{proposition}
	\label{Prop:LCAna:Bases}
	Let $R \geq 0$ and $V$ a locally convex vector space.
	If $\{e_i\}_{i \in I}$ is an absolute Schauder basis of $V$ with 
	coefficient functionals $\{\varphi^i\}_{i \in I}$, i.e. for every 
	$x \in V$ we have
	\begin{equation*}
		x
		=
		\sum\limits_{i \in I}
		\varphi^i(x) e_i
	\end{equation*}
	such that for every $p \in \algebra{P}$ there is a 
	$q \in \algebra{P}$ such that
	\begin{equation}
		\label{LCAna:AbsoluteSchauder}
		\sum\limits_{i \in I}
		|\varphi^i(x)|
		p(e_i)
		\leq
		q(x),
	\end{equation}
	then the set 
	$\{e_{i_1} \tensor \ldots \tensor e_{i_n}\}_{i_1, \ldots, i_n \in I}$ 
	defines an absolute Schauder basis of $\Tensor_R^{\bullet}(V)$ 
	together with the linear functionals $\{\varphi^{i_1} \tensor \ldots 
	\tensor 	\varphi^{i_n}\}_{i_1, \ldots, i_n \in I}$ which satisfy
	\begin{equation*}
		\sum\limits_{n=0}^{\infty}
		\sum\limits_{i_1, \ldots, i_n \in I}
		\left| 
			\left(
				\varphi^{i_1} \tensor \ldots \tensor \varphi^{i_n}
			\right)
			(x)
		\right|
		p_R \left(
			e_{i_1} \tensor \ldots \tensor e_{i_n}
		\right)
		\leq
		q_R (x)
	\end{equation*}
	for every $x \in \Tensor_R^{\bullet}(V)$ whenever $p$ and $q$ satisfy 
	\eqref{LCAna:AbsoluteSchauder}. The same statement is true for 
	$\Sym_R^{\bullet}(V)$ and for $\Sym_{R^-}^{\bullet}(V)$ (for $R > 0$) when 
	we choose a maximal linearly independent subset out of the set $\{e_{i_1} 
	\ldots e_{i_n}\}_{i_1, \ldots, i_n \in I}$.
\end{proposition}
\begin{proposition}
	\label{LCAna:Nuclearity}
	Let $V$ be a locally convex space. For $R \geq 0$ the following statements 
	are equivalent:
	\begin{propositionlist}
	  \item
		$V$ is nuclear.
	  \item
	  	$\Tensor_R^{\bullet}(V)$ is nuclear.
	  \item
	  	$\Sym_R^{\bullet}(V)$ is nuclear.	  	
	\end{propositionlist}
	If moreover $R > 0$, then the following statements are equivalent:
	\begin{propositionlist}
	  \item
		$V$ is strongly nuclear.
	  \item
	  	$\Tensor_R^{\bullet}(V)$ is strongly nuclear.
	  \item
	  	$\Sym_R^{\bullet}(V)$ is strongly nuclear.
	\end{propositionlist}
\end{proposition}




\section{Continuity results for the Gutt star product}
\label{sec:chap5_TopologyStar}

From now on, we start with an AE Lie algebra $\lie{g}$ rather than with a 
general locally convex space. We have all the tools by the hand to show the 
continuity of the Gutt star product. We can do it either via the bigger 
formula \eqref{Formulas:2MonomialsFormula2} for two monomials or via the 
smaller one \eqref{Formulas:LinearMonomial2} for a monomial with a 
vector and iterate it. The results are very similar, but a bit better for the 
first approach. Nevertheless, both approaches give strong results, and 
depending on the precise situation, each one has its advantages. This is why 
we want to give both proofs here.

There will be a very general way how most of the proofs will work, and which 
tools will be used in the following. If we want to show the continuity of a 
map $f \colon \Sym_R^{\bullet}(\lie{g}) \longrightarrow 
\Sym_R^{\bullet}(\lie{g})$, we will proceed most of the time like this:
\begin{enumerate}
	\item \label{Item:LCAna:Step1}
	First, we extend a map to the whole tensor algebra by putting the 
	symmetrizer in front: $f = f \circ \Symmetrizer$. This doesn't lead 
	to problems since the symmetrization does not affect symmetric tensors.

	\item \label{Item:LCAna:Step2}
	Then, we start with an estimate, which we do only on 
	factorizing tensors in order to use the infimum argument 
	(Lemma \ref{Lemma:LCAna:InfimumArgument}).

	\item \label{Item:LCAna:Step3}
	During the estimation process, we find symmetric products of Lie 
	brackets. Those will be split up by the continuity of the symmetric 
	product \eqref{LCAna:SymmProdCont} from Lemma 
	\ref{Lemma:LCAna:ProjTensSymm} the AE property \eqref{LCAna:AE}.
	
	\item \label{Item:LCAna:Step4}
	Finally, we rearrange the split up semi-norms to the semi-norm of a 
	factorizing tensor by \eqref{LCAna:FactorTensor}.
\end{enumerate}


\subsection{Continuity of the product}
In the first proof, we want to approach the estimate via the formula
\begin{equation*}
    \xi_1 \cdots \xi_k \star_{zG} \eta_1 \cdots \eta_{\ell}
    =
    \sum\limits_{n=0}^{k + \ell -1}
    z^n
    C_n (\xi_1 \cdots \xi_k, \eta_1 \cdots \eta_{\ell})
\end{equation*}
Since this comes from polarizing the formula
\begin{equation}
    \label{LCAna:2MonomialsSumCn}
    \xi^k \star_{zG} \eta^{\ell}
    =
    \sum\limits_{n=0}^{k + \ell -1}
    z^n
    C_n \left( \xi^k, \eta^{\ell} \right),
\end{equation}
we will just give an explicit proof for the latter one. One gets the
estimate for the first one easily in the same way, since in the end,
all Lie brackets are broken up in step $(iii)$. One will get 
sums over permutations weighted with the inverse of their quantity and, as 
their semi-norms are just numbers which commute, one ends up with the same 
estimate as for \eqref{LCAna:2MonomialsSumCn}. First, we want to extend the 
Gutt star product to the whole tensor algebra: we define
\begin{equation*}
	\ostar_{zG}
	\colon
	\Tensor^{\bullet}(\lie{g})
	\times
	\Tensor^{\bullet}(\lie{g})
	\longrightarrow
	\Tensor^{\bullet}(\lie{g})
	, \quad
	\ostar_{zG} 
	= 
	\star_{zG} \circ \Symmetrizer.
\end{equation*}
\begin{theorem}
    \label{Thm:LCAna:Continuity1}%
    Let $\lie{g}$ be an AE-Lie algebra and $R \geq 1$, then for $x, y
    \in \Tensor_R^{\bullet}(\lie{g})$, $z \in \mathbb{C}$ and each 
    continuous semi-norm $p$ on $\lie{g}$ there exists a constant $c$ 
    such that we have for an asymptotic estimate $q$ of $p$
    \begin{equation}
        \label{LCAna:Continuity1}
        p_R(x \ostar_{zG} y)
        \leq
        (c q)_R(x) (c q)_R(y).
    \end{equation}
    Hence the Gutt star product is continuous on
    $\Sym_R^{\bullet}(\lie{g})$ for all $z \in \mathbb{C}$.
\end{theorem}
\begin{proof}
    We need to give estimates on the $z^n C_n$ in order to show their
    convergence.  Let us use $r = k + \ell - n$ for brevity and recall
    that the products are taken in the symmetric algebra.
    Then we can use Equation \eqref{Formulas:2MonomialsExplicit} in the 
    proof of Lemma \ref{Lemma:Formulas:2MonomialsFormula1} and put 
    estimates on it. Let $p$ be a continuous semi-norm
    and let $q$ be an asymptotic estimate for it. By using the
    continuity estimate for the symmetric tensor product in (a), the
    AE property of the Lie bracket \eqref{LCAna:AE} in (b) and then 
    simplifying the summation by adding more terms in (c), we get
    \begin{align}
        \nonumber
        p_R \left(
            z^n C_n\left( \xi^k, \eta^{\ell} \right)
        \right)
        & =
        p_R \bigg(
        z^n
        \frac{k! \ell!}{r!}
        \sum\limits_{\substack{a_1, b_1, \ldots, a_r, b_r \geq 0 \\
            a_i + b_i \geq 1 \\
            a_1 + \ldots + a_r = k \\
            b_1 + \ldots + b_r = \ell
          }}
        \bchparts{a_1}{b_1}{\xi}{\eta}
        \cdots
        \bchparts{a_r}{b_r}{\xi}{\eta}
        \bigg)
        \\
        \nonumber
        & \ot{(a)}{\leq}
        |z|^n
        \frac{k! \ell!}{r!}
        r!^R
        \sum\limits_{\substack{a_1, b_1, \ldots, a_r, b_r \geq 0 \\
            a_i + b_i \geq 1 \\
            a_1 + \ldots + a_r = k \\
            b_1 + \ldots + b_r = \ell
          }}
        p^{a_1 + b_1} \left( \bchparts{a_1}{b_1}{\xi}{\eta} \right)
        \cdots
        p^{a_r + b_r} \left( \bchparts{a_r}{b_r}{\xi}{\eta} \right)
        \\
        \nonumber
        & \ot{(b)}{\leq}
        |z|^n
        \frac{k! \ell!}{r!^{1-R}}
        q(\xi)^k q(\eta)^{\ell}
        \sum\limits_{\substack{a_1, b_1, \ldots, a_r, b_r \geq 0 \\
            a_i + b_i \geq 1 \\
            a_1 + \ldots + a_r = k \\
            b_1 + \ldots + b_r = \ell
          }}
        |\vartheta_{a_1, b_1}| \ldots |\vartheta_{a_r, b_r}|
        \\
        \label{LCAna:2MonomialsPreEstimate}
        & \ot{(c)}{\leq}
        |z|^n
        \frac{k! \ell!}{r!^{1-R}}
        q(\xi)^k q(\eta)^{\ell}
        \sum\limits_{\substack{j_1, \ldots, j_r \geq 1 \\
            j_1 + \ldots + j_r = k + \ell}}
        |\vartheta_{j_1}| \ldots |\vartheta_{j_r}|
    \end{align}
    Now we will use the fact that Thompson gave estimate for the growth 
    of the Baker-Campbell-Hausdorff coefficientsin in 
    \cite{Thompson:Convergence:1989}. More precisely he showed 
    $|\vartheta_j| \leq \frac{2}{j}$. For us, it will be sufficient that 
    for all $j \in \mathbb{N}$ we have $|\vartheta_j| \leq 2$. Knowing this 
    and using some easy combinatoric estimates we find
    \begin{equation*}
        \sum\limits_{
          \substack{
            j_1, \ldots, j_r \geq 1 \\
            j_1 + \ldots + j_r = k + \ell
          }
        }
        |\vartheta_{j_1}| \cdots |\vartheta_{j_r}|
        \leq
        2^r \sum\limits_{
          \substack
          {j_1, \ldots, j_r \geq 1 \\
            j_1 + \cdots + j_r = k + \ell
          }
        } 1
        \leq
        2^r \binom{k + \ell + r - 1}{k + \ell}
        \leq
        2^{3 (k + \ell) - 2n - 1}.
    \end{equation*}
    We put this together with \eqref{LCAna:2MonomialsPreEstimate}
    into the estimate
    \begin{align}
        \nonumber
        p_R \left(
            \xi^{\tensor k} \ostar_{zG} \eta^{\tensor \ell}
        \right)
        & =
        p_R \left(
            \sum\limits_{n=0}^{k + \ell - 1}
            z^n
            C_n \left( \xi^k, \eta^{\ell} \right)
        \right)
        \\
        \nonumber
        & \leq
        \sum\limits_{n=0}^{k + \ell - 1}
        |z|^n
        \frac{k! \ell!}{(k + \ell - n)!^{1-R}}
        q(\xi)^k q(\eta)^{\ell}
        2^{3 (k + \ell) - 2 n - 1}
        \\
        \nonumber
        & =
        \frac{1}{2}
        \sum\limits_{n=0}^{k + \ell - 1}
        \frac{|z|^n}{4^n}
        \left( \frac{k! \ell! n!}{(k + \ell - n)! n!}
        \right)^{1-R}
        (8 q)_R \left( \xi^{\tensor k} \right)
        (8 q)_R \left( \eta^{\tensor \ell} \right)
        \\
        \label{LCAna:2MonomialEstimate}
        & \leq
        \frac{1}{2}
        \sum\limits_{n=0}^{k + \ell - 1}
        \frac{|z|^n}{4^n}
        \frac{1}{n!^{R - 1}}
        \binom{k + \ell}{k}^{R - 1}
        \binom{k + \ell}{n}^{1 - R}
        (8 q)_R \left( \xi^{\tensor k} \right)
        (8 q)_R \left( \eta^{\tensor \ell} \right)
        \\
        \nonumber
        & \leq
        \frac{1}{2}
        \sum\limits_{n=0}^{k + \ell - 1}
        \frac{|z|^n}{4^n n!^{R - 1}}
        2^{(1-R)(k + \ell)}
        (8 q)_R \left( \xi^{\tensor k} \right)
        (8 q)_R \left( \eta^{\tensor \ell} \right)
        \\
        \nonumber
        & =
        \frac{1}{2}
        \sum\limits_{n=0}^{k + \ell - 1}
        \frac{|z|^n}{4^n n!^{R - 1}}
        (8 q)_R \left( \xi^{\tensor k} \right)
        (8 q)_R \left( \eta^{\tensor \ell} \right).
    \end{align}
    Remind that $R \geq 1$, so $2^{1-R} \leq 1$. For $|z| \leq 2$ we get
    \begin{align*}
        \nonumber
        p_R \left(
            \xi^{\tensor k} \ostar_{zG} \eta^{\tensor \ell}
        \right)
        & \leq
        \frac{1}{2}
        \sum\limits_{n=0}^{\infty}
        \frac{1}{2^n}
        (8 q)_R(\xi^{\tensor k})
        (8 q)_R(\eta^{\tensor \ell})
        \\
        & =
        (8 q)_R \left( \xi^{\tensor k} \right)
        (8 q)_R \left( \eta^{\tensor \ell} \right).
    \end{align*}
    For $|z| > 1$ we have on the other hand
    \begin{align*}
        \nonumber
        p_R \left(
            \xi^{\tensor k} \ostar_{zG} \eta^{\tensor \ell}
        \right)
        & \leq
        \frac{1}{2}
        \sum\limits_{n=0}^{k + \ell - 1}
        \frac{|z|^{k + \ell}}{4^n}
        (8 q)_R(\xi^{\tensor k})
        (8 q)_R(\eta^{\tensor \ell})
        \\
        & \leq
        (8z q)_R \left( \xi^{\tensor k} \right)
        (8z q)_R \left( \eta^{\tensor \ell} \right).
    \end{align*}
    Thus we can find in both cases a continuous semi-norm which
    fulfils \eqref{LCAna:Continuity1}. The proof for factorizing
    tensors is the same and one finally gets
    \begin{equation*}
    	p_R
    	\left(
    		\xi_1 \tensor \ldots \tensor \xi_k
    		\star_{zG}
    		\eta_1 \tensor \ldots \tensor \eta_{\ell}
    	\right)
    	\leq
    	(c q)_R
    	\left(
    		\xi_1 \tensor \ldots \tensor \xi_k
    	\right)
    	(c q)_R
    	\left(
    		\eta_1 \tensor \ldots \tensor \eta_{\ell}
    	\right)
    \end{equation*}
    with the same $c$'s as before. Having established this, we can use 
    the infimum argument to get the statement on arbitrary tensors
    and we are done.
\end{proof}
\begin{remark}
	\mbox{}
	\label{Rem:LCAna:CnOperators}
	We have actually proven more than just the continuity of the star product. 
	We will come back to this proof later in order to show also the entire 
	holomorphic dependence of the star product on the formal parameter. This 
	will be possible because we put bounds on the series 
	$\sum_n z^n C_n(\cdot, \cdot)$.
	This again means, that we have proven the continuity of the $C_n$ 
	operators, too: for arbitrary tensors $x, y \in \Tensor_R^{\bullet}
	(\lie{g})$ and a continuous semi-norm $p$, there is a constant $c$ 
	independent of $z$ (and even of $R$) such that
	\begin{equation}
		\label{LCAna:CnOperators}
		p_R
		\left( C_n(x, y) \right)
		\leq
		(c q)_R (x)
		(c q)_R (y)
	\end{equation}
	for all $n \in \mathbb{N}$, where $q$ is an asymptotic estimate 
	for $p$. But we even have more than that:
\end{remark}
\begin{corollary}
	\label{Coro:LCAna:ContinuityCn}
	Let $R \geq 0$ and $n \in \mathbb{N}$. Then, for every continuous semi-
	norm $p$ there exists a constant $c > 0$ such that for an asymptotic 
	estimate $q$ and every $x,y \in \Tensor_R^{\bullet}(\lie{g})$ we get
	the estimate
	\begin{equation}
		\label{LCAna:ContinuityCn}
		p_R \left( C_n(x,y) \right)
		\leq
		n!^{1 - R}
		(c q)_R(x) (c q)_R(y).
	\end{equation}
\end{corollary}
\begin{proof}
	We go back to \eqref{LCAna:2MonomialsPreEstimate} in the proof and use the 
	estimate for the sum over the $|\vartheta_i|$:
	\begin{align*}
		p_R \left(
			C_n \left( \xi^{\tensor k}, \eta^{\tensor \ell} \right)
		\right)
		& \leq
		|z|^n
		\frac{k! \ell!}{(k + \ell - n)!^{1-R}}
		2^{3(k + \ell)}
		q(\xi)^k q(\eta)^{\ell}
		\\
		& =
		|z|^n
		\left(
			\frac{k! \ell! (k + \ell)! n!}
			{(k + \ell - n)! (k + \ell)! n!}
		\right)^{1 - R}
		2^{3(k + \ell)}
		q_R\left( \xi^{\tensor k} \right)
		q_R\left( \eta^{\tensor \ell} \right)
		\\
		& \leq
		|z|^n
		2^{(1 - R)(k + \ell)}
		n!^{1 - R}
		2^{3(k + \ell)}
		q_R\left( \xi^{\tensor k} \right)
		q_R\left( \eta^{\tensor \ell} \right)
		\\
		& \leq
		n!^{1-R}
		\left( 16 (|z| + 1) \right)^{k + \ell}
		q_R\left( \xi^{\tensor k} \right)
		q_R\left( \eta^{\tensor \ell} \right).
	\end{align*}
	We just have to absorb the constant in front into the semi-norms and the 
	proof is done.
\end{proof}
\begin{remark}
	This makes things a bit clearer: The estimate \ref{LCAna:ContinuityCn} is 
	exactly the one from \eqref{LCAna:CnOperators} for $R \geq 1$. All the 
	$C_n$ are indeed continuous for any $R \geq 0$, but only for $R \geq 1$ 
	there is something like a uniform continuity. When $R$ decreases, the 
	continuity of the $C_n$'s ''gets worse'' and the uniform continuity 
	finally breaks down when the threshold $R = 1$ is trespassed. But we need
	this uniform estimate, since we have to control the operators up to an 
	arbitrarily high order if we want to guarantee the continuity of the 
	star product. Continuity up to a formerly chosen order $n$ does not 
	suffice.
\end{remark}



Now, we want to give the second proof, which relies on 
\eqref{Formulas:LinearMonomial2}. Approaching like this, we don't account for 
the fact that we will encounter terms like $[\eta, \eta]$ which will vanish, 
but we estimate more brutally. During this procedure, we will also count the 
formal parameter $z$ more often than it is actually there. This is why we will 
have to make assumptions on $R$ and $z$ which are a bit stronger than before. 
Moreover, we will split up tensor products and put them together again various 
times, which is the reason why an AE Lie algebra will not suffice any more: we 
will need $\lie{g}$ to be locally m-convex. But if we make these assumptions, 
we get the following lemma which will finally make the proof easier.
\begin{lemma}
    \label{Lemma:LCAna:PreContinuity2}%
    Let $\lie{g}$ be a locally m-convex Lie algebra and $R \geq1$. 
    Then if $|z| < 2 \pi$ or $R >1$ there exists for $x \in
    \Tensor^{\bullet}(\lie{g})$ of degree at most $k$, $\eta \in \lie{g}$
    and each continuous submultiplicative semi-norm $p$ a constant $c_{z,R}$ 
    only depending on $z$ and $R$ such that the following estimate holds:
    \begin{equation}
        \label{LCAna:PreContinuity2}
        p_R(x \ostar_{zG} \eta)
        \leq
        c_{z,R} (k+1)^R p_R(x) q(\eta)
    \end{equation}
\end{lemma}
\begin{proof}
    We start again with factorizing tensors. Since we get the same estimate 
    for monomials and for powers of some $\xi \in \lie{g}$ via polarization, 
    it is enough to consider $\xi^{\tensor k} \ostar_{zG} \eta$. This gives
    \begin{align*}
        p_R \left( \xi^{\tensor k} \ostar_{zG} \eta \right)
        & =
        p_R \left(
        \sum\limits_{n=0}^k
        \binom{k}{n} B_n^* z^n \xi^{k-n}
        \left( \ad_\xi \right)^n (\eta)
        \right)
        \\
        & =
        \sum\limits_{n=0}^k
        \binom{k}{n} |B_n^*| |z|^n
        (k + 1 - n)!^R
        p^{k + 1 - n} 
        \left(
         	\xi^{k-n}
         	\left( \ad_\xi \right)^n (\eta)
        \right)
        \\
        & \leq
        (k + 1)^R
        \sum\limits_{n=0}^k
        |B_n^*| |z|^n
        \frac{k! (k - n)!^R}{(k - n)! n!}
        p(\xi)^k p(\eta)
        \\
        & =
        (k + 1)^R
        \sum\limits_{n=0}^k
        \frac{|B_n^*| |z|^n}{n!^R}
        \left( \frac{(k - n)! n!}{k!} \right)^{R - 1}
        p_R \left( \xi^{\tensor k} \right) p(\eta)
        \\
        & \leq
        (k + 1)^R
        p_R \left( \xi^{\tensor k} \right) p(\eta)
        \sum\limits_{n=0}^k
        \frac{|B_n^*| |z|^n}{n!^R}.
    \end{align*}
    Now if $|z| < 2 \pi$ the sum can be estimated by extending it to a
    series which converges. We end up with a constant depending on $R$
    and on $z$ such that
    \begin{equation*}
        p_R \left( \xi^{\tensor k} \ostar_{zG} \eta \right)
        \leq
        (k + 1)^R c_{z, R}
        p_R \left( \xi^{\tensor k} \right) p(\eta).
    \end{equation*}
    If on the other hand $|z| \geq 2 \pi$ and $R > 1$ we can estimate
    \begin{align*}
        p_R \left( \xi^{\tensor k} \ostar_{zG} \eta \right)
        &\leq
        (k + 1)^R
        p_R \left( \xi^{\tensor k} \right) p(\eta)
        \left(
        \sum\limits_{n=0}^k
        \frac{|B_n^*|}{n!}
        \right) \left(
        \sum\limits_{n=0}^k
        \frac{|z|^n}{n!^{R - 1}}
        \right)
        \\
        &\leq
        (k + 1)^R
        \underbrace{2 \tilde{c}_{z,R}}_{= c_{z,R}}
        p_R \left( \xi^{\tensor k} \right) p(\eta).
    \end{align*}
    We hence have the estimate on factorizing tensors and can extend
    this to generic tensors of degree at most $k$ by the infimum argument.
\end{proof}


In the following, we assume again that either $R > 1$ or $R \geq 1$
and $|z| < 2\pi$ in order the use
Lemma~\ref{Lemma:LCAna:PreContinuity2}. Now we can give a simpler proof of 
Theorem~\ref{Thm:LCAna:Continuity1} for the case of a locally m-convex 
Lie algebra:
\begin{proof}[Alternative Proof of Theorem~\ref{Thm:LCAna:Continuity1}]
    Assume that $\lie{g}$ is now even locally m-convex.  We want to
    replace $\eta$ in the foregoing lemma by an arbitrary tensor $y$
    of degree at most $\ell$. Again, we do that on factorizing tensors
    first and get
    \begin{align*}
        p_R \left(
        \xi^{\tensor k} \ostar_{zG} \eta^{\tensor \ell}
        \right)
        & =
        p_R \big(
        \xi^{\tensor k}
        \underbrace{\ostar_{zG} \eta \ostar \cdots \ostar_{zG} \eta}_{
        \ell \textrm{-times}}
        \big)
        \\
        & \leq
        c_{z,R} (k + \ell)^R
        p_R \big(
        \xi^{\tensor k}
        \underbrace{\ostar_{zG} \eta \ostar \cdots \ostar_{zG} \eta}_{
        \ell - 1 \textrm{-times}}
        \big)
        p(\eta)
        \\
        & \leq
        \quad \vdots
        \\
        & \leq
        c_{z,R}^{\ell} ((k + \ell) \cdots (k + 1))^R
        p_R \left( \xi^{\tensor k} \right)
        p(\eta)^{\ell}
        \\
        & =
        c_{z,R}^{\ell} \left(
        \frac{(k + \ell)!}{k! \ell!}
        \right)^R
        p_R \left( \xi^{\tensor k} \right)
        p_R \left( \eta^{\tensor \ell} \right)
        \\
        & \leq
        (2^R p)_R \left( \xi^{\tensor k} \right)
        (2^R c_{z,R} p)_R \left( \eta^{\tensor \ell} \right).
    \end{align*}
    Once again, we have the estimate on factorizing tensors via
    polarization and extend it via the infimum argument to the whole
    tensor algebra, since the estimate depends no longer on the degree
    of the tensors.
\end{proof}


Using this approach for continuity, it is easy to see that nilpotency
of the Lie algebra changes the estimate substantially: If we knew that
we will have at most $N$ brackets because $N + 1$ brackets vanish,
then the sum in the proof of Lemma \ref{Lemma:LCAna:PreContinuity2}
would end at $N$ instead of $k$ and would therefore be independent of
the degree of $x$.


In both proofs, it is easy to see that we need at least $R \geq 1$ to
get rid of the factorials which come up because of the combinatorics
of the star product. It is nevertheless interesting to see that this
result is sharp, that means the Gutt star product really fails
continuity, if $R < 1$:
\begin{example}
    \label{Ex:LCAna:HeisenbergAlgebra}%
    Let $0 \leq R < 1$ and $\lie{g}$ be the Heisenberg algebra in three 
    dimensions, i.e. the Lie algebra generated by the elements 
    $P$, $Q$ and $E$ with the bracket $[P,Q] = E$ and all other brackets
    vanishing. This is a very simple example for a non-abelian Lie algebra
    and if continuity of the star product fails for this one, then we
    can not expect it to hold for more complex ones. We impose on
    $\lie{g}$ the $\ell^1$-topology with the norm $n$ and $n(P) = n(Q)
    = n(E) = 1$. This will be helpful, since here we really have the equality
    \begin{equation*}
    		p^{n+m} \left( X^n Y^m \right)
    		=
    		p^n \left( X^n \right)
    		p^m \left( Y^m \right)
    \end{equation*}
    for the symmetric product. Then we consider
    \begin{equation*}
        a_k
        =
        \frac{P^k}{k!^R}
        \quad
        \textrm{and}
        \quad
        b_k
        =
        \frac{Q^k}{k!^R}.
    \end{equation*}
    It is easy to see that
    \begin{equation*}
        n_R(a_k)
        =
        n_R(b_k)
        =
        1
    \end{equation*}
    We want to show that there is no $c > 0$ such that
    \begin{equation*}
        n_R(a_k \star_{zG} b_k)
        \leq
        (c n)_R(a_k) (c n)_R(b_k)
    \end{equation*}
    With other words, $n_R(a_k \star_{zG} b_k)$ grows faster than
    exponentially. But this is the case, since with our combinatorial
    formula \eqref{Formulas:2MonomialsFormula2} we see
    \begin{align*}
        n_R(a_k \star_{zG} b_k)
        & =
        n_R \left(
        \sum\limits_{j=0}^k
        \binom{k}{j}
        \binom{k}{j}
        j! \frac{1}{k!^{2R}}
        P^{k-j} Q^{k-j} E^j
        \right)
        \\
        & =
        \sum\limits_{j=0}^k
        \frac{k!^2 j! (2k - j)!^R}{(k-j)!^2 j!^2 k!^{2R}}
        \underbrace{
        n^{2k-j}
        ( P^{k-j} Q^{k-j} E^j )
        }_{= 1}
        \\
        & =
        \sum\limits_{j=0}^k
        \underbrace{
        \binom{k}{j}^2 \binom{2k}{k} \binom{2k}{j}^{-1}
        }_{\geq 1}
        j!^{1-R}
        \\
        & \geq
        \sum\limits_{j=0}^k
        j!^{1-R}
        \\
        & \geq
        k!^{1-R},
    \end{align*}
    which is exactly what we wanted to show.
\end{example}



\subsection{Dependence on the formal parameter}

We now look at the completion $\widehat{\Sym}_R^{\bullet}(\lie{g})$ of
the symmetric algebra with the Gutt star product $\star_{zG}$ and 
get the following negative result:
\begin{proposition}
    \label{proposition:NoExponentialsSorry}%
    Let $\xi \in \lie{g}$ and $R \geq 1$, then $\exp(\xi) \not\in
    \widehat{\Sym}_R^{\bullet}(\lie{g})$, where $\exp(\xi) =
    \sum_{n=0}^{\infty} \frac{\xi^n}{n!}$.
\end{proposition}
\begin{proof}
    Take a semi-norm $p$ such that $p(\xi) \neq 0$. Then set $c =
    p(\xi)^{-1}$. For $\xi^n$ the powers in the sense of either the
    usual tensor product, or the symmetric product or the star product
    are the same. So we have for $N \in \mathbb{N}$
    \begin{equation*}
        p_R \left(
        \sum\limits_{n=0}^N
        \frac{c^n}{n!} \xi^n
        \right)
        =
        \sum\limits_{n=0}^N
        \frac{n!^R}{n!}
        c^n
        p_R \left( \xi^n
        \right)
        =
        \sum\limits_{n=0}^N
        n!^{R - 1}
        \geq
        N,
    \end{equation*}
    and clearly $\exp(\xi)$ does not converge for the semi-norm $p_R$.
\end{proof}
As already mentioned, the proof of Theorem \ref{Thm:LCAna:Continuity1}
gives more than stated before. We know, that for $R \geq 1$ the star product 
is continuous on $\Sym_R^{\bullet}(\lie{g})$ and therefore has a continuous 
extension to $\widehat{\Sym}_R^{\bullet}(\lie{g})$, but this extension is a 
priori abstract. It does not need to be the series of the $C_n$ operators 
again. Yet, this is the case.
\begin{corollary}
	\label{Coro:LCAna:StarProdAbsConv}
	Let $\lie{g}$ be an AE Lie algebra and $R \geq 1$. Then, for every $z \in 
	\mathbb{C}$, the Gutt star product converges absolutely, i.e. for every 
	continuous semi-norm $p$, there is another continuous semi-norm $q$ such 
	that for all $f,g \in \widehat{\Sym}_R^{\bullet}(\lie{g})$
	\begin{equation}
		\label{LCAna:StarProdAbsConv}
		p_R \left( f \star_{zG} g \right)
		\leq
		\sum\limits_{n = 0}^{\infty}
		p_R \left( z^n C_n(f, g) \right)
		\leq
		q_R(f) q_R(g).
	\end{equation}
\end{corollary}
\begin{proof}
	The first inequality is clear. We know that for all $z \in \mathbb{C}$ 
	and for all $p \in \algebra{P}$ there is a $q \in \algebra{P}$ such that
	\begin{equation*}
		p_R \left( z^n C_n(f, g) \right)
		\leq
		q_R(f) q_R(g)
	\end{equation*}
	holds for all $n \in \mathbb{N}$. So there is also a $q' \in 
	\algebra{P}$ such that this holds for $2z$. We hence get
	\begin{equation*}
		p_R \left( z^n C_n(f, g) \right)
		\leq
		2^{-n} q'_R(f) q'_R(g).
	\end{equation*}
	Now the conclusion follows since
	\begin{align*}
		\sum\limits_{n = 0}^{\infty}
		p_R \left( z^n C_n(f, g) \right)
		& \leq
		\sum\limits_{n = 0}^{\infty}
		2^{-n} q'_R(f) q'_R(g)
		\\
		& \leq
		2 q'_R(f) q'_R(g)
	\end{align*}
	and the $2$ in front can be absorbed in the semi-norms.
\end{proof}
The consequence of Corollary \ref{Coro:LCAna:StarProdAbsConv} is that 
$\star_{zG}$ really converges to the formal series, and not just in an 
abstract sense. So the formal series 
\begin{equation*}
	x \star_{zG} y
	=
	\sum\limits_{n = 0}^{\infty}
	z^n C_n(x,y)
\end{equation*}
remains valid for elements $x,y$ in the completion. Knowing this and using the 
fact that all the projections on the homogeneous components are continuous 
from Lemma \ref{Lemma:LCAna:Projections} 
(\ref{Item:ComponentProjectionsContinuous}), we can reinterpret the 
continuity result we found in Theorem \ref{Thm:LCAna:Continuity1}.
\begin{proposition}
    \label{Prop:LCAna:HolomorphicDependence}%
    Let $R \geq 1$, then for all $f, g \in
    \widehat{\Sym}_R^{\bullet}(\lie{g})$ the map
    \begin{equation}
        \label{LCAna:Holomorphicity}
        \mathbb{K} \ni z
        \longmapsto
        f \star_{zG} g \in
        \widehat{\Sym}_R^{\bullet}(\lie{g})
    \end{equation}
    is real-analytic if $\mathbb{K} = \mathbb{R}$ and
    entire-holomorphic if $\mathbb{K} = \mathbb{C}$ with Taylor
    expansion at $z = 0$ given by Equation 
    \eqref{Formulas:2MonomialsFormula2}. The collection of the algebras 
    $\left\{ \left( \widehat{\Sym}_R^{\bullet}(\lie{g}), \star_{zG} \right) 
    \right\}_{z \in \mathbb{C}}$ is a holomorphic deformation of 
    the completed symmetric tensor algebra 
    $\left( \widehat{\Sym}_R^{\bullet}(\lie{g}), \vee \right)$.
\end{proposition}
\begin{proof}
	The important point is that for $f,g \in \widehat{\Sym}_R^{\bullet}
	(\lie{g})$ and every continuous semi-norm $p$ we have another 
	continuous semi-norm $q$ such that
	\begin{align*}
		p_R \left( f \star_{zG} g \right)
		& =
		p_R 
		\left(
			\sum\limits_{n=0}^{\infty}
			z^n C_n(f,g)
		\right)
		\\
		& =
		\sum\limits_{n=0}^{\infty}
		|z|^n
		p_R( C_n(f, g) )
		\\
		& \leq
		\sum\limits_{n=0}^{\infty}
		|z|^n
		q_R(f)
		q_R(g).
	\end{align*}
	We already showed that for every $M > 0$, there exists a $c \geq 1$, 
	such that in the open disc $|z| < M \subset \mathbb{C}$ the 
	inequality
	\begin{equation*}
		p_R \left( f \star_{zG} g \right)
		\leq
		\sum\limits_{n=0}^{\infty}
		|z|^n
		q_R(f)
		q_R(g)
		\leq
		(c q)_R(f)
		(c q)_R(g)
		<
		\infty
	\end{equation*}
	holds. Thus the map \eqref{LCAna:Holomorphicity} is holomorphic in 
	$z$ on every open disc around $0$. This means that the map is 
	actually entire.
\end{proof}
\begin{remark}
	If $R > 1$, we even have the result
	\begin{equation*}
		p_R \left(
			C_n (f, g)
		\right)
		\leq
		\frac{1}{n!^{R - 1}}
		q_R(f) q_R(g)
	\end{equation*}
	from \eqref{LCAna:CnOperators} for every $p \in \algebra{P}$ and a 
	suitable $q \in \algebra{P}$. In this case, we get
	\begin{equation*}
		p_R \left(
			f \star_{zG} g
		\right)
		\leq
		\sum\limits_{n = 0}^{\infty}
		p_R \left(
			z^n
			C_n (f, g)
		\right)
		\leq
		\sum\limits_{n = 0}^{\infty}
		\frac{|z|^n}{n!^{R - 1}}
		q_R(f) q_R(g),
	\end{equation*}
	and we see the entire dependence easier.
\end{remark}



\section{Functorialty, Representations and an optimal result}
\label{sec:chap5_Functoriality}

\subsubsection*{An optimal result}
Let's set the formal parameter $z = 1$ for a moment and make some 
observations. So far, we found a topology on $\Sym_R^{\bullet}(\lie{g})$ which 
gives a continuous star product and which has a reasonably large completion, 
but it is always fair to ask if we can do better than that: we've seen that 
our completed algebra will not contain exponential series, which would be a 
very nice feature to have. So is it possible to put another locally convex 
topology on $\Sym_R^{\bullet}(\lie{g})$ which gives a completion with 
exponentials? The answer is no, at least under mild additional assumptions. 
\begin{proposition}
	\label{Prop:LCAna:NoBetterTopology}
	Let $\lie{g}$ be an AE Lie algebra in which one has elements $\xi, \eta$ 
	for which the Baker-Campbell-Hausdorff series does not converge. 
	Then there is no locally convex topology on $\Sym^{\bullet}(\lie{g})$ 
	such that all of the following things are fulfilled:
	\begin{propositionlist}
		\item
		The Gutt star product $\star_G$ is continuous.
		\item
		For every $\xi \in \lie{g}$ the series $\exp(\xi)$ converges 
		absolutely in the completion of $\Sym^{\bullet}(\lie{g})$.
		\item
		For all $n \in \mathbb{N}$ the projection and inclusion maps with 
		respect to the graded structure
		\begin{equation*}
			\Sym^{\bullet}(\lie{g})
	    		\ot{$\pi_n$}{\longrightarrow}
    	    		\Sym^n(\lie{g})
	    	    	\ot{$\iota_n$}{\longrightarrow}
	    		\Sym^{\bullet}(\lie{g})
		\end{equation*}
		are continuous.
	\end{propositionlist}
\end{proposition}
\begin{proof}
	We will just need the projection $\pi_1$ to the Lie algebra itself. Let 
	$\xi, \eta \in \lie{g}$ such that $\bch{\xi}{\eta}$ does not exist, i.e.
	there is a continuous semi-norm $p$ such that the limit
	\begin{equation}\label{LCAna:BCHContradiction}
		p \left(
			\lim_{N \rightarrow \infty}
			\lim_{M \rightarrow \infty}		
			\sum\limits_{n,m = 0}^{N,M}
			\bchparts{\xi}{\eta}{n}{m}
		\right)				
	\end{equation}
	does not exist. But if we assume that the Gutt star product is continuous
	and that the exponential series is absolutely convergent, then we have
	\begin{align*}
	\pi_1 \left( \exp(\xi) \star_G \exp(\eta) \right)
	& =
	\pi_1
	\left(
		\lim_{N \rightarrow \infty}
		\left(
			\sum\limits_{n=0}^N
			\frac{\xi^n}{n!}
		\right)
		\star_G
		\lim_{M \rightarrow \infty}
		\left(
			\sum\limits_{m=0}^M
			\frac{\eta^m}{m!}
		\right)
	\right)
	\\
	& \ot{(a)}{=}
	\pi_1
	\left(
		\lim_{N \rightarrow \infty}
		\lim_{M \rightarrow \infty}
		\left(
			\sum\limits_{n=0}^N
			\frac{\xi^n}{n!}
		\right)
		\star_G
		\left(
			\sum\limits_{m=0}^M
			\frac{\eta^m}{m!}
		\right)
	\right)
	\\
	& \ot{(b)}{=}
	\lim_{N \rightarrow \infty}
	\lim_{M \rightarrow \infty}
	\pi_1
	\left(	
		\left(
			\sum\limits_{n=0}^N
			\frac{\xi^n}{n!}
		\right)
		\star_G
		\left(
			\sum\limits_{m=0}^M
			\frac{\eta^m}{m!}
		\right)
	\right)
	\\
	& \ot{(c)}{=}
	\lim_{N \rightarrow \infty}
	\lim_{M \rightarrow \infty}
	\sum\limits_{n,m=0}^{N, M}
	\bchparts{\xi}{\eta}{n}{m}
	\end{align*}
	where we used the continuity of the star product in (a), the continuity of 
	the projection in (b) and evaluated the projection in (c). Since 
	$\exp(\xi)$ and $\exp(\eta)$ are elements in the completion, their star 
	product exists and hence each continuous semi-norm has a well-defined 
	value for it. But if we take now the semi-norm from 
	\eqref{LCAna:BCHContradiction} of it, we get a contradiction.
\end{proof}
Of course, there are many ways of arranging the Baker-Campbell-Hausdorff 
series. There is no need to write it like in \eqref{LCAna:BCHContradiction}, 
but this is enough to illustrate the meaning of Proposition 
\ref{Prop:LCAna:NoBetterTopology}. For finite-dimensional Lie algebras, the 
statement is even better: In a finite-dimensional Lie algebra, the 
Baker-Campbell-Hausdorff series converges if and only if it converges 
absolutely, since we can write it as a power series.



\subsection*{Functoriality}

For $z = 1$, the algebra $\Sym_1^{\bullet}(\lie{g})$ with the 
Gutt star product $\star_G$ is isomorphic to the universal enveloping 
algebra $\mathcal{U}(\lie{g})$. Since $\mathcal{U}(\lie{g})$ and 
$\Sym^{\bullet}(\lie{g})$ have universal properties and we 
endowed them with a topology, we can ask whether we get some functorial
properties with our construction. In other words: Let $\mathcal{A}$ be
an associative, locally convex algebra and $\phi \colon \lie{g}
\longrightarrow \mathcal{A}$ a continuous Lie algebra homomorphism. We
have the commuting diagram
\begin{center}
    \begin{tikzpicture}
        \matrix (m)[
        matrix of math nodes,
        row sep=6em,
        column sep=7em
        ]
        {
          \mathcal{U}_R(\lie{g})
          & \Sym_R^{\bullet}(\lie{g}) \\
          \lie{g}
          & \mathcal{A} \\
        };
        \draw
        [-stealth]
        (m-1-1) edge node [above] {$\lie{q}^{-1}$}
        (m-1-2) edge node [left] {$\Phi$}
        (m-2-2)
        (m-1-2) edge node [right] {$\widetilde{\Phi}$}
        (m-2-2)
        (m-2-1) edge node [left] {$\iota$}
        (m-1-1)
        (m-2-1) edge node [below] {$\phi$}
        (m-2-2);
    \end{tikzpicture}
\end{center}
from the algebraic theory. The important question is now whether the
algebra homomorphisms $\Phi$ and $\widetilde{\Phi}$ are continuous. This 
question is partly answered by the following result:
\begin{proposition}
    \label{Thm:LCAna:Semi-functoriality}%
    Let $\lie{g}$ be an AE-Lie algebra, $\mathcal{A}$ an associative
    AE-algebra and $\phi \colon \lie{g} \longrightarrow \mathcal{A}$
    is a continuous Lie algebra homomorphism.  If $R \geq 0$, then the
    induced algebra homomorphisms $\Phi$ and $\widetilde{\Phi}$ are
    continuous.
\end{proposition}
\begin{proof}
    We define an extension of $\Phi$ on the whole tensor algebra
    again:
    \begin{equation*}
        \Psi \colon
        \Tensor_R^{\bullet}(\lie{g})
        \longrightarrow
        \mathcal{A},
        \quad
        \Psi
        =
        \widetilde{\Phi} \circ \Symmetrizer
    \end{equation*}
    It is clear that if $\Psi$ is continuous on factorizing tensors,
    we get the continuity of $\widetilde{\Phi}$ and of $\Phi$ via the
    infimum argument. So let $p$ be a continuous semi-norm on
    $\mathcal{A}$ with its asymptotic estimate $q$ and $\xi_1, \ldots,
    \xi_n \in \lie{g}$. Since $\phi$ is continuous, we find a
    continuous semi-norm $r$ on $\lie{g}$ such that for all $\xi \in
    \lie{g}$ we have $q(\phi(\xi)) \leq r(\xi)$. Then we have
    \begin{align*}
        p \left(
        \Psi \left(
        \xi_1 \tensor \cdots \tensor \xi_n
        \right) \right)
        & =
        p \left(
        \widetilde{\Phi} \left(
        \xi_1 \star_{zG} \cdots \star_{zG} \xi_n
        \right) \right)
        \\
        & =
        p( \phi(\xi_1) \cdots \phi(\xi_n) )
        \\
        & \leq
        q( \phi(\xi_1) )
        \cdots
        q( \phi(\xi_n) )
        \\
        & \leq
        r(\xi_1) \cdots r(\xi_n)
        \\
        & \leq
        r_R(\xi_1 \tensor \cdots \tensor \xi_n),
    \end{align*}
    where the last inequality is true for all $R \geq 0$.
\end{proof}


Although this is a nice result, our construction fails to be
universal, since the universal enveloping algebra endowed with our
topology is \emph{not} AE in general. This is even very easy to see:
\begin{example}
    Take $\xi \in \lie{g}$, then we know that $\xi^{\tensor n} =
    \xi^{\ostar_G n} = \xi^n$ for $n \in \mathbb{N}$ where the formal
    parameter is $z = 1$. Let $R > 0$ and $p$ a continuous semi-norm
    in $\lie{g}$ then we find
    \begin{equation}
        p_R(\xi^n)
        =
        n!^R p(\xi)^n
        =
        \frac{n!^R}{c^n} q(\xi)^n
    \end{equation}
    for $c = \frac{p(\xi)}{q(\xi)}$ for a different semi-norm $q$ with
    $q(\xi) \neq 0$.  But since the $\frac{n!^R}{c^n}$ will always
    diverge for $n \rightarrow \infty$ we will never get an asymptotic
    estimate for $p_R$.
\end{example}
Nevertheless, we can draw a nice conclusion from
Proposition~\ref{Thm:LCAna:Semi-functoriality}:
\begin{proposition}
    \label{Thm:LCAna:ContinuousRepresentations}%
    Let $R \geq 1$ and $\mathcal{U}_R(\lie{g})$ the universal
    enveloping algebra of an AE-Lie algebra $\lie{g}$, then for every
    continuous representation $\phi$ of $\lie{g}$ into the bounded
    linear operators $\Bounded(V)$ on a Banach space $V$ the induced
    homomorphism of associative algebras $\Phi \colon
    \mathcal{U}(\lie{g}) \longrightarrow \Bounded(V)$ is continuous.
\end{proposition}
\begin{proof}
    This follows directly from
    Proposition~\ref{Thm:LCAna:Semi-functoriality} and $\Bounded(V)$
    being a Banach algebra.
\end{proof}
\begin{remark}
    \label{Rem:LCAnaBCHConvergence}
    ~
    \begin{remarklist}
    \item \label{item:FiniteDimRepsContinuous} From this, it follows
        in particular that for all finite-dimensional Lie algebras all
        finite-dimensional representations on some vector space $V$
        extend to continuous algebra homomorphisms
        $\mathcal{U}_R(\lie{g}) \longrightarrow
        \operatorname{End}(V)$. For representations on
        infinite-dimensional Banach or Hilbert spaces, the statement
        is typically rather irrelevant, since there one rarely has
        norm-continuous representations, but merely strongly
        continuous ones.
    \item In \cite{pflaum.schottenloher:1998a} Schottenloher and
        Pflaum mention an alternative topology on
        $\mathcal{U}(\lie{g})$ for finite-dimensional Lie algebras:
        They took the coarsest locally convex topology, such that all
        finite-dimensional representations of $\lie{g}$ extend to
        continuous algebra homomorphisms. This topology is in fact
        even locally m-convex. Our topology which uses the grading on
        $\Sym_R^{\bullet}(\lie{g})$ is different from that: As we have
        seen in Proposition~\ref{Thm:LCAna:ContinuousRepresentations},
        it is finer for $R \geq 0$ and even strictly finer for $R >
        0$. For the interesting case $R \geq 1$ it is ''just'' locally
        convex, but its advantage (for our purpose) is that it
        respects the grading, which is helpful for the holomorphic
        dependence on the formal parameter. 
    \end{remarklist}
\end{remark}
