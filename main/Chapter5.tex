
%
% Chapter 5 of my master thesis:
% The analysis part begins
%

\chapter{A locally convex topology for the Gutt star product}

Here should already be some stuff...

\section{Why locally convex?}
\label{sec:chap5_Prelim}

The first question one could ask is, why we want the observable algebra 
to be a \textit{locally convex} one. There are a lot of different choices 
and most of them would even make things simpler: we could think of 
locally multiplicatively convex algebras, Banach algebras, $C^*$- or even 
von Neumann algebras. All of them have much more structure than just 
locally convex algebras. We would have an entire holomorphic calculus 
within our algebra if assumed it to be locally m-convex or even a 
continuous one if we wanted it to be $C^*$.


The reason is, that all these nice features are simply not there, in 
general. Quantum mechanics shows us, that the algebra made up by 
$\hat{q}$ and $\hat{p}$ can not be locally m-convex.
\begin{proposition}
	\label{Prop:LCAna:QMnotLMC}
	Let $\algebra{A}$ be a unital associative algebra which contains the 
	quantum mechanical observables $\hat{q}$ and $\hat{p}$ and in which 
	the canonical commutation relation
	\begin{equation*}
		[\hat{q}, \hat{p}]
		=
		i \hbar
	\end{equation*}
	is fulfilled. Then the only submultiplicative semi-norm on it is 
	$p = 0$.
\end{proposition}
\begin{proof}
	First, we need to show a little lemma:
	\begin{lemma}
		\label{Lemma:LCAna:NotLMCHelp}
		In the given algebra, we have for $n \in \mathbb{N}$
		\begin{equation}
			\label{LCAna:NotLMCHelp}
			\left( \ad_{\hat{q}} \right)^n (\hat{p}^n)
			=
			(i \hbar)^n n! \Unit.
		\end{equation}
	\end{lemma}
	\begin{subproof}
		To show it, we use the fact that for $a \in \algebra{A}$ the 
		operator $\ad_a$ is a derivation, which is always true for a Lie 
		algebra which comes from an associative algebra with the 
		commutator, since for $a, b, c \in \algebra{A}$ we have
		\begin{align*}
			[a, bc]
			& =
			a b c - b c a
			\\
			& =
			a b c - b a c + b a c - b c a
			\\
			& =
			[a, b] c + b [a, c].
		\end{align*}
		Now for $n = 1$ Equation \eqref{Lemma:LCAna:NotLMCHelp} is 
		certainly true. So let's look at the step $n \rightarrow n+1$.
		We make use of the derivation property and have
		\begin{align*}
			\left( \ad_{\hat{q}} \right)^{n+1}
			\left( \hat{p}^{n+1} \right)
			& =
			\left( \ad_{\hat{q}} \right)^{n}
			\left(
				i \hbar \hat{p}^n
				+
				\hat{p} 
				\ad_{\hat{q}} \left( \hat{p}^n \right)
			\right)
			\\
			& =
			(i \hbar)^{n + 1} n!
			+
			\left( \ad_{\hat{q}} \right)^{n}
			\left(
				\hat{p}
				\ad_{\hat{q}} \left( \hat{p}^n \right)
			\right)
			\\
			& =
			(i \hbar)^{n + 1} n!
			+
			\left( \ad_{\hat{q}} \right)^{n-1}
			\left(
				[\hat{q}, \hat{p}]
				\ad_{\hat{q}} \left( \hat{p}^n \right)
				+
				\hat{p}
				\left( \ad_{\hat{q}} \right)^2
				\left( \hat{p}^n \right)
			\right)
			\\
			& =
			(i \hbar)^{n + 1} n!
			+
			i \hbar
			\left( \ad_{\hat{q}} \right)^{n}
			\left( \hat{p}^n \right)
			+
			\left( \ad_{\hat{q}} \right)^{n-1}
			\left(
				\hat{p}
				\left( \ad_{\hat{q}} \right)^2
				\left( \hat{p}^n \right)
			\right)
			\\
			& =
			2 (i \hbar)^{n+1} n!
			+
			\left( \ad_{\hat{q}} \right)^{n-1}
			\left(
				\hat{p}
				\left( \ad_{\hat{q}} \right)^2
				\left( \hat{p}^n \right)
			\right)
			\\
			& \ot{($*$)}{=} 
			\quad \vdots
			\\
			& =
			n (i \hbar)^{n+1} n!
			+
			\ad_{\hat{q}}
			\left(
				\hat{p}
				\left( \ad_{\hat{q}} \right)^n
				\left( \hat{p}^n \right)
			\right)
			\\
			& =
			n (i \hbar)^{n+1} n!
			+
			i \hbar (i \hbar)^n n!
			\\
			& =
			(i \hbar)^{n + 1} (n + 1)!.
		\end{align*}
		At ($*$) we actually used another statement which is to be 
		proven by induction over $k$ and says
		\begin{equation*}
			\left( \ad_{\hat{q}} \right)^{n + 1}
			\left( \hat{p}^{n + 1} \right)
			=
			k (i \hbar)^{n + 1} n!
			+
			\left( \ad_{\hat{q}} \right)^{n + 1 - k}
			\left(
				\hat{p}
				\left( \ad_{\hat{q}} \right)^k
				\left( \hat{p}^n \right)
			\right).
		\end{equation*}
		Since this proof is analogous to the first lines of the 
		computation before, we omit it here and the lemma is proven.
	\end{subproof}	
	Now we can go on with the actual proof. Let $\norm{\cdot}$ be a 
	submultiplicative semi-norm. Then we see from Equation 
	\eqref{Lemma:LCAna:NotLMCHelp} that
	\begin{equation*}
		\norm{
			\left( \ad_{\hat{q}} \right)^n
			(\hat{p}^n)
		}
		=
		|\hbar|^n n! \norm{ \Unit }.
	\end{equation*}
	On the other hand, we have
	\begin{align*}
		\norm{
			\left( \ad_{\hat{q}} \right)^n
			(\hat{p}^n)
		}
		& =
		\norm{
			\hat{q}
			\left( \ad_{\hat{q}} \right)^{n-1}
			(\hat{p}^n)
			-
			\left( \ad_{\hat{q}} \right)^{n-1}
			(\hat{p}^n)
			\hat{q}
		}
		\\
		& \leq
		2 \norm{\hat{q}}
		\norm{
			\left( \ad_{\hat{q}} \right)^{n-1}
			(\hat{p}^n)
		}
		\\
		& \leq
		\quad \vdots
		\\
		& \leq
		2^n \norm{\hat{q}}^n
		\norm{ \hat{p}^n }
		\\
		& \leq
		2^n 
		\norm{\hat{q}}^n
		\norm{\hat{p}}^n
	\end{align*}
	So in the end we get
	\begin{equation*}
		|\hbar|^n n! \norm{ \Unit }
		\leq
		c^n
	\end{equation*}
	for some $c \in \mathbb{R}$. This cannot be fulfilled for all 
	$n \in \mathbb{N}$ unless $\norm{ \Unit } = 0$. But then, by
	submultiplicativity, the semi-norm itself must be equal to $0$.
\end{proof}
\begin{remark}
	The so called Weyl algebra, which fulfils the properties of the 
	foregoing proposition, can be constructed from a Poisson algebra with 
	constant Poisson tensor. It is a fair question, why this 
	restriction of not being locally m-convex should also be put on 
	linear Poisson systems. On the other hand, there is no reason to 
	expect that things become easier when we make the Poisson system more 
	complex. Moreover, the Weyl algebra is actually nothing but a 
	quotient of the Universal enveloping algebra of the so called 
	Heisenberg algebra, a particular Lie algebra. So there is also no 
	reason why the original algebra should have a ''better'' analytical 
	structure than the quotient, since the ideal, which is divided out by 
	this procedure, is closed.
\end{remark}
\section{Locally convex algebras}
\label{sec:chap5_LCAlg}


\subsection{Locally convex spaces and algebras}


\subsection{A special class of algebras}

For a locally convex Lie algebra $\lie{g}$, the Lie bracket is usually 
assumed to be continuous. This means that for every continuous 
semi-norms $p$ there exists another continuous semi-norms $q$ such that 
for all $\xi, \eta \in \lie{g}$ one has
\begin{equation*}
	p([\xi, \eta])
	\leq
	q(\xi) q(\eta).
\end{equation*}
For our study of the Gutt star product, this will not be enough, since 
we will have to estimate an arbitrary high number of nested brackets. We 
will need an estimate which does not depend on the number of lie 
brackets implied. But Lie algebras are just a special case and the 
property makes sense for any kind of locally convex algebra. This 
motivates the following definition.
\begin{definition}
	\label{def:AE}
	Let $\algebra{A}$ be a Hausdorff, locally convex algebra (not 
	necessarily 	associative) with $\cdot$ denoting the multiplication 
	and 	$\{p_i\}_{i\in I}$ the set of all continuous semi-norms which 
	defines the topology. For a given continuous semi-norm $p$ we call 
	another continuous semi-norm $q$ an asymptotic estimate for $p$, if 
	there exists an $m \in \mathbb{N}$ such that for all $n \geq m$ the 
	following holds:
	\begin{equation}
		\label{Intro:AE}
		p(x_1 \cdot \ldots \cdot x_n)
		\leq
		q(x_1) \ldots q(x_n)
		\quad
		\forall_{x_1, \ldots, x_n \in \algebra{A}}.
	\end{equation}
	We call a locally convex algebra an AE-algebra, if every continuous 
	semi-norm has an asymptotic estimate.
\end{definition}
\begin{remark}
	We can without restrictions define $m = 1$ in the upper definition, 
	since this just means taking the maximum over a finite number of 
	continuous semi-norms. Clearly the result will be a continuous semi-
	norm again.
\end{remark}
\begin{remark}
	\label{rem:AE2}
	\mbox{}
	\begin{remarklist}
		\item
		The term asymptotic estimate has, to the best of our knowledge, 
		first been used by Czichowski at all in [REFERENCE]. They 
		defined asymptotic estimates in the same way we did, but their 
		notion of AE-algebra was different from ours: in their 
		definition of an AE algebra, not just one but a series of 
		asymptotic estimates has to exist which fulfils two more 
		properties. This is not the case in our definition, which is, in 
		general, weaker.
		
		\item
		In \cite{}, Neeb and Gl\"ockner used a property to which they 
		referred as $(*)$ for associative algebras. It was then used in 
		\cite{} by ... and ..., who called it the $GN$-property. It is 
		easy to see that it is equivalent to being AE. 
		
		\item
		There are, of course, a lot of example for AE (Lie) algebras. 
		All finite dimensional and Banach (Lie) algebras fulfil 
		\eqref{Intro:AE}, just as locally m-convex (Lie) algebras do. 
		The same is true for nilpotent locally convex Lie algebras, 
		since here again one just has to take the maximum of a finite 
		number of semi-norms. We are, however, not sure what is exactly 
		implied by \eqref{Intro:AE}. Are there examples for associative 
		or Lie algebras which are AE but not locally m-convex, for 
		example?
	\end{remarklist}
\end{remark}
It is at least possible to make some easy observations: an associative 
AE-algebra $\algebra{A}$ will admit an entire holomorphic calculus: let 
$f \colon \mathbb{C} \longrightarrow \mathbb{C}$ be an entire function 
and $a_n$ such that $f(z) = \sum_n a_n z^n$. Then one has 
$\forall_{x \in \algebra{A}}$
\begin{equation*}
	p(f(x))
	=
	p \left(
		\sum\limits_{n=0}^{\infty}
		a_n x^n
	\right)
	\leq
	\sum\limits_{n=0}^{\infty}
	|a_n| 
	p \left( x^n \right)
	\leq
	\sum\limits_{n=0}^{\infty}
	|a_n| q(x)^n
	<
	\infty
\end{equation*}
where $p$ is a continuous semi-norm and $q$ its asymptotic estimate. So 
in some sense, at least in the associative case, AE-algebras are very 
close to locally m-convex ones. If the algebra is even commutative, then 
it seems like they are the same. In \cite{} ... claims that an 
associative, commutative locally convex algebra admitting an entire 
calculus must already be locally m-convex.

For non-commutative algebras, the situation is different. It is a very 
interesting (and non-trivial) question, how an example of an 
associative, non locally m-convex but AE-algebra could look like.


\subsection{The projective tensor product}
 - Inequality of the symmetric tensor product



\section{A topology for the Gutt star product}
\label{sec:chap5_TopologyStar}

\subsection{Continuity of the product}
 - a counter-example
 
\subsection{Dependence on the formal parameter}

\subsection{Completion}

\subsection{Nuclearity}



\section{Alternative topologies and an optimal result}
\label{sec:chap5_AlternativeOptimal}
