
%
% Chapter 4 of my master thesis:
% The formulas For the Gutt star product
%

\chapter{Formulas for the Gutt star product}

We have seen some results on the Baker-Campbell-Hausdorff series and an 
identity for the Gutt star product. The latter one, stated in Theorem 
\ref{Alg:Thm:ThreeStarsAreOne}, will be a very useful tool in the following, 
since we want to get explicit formulas for $\star_z$. There is still a part of 
the proof missing, but this will be caught up at the beginning of the first 
section of this chapter. From there, we will come to a first easy formula for 
$\star_z$. Afterwards, we will use the same procedure to find two more 
formulas for it: the first is a rather involved one for the $n$-fold star 
product of vectors. It will not be helpful for algebraic computations, 
but very useful for estimates. The second one is a more explicit formula for 
the product of two monomials.

From those formulas, we will be able to draw some easy, but nice 
conclusion in the second section and we will prove the classical and the 
semi-classical limit. Then, we will show how to calculate the Gutt star 
product explicitly by computing two easy examples.




\section{Formulas for the Gutt Star Product}
\label{sec:chap4_Formulas}


%
% An Iterative Approach from Linear Terms
%

\subsection{A Monomial with a Linear Term}

The easiest case for which we will develop a formula is surely the 
following one: For a given Lie algebra $\lie{g}$ and $\xi, \eta \in 
\lie{g}$ we would like to compute
\begin{equation*}
    \xi^k \star_z \eta
    =
    \sum\limits_{n=0}^k
    z^n C_n(\xi^k, \eta)
\end{equation*}
We have already done this for $\star_z$ and $\widehat{\star}_z$, now we 
want to do the same for $\ast_z$. This will finish the proof of the equality of 
the star products from Theorem~\ref{Alg:Thm:ThreeStarsAreOne}. We will use that
\begin{equation}\label{Formulas:MonomialDerivative}
    \xi^k
    =
    \frac{\partial^k}{\partial t^k}
    \At{t = 0} \exp(t \xi).
\end{equation}
Now we have all the ingredients to prove the following lemma:
\begin{lemma}
    \label{Formulas:Lemma:LinearMonomial1}
    Let $\lie{g}$ be a Lie algebra and $\xi, \eta \in \lie{g}$. We
    have the following identity for $\ast_z$:
    \begin{equation}
        \label{Formulas:LinearMonomial1}
        \xi^k \ast_z \eta
        =
        \sum\limits_{j=0}^k
        \binom{k}{j} z^j B_j^*
        \xi^{k-j}(\ad_{\xi})^j (\eta).
    \end{equation}
\end{lemma}
\begin{proof}
    We start from the simplified form for the Baker-Campbell-Hausdorff 
    series from Equation \eqref{Alg:BCHFirstOrderXi} in 
    Proposition~\ref{Alg:Prop:BCHFristOrder}:
    \begin{equation*}
		\bch{\xi}{\eta}
		=
		\xi 
		+ 
		\sum\limits_{n = 0}^{\infty}
		\frac{B_n^*}{n!}
		\left( \ad_{\xi} \right)^n (\eta)
		+
		\mathcal{O}(\eta^2).
    \end{equation*}
    If we insert this into the definition of the Drinfel'd star product 
    and use Equation~\eqref{Formulas:MonomialDerivative} we get
    \begin{align*}
        \xi^k \ast_z \eta
        & =
        \frac{\partial^k}{\partial t^k}
        \frac{\partial}{\partial s}
        \At{t=0, s=0}
        \exp \left(
            \frac{1}{z} \bch{z t \xi}{z s \eta}
        \right)
        \\
        & =
        \frac{\partial^k}{\partial t^k}
        \frac{\partial}{\partial s}
        \At{t=0, s=0}
        \exp \left(
            t \xi + \sum\limits_{j=0}^{\infty}
            z^j \frac{B_j^*}{j!}
            \left( \ad_{t \xi} \right)^j
            (s \eta)
            + \mathcal{O}(\eta^2)
        \right).
    \end{align*}
    We see that only terms which have exactly $k$ of the
    $\xi$'s in them and which are linear in $\eta$ will
    contribute. This means we can cut off the sum at $j = k$ and omit 
    higher orders in $\eta$. We now use the exponential series,
    cut it at $k$ for the same reason and get
    \begin{align*}
        \xi^k \ast_z \eta
        & =
        \frac{\partial^k}{\partial t^k}
        \frac{\partial}{\partial s}
        \At{t=0, s=0}
        \sum\limits_{n=0}^{k}
        \frac{1}{n!}
        \left(
            t \xi
            +
            \sum\limits_{j=0}^{k}
            (zt)^j \frac{B_j^*}{j!}
            \left(\ad_{\xi}\right)^j
            (s \eta)
        \right)^n
        \\
        & =
        \frac{\partial^k}{\partial t^k}
        \frac{\partial}{\partial s}
        \At{t=0, s=0}
        \sum\limits_{n=0}^{k}
        \frac{1}{n!}
        \sum\limits_{m = 0}^n
        \binom{n}{m}
        (t \xi)^{n - m}
        \left(
            \sum\limits_{j=0}^{k}
            (zt)^j \frac{B_j^*}{j!}
            \left(\ad_{\xi}\right)^j
            (s \eta)
        \right)^m
        \\
        & =
        \frac{\partial^k}{\partial t^k}
        \frac{\partial}{\partial s}
        \At{t=0, s=0}
        \left(
            \sum\limits_{n=0}^{k}
            \frac{1}{n!}
            (t \xi)^n
            +
            \sum\limits_{n=0}^{k}
            \sum\limits_{j=0}^k
            \frac{1}{(n - 1)!} t^{n + j - 1}
            z^j \frac{B_j^*}{j!}
            \xi^{n - 1}
            \left( \ad_{\xi} \right)^j
            (s \eta)
        \right).
    \end{align*}
    In the last step we set $m = 1$ since the other term have either too
    many or not enough $\eta$'s and will vanish because of the differentiation 
    with respect to $s$. We can finally differentiate to get the formula
    \begin{align*}
        \xi^k \ast_z \eta
        & =
        \sum\limits_{n=0}^k
        \sum\limits_{j=0}^k
        \delta_{k, n + j - 1}
        \frac{k!}{j! (n - 1)!}
        z^j B_j^*
        \xi^{n - 1}
        \left(\ad_{\xi}\right)^j
        (\eta)
        \\
        & =
        \sum\limits_{j=0}^k
        \binom{k}{j}
        z^j B_j^*
        \xi^{k - j}
        \left( \ad_{\xi} \right)^j
        (\eta),
    \end{align*}
    which is the wanted result.
\end{proof}
\begin{remark}
    We have proven the equality of the  star products $\widehat{\star}_z$
    $\ast_z$ by deriving an easy formula for both of them. From now on, we 
    will get all other formulas from $\ast_z$, since this is the 
    one which is easier to compute.
\end{remark}
Now it is actually easy to get the formula for monomials of the form $\xi_1 \ldots \xi_k$ with $\eta \in \lie{g}$:
\begin{proposition}
	\label{Formulas:Prop:LinearMonomial2}
    Let $\lie{g}$ be a Lie algebra and $\xi_1, \ldots, \xi_k, \eta \in 
    \lie{g}$. We have
    \begin{align}
    		\label{Formulas:LinearMonomial2}
	    	\xi_1 \cdots \xi_k \star_z \eta
    		& =
    		\sum\limits_{j=0}^k
    		\frac{1}{k!} \binom{k}{j}
    		z^j B_j^*
    		\sum\limits_{\sigma \in S_k}
    		[\xi_{\sigma(1)}, 
    			[ \ldots [\xi_{\sigma(j)}, \eta] \ldots ]
    		]
    		\xi_{\sigma(j+1)} \cdots \xi_{\sigma(k)} \text{ and}
    		\\
		\label{Formulas:LinearMonomial2T}
	    	\eta \star_z \xi_1 \cdots \xi_k
    		& =
    		\sum\limits_{j=0}^k
    		\frac{1}{k!} \binom{k}{j}
    		z^j B_j
    		\sum\limits_{\sigma \in S_k}
    		[\xi_{\sigma(1)}, 
    			[ \ldots [\xi_{\sigma(j)}, \eta] \ldots ]
    		]
    		\xi_{\sigma(j+1)} \cdots \xi_{\sigma(k)}.
    \end{align}
\end{proposition}
\begin{proof}
	We get the result by just polarizing the formula from Lemma 
	\ref{Formulas:Lemma:LinearMonomial1}. Let $\xi_1, \ldots, \xi_k \in 
	\lie{g}$, then we introduce the parameters $t_i$ for $i = 
	1, \ldots, k$ and set
	\begin{equation*}
		\Xi
		=		
		\Xi(t_1, \ldots, t_k)
		=
		\sum\limits_{i=1}^k t_i \xi^i.
	\end{equation*}
	Then we see that
	\begin{equation*}
		\xi_1 \cdots \xi_k
		=
		\frac{1}{k!}
		\frac{\partial^k}{\partial t_1 \cdots \partial t_k}
		\At{t_1, \cdots, t_k = 0}
		\Xi^k
	\end{equation*}
	since for every $i = 1, \ldots, k$ we have
	\begin{equation}\label{Formulas:LittleHelp1}
		\frac{\partial}{\partial t_i}
		\At{t_i = 0} \Xi
		=
		\xi_i.
	\end{equation}
	By writing out the $\Xi$'s and using multilinearity, we find
	\begin{align*}
		\xi_1 \cdots \xi_k \star_z \eta
		& =
		\frac{1}{k!}
		\frac{\partial^k}{\partial t_1 \cdots \partial t_k}
		\At{t_1, \cdots, t_k = 0}
        \sum\limits_{j=0}^k
        \binom{k}{j} z^j B_j^*
        \Xi^{k-j}(\ad_{\Xi})^j (\eta)
        \\
        & =
		\frac{1}{k!}
        \sum\limits_{j=0}^k
        \binom{k}{j} z^j B_j^*
		\sum\limits_{
			\{i_1, \ldots, i_k\}
			\in
			\{1, \ldots, k\}^k
		}
		\frac{\partial^k}{\partial t_1 \cdots \partial t_k}
		\At{t_1, \cdots, t_k = 0}
        t_{i_1} \cdots t_{i_k}
        \\
        & \qquad
        \cdot
        \xi_{i_1} \cdots \xi_{i_{k-j}}
        \ad_{\xi_{i_{k-j+1}}} 
        \circ \cdots \circ  
        \ad_{\xi_{i_k}}
        (\eta)
        \\
        & =
    		\sum\limits_{j=0}^k
    		\frac{1}{k!} \binom{k}{j}
    		z^j B_j^*
    		\sum\limits_{\sigma \in S_k}
    		[\xi_{\sigma(1)}, 
    			[ \ldots [\xi_{\sigma(j)}, \eta] \ldots ]
    		]
    		\xi_{\sigma(j+1)} \cdots \xi_{\sigma(k)}.
	\end{align*}
	In the last step, all expression which did not contain each $\xi_i$ 
	exactly once disappeared due to the differentiation. The proof of 
	Equation~\ref{Formulas:LinearMonomial2T} works analogously.
\end{proof}
\begin{remark}
	\label{Formulas:Rem:EasierFormulaAlreadyKnown}
	This formula is actually not a new result: Gutt already gave it in her 
	paper \cite[Prop. 1]{gutt:1983a} and referred to Dixmier 
	\cite[part 2.8.12 (c)]{dixmier:1977a}, who already gave it in his 
	textbook. It can also be found in the diploma thesis of Neumaier 
	\cite[Rem. 5.2.8]{neumaier:1998a} and a work due to Kathotia 
	\cite[Eq. 2.23]{kathotia:1998a:pre}. Probably the first one to mention it 
	was Berezin in \cite[Eq. 30]{berezin:1967a}.
\end{remark}



% A first general Formula
%
\subsection{An Iterated Formula for the General Case}

Proposition \ref{Formulas:Prop:LinearMonomial2} allows theoretically 
to get a formula for the case of $\xi_1, \ldots, \xi_k \in \lie{g}$
\begin{equation*}
	\xi_1 \star_z \ldots \star_z \xi_k
	=
	\sum\limits_{j=0}^k
	C_{z,j} \left(
		\xi_1, \ldots, \xi_k
	\right)
\end{equation*}
which we will need to prove the functoriality of our later construction.
This could also be used to give an alternative proof for our main theorem.
Unluckily, this approach has a problem: iterating this formula, we get 
strangely nested Lie brackets, which would be very difficult to bring 
into a nice form with Jacobi identity. So this is not a 
good way to find a handy formula for the usual star product of two 
monomials. Nevertheless, we want to pursue it for a moment, since we 
will get an equality which will be, although rather involved looking, 
very useful in the following: for analytic observations, it will be 
enough to put (even rough) estimates on it and the exact nature of the 
combinatorics in the formula will not be important. Hence we rewrite 
Equation \eqref{Formulas:LinearMonomial2} in order to cook up such a 
formula.

Let's take $\xi_1, \ldots, \xi_k, \eta \in \lie{g}$, then we have
\begin{equation*}
	\xi_1 \cdots \xi_k \star_z \eta
	=
	\sum\limits_{n = 0}^k 
	C_n \left( \xi_1 \ldots \xi_k, \eta \right)
\end{equation*}
with the $C_n$ being as bilinear operators which are given explicitly 
on monomials by
\begin{align}
	\label{Formulas:MultipleStarDef1}
	C_n^k
	\colon
	\Sym^k(\lie{g})
	\times
	\lie{g}
	& 
	\longrightarrow
	\Sym^{k - n + 1}(\lie{g})
	\\
	\label{Formulas:MultipleStarDef2}
	(\xi_1 \cdots \xi_k, \eta)
	&
	\longmapsto
	\frac{1}{k!}
	\sum\limits_{\sigma \in S_k}
	\binom{k}{j} B_j^* z^j
	\xi_{\sigma(1)} \cdots \xi_{\sigma(k-j)}
	[\xi_{\sigma(k-j+1)}, [ \ldots, [\xi_{\sigma(k)}, \eta]]] 
\end{align}
with
\begin{equation*}
	C_n 
	= 
	\sum\limits_{k = 0}^{\infty}
	C_n^k.
\end{equation*}
This gives us a good way of writing the $n$-fold star product of vectors:
\begin{proposition}
	\label{Formulas:Prop:MultipleStars}
	Let $\lie{g}$, $2 \leq k \in \mathbb{N}$  and $\xi_1, \ldots, \xi_k 
	\in \lie{g}$. Then we have
	\begin{equation}
		\label{Formulas:MultipleStars}
		\xi_1 \star_z \ldots \star_z \xi_k		
		=
		\sum\limits_{\substack{
			1 \leq j \leq k-1 \\
			i_j \in \{0, \ldots, j\}
		}}
		z^{i_1 + \ldots + i_{k-1}}
		C_{i_{k-1}}
		\left(
			\ldots C_{i_2}
			\left(
				C_{i_1}
				\left( \xi_1, \xi_2 \right)
				, \xi_3	
			\right) 
			\ldots, \xi_{k}
		\right).
	\end{equation}
\end{proposition}
\begin{proof}
	This is an easy prof by induction over $k$. For $k = 2$ the statement is 
	clearly true. For the step $k \rightarrow k + 1$ we get
	\begin{align*}
		\xi_1 \star_z \ldots \star_z \xi_{k+1}
		& =
		\Bigg(
			\sum\limits_{\substack{
				1 \leq j \leq k-1 \\
				i_j \in \{0, \ldots, j\}
			}}
			z^{i_1 + \cdots + i_{k-1}}
			C_{i_{k-1}}
			\left(
				\ldots C_{i_2}
				\left(
					C_{i_1} 
					\left( \xi_1, \xi_2 \right)
					, \xi_3	
				\right) 
				\ldots, \xi_{k}
			\right)
		\Bigg)
		\star_z \xi_{k+1}
		\\
		& = 
		\sum\limits_{i_k = 0}^k
		z^{i_k}
		C_{i_k}
		\Bigg(
			\sum\limits_{\substack{
				1 \leq j \leq k-1 \\
				i_j \in \{0, \ldots, j\} \\
			}}
			z^{i_1 + \cdots + i_{k-1}}
			C_{i_{k-1}}
			\left(
				\ldots C_{i_2}
				\left(
					C_{i_1} 
					\left( \xi_1, \xi_2 \right)
					, \xi_3	
				\right) 
				\ldots, \xi_{k}
			\right)
			, \xi_{k+1}
		\Bigg)
		\\
		& = 
		\sum\limits_{\substack{
			1 \leq j \leq k \\
			i_j \in \{0, \ldots, j\} \\
		}}
		z^{i_1 + \cdots + i_k}
		\left(
			C_{i_{k-1}}
			\left(
				\ldots C_{i_2}
				\left(
					C_{i_1} 
					\left( \xi_1, \xi_2 \right)
					, \xi_3	
				\right) 
				\ldots, \xi_{k}
			\right)
			, \xi_{k+1}
		\right)
	\end{align*}
\end{proof}
\begin{remark}
	\mbox{}
	\begin{remarklist}
		\item
		Of course, Proposition~\ref{Formulas:Prop:MultipleStars}  is an easy 
		consequence from Proposition \ref{Formulas:Prop:LinearMonomial2}.
		It's value, however, is that we know how the $C_n$'s look like and 
		what the summation range in \eqref{Formulas:MultipleStars} is. This 
		will allow us to put estimates on things like iterated star products.
		
		\item
		As already mentioned, we would get an identity for the star product of 
		two monomials via
		\begin{equation}
			\label{Formulas:2MonomialsWeird}
			\xi_1 \cdots \xi_k \star_z \eta_1 \cdots \eta_{\ell}
			=
			\frac{1}{k! \ell!}
			\sum\limits_{\sigma \in S_k}
			\sum\limits_{\tau \in S_{\ell}}
			\xi_{\sigma(1)} \star_z \cdots \star_z \xi_{\sigma(k)}
			\star_z
			\eta_{\tau(1)} \star_z \cdots \star_z \eta_{\tau(\ell)}.
		\end{equation}
		This can be proven from the definition of the map $\mathfrak{q}_z$.
		Unfortunately, this would give a very clumsy formula to deal with.
	\end{remarklist}
\end{remark}



% A Formula for two Monomials
%

\subsection{A Formula for two Monomials}

If we want to get an identity for the star product of two monomials, we have 
to go back to Equation \eqref{Alg:DrinfeldStar}. This will not give a simple 
looking formula either, but we will at least be able to do some 
computations with concrete examples. As a first step, we must introduce 
a bit of notation:
\begin{definition}[G-Index]
	\label{Def:GuttIndex}
	Let $k, \ell, n \in \mathbb{N}$ and $r = k + \ell - n$. 
	Then we call an $r$-tuple $J$
	\begin{equation*}
		J = (J_1, \ldots, J_r) 
		= 
		((a_1, b_1), \ldots, (a_r, b_r)) 
	\end{equation*}
  	a G-index if it fulfils the following properties:
	\begin{enumerate}[(i)]
		\item
		$J_i \in \{0, 1, \ldots, k\} \times \{0, 1, \ldots, \ell\}$
		
  		\item 
		$|J_i| 
		= 
		a_i + b_i \geq 1 
		\quad \forall_{i = 1, \ldots, r}$
		
		\item 
		$\sum\limits_{i=1}^{r} a_i = k$ 
		and 
		$\sum\limits_{i=1}^{r} b_i = \ell$
		
		\item
		The tuple is ordered in the following sense:
		$i>j \Rightarrow |J_i| \geq |J_j| \quad \forall_{i,j = 1, 
		\ldots, r}$ and $|a_i| \geq |a_j|$ if $|J_i| = |J_j|$
		
		\item 
		If $a_i = 0$ [or $b_i = 0$] for some $i$, 
		then $b_i = 1$ [or $a_i = 1$].
	\end{enumerate}
	We call the set of all such $G$-indices	$\mathcal{G}_r(k,\ell)$.
\end{definition}
\begin{definition}[G-Factorial]
	\label{Def:GuttFactorial}
	Let $J = ((a_1, b_1), \ldots, (a_r, b_r)) \in \mathcal{G}_r(k,
	\ell)$ be a G-Index. We set for a given tuple $(a,b) \in \{0, 1, 
	\ldots, k\} \times \{0, 1, \ldots, \ell\}$
	\begin{equation*}
		\#_J (a,b)
		= 
		\textrm{ number of times that $(a,b)$ appears in } J.
	\end{equation*}
	Then we define the G-factorial of $J \in \{0, 1, \ldots, k\} \times 
	\{0, 1, \ldots, \ell\}$ as
	\begin{equation*}
		J!
		= 
		\prod\limits_{
			(a,b) \in 
			\{0, 1, \ldots, k\} 
			\times 
			\{0, 1, \ldots, \ell\}
		}
		\left( \#_J (a,b) \right)!
	\end{equation*}
\end{definition}
Each pair $(a,b)$ will later correspond to $\bchparts{a}{b}{\xi}{\eta}$.
Now we can state a good formula for the Gutt star product:
\begin{lemma}
	\label{Formulas:Lemma:2MonomialsFormula1}
	Let $\lie{g}$ be a Lie algebra, $\xi, \eta \in \lie{g}$ and $k, 
	\ell \in \mathbb{N}$. Then we have the following identity for the 
	Gutt star product:
	\begin{equation*}
    	\xi^k \star_z \eta^{\ell}
    	=
	    \sum\limits_{n=0}^{k + \ell -1}
    	z^n
    	C_n \left( \xi^k, \eta^{\ell} \right),
	\end{equation*}
	where the $C_n$ are given by
	\begin{align}
		\label{Formulas:2MonomialsExplicit}
        C_n \left( \xi^k, \eta^{\ell} \right)
        & =
        \frac{k! \ell!}{(k + \ell - n)!}
        \sum\limits_{\substack{a_1, b_1, \ldots, a_r, b_r \geq 0 \\
            a_i + b_i \geq 1 \\
            a_1 + \cdots + a_r = k \\
            b_1 + \cdots + b_r = \ell
        }}
        \bchparts{a_i}{b_i}{\xi}{\eta}
        \cdots
        \bchparts{a_r}{b_r}{\xi}{\eta}
        \\
        \label{Formulas:2MonomialsFormula1}
        & =
        \sum\limits_{J \in \mathcal{G}_{k + \ell - n}(k, \ell)}
        \frac{k! \ell!}{J!}
        \prod\limits_{i = 1}^{k + \ell - n}        
        \bchparts{a_i}{b_i}{\xi}{\eta}
	\end{align}
	and the product is taken in the symmetric tensor algebra.
\end{lemma}
\begin{proof}
	We want to calculate what the $C_n$ look like. Let's denote $r = k 
	+ \ell - n$ for brevity. Then we have
	\begin{equation*}
		C_n \left( \xi^k, \eta^{\ell} \right)
		\in \Sym^r(\lie{g}).
	\end{equation*}
	Of course, the only part of the series
	\begin{equation*}
		\exp\left(
			\frac{1}{z} \bch{z \xi}{z \eta}
		\right)
		=
		\sum\limits_{n = 0}^{k + \ell}
		\left(
			\frac{1}{z} \bch{z \xi}{z \eta}
		\right)^n
		+ \mathcal{O}(\xi^{k + 1}, \eta^{\ell + 1})
	\end{equation*} 
	which lies in $\Sym^r(\lie{g})$ is the summand for 
	$n = r $. We introduce the formal parameters $t$ and $s$.
	Since we differentiate with respect to them, we can omit 
	terms of higher orders in $\xi$ and $\eta$ than $k$ 
	and $\ell$ respectively.
    \begin{align*}
        z^n C_n \left( \xi^k, \eta^{\ell} \right)
        & =
        \frac{\partial^k}{\partial t^k}
        \frac{\partial^{\ell}}{\partial s^{\ell}}
        \At{t,s = 0}
        \frac{1}{z^r}
        \frac{\bch{z t \xi}{z s \eta}^r    }{r!}
        \\
        & =
        \frac{1}{z^r}
        \frac{1}{r!}
        \frac{\partial^k}{\partial t^k}
        \frac{\partial^{\ell}}{\partial s^{\ell}}
        \At{t,s = 0}
        \left(
            \sum\limits_{j = 1}^{k + \ell}
            \bchpart{j}{z t \xi}{z s \eta}
        \right)^r
        \\
        & =
        \frac{1}{z^r}
        \frac{1}{r!}
        \frac{\partial^k}{\partial t^k}
        \frac{\partial^{\ell}}{\partial s^{\ell}}
        \At{t,s = 0}
        \sum\limits_{\substack{
        	j_1, \ldots, j_r \geq 1 \\
            j_1 + \ldots + j_r = k + \ell
        }}
        \bchpart{j_1}{z t \xi}{z s \eta} 
        \cdots
        \bchpart{j_r}{z t \xi}{z s \eta}
        \\
        & =
        z^n
        \frac{k! \ell!}{r!}
        \sum\limits_{\substack{a_1, b_1, \ldots, a_r, b_r \geq 0 \\
            a_i + b_i \geq 1 \\
            a_1 + \cdots + a_r = k \\
            b_1 + \cdots + b_r = \ell
        }}
        \bchparts{a_i}{b_i}{\xi}{\eta}
        \cdots
        \bchparts{a_r}{b_r}{\xi}{\eta}
    \end{align*}
    We sum over all possible arrangements of the $(a_i, b_i)$. In order 
    to find n easier summation range, we put the ordering from Definition 
    \ref{Def:GuttIndex} on these multi-indices and avoid therefore double 
    counting. We loose the freedom of arranging the $(a_i, b_i)$ and 
    need to count the number of multi-indices $((a_1, b_1), \ldots, 
    (a_r, b_r))$ which belong to the same G-index $J$. This number will 
    be $\frac{r!}{J!}$, since we can not interchange the $(a_i, b_i)$ any 
    more (therefore $r!$), unless they are equal (therefore $J!^{-1}$). 
    Since the ranges of the $(a_i, b_i)$ in Equation 
    \eqref{Formulas:2MonomialsExplicit} and of  the elements in 
    $\mathcal{G}_r(k, \ell)$ are the same, we can change the summation there 
    to $J \in \mathcal{G}_r(k, \ell)$ and need to multiply by $\frac{r!}{J!}$. 
    This gives
    \begin{equation*}
    	z^n C_n \left( \xi^k, \eta^{\ell} \right)
    	=
    	z^n \frac{k! \ell!}{J!}
    	\sum\limits_{J \in \mathcal{G}_r(k, \ell)}
    	\bchparts{a_i}{b_i}{\xi}{\eta}
        \cdots
        \bchparts{a_r}{b_r}{\xi}{\eta}
    \end{equation*}
    which is equivalent to Equation \eqref{Formulas:2MonomialsFormula1}.
\end{proof}
Now we need to generalize this to factorizing tensors. To do so, we 
need a last definition:
\begin{definition}
	\label{Def:BCHTilde}
	Let $a,b \in \mathbb{N}$ and $\xi_1, \ldots, \xi_a, \eta_1, \ldots, 
	\eta_b \in \lie{g}$. Then we define by
	\begin{equation*}
		\widetilde{\mathrm{BCH}}_{a,b}
		\colon
		\lie{g}^{a + b}
		\longrightarrow
		\lie{g}
	\end{equation*}
	the map which we get when we replace in$\bchparts{a}{b}{\xi}{\eta}$ 
	the $i$-th $\xi$ by $\xi_i$ and the $j$-th $\eta$ by $\eta_j$
	for $i = 1, \ldots, a$ and $j = 1, \ldots, b$.
\end{definition}
\begin{proposition}
	\label{Formulas:Prop:2MonomialsFormula2}
	Let $\lie{g}$ be a Lie algebra, $k, \ell \in \mathbb{N}$ and $\xi_1, 
	\ldots, \xi_k, \eta_1, \ldots, \eta_{\ell} \in \lie{g}$. Then we 
	have the following identity for the Gutt star product:
	\begin{equation*}
    		\xi_1 \ldots \xi_k \star_z \eta_1 \ldots \eta_{\ell}
    		=
		\sum\limits_{n=0}^{k + \ell -1}
    		z^n C_n
    		\left( 
    			\xi_1 \cdots \xi_k, \eta_1 \cdots \eta_{\ell}
    		\right),
	\end{equation*}
	where the $C_n$ are given by
	\begin{align}
		\nonumber
        C_n
        \left( 
    			\xi_1 \cdots \xi_k, \eta_1 \cdots \eta_{\ell}
    		\right)
        & =
        \sum\limits_{J \in \mathcal{G}_{k + \ell - n}(k, \ell)}
        \frac{1}{J!}
        \sum\limits_{\sigma \in S_k}
        \sum\limits_{\tau \in S_{\ell}}
        \prod_{i=1}^{l + \ell - n}
        \widetilde{\mathrm{BCH}}_{a_i, b_i}
        \big( \xi_{\sigma(a_1 + \cdots + a_{i - 1} + 1)}, 
            \ldots, 
        \\
        \label{Formulas:2MonomialsFormula2}
        & \qquad            
            \ldots, \xi_{\sigma(a_1 + \cdots + a_i)} \big)
        \big( \eta_{\tau(b_1 + \cdots + b_{i - 1} + 1)}, 
            \ldots, \eta_{\tau(b_1 + \cdots + b_i)} \big).
        \\
        \nonumber
        & =
        \frac{1}{(k + \ell - n)!}
        \sum\limits_{\sigma \in S_k, \tau \in S_{\ell}}
        \sum\limits_{\substack{a_1, b_1, \ldots, a_r, b_r \geq 0 \\
            a_i + b_i \geq 1 \\
            a_1 + \cdots + a_r = k \\
            b_1 + \cdots + b_r = \ell
          }
        }
        \\
        \nonumber
        & \qquad
        \bchtilde{a_i}{b_i}
        {\xi_{\sigma(1)}, \ldots, \xi_{\sigma(a_1)}}
        {\eta_{\tau(1)}, \ldots, \eta_{\tau(b_1)}}
        \cdots
        \\
        \label{Formulas:2MonomialsFormula22}
        & \qquad
        \bchtilde{a_r}{b_r}
        {\xi_{\sigma(k - a_r + 1)}, \ldots, \xi_{\sigma(k)}}
        {\eta_{\tau(\ell - b_r + 1)}, \ldots, \eta_{\tau(\ell)}}.
	\end{align}
\end{proposition}
\begin{proof}
	The proof relies on polarization again and is completely analogous 
	to the one of Proposition \ref{Formulas:Prop:LinearMonomial2}. We set
	\begin{equation*}
		\Xi
		=
		\sum\limits_{i=1}^k t_i \xi^i
		\quad \text{ and } \quad
		\Eta
		=
		\sum\limits_{i=1}^{\ell} t_j \eta^j
		.
	\end{equation*}
	Then it is easy to see that we will get rid of the factorials in Equation 
	\eqref{Formulas:2MonomialsFormula1} since
	\begin{equation*}
		\xi_1 \cdots \xi_k \star_z \eta_1 \cdots \eta_{\ell}
		=
		\frac{1}{k! \ell!}
		\frac{\partial^{k + \ell}}
		{\partial_{t_1} \cdots \partial_{s_{\ell}}}
		\At{t_1, \ldots, s_{\ell} = 0}
		\Xi^k \star_z \Eta^{\ell}.
	\end{equation*}
	Instead of the factorials, we get symmetrizations over the 
	$\xi_i$ and the $\eta_j$ as we did in Proposition 
	\ref{Formulas:Prop:LinearMonomial2}, which gives the wanted result.
\end{proof}



\section{Consequences and examples}
\label{sec:chap4_Consequences}

\subsubsection*{Some consequences}
Proposition \ref{Formulas:Prop:2MonomialsFormula2} allows us to get some 
algebraic results. For example, we would like to see that the Gutt star 
product fulfils the classical and the semi-classical limit from Definition 
\ref{Def:StarProduct}. We can prove this using Proposition 
\ref{Formulas:Prop:LinearMonomial2}. This will finish the proof of 
Theorem~\ref{Alg:Thm:ThreeStarsAreOne}.
\begin{corollary}
	\label{Formulas:Cor:LimitCases}
	Let $\lie{g}$ be a Lie algebra and $\Sym^{\bullet}(\lie{g})$ edowed with 
	the Gutt star product
	\begin{equation*}
		x \star_z y
		= 
		\sum\limits_{n = 0}^{\infty}
		z^n C_n(x,y).
	\end{equation*}	
	\begin{corollarylist}
		\item
		On factorizing tensors $\xi_1 \ldots \xi_k$ and $\eta_1 \ldots 
		\eta_{\ell}$, $C_0$ and $C_1$ give
		\begin{align}
			\label{Formulas:ClassicalLimit}
			C_0 \left(
				\xi_1 \cdots \xi_k, \eta_1 \cdots \eta_{\ell}
			\right)
			& = 
			\xi_1 \cdots \xi_k \eta_1 \cdots \eta_{\ell}
			\\
			\label{Formulas:SemiClassicalLimit}
			C_1 \left(
				\xi_1 \cdots \xi_k, \eta_1 \cdots \eta_{\ell}
			\right)
				& =
			\frac{1}{2}	
			\sum\limits_{i = 1}^k
			\sum\limits_{j = 1}^{\ell}
			\xi_1 \cdots \widehat{\xi_i} \cdots \xi_k
			\eta_1 \cdots \widehat{\eta_j} \cdots \eta_{\ell}
			[\xi_i \eta_j],
		\end{align}
		where the hat denotes elements which are left out.
		
		\item
		For $\lie{g}$ finite-dimensional and the canonical isomorphism 
		$\mathcal{J} \colon \Sym^{\bullet}(\lie{g}) \longrightarrow 
		\Pol^{\bullet}(\lie{g}^*)$ from Proposition~\ref{Alg:Prop:PolIsSym},
		we have for $f,g \in \Pol^{\bullet}(\lie{g}^*)$
		\begin{equation*}
			C_1 \left(
				\mathcal{J}^{-1} (f),
				\mathcal{J}^{-1} (g)
			\right)
			-
			C_1 \left(
				\mathcal{J}^{-1} (f),
				\mathcal{J}^{-1} (g)
			\right)
			= 
			\mathcal{J}^{-1} \left( 
				\{ f, g \}_{KKS}
			\right)
		\end{equation*}
		where $\{ \cdot, \cdot \}_{KKS}$ is the Kirillov-Kostant-Souriau 
		bracket.
		
		\item
		The map $\star_z$ fulfils the classical and the semi-classical limit 
		and is therefore a star product.
	\end{corollarylist}
\end{corollary}
\begin{proof}
	We take $\xi_1 \cdots \xi_k, \eta_1 \cdots \eta_{\ell} \in \Sym^{\bullet}
	(\lie{g})$ and consider the 	G-indices in $\mathcal{G}_{k + \ell}(k, \ell)$ 
	first. This is easy, since there is just one element inside:
	\begin{equation*}
		\mathcal{G}_{k + \ell}(k, \ell)
		=
		\Big\{
			( 
				\underbrace{(0,1), \ldots, (0,1)}_{
				\ell \text{ times}
				}
				,
				\underbrace{(1,0), \ldots, (1,0)}_{
				k \text{ times}
				}
			)
		\Big\}.
	\end{equation*}
	So we find
	\begin{align*}
		C_0 
		\left(
			\xi_1 \cdots \xi_k, \eta_1 \cdots \eta_{\ell}
		\right)
		& =
		\sum\limits_{\sigma \in S_k}
		\sum\limits_{\tau \in S_{\ell}}
		\frac{1}{J!}
		\bchparts{0}{1}{\varnothing}{\xi_{\sigma(1)}}
		\cdots
		\bchparts{0}{1}{\varnothing}{\xi_{\sigma(k)}}
		\\
		& \quad \cdot
		\bchparts{1}{0}{\eta_{\tau(1)}}{\varnothing}
		\cdots
		\bchparts{1}{0}{\eta_{\tau(\ell)}}{\varnothing}
		\\
		& =
		\sum\limits_{\sigma \in S_k}
		\sum\limits_{\tau \in S_{\ell}}
		\frac{1}{k! \ell!}
		\xi_{\sigma(1)} \cdots \xi_{\sigma(k)}
		\eta_{\tau(1)} \cdots \eta_{\tau(\ell)}
		\\
		& =
		\xi_1 \cdots \xi_k
		\eta_1 \cdots \eta_{\ell}
	\end{align*}
	where we used $J! = k! \ell!$  according to 
	Definition~\ref{Def:GuttFactorial}. We do the same for $C_1$. Also 
	here, we have just one element in $\mathcal{G}_{k + \ell - 1}(k, \ell)$:
		\begin{equation*}
		\mathcal{G}_{k + \ell}(k, \ell)
		=
		\Big\{
			( 
				\underbrace{(0,1), \ldots, (0,1)}_{
				\ell-1 \text{ times}
				}
				,
				\underbrace{(1,0), \ldots, (1,0)}_{
				k-1 \text{ times}
				}
				,
				(1,1)
			)
		\Big\}.
	\end{equation*}
	Using
	\begin{equation*}
		\bchparts{1}{1}{\xi}{\eta}
		=
		\frac{1}{2} [\xi, \eta]
	\end{equation*}
	and $J! = (k - 1)! (\ell - 1)!$, we find
	\begin{align*}
		C_1 
		\left(
			\xi_1 \cdots \xi_k, \eta_1 \cdots \eta_{\ell}
		\right)
		& =
		\frac{1}{2}
		\sum\limits_{\sigma \in S_k}
		\sum\limits_{\tau \in S_{\ell}}
		\frac{1}{(k-1)! (\ell-1)!}
		\xi_{\sigma(1)} \cdots \xi_{\sigma(k-1)}
		\eta_{\tau(1)} \cdots \eta_{\tau(\ell-1)}
		[\xi_{\sigma(k)}, \eta_{\tau(\ell)}]
		\\
		& =
		\frac{1}{2}
		\sum\limits_{i = 0}^k
		\sum\limits_{j = 0}^{\ell}
		\xi_1 \cdots \widehat{\xi_i} \cdots \xi_k
		\eta_1 \cdots \widehat{\eta_j} \cdots \eta_{\ell}
		[\xi_i, \eta_j].
	\end{align*}
	This finishes part one. From this, the anti-symmetry of the Lie bracket 
	yields
	\begin{equation*}
		C_1 
		\left(
			\xi_1 \cdots \xi_k, \eta_1 \cdots \eta_{\ell}
		\right)
		-
		C_1 
		\left(
			\eta_1 \cdots \eta_{\ell}, \xi_1 \cdots \xi_k
		\right)
		=
		\sum\limits_{i = 0}^k
		\sum\limits_{j = 0}^{\ell}
		\xi_1 \cdots \widehat{\xi_i} \cdots \xi_k
		\eta_1 \cdots \widehat{\eta_j} \cdots \eta_{\ell}
		[\xi_i, \eta_j].
	\end{equation*}
	We now need to compute the KKS brackets on polynomials. Because of the 
	linearity in both arguments, it is sufficient to check it on monomials of 
	coordinates. Let $e_1, \ldots, e_n$ be a basis of $\lie{g}$ with linear 
	coordinates 	$x_1, \ldots, x_n$ on $\lie{g}^*$. Now take $\mu_1, \ldots, 
	\mu_n, \nu_1, \ldots \nu_n \in \mathbb{N}$ and consider the monomials 
	$f = x_1^{\mu_1} \cdots x_n^{\mu_n}$ and $g = x_1^{\nu_1} \cdots 
	x_n^{\nu_n}$. We use the notation from 
	Proposition~\ref{Alg:Prop:LinPoissonIsLieAlg} and find for $x \in 
	\lie{g}^*$
	\begin{align*}
		\{ f, g \}_{KKS}(x)
		& =
		x_k c_{ij}^k
		\frac{\partial f}{\partial x_i}
		\frac{\partial g}{\partial x_j}
		\\
		& =
		\mu_i \nu_j c_{ij}^k
		x_k 
		x_1^{\mu_1} \cdots x_i^{\mu_i - 1} \cdots x_n^{\mu_n}
		x_1^{\nu_1} \cdots x_j^{\nu_j - 1} \cdots x_n^{\nu_n}.
	\end{align*}
	Applying $\mathcal{J}^{-1}$ to it gives
	\begin{equation}
		\label{Formulas:SemiClassicalIntermediate}
		\mathcal{J}^{-1}
		\left(
			\{ f, g \}_{KKS}
		\right)
		=
		\sum\limits_{i=0}^n
		\sum\limits_{j=0}^n
		\mu_i \nu_j
		e_1^{\mu_1} \cdots e_i^{\mu_i - 1} \cdots e_n^{\mu_n}
		e_1^{\nu_1} \cdots e_j^{\nu_j - 1} \cdots e_n^{\nu_n}
		[e_i, e_j].
	\end{equation}
	On the other hand, we have
	\begin{equation*}
		\mathcal{J}^{-1}(f)
		=
		e_1^{\mu_1} \cdots e_n^{\mu_n}
		\quad \text{ and } \quad
		\mathcal{J}^{-1}(g)
		=
		e_1^{\nu_1} \cdots e_n^{\nu_n}.
	\end{equation*}
	Together with \eqref{Formulas:SemiClassicalLimit} this gives
	\eqref{Formulas:SemiClassicalIntermediate} and proves part two.
	Due to the bilinearity of the $C_n$, the third part follows.
\end{proof}
It is clear, that the formulas from 
Proposition~\ref{Formulas:Prop:2MonomialsFormula2} and 
Proposition~\ref{Formulas:Prop:LinearMonomial2} should coincide. However, 
we want to check it, to have the evidence that everything works as we wanted.
\begin{corollary}
	\label{Formulas:Cor:FormulasCoincide}
	Given $\xi_1, \ldots, \xi_k, \eta \in \lie{g}$, the results of the 
	Equations \eqref{Formulas:2MonomialsFormula2} and 
	\eqref{Formulas:LinearMonomial2} are compatible.
\end{corollary}
\begin{proof}
	We have to compute sets of G-indices for $\xi_1, \ldots, \xi_k, 
	\eta_{\ell} \in \lie{g}$. Again, they only have one element:
	\begin{equation*}
		\mathcal{G}_{k + 1 - n}(k, 1)
		=
		\Big\{
			( 
				\underbrace{(1,0), \ldots, (1,0)}_{
				k - n \text{ times}
				}
				,
				(n,1)
			)
		\Big\}.
	\end{equation*}
	So we have with $J! = (k - n)!$ and $\bchparts{n}{1}{\xi}{\eta} = 
	\frac{B_n^*}{n!} \left( \ad_{\xi} \right)^n (\eta)$
	\begin{align*}
		z^n C_n
		\left(
			\xi_1 \ldots \xi_k, \eta_{\ell}
		\right)
		& =
		z^n
		\sum\limits_{\sigma \in S_k}
		\frac{1}{(k - n)!}
		\frac{B_n^*}{n!}
		\xi_{\sigma(1)} \ldots \xi_{\sigma(k-n)}
		[\xi_{\sigma(k-n+1)}, [
			\ldots, [\xi_{\sigma(k), \eta} ] \ldots 
		]]
		\\
		& =
		z^n
		\frac{1}{k!}
		\sum\limits_{\sigma \in S_k}
		\binom{k}{n} B_n^*
		\xi_{\sigma(1)} \ldots \xi_{\sigma(k-n)}
		[\xi_{\sigma(k-n+1)}, [
			\ldots, [\xi_{\sigma(k), \eta} ] \ldots 
		]]		
	\end{align*}
	Summing up over all $n$ gives Equation~\eqref{Formulas:LinearMonomial2}.
\end{proof}



\subsubsection*{Two examples}
Equation \eqref{Formulas:2MonomialsFormula2} is useful if one wants to do 
real computations with the star product, but it is maybe not intuitive to 
apply. This is why we will give two examples here. The easiest one which is 
not covered by the simpler formula~\eqref{Formulas:LinearMonomial2} will be 
the star product of two quadratic terms. The second one should be the a bit 
more complex case of a cubic term with a quadratic term.

\subsubsection*{Two quadratic terms}
Let $\xi_1, \xi_2, \eta_1, \eta_2 \in \mathfrak{g}$. We want to compute
\begin{equation*}
	\xi_1 \xi_2 \star_z \eta_1 \eta_2
	=
	C_0(\xi_1 \xi_2, \eta_1 \eta_2) 
	+ 
	z C_1(\xi_1 \xi_2, \eta_1 \eta_2) 
	+ 
	z^2 C_2(\xi_1 \xi_2, \eta_1 \eta_2) 
	+ 
	z^3 C_3(\xi_1 \xi_2, \eta_1 \eta_2).
\end{equation*}
The very first thing we have to do is computing the set of G-indices. Then we 
calculate the G-factorial and finally go through the permutations.
\begin{itemize}
	\item[$C_0$:]
	We already did this in Corollary \ref{Formulas:Cor:LimitCases}, and know 
	that the zeroth order in $z$ is just the symmetric product. Therefore we 
	have
	\begin{equation*}
		C_0(\xi_1 \xi_2, \eta_1 \eta_2)
		=
		\xi_1 \xi_2 \eta_1 \eta_2
	\end{equation*}
	
	\item[$C_1$:]
	We also did this one in Corollary \ref{Formulas:Cor:LimitCases}: There is 
	just one G-index and we finally get
	\begin{equation*}
		C_1(\xi_1 \xi_2, \eta_1 \eta_2)
		=
		\frac{1}{2} \left(
			\xi_2 \eta_2 [\xi_1, \eta_1] +
			\xi_2 \eta_1 [\xi_1, \eta_2] +
			\xi_1 \eta_2 [\xi_2, \eta_1] +
			\xi_1 \eta_1 [\xi_2, \eta_2]
		\right).
	\end{equation*}
	
	\item[$C_2$:]
	This is the first time, something interesting happens. We have three 
	G-indices:
	\begin{equation*}
		\mathcal{G}_2(2,2) 
		=
		\left\{ J_1, J_2, J_3 \right\}
		= 
		\big\{ 
			\big((0,1), (2,1)\big), 
			\big((1,0), (1,2)\big), 
			\big((1,1), (1,1)\big) 
		\big\}.
	\end{equation*}
	The G-factorials give $J_1! = J_2! = 1$ and $J^3! = 2$, since the index 
	$(1,1)$ appears twice in $J_3$. We take $\bchparts{a}{b}{\xi}{\eta}$ 
	from Equation \eqref{Alg:BCHSeriesLong} for $(a,b) \in \{(1,2), (2,1)\}$:
	\begin{equation*}
		\bchparts{1}{2}{\xi}{\eta}
		=
		\frac{1}{12}[[\xi, \eta], \eta]
		\quad \text{ and } \quad
		\bchparts{2}{1}{\xi}{\eta}
		=
		\frac{1}{12}[[\eta, \xi], \xi].
	\end{equation*}
	So we have to insert the $\xi_i$ and the $\eta_j$ into $\frac{1}{12} 
	\xi [[\xi, \eta], \eta]$ and $\frac{1}{12} \eta [[\eta, \xi], \xi]$ 
	respectively and then we go on with the last one, which is
	\begin{equation*}
		\frac{1}{2}
		\bchparts{1}{1}{\xi}{\eta}
		\bchparts{1}{1}{\xi}{\eta}
		=
		\frac{1}{8}
		[\xi, \eta][\xi, \eta].
	\end{equation*}
	We hence get
	\begin{align*}
		C_2(\xi_1, \xi_2, \eta_1, \eta_2) 
		& = 
		\frac{1}{12}
		\big( 
			\eta_1 [[\eta_2, \xi_1],\xi_2] + 
			\eta_1 [[\eta_2, \xi_2],\xi_1] + 
			\eta_2 [[\eta_1, \xi_1],\xi_2] + 
			\eta_2 [[\eta_1, \xi_2],\xi_1] +
		\\
		& \quad
			\xi_1 [[\xi_2, \eta_1],\eta_2] + 
			\xi_1 [[\xi_2, \eta_2],\eta_1] + 
			\xi_2 [[\xi_1, \eta_1],\eta_2] + 
			\xi_2 [[\xi_1, \eta_2],\eta_1] 
 		\big) +
 		\\
 		& \quad
		\frac{1}{4} 
		\big( 
			[\xi_1,\eta_1][\xi_2,\eta_2] + 
			[\xi_1,\eta_2][\xi_2,\eta_1] 
		\big)
	\end{align*}
	
	\item[$C_3$:]
	Here, we only have one G-index:
	\begin{equation*}
		\mathcal{G}_1(2,2) 
		=
		\big\{ 
			\big( (2,2) \big) 
		\big\}
	\end{equation*}
	The G-factorial is $1$. We take again Equation \eqref{Alg:BCHSeriesLong} 
	and see
	\begin{equation*}
		\bchparts{2}{2}{\xi}{\eta}
		=
		\frac{1}{24}
		[[[\eta, \xi], \xi], \eta].
	\end{equation*}
	This gives
	\begin{align*}
		C_3(\xi_1, \xi_2, \eta_1, \eta_2) 
		& = 
		\frac{1}{24}
		\big( 
			[[[\eta_1,\xi_1],\xi_2],\eta_2] + 
			[[[\eta_1,\xi_2],\xi_1],\eta_2] +
			[[[\eta_2,\xi_1],\xi_2],\eta_1] + 
			[[[\eta_2,\xi_2],\xi_1],\eta_1] 
		\big)
	\end{align*}
\end{itemize}
We just have to put all the four terms together and have the star product.


\subsubsection*{A cubic and a quadratic term}
Let $\xi_1, \xi_2, \xi_3, \eta_1, \eta_2 \in \mathfrak{g}$. We compute
\begin{equation*}
	\xi_1 \xi_2 \xi_3 \star_G \eta_1 \eta_2
	= 
	\sum\limits_{n = 0}^4
	z^n C_n(\xi_1 \xi_2 \xi_3, \eta_1 \eta_2)
\end{equation*}
\begin{itemize}
	\item[$C_0$:]
	The first part is again just the symmetric product:
	\begin{equation*}
		C_0(\xi_1 \xi_2 \xi_3, \eta_1 \eta_2)
		=
		\xi_1 \xi_2 \xi_3 \eta_1 \eta_2.
	\end{equation*}
	
	\item[$C_1$:]
	Here we have again the term from Corollary \ref{Formulas:Cor:LimitCases}:
	\begin{align*}
		C_1(\xi_1 \xi_2 \xi_3, \eta_1 \eta_2)
		& =
		\frac{1}{2} 
		\big( 
			\xi_2 \xi_3 \eta_2 [\xi_1, \eta_1] + 
			\xi_2 \xi_3 \eta_1 [\xi_1, \eta_2] + 
			\xi_1 \xi_3 \eta_2 [\xi_2, \eta_1] +
		\\
		& \quad 
			\xi_1 \xi_3 \eta_1 [\xi_2, \eta_2] + 
			\xi_1 \xi_2 \eta_2 [\xi_3, \eta_1] + 
			\xi_1 \xi_2 \eta_1 [\xi_3, \eta_2] 
 		\big)
	\end{align*}

	\item[$C_2$:]
	Here the calculation is very similar to the one of $C_2$ in the example 
	before. We have three G-indices:
	\begin{equation*}
		\mathcal{G}_3(3,2) 
		=
		\left\{
			J_1, J_2, J_3
		\right\}
		= 
		\big\{ 
			\big( (0,1), (1,0), (2,1) \big), 
			\big( (1,0), (1,0), (1,2) \big), 
			\big( (1,0), (1,1), (1,1) \big) 
		\big\}.
	\end{equation*}
	The G-factorials are now $J_1! = 1$ and $J_2! = J_3! = 2$. Again, we take 
	the BCH terms from Equation \eqref{Alg:BCHSeriesLong} and see, that we 
	must insert the $\xi_i$ and the $\eta_j$ into
	\begin{equation*}
		\frac{1}{12} \xi \eta [[\eta, \xi], \xi] +
		\frac{1}{24} \xi \xi [[\xi, \eta], \eta] +
		\frac{1}{8} \xi [\xi, \eta] [\xi, \eta].
	\end{equation*}
	Now we go through all the possible permutations and get
	\begin{align*}
		C_2(\xi_1 \xi_2 \xi_3, \eta_1 \eta_2) 
		& = 
		\frac{1}{12} 
		\big( 
			\xi_1 \xi_2 [[\xi_3, \eta_1], \eta_2] + 
			\xi_1 \xi_2 [[\xi_3, \eta_2], \eta_1] + 
			\xi_1 \xi_3 [[\xi_2, \eta_1], \eta_2] +
		\\
		& \quad 
			\xi_1 \xi_3 [[\xi_2, \eta_2], \eta_1] + 
			\xi_2 \xi_3 [[\xi_1, \eta_1], \eta_2] + 
			\xi_2 \xi_3 [[\xi_1, \eta_2], \eta_1] 
		\big) +
		\\ 
		& \quad
		\frac{1}{12} 
		\big( 
			\xi_1 \eta_1 [[\eta_2, \xi_2], \xi_3] + 
			\xi_1 \eta_2 [[\eta_1, \xi_2], \xi_3] + 
			\xi_1 \eta_1 [[\eta_2, \xi_3], \xi_2] +
		\\
		& \quad
			\xi_1 \eta_2 [[\eta_1, \xi_3], \xi_2] + 
			\xi_2 \eta_1 [[\eta_2, \xi_1], \xi_3] + 
			\xi_2 \eta_2 [[\eta_1, \xi_1], \xi_3] +
		\\
		& \quad
			\xi_2 \eta_1 [[\eta_2, \xi_3], \xi_1] + 
			\xi_2 \eta_2 [[\eta_1, \xi_3], \xi_1] + 
			\xi_3 \eta_1 [[\eta_2, \xi_2], \xi_1] +
		\\
		& \quad
			\xi_3 \eta_2 [[\eta_1, \xi_2], \xi_1] + 
			\xi_3 \eta_1 [[\eta_2, \xi_1], \xi_2] + 
			\xi_3 \eta_2 [[\eta_1, \xi_1], \xi_2] 
		\big) +
		\\
		& \quad
		\frac{1}{4} 
		\big( 
			\xi_1 [\xi_2, \eta_1] [\xi_3, \eta_2] + 
			\xi_1 [\xi_3, \eta_1] [\xi_2, \eta_2] + 
			\xi_2 [\xi_1, \eta_1] [\xi_3, \eta_2] + 
		\\
		& \quad
			\xi_2 [\xi_3, \eta_1] [\xi_1, \eta_2] + 
			\xi_3 [\xi_1, \eta_1] [\xi_2, \eta_2] + 
			\xi_3 [\xi_2, \eta_1] [\xi_1, \eta_2] 
 		\big).
	\end{align*}

	\item[$C_3$:]
	We first calculate the G-indices:
	\begin{equation*}
		\mathcal{G}_2(3,2) 
		= 
		\left\{
			J_1, J_2, J_3
		\right\}
		=
		\left\{ 
			\big( (0,1), (3,1) \big), 
			\big( (1,0), (2,2) \big), 
			\big( (1,1), (2,1) \big) 
		\right\}.
	\end{equation*}
	We can omit $J_1$, since $\bchparts{3}{1}{\xi}{\eta} = 0$. The 
	G-factorials for the other two indices are $1$. The BCH terms have been 
	computed before. So we have to fill in the expression
	\begin{equation*}
		\frac{1}{24} \xi [[[\eta, \xi], \xi], \eta] +
		\frac{1}{2 \cdot 12} [\xi, \eta] [[\eta, \xi], \xi].
	\end{equation*}
	Going through the permutations we get
	\begin{align*}
		C_3(\xi_1 \xi_2 \xi_3, \eta_1 \eta_2)
		& =
		\frac{1}{24}
		\big( 
			\xi_1[[[\eta_1, \xi_2], \xi_3], \eta_2] + 
			\xi_1[[[\eta_2, \xi_2], \xi_3], \eta_1] + 
			\xi_1[[[\eta_1, \xi_3], \xi_2], \eta_2] +
		\\
		& \quad
			\xi_1[[[\eta_2, \xi_3], \xi_2], \eta_1] + 
			\xi_2[[[\eta_1, \xi_1], \xi_3], \eta_2] + 
			\xi_2[[[\eta_2, \xi_1], \xi_3], \eta_1] +
		\\
		& \quad
			\xi_2[[[\eta_1, \xi_3], \xi_1], \eta_2] + 
			\xi_2[[[\eta_2, \xi_3], \xi_1], \eta_1] + 
			\xi_3[[[\eta_1, \xi_2], \xi_1], \eta_2] +
		\\
		& \quad
			\xi_3[[[\eta_2, \xi_2], \xi_1], \eta_1] + 
			\xi_3[[[\eta_1, \xi_1], \xi_2], \eta_2] + 
			\xi_3[[[\eta_2, \xi_1], \xi_2], \eta_1]
		\big) +
		\\
		& \quad
		\frac{1}{24}
		\big( 
			[\xi_1, \eta_1][[\eta_2, \xi_2], \xi_3] + 
			[\xi_1, \eta_2][[\eta_1, \xi_2], \xi_3] + 
			[\xi_1, \eta_1][[\eta_2, \xi_3], \xi_2] +
		\\
		& \quad
			[\xi_1, \eta_2][[\eta_1, \xi_3], \xi_2] + 
			[\xi_2, \eta_1][[\eta_2, \xi_1], \xi_3] + 
			[\xi_2, \eta_2][[\eta_1, \xi_1], \xi_3] +
		\\
		& \quad
			[\xi_2, \eta_1][[\eta_2, \xi_3], \xi_1] + 
			[\xi_2, \eta_2][[\eta_1, \xi_3], \xi_1] + 
			[\xi_3, \eta_1][[\eta_2, \xi_2], \xi_1] +
		\\
		& \quad
			[\xi_3, \eta_2][[\eta_1, \xi_2], \xi_1] + 
			[\xi_3, \eta_1][[\eta_2, \xi_1], \xi_2] + 
			[\xi_3, \eta_2][[\eta_1, \xi_1], \xi_2] 
		\big).
	\end{align*}

	\item[$C_4$:]
	Now there is only $C_4$ left. We have one G-index:
	\begin{equation*}
		\mathcal{G}_1(3,2) 
		= 
		\left\{ \big( (3,2) \big) \right\},
	\end{equation*}
	but there are more terms which belong to it. We have to go through
	\begin{equation*}
		\bchparts{3}{2}{\xi}{\eta}
		=
		\frac{1}{120} [[[[\xi, \eta], \xi], \eta], \xi] +
		\frac{1}{360} [[[[\eta, \xi], \xi], \xi], \eta].
	\end{equation*}
	So we permute and get
	\begin{align*}
		C_4(\xi_1 \xi_2 \xi_3, \eta_1 \eta_2)
		& =
		\frac{1}{120} 
		\big( 
			[[[[\xi_1, \eta_1], \xi_2], \eta_2], \xi_3] + 
			[[[[\xi_1, \eta_2], \xi_2], \eta_1], \xi_3] + 
			[[[[\xi_1, \eta_1], \xi_3], \eta_2], \xi_2] +
		\\
		& \quad
			[[[[\xi_1, \eta_2], \xi_3], \eta_1], \xi_2] + 
			[[[[\xi_2, \eta_1], \xi_1], \eta_2], \xi_3] + 
			[[[[\xi_2, \eta_2], \xi_1], \eta_1], \xi_3] +
		\\
		& \quad
			[[[[\xi_2, \eta_1], \xi_3], \eta_2], \xi_1] + 
			[[[[\xi_2, \eta_2], \xi_3], \eta_1], \xi_1] + 
			[[[[\xi_3, \eta_1], \xi_2], \eta_2], \xi_1] +
		\\
		& \quad
			[[[[\xi_3, \eta_2], \xi_2], \eta_1], \xi_1] + 
			[[[[\xi_3, \eta_1], \xi_1], \eta_2], \xi_2] + 
			[[[[\xi_3, \eta_2], \xi_1], \eta_1], \xi_2] 
		\big) +
		\\
		& \quad
		\frac{1}{360} 
		\big( 
			[[[[\eta_1, \xi_1], \xi_2], \xi_3], \eta_2] + 
			[[[[\eta_2, \xi_1], \xi_2], \xi_3], \eta_1] + 
			[[[[\eta_1, \xi_1], \xi_3], \xi_2], \eta_2] +
		\\
		& \quad
			[[[[\eta_2, \xi_1], \xi_3], \xi_2], \eta_1] + 
			[[[[\eta_1, \xi_2], \xi_1], \xi_3], \eta_2] + 
			[[[[\eta_2, \xi_2], \xi_1], \xi_3], \eta_1] +
		\\
		& \quad
			[[[[\eta_1, \xi_2], \xi_3], \xi_1], \eta_2] + 
			[[[[\eta_2, \xi_2], \xi_3], \xi_1], \eta_1] + 
			[[[[\eta_1, \xi_3], \xi_2], \xi_1], \eta_2] +
		\\
		& \quad
			[[[[\eta_2, \xi_3], \xi_2], \xi_1], \eta_1] + 
			[[[[\eta_1, \xi_3], \xi_1], \xi_2], \eta_2] + 
			[[[[\eta_2, \xi_3], \xi_1], \xi_2], \eta_1] 
		\big).
	\end{align*}
\end{itemize}
Now we only have to add up all those terms and we have finally computed the 
star product.
