
%
% Chapter 4 of my master thesis:
% The formulas For the Gutt star product
%

\chapter{Formulas for the Gutt star product}

We have seen some results on the Baker-Campbell-Hausdorff series and an 
identity for the Gutt star product. The latter one, stated in Theorem 
\ref{Thm:Alg:GuttIsBCH}, will be a very useful tool in the following, since 
we want to get explicit formulas for $\star_{zG}$. There is still one part of 
the proof missing, but this will be caught up at the beginning of the first 
section of this chapter. From there, we will come to a first easy formula for 
$\star_{zG}$. Afterwards, we will use the same procedure to find two more 
formulas for it: a rather involved one for the $n$-fold star product of 
vectors, which will not necessarily be helpful for algebraic computations, 
but will turn out very useful for estimates, and a more explicit one for the 
product of two monomials.

From those formulas, we will be able to draw some easy, but nice 
consequences in the next section and we will show how to compute the 
Gutt star product explicitly by calculating two easy examples.

At the end of this chapter, we will give an easy Mathematica code, which 
can be used to verify the correctness of our formulas for polynomials of low orders.



\section{Formulas for the Gutt star product}
\label{sec:chap4_Formulas}


%
% An Iterative Approach from Linear Terms
%

\subsection{An Iterative Approach from Linear Terms}

The easiest case for which we will to develop a formula is surely the 
following one: For a given Lie algebra $\lie{g}$ and $\xi, \eta \in 
\lie{g}$ we would like to compute
\begin{equation*}
    \xi^k \star_{zG} \eta
    =
    \sum\limits_{n=0}^k
    z^n C_n(\xi^k, \eta)
\end{equation*}
We have already done this for the Gutt star product, now we 
want to do the same for the BCH star product. This will finish
the proof of Theorem \ref{Thm:Alg:GuttIsBCH}. For this purpose, 
we will use that
\begin{equation}\label{Formulas:MonomialDerivative}
    \xi^k
    =
    \frac{\partial^k}{\partial t^k}
    \At{t = 0} \exp(t \xi).
\end{equation}
We now have all the ingredients to prove the following proposition:
\begin{lemma}
    \label{Formulas:Lemma:LinearMonomial1}
    Let $\lie{g}$ be a Lie algebra and $\xi, \eta \in \lie{g}$. We
    have the following identity for the BCH star product $\star_{zG}$
    \begin{equation}
        \label{Formulas:LinearMonomial1}
        \xi^k \star_{zH} \eta
        =
        \sum\limits_{j=0}^k
        \binom{k}{j} z^j B_j^*
        \xi^{k-j}(\ad_{\xi})^j (\eta).
    \end{equation}
\end{lemma}
\begin{proof}
    We start from the simplified form for the Baker-Campbell-Hausdorff 
    series from Equation \eqref{Alg:BCH1stOrder}:
    \begin{equation*}
		\bch{\xi}{\eta}
		=
		\xi 
		+ 
		\sum\limits_{n = 0}^{\infty}
		\frac{B_n^*}{n!}
		\left( \ad_{\xi} \right)^n (\eta)
		+
		\mathcal{O}(\eta^2).
    \end{equation*}
    Putting things together with the definition of the BCH star product and 
    Equation \eqref{Formulas:MonomialDerivative} we get
    \begin{align*}
        \xi^k \star_{zH} \eta
        & =
        \frac{\partial^k}{\partial t^k}
        \frac{\partial}{\partial s}
        \At{t=0, s=0}
        \exp \left(
            \frac{1}{z} \bch{z t \xi}{z s \eta}
        \right)
        \\
        & =
        \frac{\partial^k}{\partial t^k}
        \frac{\partial}{\partial s}
        \At{t=0, s=0}
        \exp \left(
            t \xi + \sum\limits_{j=0}^{\infty}
            z^j \frac{B_j^*}{j!}
            \left( \ad_{t \xi} \right)^j
            (s \eta)
        \right).
    \end{align*}
    From this, we see that only terms which have exactly $k$ of the
    $\xi$'s in them and which are linear in $\eta$ will
    contribute. This means we can cut off the sum at $j = k$. If we
    now write out the exponential series which we can also cut for the
    same reason. We have
    \begin{align*}
        \xi^k \star_{zH} \eta
        & =
        \frac{\partial^k}{\partial t^k}
        \frac{\partial}{\partial s}
        \At{t=0, s=0}
        \sum\limits_{n=0}^{k}
        \frac{1}{n!}
        \left(
            t \xi
            +
            \sum\limits_{j=0}^{k}
            (zt)^j \frac{B_j^*}{j!}
            \left(\ad_{\xi}\right)^j
            (s \eta)
        \right)^n
        \\
        & =
        \frac{\partial^k}{\partial t^k}
        \frac{\partial}{\partial s}
        \At{t=0, s=0}
        \sum\limits_{n=0}^{k}
        \frac{1}{n!}
        \sum\limits_{m = 0}^n
        \binom{n}{m}
        (t \xi)^{n - m}
        \left(
            \sum\limits_{j=0}^{k}
            (zt)^j \frac{B_j^*}{j!}
            \left(\ad_{\xi}\right)^j
            (s \eta)
        \right)^m
        \\
        & =
        \frac{\partial^k}{\partial t^k}
        \frac{\partial}{\partial s}
        \At{t=0, s=0}
        \left(
            \sum\limits_{n=0}^{k}
            \frac{1}{n!}
            (t \xi)^n
            +
            \sum\limits_{n=0}^{k}
            \sum\limits_{j=0}^k
            \frac{1}{(n - 1)!} t^{n + j - 1}
            z^j \frac{B_j^*}{j!}
            \xi^{n - 1}
            \left( \ad_{\xi} \right)^j
            (s \eta)
        \right).
    \end{align*}
    In the last step we just cut off the sum over $m$ since the terms
    for $m >1$ will vanish because of the differentiation with respect
    to $s$. We can finally differentiate to get the formula
    \begin{align*}
        \xi^k \star_{zH} \eta
        & =
        \sum\limits_{n=0}^k
        \sum\limits_{j=0}^k
        \delta_{k, n + j - 1}
        \frac{k!}{j! (n - 1)!}
        z^j B_j^*
        \xi^{n - 1}
        \left(\ad_{\xi}\right)^j
        (\eta)
        \\
        & =
        \sum\limits_{j=0}^k
        \binom{k}{j}
        z^j B_j^*
        \xi^{k - j}
        \left( \ad_{\xi} \right)^j
        (\eta),
    \end{align*}
    which is the wanted result.
\end{proof}
\begin{remark}
    We have now proven the equality of the two star products 
    $\star_{zG}$ and $\star_{zH}$ by deriving an easy formula from both 
    of them. From now on, we will derive all the other formulas from 
    $\star_{zH}$, since this is the one which is easier to compute.
\end{remark}
Once this is done, it is actually easy to get the formula for
monomials of the form $\xi_1 \ldots \xi_k$ with $\eta \in \lie{g}$:
\begin{proposition}
	\label{Prop:Formulas:LinearMonomial2}
    Let $\lie{g}$ be a Lie algebra and $\xi_1, \ldots, \xi_k, \eta \in 
    \lie{g}$. We have
    \begin{equation}\label{Formulas:LinearMonomial2}
    	\xi_1 \ldots \xi_k \star_{zG} \eta
    	=
    	\sum\limits_{j=0}^k
    	\frac{1}{k!} \binom{k}{j}
    	z^j B_j^*
    	\sum\limits_{\sigma \in S_k}
    	[\xi_{\sigma(1)}, 
    		[ \ldots [\xi_{\sigma(j)}, \eta] \ldots ]
    	]
    	\xi_{\sigma(j+1)} \ldots \xi_{\sigma(k)}.
    \end{equation}
\end{proposition}
\begin{proof}
	We get the result by just polarizing the formula from Lemma 
	\ref{Lemma:Formulas:LinearMonomial1}. Let $\xi_1, \ldots, \xi_k \in 
	\lie{g}$ be given, then we introduce the parameters $t_i$ for $i = 
	1, \ldots, k$ and set
	\begin{equation*}
		\Xi
		=		
		\Xi(t_1, \ldots, t_k)
		=
		\sum\limits_{i=1}^k t_i \xi^i.
	\end{equation*}
	Then it is immediate to see that
	\begin{equation*}
		\xi_1 \ldots \xi_k
		=
		\frac{1}{k!}
		\frac{\partial^k}{\partial t_1 \ldots \partial t_k}
		\At{t_1, \ldots, t_k = 0}
		\Xi^k
	\end{equation*}
	since for every $i = 1, \ldots, k$ we have
	\begin{equation}\label{Formulas:LittleHelp1}
		\frac{\partial}{\partial t_i}
		\At{t_i = 0} \Xi
		=
		\xi_i.
	\end{equation}
	We also find for every $\eta \in \lie{g}$
	\begin{equation}\label{Formulas:LittleHelp2}
		\frac{\partial}{\partial t_i}
		\At{t_i = 0} \ad_{\Xi}(\eta)
		=
		\ad_{\xi_i}(\eta).
	\end{equation}
	Now we just need to calculate $\Xi^k \star_{zG} \eta$ and 
	differentiate with respect to the $t_i$. In order to do this 
	properly, we define
	\begin{equation*}
		\gamma_n^k
		(\xi_1, \ldots, \xi_k; \eta)
		=
		z^n \binom{k}{n} B_n^*
		\left( 
			\ad_{\xi_1} 
			\circ \ldots \circ 
			\ad_{\xi_n}
		\right)
		(\eta)
		\xi_{n+1} \ldots \xi_k
	\end{equation*}
	and
	\begin{equation*}
		\gamma^k
		(\xi_1, \ldots, \xi_k; \eta)
		=
		\sum\limits_{n = 0}^k
		\gamma_n^k
		(\xi_1, \ldots, \xi_k; \eta).
	\end{equation*}
	We see that
	\begin{equation*}
		\Xi^k \star_{zG} \eta
		=
		\gamma^k
		(\Xi, \ldots, \Xi; \eta)
	\end{equation*}
	and can now differentiate this expression, which is linear in the 
	every argument, with respect to the $t_i$. From the Equations 
	\eqref{Formulas:LittleHelp1} and \eqref{Formulas:LittleHelp2} we get 
	with the Leibniz rule
	\begin{equation*}
		\frac{\partial}{\partial t_1}
		\gamma^k
		(\Xi, \ldots, \Xi; \eta)
		=
		\sum\limits_{j = 1}^k
		\gamma^k
		(
			\underbrace{\Xi, \ldots, \Xi}_{
				j-1 \text{ times}
			}
			, \xi_1, 
			\underbrace{\Xi, \ldots, \Xi}_{
				k-j-1 \text{ times}
			}
			; \eta
		)
	\end{equation*}
	Differentiating now with respect to $t_2$, we get a second sum, 
	where $\xi_2$ will be put once in every ''free'' position, and so 
	on. One by one, all the slots will be taken by $\xi_i$'s. We just 
	need to divide by $k!$, and we finally find the formula from 
	Equation \eqref{Formulas:LinearMonomial2}.
\end{proof}



% A first general Formula
%
\subsection{A first general Formula}

Proposition \ref{Prop:Formulas:LinearMonomial2} allows us basically 
to get a formula for the case of $\xi_1, \ldots, \xi_k \in \lie{g}$
\begin{equation*}
	\xi_1 \star_{zG} \ldots \star_{zG} \xi_k
	=
	\sum\limits_{j=0}^k
	C_{z,j} \left(
		\xi_1, \ldots, \xi_k
	\right)
\end{equation*}
which we will need to prove the continuity of the coproduct, but which 
can also help to prove the continuity of the product in a different way.

Unluckily, this approach has a problem: iterating this formula, we get 
strangely nested Lie brackets, which would be very difficult to bring 
into a nice form with Jacobi and higher identities. So this is not a 
good way to find an handy formula for the usual star product of two 
monomials. Nevertheless, we want to pursue it for a moment, since we 
will get an equality which will be, although rather unfriendly looking, 
very useful in the following: for analytic observations, it will be 
enough to put (even brutal) estimates on it and the exact nature of the 
combinatorics in the formula will not be important. Hence we rewrite 
Equation \eqref{Formulas:LinearMonomial2} in order to cook up such a 
formula.
\begin{definition}
	\label{Def:Formulas:BMaps}
	Let $j, k \in \mathbb{N}_0$, $j \leq k$ and $B_j^*$ as usual, then 
	we define bilinear maps via
	\begin{equation*}
		\begin{array}{rcl}
			B_z^{k, j} 
			\colon
			\Sym^k(\lie{g}) \times \lie{g}
			&
			\longrightarrow
			&
			\Sym^{k - j + 1}(\lie{g})
			\\
			(\xi_1 \ldots \xi_k, \eta)
			&
			\longmapsto
			&
			\frac{1}{k!}
			\sum\limits_{\sigma \in S_k}
			\binom{k}{j} B_j^* z^j
			[\xi_{\sigma(1)}, [ \ldots, [\xi_{\sigma(j)}, \eta]]] 
			\xi_{\sigma(j + 1)} \ldots \xi_{\sigma(k)}		
		\end{array}
	\end{equation*}
	and
	\begin{equation*}
		B_z^j
		\colon
		\Sym^{\bullet}(\lie{g}) \times \lie{g}
		\longrightarrow
		\Sym^{\bullet}(\lie{g})
		, \quad
		B_z^j 
		= 
		\sum\limits_{k=0}^{\infty}
		B_z^{k, j}
	\end{equation*}
	where we set $B_z^j(x) = 0$ if $\deg(x) < j$ .
\end{definition}
We immediately get an easier identity for Equation 
\eqref{Formulas:LinearMonomial2}:
\begin{equation}\label{Formulas:LinearMonomial3}
	\xi_1 \ldots \xi_k \star_{zG} \eta
	=
	\sum\limits_{j = 0}^k
	B_z^j(\xi_1 \ldots \xi_k, \eta).
\end{equation}
More than that: We can extend it to 
arbitrary symmetric tensors:
\begin{lemma}
	\label{Lemma:LinearMonomial4}
	Let $\lie{g}$ be a Lie-algebra an $x \in \Sym^{\bullet}(\lie{g})$. 
	Then we have the formula
	\begin{equation}
		\label{Formulas:LinearMonomial4}
		x \star_{zG} \eta
    	=
    	\sum\limits_{j=0}^{\infty}
    	B_z^j(x, \eta).
	\end{equation}
\end{lemma}
\begin{proof}
	First it is clear that the sum over $j$ in Equation 
	\eqref{Formulas:LinearMonomial4} is actually finite, since for $j 
	> \deg(x)$ there is no further contribution. Using the grading we 
	can write
	\begin{equation*}
		x 
		= 
		\sum\limits_{k=0}^{\deg(x)}
		\sum_i x_i^{(k)}.
	\end{equation*}
	The $B_z^i$-maps are linear in the first argument and the 
	$x_i^{(k)}$ can be chosen to be factorizing tensors. But on 
	factorizing tensors, this is just Equation 
	\eqref{Formulas:LinearMonomial3}. We hence have by the linearity 
	of $\star_{zG}$
	\begin{align*}
		x \star_{zG} \eta
		& =
		\sum\limits_{k=0}^{\deg(x)}
		\sum_i 
		x_i^{(k)}
		\star_{zG} \eta
		\\
		& =
		\sum\limits_{k=0}^{\deg(x)}
		\sum_i 
		\sum\limits_{j=0}^{\infty}
    	B_z^j \left( x_i^{(k)}, \eta \right)
		\\
		& =
		\sum\limits_{j=0}^{\infty}
    	B_z^j
    	\left(
			\sum\limits_{k=0}^{\deg(x)}
			\sum_i 
			x_i^{(k)}, \eta
		\right)
		\\
		& =
    	\sum\limits_{j=0}^{\infty}
    	B_z^j(x, \eta).
	\end{align*}
\end{proof}
We can use this approach to go on:
\begin{proposition}
	\label{Prop:Formulas:MultipleStars}
	Let $\lie{g}$, $2 \leq k \in \mathbb{N}$  and $\xi_1, \ldots, \xi_k 
	\in \lie{g}$. Then we have
	\begin{equation}\label{Formulas:MultipleStars}
		\xi_1 \star_{zG} \ldots \star_{zG} \xi_k		
		=
		\sum\limits_{\substack{
			1 \leq j \leq k-1 \\
			i_j \in \{0, \ldots, j\}
		}}
		B_z^{i_{k-1}}
		\left(
			\ldots B_z^{i_2}
			\left(
				B_z^{i_1} 
				\left( \xi_1, \xi_2 \right)
				, \xi_3	
			\right) 
			\ldots, \xi_{k}
		\right).
	\end{equation}
\end{proposition}
\begin{proof}
	We will prove this by induction over $k$. For $k = 2$ we get
	\begin{equation*}
		\xi_1 \star_{zG} \xi_2
		=
		B_z^0 \left( \xi_1, \xi_2 \right)
		+
		B_z^1 \left( \xi_1, \xi_2 \right)
		=
		\xi_1 \xi_2 +
		\frac{1}{2} [\xi_1, \xi_2]
	\end{equation*}
	Which is clearly true. For the step $k \rightarrow k + 1$ we can directly 
	apply Equation \eqref{Formulas:LinearMonomial4}:
	\begin{align*}
		\xi_1 \star_{zG} \ldots \star_{zG} \xi_{k+1}
		& =
		\Bigg(
			\sum\limits_{\substack{
				1 \leq j \leq k-1 \\
				i_j \in \{0, \ldots, j\}
			}}
			B_z^{i_{k-1}}
			\left(
				\ldots B_z^{i_2}
				\left(
					B_z^{i_1} 
					\left( \xi_1, \xi_2 \right)
					, \xi_3	
				\right) 
				\ldots, \xi_{k}
			\right)
		\Bigg)
		\star_{zG} \xi_{k+1}
		\\
		& = 
		\sum\limits_{i_k = 0}^k
		B_z^{i_k}
		\Bigg(
			\sum\limits_{\substack{
				1 \leq j \leq k-1 \\
				i_j \in \{0, \ldots, j\} \\
			}}
			B_z^{i_{k-1}}
			\left(
				\ldots B_z^{i_2}
				\left(
					B_z^{i_1} 
					\left( \xi_1, \xi_2 \right)
					, \xi_3	
				\right) 
				\ldots, \xi_{k}
			\right)
			, \xi_{k+1}
		\Bigg)
		\\
		& = 
		\sum\limits_{\substack{
			1 \leq j \leq k \\
			i_j \in \{0, \ldots, j\} \\
		}}
		B_z^{i_k}
		\left(
			B_z^{i_{k-1}}
			\left(
				\ldots B_z^{i_2}
				\left(
					B_z^{i_1} 
					\left( \xi_1, \xi_2 \right)
					, \xi_3	
				\right) 
				\ldots, \xi_{k}
			\right)
			, \xi_{k+1}
		\right)
	\end{align*}
\end{proof}

\begin{remark}
	Our final goal in this chapter is actually a nice identity for the case 
	of two monomials $\xi_1 \ldots \xi_k \star_{zG} \eta_1 \ldots 
	\eta_{\ell}$ with $\xi_1, \ldots, \xi_k, \eta_1, \ldots, \eta_{\ell} \in 
	\lie{g}$. Theoretically, we could use Equation 
	\eqref{Formulas:MultipleStars} for it, since
	\begin{equation}\label{Formulas:2MonomialsWeird}
		\xi_1 \ldots \xi_k \star_{zG} \eta_1 \ldots \eta_{\ell}
		=
		\frac{1}{k! \ell!}
		\sum\limits_{\sigma \in S_k}
		\sum\limits_{\tau \in S_{\ell}}
		\xi_{\sigma(1)} \star_{zG} \ldots \star_{zG} \xi_{\sigma(k)}
		\star_{zG}
		\eta_{\tau(1)} \star_{zG} \ldots \star_{zG} \eta_{\tau(\ell)}.
	\end{equation}
	This equality can easily been proven from the definition of the map 
	$\mathfrak{q}$. The only flaw in the plan is, however, that we're looking 
	for something \textit{nice}. So we have to go for something different.
\end{remark}



% A Formula for two Monomials
%

\subsection{A Formula for two Monomials}

If we want to get an identity for the star product of two monomials, we have 
to get back to Equation \eqref{Alg:GuttIsBCH}. The result will not be very 
explicit either, but still by far better than Equation 
\eqref{Formulas:2MonomialsWeird}. We will at least be able to do some 
computations with concrete examples. As a first step, we must introduce 
a bit of notation:
\begin{definition}[G-Index]
	\label{Def:GuttIndex}
	Let $k, \ell, n \in \mathbb{N}$ and $r = k + \ell - n$. 
	Then we call an $r$-tuple $J$
	\begin{equation*}
		J = (J_1, \ldots, J_r) 
		= 
		((a_1, b_1), \ldots, (a_r, b_r)) 
	\end{equation*}
  	a G-index if it fulfils the following properties:
	\begin{enumerate}[(i)]
		\item
		$J_i \in \{0, 1, \ldots, k\} \times \{0, 1, \ldots, \ell\}$
		
  		\item 
		$|J_i| 
		= 
		a_i + b_i \geq 1 
		\quad \forall_{i = 1, \ldots, r}$
		
		\item 
		$\sum\limits_{i=1}^{r} a_i = k$ 
		and 
		$\sum\limits_{i=1}^{r} b_i = \ell$
		
		\item
		The tuple is ordered in the following sense:
		$i>j \Rightarrow |J_i| \geq |J_j| \quad \forall_{i,j = 1, 
		\ldots, r}$ and $|a_i| \geq |a_j|$ if $|J_i| = |J_j|$
		
		\item 
		If $a_i = 0$ [or $b_i = 0$] for some $i$, 
		then $b_i = 1$ [or $a_i = 1$].
	\end{enumerate}
	We call the set of all such $G$-indices	$\mathcal{G}_r(k,\ell)$.
\end{definition}
\begin{definition}[G-Factorial]
	\label{Def:GuttFactorial}
	Let $J = ((a_1, b_1), \ldots, (a_r, b_r)) \in \mathcal{G}_r(k,
	\ell)$ be a G-Index. We set for a given tuple $(a,b) \in \{0, 1, 
	\ldots, k\} \times \{0, 1, \ldots, \ell\}$
	\begin{equation*}
		\#_J (a,b)
		= 
		\textrm{ number of times that $(a,b)$ appears in } J.
	\end{equation*}
	Then we define the G-factorial of $J \in \{0, 1, \ldots, k\} \times 
	\{0, 1, \ldots, \ell\}$ as
	\begin{equation*}
		J!
		= 
		\prod\limits_{
			(a,b) \in 
			\{0, 1, \ldots, k\} 
			\times 
			\{0, 1, \ldots, \ell\}
		}
		\left( \#_J (a,b) \right)!
	\end{equation*}
\end{definition}
This allows us to state an explicit formula for the Gutt star product:
\begin{lemma}
	\label{Lemma:Formulas:2MonomialsFormula1}
	Let $\lie{g}$ be a Lie algebra, $\xi, \eta \in \lie{g}$ and $k, 
	\ell \in \mathbb{N}$. Then we have the following identity for the 
	Gutt star product:
	\begin{equation*}
    	\xi^k \star_{zG} \eta^{\ell}
    	=
	    \sum\limits_{n=0}^{k + \ell -1}
    	z^n
    	C_n \left( \xi^k, \eta^{\ell} \right),
	\end{equation*}
	where the $C_n$ are given by
	\begin{equation}
		\label{Formulas:2MonomialsFormula1}
        C_n \left( \xi^k, \eta^{\ell} \right)
        =
        \sum\limits_{J \in \mathcal{G}_{k + \ell - n}(k, \ell)}
        \frac{k! \ell!}{J!}
        \prod\limits_{i = 1}^{k + \ell - n}        
        \bchparts{a_i}{b_i}{\xi}{\eta}
	\end{equation}
	and the product is taken in the symmetric tensor algebra.
\end{lemma}
\begin{proof}
	We want to calculate what the $C_n$ look like. Let's denote $r = k 
	+ \ell - n$ for brevity. Then we have
	\begin{equation*}
		C_n \left( \xi^k, \eta^{\ell} \right)
		\in \Sym^r(\lie{g}).
	\end{equation*}
	Of course, the only part of the series
	\begin{equation*}
		\exp\left(
			\frac{1}{z} \bch{z \xi}{z \eta}
		\right)
		=
		\sum\limits_{n = 0}^{k + \ell}
		\left(
			\frac{1}{z} \bch{z \xi}{z \eta}
		\right)^n
		+ \mathcal{O}(\xi^{k + 1}, \eta^{\ell + 1})
	\end{equation*} which lies in $\Sym^r(\lie{g})$ is the summand for 
	$n = r $. Since we introduce the formal parameters $t$ and $s$, we don't 
	need to care about terms of higher orders in $\xi$ and $\eta$ than $k$ 
	and $\ell$ respectively.
    \begin{align}
        \nonumber
        z^n C_n \left( \xi^k, \eta^{\ell} \right)
        & =
        \frac{\partial^k}{\partial t^k}
        \frac{\partial^{\ell}}{\partial s^{\ell}}
        \At{t,s = 0}
        \frac{1}{z^r}
        \frac{\bch{z t \xi}{z s \eta}^r    }{r!}
        \\
        \nonumber
        & =
        \frac{1}{z^r}
        \frac{1}{r!}
        \frac{\partial^k}{\partial t^k}
        \frac{\partial^{\ell}}{\partial s^{\ell}}
        \At{t,s = 0}
        \left(
            \sum\limits_{j = 1}^{k + \ell}
            \bchpart{j}{z t \xi}{z s \eta}
        \right)^r
        \\
        \nonumber
        & =
        \frac{1}{z^r}
        \frac{1}{r!}
        \frac{\partial^k}{\partial t^k}
        \frac{\partial^{\ell}}{\partial s^{\ell}}
        \At{t,s = 0}
        \sum\limits_{\substack{
        	j_1, \ldots, j_r \geq 1 \\
            j_1 + \ldots + j_r = k + \ell
        }}
        \bchpart{j_1}{z t \xi}{z s \eta} 
        \cdots
        \bchpart{j_r}{z t \xi}{z s \eta}
        \\
        \label{Formulas:2MonomialsExplicit}
        & =
        z^n
        \frac{k! \ell!}{r!}
        \sum\limits_{\substack{a_1, b_1, \ldots, a_r, b_r \geq 0 \\
            a_i + b_i \geq 1 \\
            a_1 + \ldots + a_r = k \\
            b_1 + \ldots + b_r = \ell
        }}
        \bchparts{a_i}{b_i}{\xi}{\eta}
        \ldots
        \bchparts{a_r}{b_r}{\xi}{\eta}
    \end{align}
    We sum over all possible arrangements of the $(a_i, b_i)$. In order 
    to find a nicer form of the sum, we put the ordering from definition 
    \ref{Def:GuttIndex} on these multi-indices and avoid therefore double 
    counting. We loose the freedom of arranging the $(a_i, b_i)$ and 
    need to count the number of multi-indices $((a_1, b_1), \ldots, 
    (a_r, b_r))$ which belong to the same G-index $J$. This number will 
    be $\frac{r!}{J!}$, since we can't interchange the $(a_i, b_i)$ any 
    more (therefore $r!$), unless they are equal (therefore $J!^{-1}$). 
    Since the ranges of the $(a_i, b_i)$ in Equation 
    \eqref{Formulas:2MonomialsExplicit} and of  the elements in 
    $\mathcal{G}_r(k, \ell)$ are the same. we can change the summation there 
    to $J \in \mathcal{G}_r(k, \ell)$ and multiply by $\frac{r!}{J!}$. We 
    find
    \begin{equation*}
    	z^n C_n \left( \xi^k, \eta^{\ell} \right)
    	=
    	z^n \frac{k! \ell!}{J!}
    	\sum\limits_{J \in \mathcal{G}_r(k, \ell)}
    	\bchparts{a_i}{b_i}{\xi}{\eta}
        \ldots
        \bchparts{a_r}{b_r}{\xi}{\eta}
    \end{equation*}
    which is precisely Equation \eqref{Formulas:2MonomialsFormula1}.
\end{proof}
Now we just need to generalize this to factorizing tensors. To do so, we 
need a last definition:
\begin{definition}
	\label{Def:GuttInsert}
	Let $k, \ell, n \in \mathbb{N}$ and $J \in \mathcal{G}_{k + \ell - 
	n}(k, \ell)$. Then for $\xi_1, \ldots, \xi_k, \eta_1, \ldots, 
	\eta_{\ell}$ from a Lie algebra $\lie{g}$ we set
	\begin{equation}
		\label{Formulas:GuttInsert}
		\Gamma_J(\xi_1, \ldots, \xi_k; \eta_1, \ldots, \eta_{\ell})
		=
		\frac{1}{J!}
		\prod\limits_{i=1}^{k + \ell - n}
		\bchparts{a_i}{b_i}{\xi^{(a_i)}}{\eta^{(b_i)}}
	\end{equation}
	where the notation $\bchparts{a_i}{b_i}{\xi^{(a_i)}}{\eta^{(b_i)}}$ 
	means that we have taken $\prod\limits_{i=1}^{k + \ell - n} 
	\bchparts{a_i}{b_i}{\xi^{(a_i)}}{\eta^{(b_i)}}$ and replaced the 
	$j$-th $\xi$ appearing in it with $\xi_j$ for $j = 1, \ldots, k$ and 
	analogously with the $\eta$'s.
\end{definition}
\begin{proposition}
	\label{Prop:2MonomialsFormula2}
	Let $\lie{g}$ be a Lie algebra, $k, \ell \in \mathbb{N}$ and $\xi_1, 
	\ldots, \xi_k, \eta_1, \ldots, \eta_{\ell} \in \lie{g}$. Then we 
	have the following identity for the Gutt star product:
	\begin{equation*}
    	\xi_1 \ldots \xi_k \star_{zG} \eta_1 \ldots \eta_{\ell}
    	=
	    \sum\limits_{n=0}^{k + \ell -1}
    	z^n C_n
    	\left( 
    		\xi_1 \ldots \xi_k, \eta_1 \ldots \eta_{\ell}
    	\right),
	\end{equation*}
	where the $C_n$ are given by
	\begin{equation}
		\label{Formulas:2MonomialsFormula2}
        C_n
        \left( 
    		\xi_1 \ldots \xi_k, \eta_1 \ldots \eta_{\ell}
    	\right)
        =
        \sum\limits_{J \in \mathcal{G}_{k + \ell - n}(k, \ell)}
        \sum\limits_{\sigma \in S_k}
        \sum\limits_{\tau \in S_{\ell}}
        \Gamma_J 
        \left(
        	\xi_{\sigma(1)}, \ldots, \xi_{\sigma(k)};
        	\eta_{\tau(1)}, \ldots, \eta_{\tau(\ell)}
        \right)
	\end{equation}
	and the product is taken in the symmetric tensor algebra.
\end{proposition}
\begin{proof}
	The proof relies on polarization again and is completely analogous 
	to the one of proposition \ref{Prop:Formulas:LinearMonomial2}. We set
	\begin{equation*}
		\Xi
		=
		\sum\limits_{i=1}^k t_i \xi^i
		\quad \text{ and } \quad
		\Eta
		=
		\sum\limits_{i=1}^{\ell} t_j \eta^j
		.
	\end{equation*}
	Then it is easy to see that we will get rid of the factorials in Equation 
	\eqref{Formulas:2MonomialsFormula1} since
	\begin{equation*}
		\xi_1 \ldots \xi_k \star_{zG} \eta_1 \ldots \eta_{\ell}
		=
		\frac{1}{k! \ell!}
		\frac{\partial^{k + \ell}}
		{\partial_{t_1} \ldots \partial_{s_{\ell}}}
		\At{t_1, \ldots, s_{\ell} = 0}
		\Xi^k \star_{zG} \Eta^{\ell}.
	\end{equation*}
	Instead of the factorials, we get symmetrizations over the 
	$\xi_i$ and the $\eta_j$ as we did in Proposition 
	\ref{Prop:Formulas:LinearMonomial2}, which gives the wanted result.
\end{proof}



\section{Consequences and examples}
\label{sec:chap4_Consequences}

\subsubsection*{Some consequences}
Proposition \ref{Prop:2MonomialsFormula2} allows us to get some easy 
algebraic results. For example, we know that the Gutt star product should 
fulfil the classical and the semi-classical limit from Definition 
\ref{Def:StarProduct} and this was also proven by Simone Gutt in the paper 
\cite{Gutt:83-Gst} where she discovered it, but it is good to see that the 
formula we set up really gives the same result.

\begin{corollary}
	\label{Formulas:Cor:LimitCases}
	Let $\lie{g}$ be a Lie algebra and $\Sym^{\bullet}(\lie{g})$ edowed with 
	the Gutt star product
	\begin{equation*}
		x \star_z y
		= 
		\sum\limits_{n = 0}^{\infty}
		z^n C_n(x,y).
	\end{equation*}	
	\begin{corollarylist}
		\item
		On factorizing tensors $\xi_1 \ldots \xi_k$ and $\eta_1 \ldots 
		\eta_{\ell}$, $C_0$ and $C_1$ give
		\begin{align}
			\label{Formulas:ClassicalLimit}
			C_0 \left(
				\xi_1 \ldots \xi_k, \eta_1 \ldots \eta_{\ell}
			\right)
			& = 
			\xi_1 \ldots \xi_k \eta_1 \ldots \eta_{\ell}
			\\
			\label{Formulas:SemiClassicalLimit}
			C_1 \left(
				\xi_1 \ldots \xi_k, \eta_1 \ldots \eta_{\ell}
			\right)
				& =
			\frac{1}{2}	
			\sum\limits_{i = 1}^k
			\sum\limits_{j = 1}^{\ell}
			\xi_1 \ldots \widehat{\xi_i} \ldots \xi_k
			\eta_1 \ldots \widehat{\eta_j} \ldots \eta_{\ell}
			[\xi_i \eta_j],
		\end{align}
		where the hat denotes elements which are left out.
		
		\item
		If $\lie{g}$ is finite-dimensional and we use the canonical 
		isomorphism $\mathcal{J} \colon \Sym^{\bullet}(\lie{g}) 
		\longrightarrow \Pol^{\bullet}(\lie{g}^*)$, we have for
		$f,g \in \Pol^{\bullet}(\lie{g}^*)$
		\begin{equation*}
			C_1 \left(
				\mathcal{J} (f),
				\mathcal{J} (g)
			\right)
			-
			C_1 \left(
				\mathcal{J} (f),
				\mathcal{J} (g)
			\right)
			= 
			\mathcal{J}^{-1} \left( 
				\{ f, g \}_{KKS}
			\right)
		\end{equation*}
		where $\{ \cdot, \cdot \}_{KKS}$ is the Kirillov-Kostant-Souriau 
		bracket.
		
		\item
		The map $\star_z$ fulfils the classical and the semi-classical limit 
		and is therefore a star product.
	\end{corollarylist}
\end{corollary}
\begin{proof}
	We take $\xi_1 \ldots \xi_k, \eta_1 \ldots \eta_{\ell} \in \Sym^{\bullet}
	(\lie{g})$ and consider the 	G-indices in $\mathcal{G}_{k + \ell}(k, \ell)$ 
	first. This is easy, since there is just one element inside:
	\begin{equation*}
		\mathcal{G}_{k + \ell}(k, \ell)
		=
		\Big\{
			( 
				\underbrace{(0,1), \ldots, (0,1)}_{
				\ell \text{ times}
				}
				,
				\underbrace{(1,0), \ldots, (1,0)}_{
				k \text{ times}
				}
			)
		\Big\}.
	\end{equation*}
	So we find
	\begin{align*}
		C_0 
		\left(
			\xi_1 \ldots \xi_k, \eta_1 \ldots \eta_{\ell}
		\right)
		& =
		\sum\limits_{\sigma \in S_k}
		\sum\limits_{\tau \in S_{\ell}}
		\frac{1}{J!}
		\bchparts{0}{1}{\varnothing}{\xi_{\sigma(1)}}
		\ldots
		\bchparts{0}{1}{\varnothing}{\xi_{\sigma(k)}}
		\\
		& \quad \cdot
		\bchparts{1}{0}{\eta_{\tau(1)}}{\varnothing}
		\ldots
		\bchparts{1}{0}{\eta_{\tau(\ell)}}{\varnothing}
		\\
		& =
		\sum\limits_{\sigma \in S_k}
		\sum\limits_{\tau \in S_{\ell}}
		\frac{1}{k! \ell!}
		\xi_{\sigma(1)} \ldots \xi_{\sigma(k)}
		\eta_{\tau(1)} \ldots \eta_{\tau(\ell)}
		\\
		& =
		\xi_{\sigma(1)} \ldots \xi_{\sigma(k)}
		\eta_{\tau(1)} \ldots \eta_{\tau(\ell)}
	\end{align*}
	where we used $J! = k! \ell!$  according to 
	Definition~\ref{Def:GuttFactorial}. We do the same for $C_1(\ldots)$. Also 
	here, we have just one element in $\mathcal{G}_{k + \ell - 1}(k, \ell)$:
		\begin{equation*}
		\mathcal{G}_{k + \ell}(k, \ell)
		=
		\Big\{
			( 
				\underbrace{(0,1), \ldots, (0,1)}_{
				\ell-1 \text{ times}
				}
				,
				\underbrace{(1,0), \ldots, (1,0)}_{
				k-1 \text{ times}
				}
				,
				(1,1)
			)
		\Big\}.
	\end{equation*}
	Using
	\begin{equation*}
		\bchparts{1}{1}{\xi}{\eta}
		=
		\frac{1}{2} [\xi, \eta]
	\end{equation*}
	and $J! = (k - 1)! (\ell - 1)!$, we find
	\begin{align*}
		C_1 
		\left(
			\xi_1 \ldots \xi_k, \eta_1 \ldots \eta_{\ell}
		\right)
		& =
		\frac{1}{2}
		\sum\limits_{\sigma \in S_k}
		\sum\limits_{\tau \in S_{\ell}}
		\frac{1}{(k-1)! (\ell-1)!}
		\xi_{\sigma(1)} \ldots \xi_{\sigma(k-1)}
		\eta_{\tau(1)} \ldots \eta_{\tau(\ell-1)}
		[\xi_{\sigma(k)}, \eta_{\tau(\ell)}]
		\\
		& =
		\frac{1}{2}
		\sum\limits_{i = 0}^k
		\sum\limits_{j = 0}^{\ell}
		\xi_1 \ldots \widehat{\xi_i} \ldots \xi_k
		\eta_1 \ldots \widehat{\eta_j} \ldots \eta_{\ell}
		[\xi_i, \eta_j].
	\end{align*}
	This finishes part one. From this, the anti-symmetry of the Lie bracket 
	yields
	\begin{equation*}
		C_1 
		\left(
			\xi_1 \ldots \xi_k, \eta_1 \ldots \eta_{\ell}
		\right)
		-
		C_1 
		\left(
			\eta_1 \ldots \eta_{\ell}, \xi_1 \ldots \xi_k
		\right)
		=
		\sum\limits_{i = 0}^k
		\sum\limits_{j = 0}^{\ell}
		\xi_1 \ldots \widehat{\xi_i} \ldots \xi_k
		\eta_1 \ldots \widehat{\eta_j} \ldots \eta_{\ell}
		[\xi_i, \eta_j].
	\end{equation*}
	We now need to compute the KKS brackets on polynomials. Because of the 
	linearity in both arguments, it is sufficient to check it on monomials of 
	coordinates. Let $e_1, \ldots, e_n$ be a basis with linear coordinates 
	$x_1, \ldots, x_n$. Now take $\mu_1, \ldots, \mu_n, \nu_1, \ldots \nu_n 
	\in \mathbb{N}$ and 	consider the monomials $f = x_1^{\mu_1} \ldots 
	x_n^{\mu_n}$ and $g = x_1^{\nu_1} \ldots x_n^{\nu_n}$. We use the 
	notation fronm Proposition~\ref{Alg:Prop:LinPoissonIsLieAlg} and find
	for $x \in \lie{g}^*$
	\begin{align*}
		\{ f, g \}_{KKS}(x)
		& =
		x_k c_{ij}^k
		\frac{\partial f}{\partial x_i}
		\frac{\partial g}{\partial x_j}
		\\
		& =
		\mu_i \nu_j c_{ij}^k
		x_k 
		x_1^{\mu_1} \ldots x_i^{\mu_i - 1} \ldots x_n^{\mu_n}
		x_1^{\nu_1} \ldots x_j^{\nu_j - 1} \ldots x_n^{\nu_n}.
	\end{align*}
	Applying $\mathcal{J}^{-1}$ to it gives
	\begin{equation}
		\label{Formulas:SemiClassicalIntermediate}
		\mathcal{J}^{-1}
		\left(
			\{ f, g \}_{KKS}
		\right)
		=
		\sum\limits_{i=0}^n
		\sum\limits_{j=0}^n
		\mu_i \nu_j
		e_1^{\mu_1} \ldots e_i^{\mu_i - 1} \ldots e_n^{\mu_n}
		e_1^{\nu_1} \ldots e_j^{\nu_j - 1} \ldots e_n^{\nu_n}
		[e_i, e_j].
	\end{equation}
	On the other hand, we have
	\begin{equation*}
		\mathcal{J}(f)
		=
		e_1^{\mu_1} \ldots e_n^{\mu_n}
		\quad \text{ and } \quad
		\mathcal{J}(g)
		=
		e_1^{\nu_1} \ldots e_n^{\nu_n}.
	\end{equation*}
	Together with \eqref{Formulas:SemiClassicalLimit} this is exactly 
	\eqref{Formulas:SemiClassicalIntermediate} and proves part two.
	From this, the third part is immediate, due to the bilinearity of 
	the $C_n$.
\end{proof}
Moreover, we have compatibility of the bigger formula from Proposition 
\ref{Prop:2MonomialsFormula2} with the smaller one from Proposition 
\ref{Prop:Formulas:LinearMonomial2}.
\begin{corollary}
	\label{Formulas:Cor:FormulasCoincide}
	Given $\xi_1, \ldots, \xi_k, \eta \in \lie{g}$, the results of the two 
	Equations \eqref{Formulas:2MonomialsFormula2} and 
	\eqref{Formulas:LinearMonomial2} coincide.
\end{corollary}
\begin{proof}
	We have to compute sets of G-indices for $\xi_1, \ldots, \xi_k, 
	\eta_{\ell} \in \lie{g}$. Again, they only have one element:
	\begin{equation*}
		\mathcal{G}_{k + 1 - n}(k, 1)
		=
		\Big\{
			( 
				\underbrace{(1,0), \ldots, (1,0)}_{
				k - n \text{ times}
				}
				,
				(n,1)
			)
		\Big\}.
	\end{equation*}
	So we have with $J! = (k - n)!$
	\begin{equation*}
		z^n C_n
		\left(
			\xi_1 \ldots \xi_k, \eta_{\ell}
		\right)
		=
		z^n
		\sum\limits_{\sigma \in S_k}
		\frac{1}{(k - n)!}
		\frac{B_n^*}{n!}
		\xi_{\sigma(1)} \ldots \xi_{\sigma(k-n)}
		[\xi_{\sigma(k-n+1)}, [
			\ldots, [\xi_{\sigma(k), \eta} ] \ldots 
		]]
	\end{equation*}
	which gives, after a light reordering
	\begin{equation*}
		z^n C_n
		\left(
			\xi_1 \ldots \xi_k, \eta_{\ell}
		\right)
		=
		z^n
		\frac{1}{k!}
		\sum\limits_{\sigma \in S_k}
		\binom{k}{n} B_n^*
		[\xi_{\sigma(1)}, [
			\ldots, [\xi_{\sigma(n), \eta} ] \ldots 
		]]
		\xi_{\sigma(n+1)} \ldots \xi_{\sigma(k)}.
	\end{equation*}
	Summing up over all $n$ now gives Equation 
	\eqref{Formulas:LinearMonomial2}.
\end{proof}
Just to make it it complete, we also want to state what it looks like when we 
change the left and the right hand side.
\begin{proposition}
	\label{Prop:Formulas:LinearMonomial2T}
    Let $\lie{g}$ be a Lie algebra and $\xi_1, \ldots, \xi_k, \eta \in 
    \lie{g}$. We have
    \begin{equation}\label{Formulas:LinearMonomial2T}
    	\eta \star_{zG} \xi_1 \ldots \xi_k
    	=
    	\sum\limits_{j=0}^k
    	\frac{1}{k!} \binom{k}{j}
    	z^j B_j
    	\sum\limits_{\sigma \in S_k}
    	[\xi_{\sigma(1)}, 
    		[ \ldots [\xi_{\sigma(j)}, \eta] \ldots ]
    	]
    	\xi_{\sigma(j+1)} \ldots \xi_{\sigma(k)}.
    \end{equation}
\end{proposition}
\begin{proof}
	The proof is completely analogue to the one of Proposition 
	\ref{Prop:Formulas:LinearMonomial2}. The only difference is that we take 
	Equation \eqref{Alg:BCHFirstOrderT} which gives the BCH series up to 
	first order in the first and not in the second argument.
\end{proof}



\subsubsection*{Two examples}
Equation \eqref{Formulas:2MonomialsFormula2} is useful if one wants to do 
real computations with the star product, but it does not look very easy to 
apply on the first sight. This is why we will give two examples here. The 
easiest one which is not covered by the simpler formula 
\eqref{Formulas:LinearMonomial2} will be the star product of two quadratic 
terms. The second one should be the a bit more complex case of a cubic term 
with a quadratic term.

\subsubsection*{Two quadratic terms}
Let's take $\xi_1, \xi_2, \eta_1, \eta_2 \in \mathfrak{g}$. We want to 
compute
\begin{equation*}
	\xi_1 \xi_2 \star_{zG} \eta_1 \eta_2
	=
	C_0(\xi_1, \xi_2, \eta_1, \eta_2) 
	+ 
	z C_1(\xi_1, \xi_2, \eta_1, \eta_2) 
	+ 
	z^2 C_2(\xi_1, \xi_2, \eta_1, \eta_2) 
	+ 
	z^3 C_3(\xi_1, \xi_2, \eta_1, \eta_2).
\end{equation*}
The very first thing we have to do is computing the set of G-indices. Then we 
calculate the G-factorial and finally go through the permutations.
\begin{itemize}
	\item[$C_0$:]
	We already did this in Corollary \ref{Cor:Formulas:LimitCases} and know, 
	that the zeroth order in $z$ is just the symmetric product. Therefore we 
	have
	\begin{equation*}
		C_0(\xi_1 \xi_2, \eta_1 \eta_2)
		=
		\xi_1 \xi_2 \eta_1 \eta_2
	\end{equation*}
	
	\item[$C_1$:]
	We also did this one in Corollary \ref{Cor:Formulas:LimitCases}: There is 
	just one G-index and we finally get
	\begin{equation*}
		C_1(\xi_1 \xi_2, \eta_1 \eta_2)
		=
		\frac{1}{2} \left(
			\xi_2 \eta_2 [\xi_1, \eta_1] +
			\xi_2 \eta_1 [\xi_1, \eta_2] +
			\xi_1 \eta_2 [\xi_2, \eta_1] +
			\xi_1 \eta_1 [\xi_2, \eta_2]
		\right).
	\end{equation*}
	
	\item[$C_2$:]
	This is the first time, something interesting happens. We have three 
	G-indices:
	\begin{equation*}
		\mathcal{G}_2(2,2) 
		=
		\left\{ J^1, J^2, J^3 \right\}
		= 
		\big\{ 
			\big((0,1), (2,1)\big), 
			\big((1,0), (1,2)\big), 
			\big((1,1), (1,1)\big) 
		\big\}.
	\end{equation*}
	The G-factorials give $J^1! = J^2! = 1$ and $J^3! = 2$, since the index 
	$(1,1)$ appears twice in $J_3$. We take $\bchparts{1}{2}{X}{Y}$ and 
	$\bchparts{2}{1}{X}{Y}$ for two variables $X$ and $Y$ from Equation 
	\eqref{Alg:BCHSeriesLong}:
	\begin{equation*}
		\bchparts{1}{2}{X}{Y}
		=
		\frac{1}{12}[Y, [Y, X]]
		\quad \text{ and } \quad
		\bchparts{2}{1}{X}{Y}
		=
		\frac{1}{12}[X, [X, Y]].
	\end{equation*}
	So we have to insert the $\xi_i$ and the $\eta_j$ into $\frac{1}{12} 
	X [Y, [Y, X]]$ and $\frac{1}{12} Y [X, [X, Y]]$ respectively and then 
	we go on with the last one, which is
	\begin{equation*}
		\frac{1}{2}
		\bchparts{1}{1}{X}{Y}
		\bchparts{1}{1}{X}{Y}
		=
		\frac{1}{8}
		[X, Y][X, Y].
	\end{equation*}
	We hence get
	\begin{align*}
		C_2(\xi_1, \xi_2, \eta_1, \eta_2) 
		& = 
		\frac{1}{12}
		\big( 
			\eta_1 [[\eta_2, \xi_1],\xi_2] + 
			\eta_1 [[\eta_2, \xi_2],\xi_1] + 
			\eta_2 [[\eta_1, \xi_1],\xi_2] + 
			\eta_2 [[\eta_1, \xi_2],\xi_1] +
		\\
		& \quad
			\xi_1 [[\xi_2, \eta_1],\eta_2] + 
			\xi_1 [[\xi_2, \eta_2],\eta_1] + 
			\xi_2 [[\xi_1, \eta_1],\eta_2] + 
			\xi_2 [[\xi_1, \eta_2],\eta_1] 
 		\big) +
 		\\
 		& \quad
		\frac{1}{4} 
		\big( 
			[\xi_1,\eta_1][\xi_2,\eta_2] + 
			[\xi_1,\eta_2][\xi_2,\eta_1] 
		\big)
	\end{align*}
	
	\item[$C_3$:]
	Here, we only have one G-index:
	\begin{equation*}
		\mathcal{G}_1(2,2) 
		=
		\big\{ 
			\big( (2,2) \big) 
		\big\}
	\end{equation*}
	The G-factorial is $1$. We take again Equation \eqref{Alg:BCHSeriesLong} 
	and see
	\begin{equation*}
		\bchparts{2}{2}{X}{Y}
		=
		\frac{1}{24}
		[Y, [X, [Y, X]]].
	\end{equation*}
	This gives
	\begin{align*}
		C_3(\xi_1, \xi_2, \eta_1, \eta_2) 
		& = 
		\frac{1}{24}
		\big( 
			[[[\eta_1,\xi_1],\xi_2],\eta_2] + 
			[[[\eta_1,\xi_2],\xi_1],\eta_2] +
			[[[\eta_2,\xi_1],\xi_2],\eta_1] + 
			[[[\eta_2,\xi_2],\xi_1],\eta_1] 
		\big)
	\end{align*}
\end{itemize}
We just have to put all the four terms together and have the star product.


\subsubsection*{A cubic and a quadratic term}
Let $\xi_1, \xi_2, \xi_3, \eta_1, \eta_2 \in \mathfrak{g}$. We compute
\begin{equation*}
	\xi_1 \xi_2 \xi_3 \star_G \eta_1 \eta_2
	= 
	\sum\limits_{n = 0}^4
	z^n C_n(\xi_1, \xi_2, \xi_3, \eta_1, \eta_2)
\end{equation*}
\begin{itemize}
	\item[$C_0$:]
	The first part is again just the symmetric product:
	\begin{equation*}
		C_0(\xi_1 \xi_2 \xi_3, \eta_1 \eta_2)
		=
		\xi_1 \xi_2 \xi_3 \eta_1 \eta_2.
	\end{equation*}
	
	\item[$C_1$:]
	Here we have again the term from Corollary \ref{Cor:Formulas:LimitCases}:
	\begin{align*}
		C_1(\xi_1 \xi_2 \xi_3, \eta_1 \eta_2)
		& =
		\frac{1}{2} 
		\big( 
			\xi_2 \xi_3 \eta_2 [\xi_1, \eta_1] + 
			\xi_2 \xi_3 \eta_1 [\xi_1, \eta_2] + 
			\xi_1 \xi_3 \eta_2 [\xi_2, \eta_1] +
		\\
		& \quad 
			\xi_1 \xi_3 \eta_1 [\xi_2, \eta_2] + 
			\xi_1 \xi_2 \eta_2 [\xi_3, \eta_1] + 
			\xi_1 \xi_2 \eta_1 [\xi_3, \eta_2] 
 		\big)
	\end{align*}

	\item[$C_2$:]
	Here the calculation is very similar to the one of $C_2$ in the example 
	before. We have three G-indices:
	\begin{equation*}
		\mathcal{G}_3(3,2) 
		=
		\left\{
			J^1, J^2, J^3
		\right\}
		= 
		\big\{ 
			\big( (0,1), (1,0), (2,1) \big), 
			\big( (1,0), (1,0), (1,2) \big), 
			\big( (1,0), (1,1), (1,1) \big) 
		\big\}.
	\end{equation*}
	The G-factorials are now $J^1! = 1$ and $J^2! = J^3! = 2$. Again, we take 
	the BCH terms from Equation \eqref{Alg:BCHSeriesLong} and see, that we 
	must insert the $\xi_i$ and the $\eta_j$ into
	\begin{equation*}
		\frac{1}{12} X Y [X, [X, Y]] +
		\frac{1}{24} X X [Y, [Y, X]] +
		\frac{1}{8} X [X, Y] [X, Y].
	\end{equation*}
	Now we go through all the possible permutations and get
	\begin{align*}
		C_2(\xi_1 \xi_2 \xi_3, \eta_1 \eta_2) 
		& = 
		\frac{1}{12} 
		\big( 
			\xi_1 \xi_2 [[\xi_3, \eta_1], \eta_2] + 
			\xi_1 \xi_2 [[\xi_3, \eta_2], \eta_1] + 
			\xi_1 \xi_3 [[\xi_2, \eta_1], \eta_2] +
		\\
		& \quad 
			\xi_1 \xi_3 [[\xi_2, \eta_2], \eta_1] + 
			\xi_2 \xi_3 [[\xi_1, \eta_1], \eta_2] + 
			\xi_2 \xi_3 [[\xi_1, \eta_2], \eta_1] 
		\big) +
		\\ 
		& \quad
		\frac{1}{12} 
		\big( 
			\xi_1 \eta_1 [[\eta_2, \xi_2], \xi_3] + 
			\xi_1 \eta_2 [[\eta_1, \xi_2], \xi_3] + 
			\xi_1 \eta_1 [[\eta_2, \xi_3], \xi_2] +
		\\
		& \quad
			\xi_1 \eta_2 [[\eta_1, \xi_3], \xi_2] + 
			\xi_2 \eta_1 [[\eta_2, \xi_1], \xi_3] + 
			\xi_2 \eta_2 [[\eta_1, \xi_1], \xi_3] +
		\\
		& \quad
			\xi_2 \eta_1 [[\eta_2, \xi_3], \xi_1] + 
			\xi_2 \eta_2 [[\eta_1, \xi_3], \xi_1] + 
			\xi_3 \eta_1 [[\eta_2, \xi_2], \xi_1] +
		\\
		& \quad
			\xi_3 \eta_2 [[\eta_1, \xi_2], \xi_1] + 
			\xi_3 \eta_1 [[\eta_2, \xi_1], \xi_2] + 
			\xi_3 \eta_2 [[\eta_1, \xi_1], \xi_2] 
		\big) +
		\\
		& \quad
		\frac{1}{4} 
		\big( 
			\xi_1 [\xi_2, \eta_1] [\xi_3, \eta_2] + 
			\xi_1 [\xi_3, \eta_1] [\xi_2, \eta_2] + 
			\xi_2 [\xi_1, \eta_1] [\xi_3, \eta_2] + 
		\\
		& \quad
			\xi_2 [\xi_3, \eta_1] [\xi_1, \eta_2] + 
			\xi_3 [\xi_1, \eta_1] [\xi_2, \eta_2] + 
			\xi_3 [\xi_2, \eta_1] [\xi_1, \eta_2] 
 		\big).
	\end{align*}

	\item[$C_3$:]
	We first calculate the G-indices:
	\begin{equation*}
		\mathcal{G}_2(3,2) 
		= 
		\left\{
			J^1, J^2, J^3
		\right\}
		=
		\left\{ 
			\big( (0,1), (3,1) \big), 
			\big( (1,0), (2,2) \big), 
			\big( (1,1), (2,1) \big) 
		\right\}.
	\end{equation*}
	We don't have to care about $J^1$, since $\bchparts{3}{1}{X}{Y} = 0$. The 
	G-factorials for the other two indices are $1$. The BCH terms have been 
	computed before. So we have to fill in the expression
	\begin{equation*}
		\frac{1}{24} X [Y, [X, [Y, X]]] +
		\frac{1}{2 \cdot 12} [X, Y] [X, [X, Y]].
	\end{equation*}
	Doing the permutations, we get
	\begin{align*}
		C_3(\xi_1 \xi_2 \xi_3, \eta_1 \eta_2)
		& =
		\frac{1}{24}
		\big( 
			\xi_1[[[\eta_1, \xi_2], \xi_3], \eta_2] + 
			\xi_1[[[\eta_2, \xi_2], \xi_3], \eta_1] + 
			\xi_1[[[\eta_1, \xi_3], \xi_2], \eta_2] +
		\\
		& \quad
			\xi_1[[[\eta_2, \xi_3], \xi_2], \eta_1] + 
			\xi_2[[[\eta_1, \xi_1], \xi_3], \eta_2] + 
			\xi_2[[[\eta_2, \xi_1], \xi_3], \eta_1] +
		\\
		& \quad
			\xi_2[[[\eta_1, \xi_3], \xi_1], \eta_2] + 
			\xi_2[[[\eta_2, \xi_3], \xi_1], \eta_1] + 
			\xi_3[[[\eta_1, \xi_2], \xi_1], \eta_2] +
		\\
		& \quad
			\xi_3[[[\eta_2, \xi_2], \xi_1], \eta_1] + 
			\xi_3[[[\eta_1, \xi_1], \xi_2], \eta_2] + 
			\xi_3[[[\eta_2, \xi_1], \xi_2], \eta_1]
		\big) +
		\\
		& \quad
		\frac{1}{24}
		\big( 
			[\xi_1, \eta_1][[\eta_2, \xi_2], \xi_3] + 
			[\xi_1, \eta_2][[\eta_1, \xi_2], \xi_3] + 
			[\xi_1, \eta_1][[\eta_2, \xi_3], \xi_2] +
		\\
		& \quad
			[\xi_1, \eta_2][[\eta_1, \xi_3], \xi_2] + 
			[\xi_2, \eta_1][[\eta_2, \xi_1], \xi_3] + 
			[\xi_2, \eta_2][[\eta_1, \xi_1], \xi_3] +
		\\
		& \quad
			[\xi_2, \eta_1][[\eta_2, \xi_3], \xi_1] + 
			[\xi_2, \eta_2][[\eta_1, \xi_3], \xi_1] + 
			[\xi_3, \eta_1][[\eta_2, \xi_2], \xi_1] +
		\\
		& \quad
			[\xi_3, \eta_2][[\eta_1, \xi_2], \xi_1] + 
			[\xi_3, \eta_1][[\eta_2, \xi_1], \xi_2] + 
			[\xi_3, \eta_2][[\eta_1, \xi_1], \xi_2] 
		\big).
	\end{align*}

	\item[$C_4$:]
	Now there's only $C_4$ left. We only have one G-index:
	\begin{equation*}
		\mathcal{G}_1(3,2) 
		= 
		\left\{ \big( (3,2) \big) \right\},
	\end{equation*}
	but there are more terms which belong to it. We have to go through
	\begin{equation*}
		\bchparts{3}{2}{X}{Y}
		=
		\frac{1}{120} [[[[X,Y],X],Y],X] +
		\frac{1}{360} [[[[Y,X],X],X],Y].
	\end{equation*}
	So we permute and get
	\begin{align*}
		C_4(\xi_1 \xi_2 \xi_3, \eta_1 \eta_2)
		& =
		\frac{1}{120} 
		\big( 
			[[[[\xi_1, \eta_1], \xi_2], \eta_2], \xi_3] + 
			[[[[\xi_1, \eta_2], \xi_2], \eta_1], \xi_3] + 
			[[[[\xi_1, \eta_1], \xi_3], \eta_2], \xi_2] +
		\\
		& \quad
			[[[[\xi_1, \eta_2], \xi_3], \eta_1], \xi_2] + 
			[[[[\xi_2, \eta_1], \xi_1], \eta_2], \xi_3] + 
			[[[[\xi_2, \eta_2], \xi_1], \eta_1], \xi_3] +
		\\
		& \quad
			[[[[\xi_2, \eta_1], \xi_3], \eta_2], \xi_1] + 
			[[[[\xi_2, \eta_2], \xi_3], \eta_1], \xi_1] + 
			[[[[\xi_3, \eta_1], \xi_2], \eta_2], \xi_1] +
		\\
		& \quad
			[[[[\xi_3, \eta_2], \xi_2], \eta_1], \xi_1] + 
			[[[[\xi_3, \eta_1], \xi_1], \eta_2], \xi_2] + 
			[[[[\xi_3, \eta_2], \xi_1], \eta_1], \xi_2] 
		\big) +
		\\
		& \quad
		\frac{1}{360} 
		\big( 
			[[[[\eta_1, \xi_1], \xi_2], \xi_3], \eta_2] + 
			[[[[\eta_2, \xi_1], \xi_2], \xi_3], \eta_1] + 
			[[[[\eta_1, \xi_1], \xi_3], \xi_2], \eta_2] +
		\\
		& \quad
			[[[[\eta_2, \xi_1], \xi_3], \xi_2], \eta_1] + 
			[[[[\eta_1, \xi_2], \xi_1], \xi_3], \eta_2] + 
			[[[[\eta_2, \xi_2], \xi_1], \xi_3], \eta_1] +
		\\
		& \quad
			[[[[\eta_1, \xi_2], \xi_3], \xi_1], \eta_2] + 
			[[[[\eta_2, \xi_2], \xi_3], \xi_1], \eta_1] + 
			[[[[\eta_1, \xi_3], \xi_2], \xi_1], \eta_2] +
		\\
		& \quad
			[[[[\eta_2, \xi_3], \xi_2], \xi_1], \eta_1] + 
			[[[[\eta_1, \xi_3], \xi_1], \xi_2], \eta_2] + 
			[[[[\eta_2, \xi_3], \xi_1], \xi_2], \eta_1] 
		\big).
	\end{align*}
\end{itemize}
Now we only have to add up all those terms and we have finally computed the 
star product.




\section{Low-Verifications of the formulas}
\label{sec:chap4_Mathematica}

\subsection{First verifications with Mathematica}

\subsection{Ideas for an algorithm beyond}
