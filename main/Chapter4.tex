
%
% Chapter 4 of my master thesis:
% The formulas For the Gutt star product
%

\chapter{Formulas for the Gutt star product}

We have seen some results on the Baker-Campbell-Hausdorff series and one 
on an identity for the Gutt star product. In Theorem 
\ref{Thm:Alg:GuttIsBCH} we found a very useful tool to get explicit 
formulas for the Gutt star product. We moreover may not forget that we 
still have to prove one part of it. We will do at the beginning of the 
first section of this chapter. This will then lead to a first easy 
formula for $\star_{zG}$. Afterwards, we will use the same procedure to 
find two more formulas for it: a rather involved one for the $n$-fold 
star product of vectors, which will not necessarily be helpful for 
algebraic computations, but will turn out very useful for estimates, and 
a more explicit one for the product of two monomials.

From those formulas, we will be able to draw some easy, but nice 
consequences in the next section and we will show how to compute the 
Gutt star product explicitly by doing two easy examples.

At the end of this chapter, we will give an easy Mathematica code, which 
can be used to verify the correctness of our formulas for low orders.



\section{Formulas for the Gutt star product}
\label{sec:chap4_Formulas}


%
% An Iterative Approach from Linear Terms
%

\subsection{An Iterative Approach from Linear Terms}

The easiest case for which we will to develop a formula is surely the 
following one: For a given Lie algebra $\lie{g}$ and $\xi, \eta \in 
\lie{g}$ we would like to compute
\begin{equation*}
    \xi^k \star_{zG} \eta
    =
    \sum\limits_{n=0}^k
    z^n C_n(\xi^k, \eta)
\end{equation*}
We have already done this already for the Gutt star product, now we 
want to do the same for the BCH star product. This will finish
the proof of Theorem \ref{Thm:Alg:GuttIsBCH}. For this purpose, we will
use
\begin{equation*}
    \xi^k
    =
    \frac{\partial^k}{\partial t^k}
    \At{t = 0} \exp(t \xi).
\end{equation*}
We now have all the ingredients to prove the following proposition:
\begin{lemma}
    \label{Lemma:Formulas:LinearMonomial1}
    Let $\lie{g}$ be a Lie algebra and $\xi, \eta \in \lie{g}$. We
    have the following identity for the BCH star product $\star_{zG}$
    \begin{equation}
        \label{Formulas:LinearMonomial1}
        \xi^k \star_{zH} \eta
        =
        \sum\limits_{j=0}^k
        \binom{k}{j} z^j B_j^*
        \xi^{k-j}(\ad_{\xi})^j (\eta).
    \end{equation}
\end{lemma}
\begin{proof}
    We start from the simplified form for the Baker-Campbell-Hausdorff 
    series from Equation \eqref{Alg:BCH1stOrder}. Putting things 
    together we get
    \begin{align*}
        \xi^k \star_{zH} \eta
        & =
        \frac{\partial^k}{\partial t^k}
        \frac{\partial}{\partial s}
        \At{t=0, s=0}
        \exp \left(
            \frac{1}{z} \bch{z t \xi}{z s \eta}
        \right)
        \\
        & =
        \frac{\partial^k}{\partial t^k}
        \frac{\partial}{\partial s}
        \At{t=0, s=0}
        \exp \left(
            t \xi + \sum\limits_{j=0}^{\infty}
            z^j \frac{B_j^*}{j!}
            \left( \ad_{t \xi} \right)^j
            (s \eta)
        \right).
    \end{align*}
    From this, we see that only terms which have exactly $k$ of the
    $\xi$'s in them and which are linear in $\eta$ will
    contribute. This means we can cut off the sum at $j = k$. If we
    now write out the exponential series, we can cut it, too, for the
    same reason. We have
    \begin{align*}
        \xi^k \star_{zH} \eta
        & =
        \frac{\partial^k}{\partial t^k}
        \frac{\partial}{\partial s}
        \At{t=0, s=0}
        \sum\limits_{n=0}^{k}
        \frac{1}{n!}
        \left(
            t \xi
            +
            \sum\limits_{j=0}^{k}
            (zt)^j \frac{B_j^*}{j!}
            \left(\ad_{\xi}\right)^j
            (s \eta)
        \right)^n
        \\
        & =
        \frac{\partial^k}{\partial t^k}
        \frac{\partial}{\partial s}
        \At{t=0, s=0}
        \sum\limits_{n=0}^{k}
        \frac{1}{n!}
        \sum\limits_{m = 0}^n
        \binom{n}{m}
        (t \xi)^{n - m}
        \left(
            \sum\limits_{j=0}^{k}
            (zt)^j \frac{B_j^*}{j!}
            \left(\ad_{\xi}\right)^j
            (s \eta)
        \right)^m
        \\
        & =
        \frac{\partial^k}{\partial t^k}
        \frac{\partial}{\partial s}
        \At{t=0, s=0}
        \left(
            \sum\limits_{n=0}^{k}
            \frac{1}{n!}
            (t \xi)^n
            +
            \sum\limits_{n=0}^{k}
            \sum\limits_{j=0}^k
            \frac{1}{(n - 1)!} t^{n + j - 1}
            z^j \frac{B_j^*}{j!}
            \xi^{n - 1}
            \left( \ad_{\xi} \right)^j
            (s \eta)
        \right).
    \end{align*}
    In the last step we just cut off the sum over $m$ since the terms
    for $m >1$ will vanish because of the differentiation with respect
    to $s$. We can finally differentiate to get the formula
    \begin{align*}
        \xi^k \star_{zH} \eta
        & =
        \sum\limits_{n=0}^k
        \sum\limits_{j=0}^k
        \delta_{k, n + j - 1}
        \frac{k!}{j! (n - 1)!}
        z^j B_j^*
        \xi^{n - 1}
        \left(\ad_{\xi}\right)^j
        (\eta)
        \\
        & =
        \sum\limits_{j=0}^k
        \binom{k}{j}
        z^j B_j^*
        \xi^{k - j}
        \left( \ad_{\xi} \right)^j
        (\eta),
    \end{align*}
    which is the wanted result.
\end{proof}
\begin{remark}
    We have now proven the equality of the two star products 
    $\star_{zG}$ and $\star_{zH}$, by deriving an easy formula from both 
    of them. From now on, we will derive all the other formulas from $
    \star_{zH}$, since this is the one which is easier to compute.
\end{remark}
Once this is done, it is actually easy to get the formula for
monomials of the form $\xi_1 \ldots \xi_k$ with $\eta \in \lie{g}$:
\begin{proposition}
	\label{Prop:Formulas:LinearMonomial2}
    Let $\lie{g}$ be a Lie algebra and $\xi_1, \ldots, \xi_k, \eta \in 
    \lie{g}$. We have
    \begin{equation}\label{Formulas:LinearMonomial2}
    	\xi_1 \ldots \xi_k \star_{zG} \eta
    	=
    	\sum\limits_{j=0}^k
    	\frac{1}{k!} \binom{k}{j}
    	z^j B_j^*
    	\sum\limits_{\sigma \in S_k}
    	[\xi_{\sigma(1)}, 
    		[ \ldots [\xi_{\sigma(j)}, \eta] \ldots ]
    	]
    	\xi_{\sigma(j+1)} \ldots \xi_{\sigma(k)}.
    \end{equation}
\end{proposition}
\begin{proof}
	We get the result by just polarizing the formula from Lemma 
	\ref{Thm:Formulas:LinearMonomial1}. Let $\xi_1, \ldots, \xi_k \in 
	\lie{g}$ be given, then we introduce the parameters $t_i$ for $i = 
	1, \ldots, k$ and set
	\begin{equation*}
		\Xi
		=		
		\Xi(t_1, \ldots, t_k)
		=
		\sum\limits_{i=1}^k t_i \xi^k.
	\end{equation*}
	Then it is immediate to see that
	\begin{equation*}
		\xi_1 \ldots \xi_k
		=
		\frac{1}{k!}
		\frac{\partial^k}{\partial t_1 \ldots \partial t_k}
		\At{t_1, \ldots, t_k = 0}
		\Xi^k
	\end{equation*}
	since for every $i = 1, \ldots, k$ we have
	\begin{equation}\label{Formulas:LittleHelp1}
		\frac{\partial}{\partial t_i}
		\At{t_i = 0} \Xi
		=
		\xi_i.
	\end{equation}
	We also find for every $\eta \in \lie{g}$
	\begin{equation}\label{Formulas:LittleHelp2}
		\frac{\partial}{\partial t_i}
		\At{t_i = 0} \ad_{\Xi}(\eta)
		=
		\ad_{\xi_i}(\eta).
	\end{equation}
	Now we just need to calculate $\Xi^k \star_{zG} \eta$ and 
	differentiate with respect to the $t_i$. In order to do this 
	properly, we define
	\begin{equation*}
		\gamma_n^k
		(\xi_1, \ldots, \xi_k; \eta)
		=
		z^n \binom{k}{n} B_n^*
		\left( 
			\ad_{\xi_1} 
			\circ \ldots \circ 
			\ad_{\xi_n}
		\right)
		(\eta)
		\xi_{n+1} \ldots \xi_k
	\end{equation*}
	and
	\begin{equation*}
		\gamma^k
		(\xi_1, \ldots, \xi_k; \eta)
		=
		\sum\limits_{n = 0}^k
		\gamma_n^k
		(\xi_1, \ldots, \xi_k; \eta).
	\end{equation*}
	We see that
	\begin{equation*}
		\Xi^k \star_{zG} \eta
		=
		\gamma^k
		(\Xi, \ldots, \Xi; \eta)
	\end{equation*}
	and can now differentiate this expression, which is linear in the 
	every argument, with respect to the $t_i$. From the Equations 
	\eqref{Formulas:LittleHelp1} and \eqref{Formulas:LittleHelp2} we get 
	with the Leibniz rule
	\begin{equation*}
		\frac{partial}{\partial t_1}
		\gamma^k
		(\Xi, \ldots, \Xi; \eta)
		=
		\sum\limits_{j = 1}^k
		\gamma^k
		(
			\underbrace{\Xi, \ldots, \Xi}_{
				j-1 \text{ times}
			}
			, \xi_1, 
			\underbrace{\Xi, \ldots, \Xi}_{
				k-j-1 \text{ times}
			}
			; \eta
		)
	\end{equation*}
	Differentiating now with respect to $t_2$, we get a second sum, 
	where $\xi_2$ will be put once in every ''free'' position, and so 
	on. One by one, all the slots will be taken by $\xi_i$'s. We just 
	need to divide by $k!$, and we finally find the formula from 
	Equation \eqref{Formulas:LinearMonomial2}.
\end{proof}



% A first general Formula
%
\subsection{A first general Formula}

Proposition \ref{Prop:Formulas:LinearMonomial2} allows us basically 
to get a formula for the case of $\xi_1, \ldots, \xi_k \in \lie{g}$
\begin{equation*}
	\xi_1 \star_{zG} \ldots \star_{zG} \xi_k
	=
	\sum\limits_{j=0}^k
	C_{z,j} \left(
		\xi_1, \ldots, \xi_k
	\right)
\end{equation*}
which we will need to prove the continuity of the coproduct, but which 
can also help to prove the continuity of the product in a different way.

Unluckily, this approach has a problem: iterating this formula, we get 
strangely nested Lie brackets, which would be very difficult to bring 
into a nice form with Jacobi and higher identities. So this is not a 
good way to find an handy formula for the usual star product of two 
monomials. Nevertheless, we want to pursue it for a moment, since we 
will get an equality which will be, although rather unfriendly looking, 
very useful in the following: for analytic observations, it will be 
enough to put (even brutal) estimates on it and the exact nature of the 
combinatorics in the formula is just not important. Hence we rewrite 
Equation \eqref{Formulas:LinearMonomial2} in order to cook up such a 
formula.
\begin{definition}
	\label{Def:Formulas:BMaps}
	Let $j, k \in \mathbb{N}_0$, $j \leq k$ and $B_j^*$ as usual, then 
	we define bilinear maps via
	\begin{equation*}
		\begin{array}{rcl}
			B_z^{k, j} 
			\colon
			\Sym^k(\lie{g}) \times \lie{g}
			&
			\longrightarrow
			&
			\Sym^{k - j + 1}(\lie{g})
			\\
			(\xi_1 \ldots \xi_k, \eta)
			&
			\longmapsto
			&
			\frac{1}{k!}
			\sum\limits_{\sigma \in S_k}
			\binom{k}{j} B_j^* z^j
			[\xi_{\sigma(1)}, [ \ldots, [\xi_{\sigma(j)}, \eta]]] 
			\xi_{\sigma(j + 1)} \ldots \xi_{\sigma(k)}		
		\end{array}
	\end{equation*}
	and
	\begin{equation*}
		B_z^j
		\colon
		\Sym^{\bullet}(\lie{g}) \times \lie{g}
		\longrightarrow
		\Sym^{\bullet}(\lie{g})
		, \quad
		B_z^j 
		= 
		\sum\limits_{k=0}^{\infty}
		B_z^{k, j}
	\end{equation*}
	where we set $B_z^j(x) = 0$ if $\deg(x) < j$ .
\end{definition}
We immediately get an easier identity for Equation 
\eqref{Formulas:LinearMonomial2}. More than that: We can extend it to 
arbirtry symmetric tensors:
\begin{lemma}
	\label{Lemma:LinearMonimial3}
	Let $\lie{g}$ be a Lie-algebra an $x \in \Sym^{\bullet}(\lie{g})$. 
	Then we have the formula
	\begin{equation}
		\label{Formulas:LinearMonomial3}
		x \star_{zG} \eta
    	=
    	\sum\limits_{j=0}^{\infty}
    	B_z^j(x, \eta).
	\end{equation}
\end{lemma}
\begin{proof}
	First it is clear that the sum over $j$ in Equation 
	\eqref{Formulas:LinearMonomial3} is actually finite, since for $j 
	> \deg(x)$ there is no further contribution. Using the grading we 
	can write
	\begin{equation*}
		x 
		= 
		\sum\limits_{k=0}^{\deg(x)}
		\sum_i x_i^{(k)}.
	\end{equation*}
	The $B_z^i$-maps are linear in the first argument and the 
	$x_i^{(k)}$ can be chosen to be factorizing tensors. But on 
	factorizing tensors, Equation \eqref{Formulas:LinearMonomial3} is 
	clearly true. We hence have by the linearity of $\star_{zG}$
	\begin{align*}
		x \star_{zG} \eta
		& =
		\sum\limits_{k=0}^{\deg(x)}
		\sum_i 
		x_i^{(k)}
		\star_{zG} \eta
		\\
		& =
		\sum\limits_{k=0}^{\deg(x)}
		\sum_i 
		\sum\limits_{j=0}^{\infty}
    	B_z^j(x_i^{(k)}, \eta)
		\\
		& =
		\sum\limits_{j=0}^{\infty}
    	B_z^j
    	\left(
			\sum\limits_{k=0}^{\deg(x)}
			\sum_i 
			x_i^{(k)}, \eta
		\right)
		\\
		& =
    	\sum\limits_{j=0}^{\infty}
    	B_z^j(x, \eta).
	\end{align*}
\end{proof}
Now we want to use this approach to go on:
\begin{proposition}
	\label{Prop:Formulas:MultipleStars}
	Let $\lie{g}$, $2 \leq k \in \mathbb{N}$  and $\xi_1, \ldots, \xi_k 
	\in \lie{g}$. Then we have
	\begin{equation*}
		\xi_1 \star_{zG} \ldots \star_{zG} \xi_k		
		=
		\sum\limits_{\substack{
			1 \leq j \leq k-1 \\
			i_j \in \{0, \ldots, j\}
		}}
		B_z^{i_{k-1}}
		\left(
			\ldots B_z^{i_2}
			\left(
				B_z^{i_1} 
				\left( \xi_1, \xi_2 \right)
				, \xi_3	
			\right) 
			\ldots, \xi_{k}
		\right).
	\end{equation*}
\end{proposition}
\begin{proof}
	We will prove this by induction over $k$. For $k = 2$ we get
	\begin{equation*}
		\xi_1 \star_{zG} \xi_2
		=
		B_z^0 \left( \xi_1, \xi_2 \right)
		+
		B_z^1 \left( \xi_1, \xi_2 \right)
		=
		\xi_1 \xi_2 +
		\frac{1}{2} [\xi_1, \xi_2].
	\end{equation*}
	For the step $k \rightarrow k + 1$ we can directly apply Equation 
	\eqref{Formulas:LinearMonomial3}:
	\begin{align*}
		\xi_1 \star_{zG} \ldots \star_{zG} \xi_{k+1}
		& =
		\Bigg(
			\sum\limits_{\substack{
				1 \leq j \leq k-1 \\
				i_j \in \{0, \ldots, j\}
			}}
			B_z^{i_{k-1}}
			\left(
				\ldots B_z^{i_2}
				\left(
					B_z^{i_1} 
					\left( \xi_1, \xi_2 \right)
					, \xi_3	
				\right) 
				\ldots, \xi_{k}
			\right)
		\Bigg)
		\star_{zG} \xi_{k+1}
		\\
		& = 
		\sum\limits_{i_k = 0}^k
		B_z^{i_k}
		\Bigg(
			\sum\limits_{\substack{
				1 \leq j \leq k-1 \\
				i_j \in \{0, \ldots, j\} \\
			}}
			B_z^{i_{k-1}}
			\left(
				\ldots B_z^{i_2}
				\left(
					B_z^{i_1} 
					\left( \xi_1, \xi_2 \right)
					, \xi_3	
				\right) 
				\ldots, \xi_{k}
			\right)
			, \xi_{k+1}
		\Bigg)
		\\
		& = 
		\sum\limits_{\substack{
			1 \leq j \leq k \\
			i_j \in \{0, \ldots, j\} \\
		}}
		B_z^{i_k}
		\left(
			B_z^{i_{k-1}}
			\left(
				\ldots B_z^{i_2}
				\left(
					B_z^{i_1} 
					\left( \xi_1, \xi_2 \right)
					, \xi_3	
				\right) 
				\ldots, \xi_{k}
			\right)
			, \xi_{k+1}
		\right)
	\end{align*}
\end{proof}
Theoretically, we could now use this formula to get another one for the 
case of two monomials with $\xi_1, \ldots, \xi_k, \eta_1, \ldots, 
\eta_{\ell} \in \lie{g}$ since
\begin{equation*}
	\xi_1 \ldots \xi_k \star_{zG} \eta_1 \ldots \eta_{\ell}
	=
	\frac{1}{k! \ell!}
	\sum\limits_{\sigma \in S_k}
	\sum\limits_{\tau \in S_{\ell}}
	\xi_{\sigma(1)} \star_{zG} \ldots \star_{zG} \xi_{\sigma(k)}
	\star_{zG}
	\eta_{\tau(1)} \star_{zG} \ldots \star_{zG} \eta_{\tau(\ell)}
\end{equation*}
This equality can easily be gotten from the definition of the map 
$\mathfrak{q}$.



% A Formula for two Monomials
%

\subsection{A Formula for two Monomials}

Now we need a formula for two monomials, which will be useful to prove 
the continuity of $\star_{zG}$. It will not be very explicit, but it will 
good enough to do computations with concrete examples. As a first step, 
we must introduce a bit of notation:
\begin{definition}[G-Index]
	\label{Def:GuttIndex}
	Let $k, \ell, n \in \mathbb{N}$ and $r = k + \ell - n$. 
	Then we call an $r$-tuple $J$
	\begin{equation*}
		J = (J_1, \ldots, J_r) 
		= 
		((a_1, b_1), \ldots, (a_r, b_r)) 
	\end{equation*}
  	a G-index if it fulfils the following properties:
	\begin{enumerate}[(i)]
		\item
		$J_i \in \{0, 1, \ldots, k\} \times \{0, 1, \ldots, \ell\}$
		
  		\item 
		$|J_i| 
		= 
		a_i + b_i \geq 1 
		\quad \forall_{i = 1, \ldots, r}$
		
		\item 
		$\sum\limits_{i=1}^{r} a_i = k$ 
		and 
		$\sum\limits_{i=1}^{r} b_i = \ell$
		
		\item
		The tuple is ordered in the following sense:
		$i>j \Rightarrow |J_i| \geq |J_j| \quad \forall_{i,j = 1, 
		\ldots, r}$ and $|a_i| \geq |a_j|$ if $|J_i| = |J_j|$
		
		\item 
		If $a_i = 0$ [or $b_i = 0$] for some $i$, 
		then $b_i = 1$ [or $a_i = 1$].
	\end{enumerate}
	We call the set of all such $G$-indices	$\mathcal{G}_r(k,\ell)$.
\end{definition}
\begin{definition}[G-Factorial]
	\label{Def:GuttFactorial}
	Let $J = ((a_1, b_1), \ldots, (a_r, b_r)) \in \mathcal{G}_r(k,
	\ell)$ be a G-Index. We set for a given tuple $(a,b) \in \{0, 1, 
	\ldots, k\} \times \{0, 1, \ldots, \ell\}$
	\begin{equation*}
		\#_J (a,b)
		= 
		\textrm{ number of times that $(a,b)$ appears in } J.
	\end{equation*}
	Then we define the G-factorial of $J \in \{0, 1, \ldots, k\} \times 
	\{0, 1, \ldots, \ell\}$ as
	\begin{equation*}
		J!
		= 
		\prod\limits_{
			(a,b) \in 
			\{0, 1, \ldots, k\} 
			\times 
			\{0, 1, \ldots, \ell\}
		}
		\left( \#_J (a,b) \right)!
	\end{equation*}
\end{definition}
This allows us to state an explicit formula for the Gutt star product:
\begin{lemma}
	\label{lemma:2MonomialsFormula1}
	Let $\lie{g}$ be a Lie algebra, $\xi, \eta \in \lie{g}$ and $k, 
	\ell \in \mathbb{N}$. Then we have the following identity for the 
	Gutt star product:
	\begin{equation*}
    	\xi^k \star_{zG} \eta^{\ell}
    	=
	    \sum\limits_{n=0}^{k + \ell -1}
    	z^n
    	C_n \left( \xi^k, \eta^{\ell} \right),
	\end{equation*}
	where the $C_n$ are given by
	\begin{equation}
		\label{Formulas:2MonomialsFormula1}
        C_n \left( \xi^k, \eta^{\ell} \right)
        =
        \sum\limits_{J \in \mathcal{G}_{k + \ell - n}(k, \ell)}
        \frac{k! \ell!}{J!}
        \prod\limits_{i = 1}^{k + \ell - n}        
        \bchparts{a_i}{b_i}{\xi}{\eta}
	\end{equation}
	and the product is taken in the symmetric tensor algebra.
\end{lemma}
\begin{proof}
	We want to calculate what the $C_n$ look like. Let's denote $r = k 
	+ \ell - n$ for brevity. Then we have
    \begin{align}
        \nonumber
        z^n C_n(\xi^k, \eta^{\ell})
        & =
        \frac{\partial^k}{\partial t^k}
        \frac{\partial^{\ell}}{\partial s^{\ell}}
        \At{t,s = 0}
        \frac{1}{z^r}
        \frac{\bch{z t \xi}{z s \eta}^r    }{r!}
        \\
        \nonumber
        & =
        \frac{1}{z^r}
        \frac{1}{r!}
        \frac{\partial^k}{\partial t^k}
        \frac{\partial^{\ell}}{\partial s^{\ell}}
        \At{t,s = 0}
        \left(
            \sum\limits_{j = 1}^{k + \ell}
            \bchpart{j}{z t \xi}{z s \eta}
        \right)^r
        \\
        \nonumber
        & =
        \frac{1}{z^r}
        \frac{1}{r!}
        \frac{\partial^k}{\partial t^k}
        \frac{\partial^{\ell}}{\partial s^{\ell}}
        \At{t,s = 0}
        \sum\limits_{\substack{
        	j_1, \ldots, j_r \geq 1 \\
            j_1 + \ldots + j_r = k + \ell
        }}
        \bchpart{j_1}{z t \xi}{z s \eta} 
        \cdots
        \bchpart{j_r}{z t \xi}{z s \eta}
        \\
        \label{Formulas:2MonomialsExplicit}
        & =
        z^n
        \frac{k! \ell!}{r!}
        \sum\limits_{\substack{a_1, b_1, \ldots, a_r, b_r \geq 0 \\
            a_i + b_i \geq 1 \\
            a_1 + \ldots + a_r = k \\
            b_1 + \ldots + b_r = \ell
        }}
        \bchparts{a_i}{b_i}{\xi}{\eta}
        \cdots
        \bchparts{a_r}{b_r}{\xi}{\eta}
    \end{align}
    We sum over all possible arrangements of the $(a_i, b_i)$. In order 
    to get rid of redundancies, we put an ordering on these multi-
    indices: We use the one from definition \ref{Def:GuttIndex}. 
    Obviously, we loose the freedom of arranging the $(a_i, b_i)$ a we 
    need to count the number of multi-indices $((a_1, b_1), \ldots, 
    (a_r, b_r))$ which belong to the same G-index $J$. This number will 
    be $\frac{r!}{J!}$, since we can't interchange the $(a_i, b_i)$ any 
    more (therefore $r!$), unless they are equal (therefore $J!^{-1}$). 
    Changing the summation to $J \in \mathcal{G}_r(k, \ell)$ and 
    multiplying by $\frac{r!}{J!}$ we get the identity 
    \eqref{Formulas:2MonomialsFormula1}.
\end{proof}
Now we just need to generalize this to factorizing tensors. To do so, we 
need a last definition:
\begin{definition}
	\label{Def:GuttInsert}
	Let $\k, \ell, n \in \mathbb{N}$ and $J \in \mathcal{G}_{k + \ell - 
	n}(k, \ell)$. Then for $\xi_1, \ldots, \xi_k, \eta_1, \ldots, 
	\eta_{\ell}$ from a Lie algebra $\lie{g}$ we set
	\begin{equation}
		\label{Formulas:GuttInsert}
		\Gamma_J(\xi_1, \ldots, \xi_k; \eta_1, \ldots, \eta_{\ell})
		=
		\frac{1}{J!}
		\prod\limits_{i=1}^{k + \ell - n}
		\bchparts{a_i}{b_i}{\xi^{(a_i)}}{\eta^{(b_i)}}
	\end{equation}
	where the notation $\bchparts{a_i}{b_i}{\xi^{(a_i)}}{\eta^{(b_i)}}$ 
	means that we have taken $\prod\limits_{i=1}^{k + \ell - n} 
	\bchparts{a_i}{b_i}{\xi^{(a_i)}}{\eta^{(b_i)}}$ and replaced the 
	$j$-th $\xi$ appearing in it with $\xi_j$ for $j = 1, \ldots, k$ and 
	analogously with the $\eta$'s.
\end{definition}
\begin{proposition}
	\label{proposition:2MonomialsFormula2}
	Let $\lie{g}$ be a Lie algebra, $k, \ell \in \mathbb{N}$ and $\xi_1, 
	\ldots, \xi_k, \eta_1, \ldots, \eta_{\ell} \in \lie{g}$. Then we 
	have the following identity for the Gutt star product:
	\begin{equation*}
    	\xi_1 \ldots \xi_k \star_{zG} \eta_1 \ldots \eta_{\ell}
    	=
	    \sum\limits_{n=0}^{k + \ell -1}
    	z^n C_n
    	\left( 
    		\xi_1 \ldots \xi_k, \eta_1 \ldots \eta_{\ell}
    	\right),
	\end{equation*}
	where the $C_n$ are given by
	\begin{equation}
		\label{Formulas:2MonomialsFormula2}
        C_n
        \left( 
    		\xi_1 \ldots \xi_k, \eta_1 \ldots \eta_{\ell}
    	\right)
        =
        \sum\limits_{J \in \mathcal{G}_{k + \ell - n}(k, \ell)}
        \sum\limits_{\sigma \in S_k}
        \sum\limits_{\tau \in S_{\ell}}
        \Gamma_J 
        \left(
        	\xi_{\sigma(1)}, \ldots, \xi_{\sigma(k)};
        	\eta_{\tau(1)}, \ldots, \eta_{\tau(\ell)}
        \right)
	\end{equation}
	and the product is taken in the symmetric tensor algebra.
\end{proposition}
\begin{proof}
	The proof relies on polarization again and is completely analogous 
	to the one of proposition \ref{Thm:Formulas:LinearMonomial2}. We get 
	rid of the factorials in Equation 
	\eqref{Formulas:2MonomialsFormula1} and get symmetrizations over the 
	$\xi_i$ and the $\eta_j$ instead, which gives the wanted result.
\end{proof}
\begin{remark}
	Equation \eqref{Formulas:2MonomialsFormula2} allows us to do some 
	explicit computations. One can check for example the following 
	things:
	\begin{remarklist}
		\item
		The classical limit holds:
		\begin{equation*}
			C_0(\xi_1 \ldots \xi_k, \eta_1 \ldots \eta_{\ell})
			=
			\xi_1 \ldots \xi_k \eta_1 \ldots \eta_{\ell}
		\end{equation*}
		
		\item
		The semi-classical limit holds:
		\begin{equation*}
			\xi_1 \ldots \xi_k \star_{zG} \eta_1 \ldots \eta_{\ell}
			-
			\eta_1 \ldots \eta_{\ell} \star_{zG} \xi_1 \ldots \xi_k
			=
			z 
			\{ \xi_1 \ldots \xi_k, \eta_1 \ldots \eta_{\ell} \}_{KKS}
		\end{equation*}
		with $\{\cdot, \cdot\}_{KKS}$ being the Kirillov-Konstant-
		Souriau bracket.
		
		\item
		For $\xi_1 \ldots \xi_k \star_{zG} \eta$ we get the smaller 
		formula in Equation \eqref{Formulas:LinearMonomial2} again.
	\end{remarklist}
	We don't want to check this here, since these are easy calculations, 
	but it is interesting to see that the formulas really give 
	the wanted identities for the Gutt star product.
\end{remark}


\section{Consequences and examples}
\label{sec:chap4_Consequences}


\section{Low-Verifications of the formulas}
\label{sec:chap4_Mathematica}

\subsection{First verifications with Mathematica}

\subsection{Ideas for an algorithm beyond}
