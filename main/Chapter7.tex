
%
% Chapter 7 of my master thesis:
% The Hopf algebra structure
%

\chapter{The Hopf algebra structure}

\section{Everything works}

In the previous chapters we discussed a locally convex topology on $\Sym^{\bullet}
(\lie{g})$ or on the universal enveloping algebra with scaled Lie bracket, 
considered as an associative algebra. The Gutt star product (or just the 
multiplication in the case of $\algebra{U}_R(\lie{g}_z)$) was continuous.
In the following, we investigate the continuity of the other maps which belong to 
the Hopf structure on $\algebra{U}_R(\lie{g}_z)$. For this purpose, we need an 
explicit formula for the antipode
\begin{equation}
    \label{eq:Antipode}
    S_z \colon
    \algebra{U}(\lie{g}_z)
    \longrightarrow
    \algebra{U}(\lie{g}_z)
\end{equation}
and the coproduct
\begin{equation}
    \label{eq:CoProduct}
    \coproduct_z \colon
    \algebra{U}(\lie{g}_z)
    \longrightarrow
    \algebra{U}(\lie{g}_z)
    \tensor
    \algebra{U}(\lie{g}_z)
\end{equation}
in the $\algebra{U}_R(\lie{g}_z)$ and in $\Sym^{\bullet}(\lie{g})$. We pull them 
back to the symmetric algebra and extend them to the whole tensor algebra by 
symmetrizing beforehand. We define
\begin{equation}
    \label{eq:AntipodeOnTensor}
    \widetilde{S}_z \colon
    \Tensor^\bullet(\lie{g})
    \longrightarrow
    \Sym^\bullet(\lie{g})
    , \quad
    \widetilde{S}_z
    =
    \mathfrak{q}_z^{-1}
    \circ
    S_z
    \circ
    \mathfrak{q}_z
    \circ
    \Symmetrizer.
\end{equation}
and
\begin{equation}
    \label{eq:CoProductOnTensor}
    \widetilde{\ocoproduct}_z \colon
    \Tensor^\bullet(\lie{g})
    \longrightarrow
    \Sym^\bullet(\lie{g})
    \tensor
    \Sym^\bullet(\lie{g})
    , \quad
    \widetilde{\ocoproduct}_z
    =
    (\mathfrak{q}_z^{-1} \tensor \mathfrak{q}_z^{-1})
    \circ
    \coproduct_z
    \circ
    \mathfrak{q}_z
    \circ
    \Symmetrizer,
\end{equation}
to avoid that the maps on $\Sym^{\bullet}(\lie{g})$ and on $\algebra{U}
(\lie{g}_z)$ are denoted by the same symbols. The next lemma gives us the two 
explicit formulas we need.
\begin{lemma}
    \label{Thm:Hopf:Formulas}%
    For $\xi_1, \ldots, \xi_n \in \lie{g}$ we have the identities
    \begin{equation}
        \label{Hopf:AntipodeFormula}
        S_z(\xi_1 \tensor \cdots \tensor \xi_n)
        =
        (-1)^n
        \xi_1 \cdots \xi_n
    \end{equation}
    and
    \begin{equation}
        \label{Hopf:CoproductFormula}
        \ocoproduct_z(\xi_1 \tensor \cdots \tensor \xi_n)
        =
        \sum\limits_{
        	I \subseteq
        	\{1, \ldots, n\}
        }
        \xi_I
        \tensor
        \xi_1 \cdots
        \widehat{\xi_I}
        \cdots \xi_n
    \end{equation}
    where $\xi_I$ denotes the symmetric tensor product of all $\xi_i$ with 
    $i \in I$ and $\widehat{\xi_I}$ means that the $\xi_i$ with $i \in I$ 
    are left out.
\end{lemma}
\begin{proof}
	First, we derive Formula \ref{Hopf:AntipodeFormula}: the antipode gives
	$S_z(\xi) = - \xi$ for $\xi \in \lie{g}$ and extends to $\algebra{U}
	(\lie{g}_z)$ by algebra antihomomorphism, hence
	\begin{equation*}
		S_z \left( 
			\xi_1 \odot \cdots \odot \xi_n
		\right)
		=
		(-1)^n
		\xi_n \odot \cdots \odot \xi_1
	\end{equation*}
	in $\algebra{U}(\lie{g}_z)$ for $\xi_1, \ldots, \xi_n \in \lie{g}$.
	This means
	\begin{equation*}
		S_z \left( 
			\xi_1 \star_z \cdots \star_z \xi_n
		\right)
		=
		(-1)^n
		\xi_n \star_z \cdots \star_z \xi_1
	\end{equation*}	
	in $\Sym^{\bullet}(\lie{g})$. But now, using the linearity of $S_z$ 
	we get
	\begin{align*}
		S_z \left( \xi_1 \cdots \xi_n \right)
		& =
		S_z \left( 
		\frac{1}{n!}
			\sum\limits_{\sigma \in S_n}
			\xi_{\sigma(1)} 
			\star_z \cdots \star_z 
			\xi_{\sigma(n)}
		\right)
		\\
		& =
		\frac{1}{n!}
		\sum\limits_{\sigma \in S_n}
		S_z \left( 
			\xi_{\sigma(1)} 
			\star_z \cdots \star_z 
			\xi_{\sigma(n)}
		\right)
		\\
		& =
		\frac{1}{n!}
		\sum\limits_{\sigma \in S_n}
		(-1)^n
		\xi_{\sigma(n)} 
		\star_z \cdots \star_z 
		\xi_{\sigma(1)}
		\\
		& =
		(-1)^n
		\xi_1 \cdots \xi_n.
	\end{align*}
	For the coproduct, we have well-known formula with shuffle permutations:
	\begin{equation*}
		\ocoproduct_z \left(
			\xi_1 \odot \cdots \odot \xi_n
		\right)
		=
		\sum\limits_{k=0}^n
		\sum\limits_{\substack{
			1 \leq i_1
			< \ldots <
			i_k \leq n
			\\
			I = \{i_1, \ldots, i_k\}
		}}
		\xi_{i_1} \odot \cdots \odot \xi_{i_k}
		\tensor
		\xi_1 
		\odot \cdots 
		\widehat{\xi_I}
		\cdots \odot
		\xi_n.
	\end{equation*}
	Pulling this back to $\Sym^{\bullet}(\lie{g}_z)$, we get star products
	instead of $\odot$. For symmetric tensor we have by linearity
	\begin{align*}
		\ocoproduct_z \left( \xi_1 \cdots \xi_n \right)
		& =
		\ocoproduct_z \left( 
			\frac{1}{n!}
			\sum\limits_{\sigma \in S_n}
			\xi_{\sigma(1)} 
			\star_z \cdots \star_z 
			\xi_{\sigma(n)}
		\right)
		\\
		& =
		\frac{1}{n!}
		\sum\limits_{\sigma \in S_n}
		\ocoproduct_z \left( 
			\xi_{\sigma(1)} 
			\star_z \cdots \star_z 
			\xi_{\sigma(n)}
		\right)
		\\
		& =
		\frac{1}{n!}
		\sum\limits_{\sigma \in S_n}
		\sum\limits_{k=0}^n
		\sum\limits_{\substack{
			1 \leq i_1
			< \ldots <
			i_k \leq n
			\\
			I = \{i_1, \ldots, i_k\}
		}}
		\xi_{i_{\sigma(1)}} 
		\star_z \cdots \star_z
		\xi_{i_{\sigma(k)}}
		\tensor
		\xi_{\sigma(1)}
		\star_z \cdots
		\widehat{\xi_{\sigma(I)}}
		\cdots \star_z 
		\xi_{\sigma(n)}
		\\
		& =
		\frac{1}{n!}
		\sum\limits_{k=0}^n
		\sum\limits_{\sigma \in S_k}
		\sum\limits_{\tau \in S_{n - k}}
		\sum\limits_{
			\{i_1, \ldots, i_k\}
			\subseteq
			\{1, \ldots, n\}
		}
		\frac{n!}{k! (n - k)!}
		\\
		& \qquad
		\cdot
		\xi_{i_{\sigma(1)}} 
		\star_z \cdots \star_z 
		\xi_{i_{\sigma(k)}}
		\tensor
		\xi_{\tau(1)}
		\star_z \cdots 
		\widehat{\xi_I}
		\cdots \star_z 
		\xi_{\tau(n)}
		\\
		& =
		\sum\limits_{k=0}^n
		\sum\limits_{
			\{i_1, \ldots, i_k\}
			\subseteq
			\{1, \ldots, n\}
		}
		\xi_{i_1} \cdots \xi_{i_k}
		\tensor
		\xi_1 \cdots \widehat{\xi_I} \cdots \xi_n.
	\end{align*}
\end{proof}
\begin{remark}
	From Equations \ref{Hopf:AntipodeFormula} and 
	\ref{Hopf:CoproductFormula}, we see that the maps $S_z$ and 
	$\ocoproduct_z$ do not depend on $z$, since only symmetric tensor products 
	are involved. Hence $S_z = S_0$ and $\ocoproduct_z = \ocoproduct_0$. 
	This means that the product is the only map of the Hopf algebra, which is 
	deformed.
\end{remark}
% AE-Lie Algebras
%


We need a topology on the tensor product in \eqref{Hopf:CoproductFormula}, 
for which we take again the projective tensor product. The continuity of the 
two maps is now easy to prove.
\begin{proposition}
    \label{Prop:Hopf:CoproductContinuity}%
    Let $\lie{g}$ be an AE-Lie algebra and $R \geq 0$. For every continuous 
    seminorm $p$ and all $x \in \widehat{\Tensor}_R^\bullet(\lie{g})$
    the following estimates hold:
    \begin{equation}
        \label{Hopf:AntipodeContinuity}
        p_R \left( S_z(x) \right)
        \leq
        p_R (x)
    \end{equation}
    and
    \begin{equation}
        \label{Hopf:CoproductContinuity}
        (p_R \tensor p_R)
        \left(\ocoproduct_z(x)\right)
        \leq
        (2 p)_R (x).
    \end{equation}
\end{proposition}
\begin{proof}
	We use the extension to the whole tensor algebra. Inequality
	\eqref{Hopf:AntipodeContinuity} is clear on factorizing tensors 
	and extends to all tensors by the infimum argument. 
	To get the estimate~\eqref{Hopf:CoproductContinuity}, 
	we compute it on factorizing tensors:
	\begin{align*}
		(p_R \tensor p_R)
        \left( \ocoproduct_z
        	\left(
        		\xi_1 \tensor \cdots \tensor \xi_n
        	\right)
        \right)
        & =
		(p_R \tensor p_R)
        \left( 
			\sum\limits_{
        		I \subseteq
        		\{1, \ldots, n\}
        	}
        	\xi_I
        	\tensor
        	\xi_1 \cdots
        	\widehat{\xi_I}
        	\cdots \xi_n
        \right)
        \\
        & \leq
        \sum\limits_{
        	I \subseteq
        	\{1, \ldots, n\}
        }
        |I|!^R (n - |I|)!^R
        p^{|I|} \left( \xi_I \right)
        p^{n - |I|}
        \left( 
        	\xi_1 \cdots \widehat{\xi_I} \cdots \xi_n 
        \right)
        \\
        & \leq
        \sum\limits_{
        	I \subseteq
        	\{1, \ldots, n\}
        }
        |I|!^R (n - |I|)!^R
        p(\xi_1) \cdots p(\xi_n)
        \\
        & \leq
        \sum\limits_{
        	I \subseteq
        	\{1, \ldots, n\}
        }
        n!^R
        p(\xi_1) \cdots p(\xi_n)
        \\
        & =
        2^n n!^R
        p(\xi_1) \cdots p(\xi_n)
        \\
        & =
        (2p)_R \left(
        	\xi_1 \tensor \cdots \tensor \xi_n
        \right).
	\end{align*}
	This extends to all tensors by the infimum argument.
\end{proof}

Since the continuity of the unit and the counit is clear by the definition of 
our topology, we have the following result.
\begin{proposition}
    \label{Prop:Hopf:ContinuousHopf}%
    Let $\lie{g}$ be an AE-Lie algebra and $z \in \mathbb{C}$. Then, 
    if $R \geq 1$, $\widehat{\Sym}_R^\bullet (\lie{g})$ is a topological Hopf 
    algebra. The same holds for $\widehat{\Sym}_{1^-}^{\bullet}(\lie{g})$,
    if $\lie{g}$ is a nilpotent locally convex Lie algebra with continuous
    Lie bracket.
\end{proposition}
