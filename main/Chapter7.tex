
%
% Chapter 7 of my master thesis:
% The Hopf algebra structure
%

\chapter{The Hopf Algebra Structure}

As already pointed out in Chapter 3, universal enveloping algebras of a Lie 
algebras are more than just associative, unital algebras: they constitute one of 
the most important types of \emph{Hopf algebras}, which are very common 
structures in mathematics. Hopf algebras are a particular \emph{bialgebras}, 
which in turn are on one hand associative, unital algebras, and on the other hand 
coassociative, counital coalgebras. Those two substructures of bialgebras must of 
course fulfil certain compatibility conditions. Note at this point, that 
coalgebras play a crucial role in the theory of formal deformation quantization, 
which is due to Kontsevich, but which has been a lot further developed since. 
Good references on this so-called formality theory are given by Esposito 
\cite{esposito:2015a}.

In a Hopf algebra however, we have an 
additional map, called the \emph{antipode}, which again must fulfil some 
compatibility relations. So, in brief, a Hopf algebra over a field $\mathbb{K}$ 
is a tuple $(H, \cdot, \eta, \Delta, \varepsilon, S)$ of a vector space $H$ 
together with the following maps
\begin{equation*}
\begin{array}{rlrl}
	\cdot \colon
	&
	H \tensor H 
	\longrightarrow
	H, 
	& \quad &
	\text{ multiplication}
	\\
	\eta \colon
	&
	\mathbb{K}
	\longrightarrow
	H
	, 
	& \quad &
	\text{ unit}
	\\
	\Delta \colon
	&
	H
	\longrightarrow
	H \tensor H
	, & \quad &
	\text{ coproduct}
	\\
	\varepsilon \colon
	&
	H
	\longrightarrow
	\mathbb{K}
	, & \quad &
	\text{ counit}
	\\
	S \colon
	&
	H
	\longrightarrow
	H
	, & \quad &
	\text{ antipode},
\end{array}
\end{equation*}
such that we have a certain set of commuting diagrams. A very nice introduction 
to the theory of Hopf algebras with a lot of examples can e.g. be found in 
\cite{schweigert:2015a:script}.


In the previous chapters, we have mostly studied the properties of the 
multiplication in $\algebra{U}(\lie{g}_z)$, endowed with a particular topology. 
In this last chapter of this thesis, we will treat the remaining Hopf algebra 
structure maps. In the first section, we will see that the comultiplication and 
the antipode are not touched by our deformation procedure. Hence, it will be 
enough to show the continuity of the undeformed versions of those maps, which we 
will do in second section.



\section{An Undeformed Hopf Structure}

To get show the continuity of the remaining structure maps of 
$\algebra{U}_R(\lie{g}_z)$, we have to put estimates on seminorms again. For this 
purpose, we need explicit formulas for the antipode
\begin{equation}
    \label{eq:Antipode}
    S_z \colon
    \algebra{U}(\lie{g}_z)
    \longrightarrow
    \algebra{U}(\lie{g}_z)
\end{equation}
and the coproduct
\begin{equation}
    \label{eq:CoProduct}
    \coproduct_z \colon
    \algebra{U}(\lie{g}_z)
    \longrightarrow
    \algebra{U}(\lie{g}_z)
    \tensor
    \algebra{U}(\lie{g}_z)
\end{equation}
in the $\algebra{U}_R(\lie{g}_z)$ and in $\Sym^{\bullet}(\lie{g})$. We pull them 
back to the symmetric algebra and extend them to the whole tensor algebra by 
symmetrizing beforehand. We define
\begin{equation}
    \label{eq:AntipodeOnTensor}
    \widetilde{S}_z \colon
    \Tensor^\bullet(\lie{g})
    \longrightarrow
    \Sym^\bullet(\lie{g})
    , \quad
    \widetilde{S}_z
    =
    \mathfrak{q}_z^{-1}
    \circ
    S_z
    \circ
    \mathfrak{q}_z
    \circ
    \Symmetrizer
\end{equation}
and
\begin{equation}
    \label{eq:CoProductOnTensor}
    \widetilde{\ocoproduct}_z \colon
    \Tensor^\bullet(\lie{g})
    \longrightarrow
    \Sym^\bullet(\lie{g})
    \tensor
    \Sym^\bullet(\lie{g})
    , \quad
    \widetilde{\ocoproduct}_z
    =
    (\mathfrak{q}_z^{-1} \tensor \mathfrak{q}_z^{-1})
    \circ
    \coproduct_z
    \circ
    \mathfrak{q}_z
    \circ
    \Symmetrizer,
\end{equation}
to avoid that the maps on $\Sym^{\bullet}(\lie{g})$ and on $\algebra{U}
(\lie{g}_z)$ are denoted by the same symbols. The next lemma gives us the two 
explicit formulas we need.
\begin{lemma}
    \label{Thm:Hopf:Formulas}%
    For $\xi_1, \ldots, \xi_n \in \lie{g}$ we have the identities
    \begin{equation}
        \label{Hopf:AntipodeFormula}
        \widetilde{S}_z
        \left( \xi_1 \tensor \cdots \tensor \xi_n \right)
        =
        (-1)^n
        \xi_1 \cdots \xi_n
    \end{equation}
    and
    \begin{equation}
        \label{Hopf:CoproductFormula}
        \widetilde{\ocoproduct}_z
        \left(\xi_1 \tensor \cdots \tensor \xi_n \right)
        =
        \sum\limits_{
        	I \subseteq
        	\{1, \ldots, n\}
        }
        \xi_I
        \tensor
        \xi_1 \cdots
        \widehat{\xi_I}
        \cdots \xi_n
    \end{equation}
    where $\xi_I$ denotes the symmetric tensor product of all $\xi_i$ with 
    $i \in I$ and $\widehat{\xi_I}$ means that the $\xi_i$ with $i \in I$ 
    are left out.
\end{lemma}
\begin{proof}
	First, we derive Formula \ref{Hopf:AntipodeFormula}: the antipode gives
	$S_z(\xi) = - \xi$ for $\xi \in \lie{g}$ and extends to $\algebra{U}
	(\lie{g}_z)$ by algebra antihomomorphism, hence
	\begin{equation*}
		S_z \left( 
			\xi_1 \odot \cdots \odot \xi_n
		\right)
		=
		(-1)^n
		\xi_n \odot \cdots \odot \xi_1
	\end{equation*}
	in $\algebra{U}(\lie{g}_z)$ for $\xi_1, \ldots, \xi_n \in \lie{g}$.
	This means
	\begin{equation*}
		\widetilde{S}_z \left( 
			\xi_1 \star_z \cdots \star_z \xi_n
		\right)
		=
		(-1)^n
		\xi_n \star_z \cdots \star_z \xi_1
	\end{equation*}	
	in $\Sym^{\bullet}(\lie{g})$. But now, using the linearity of 
	$\widetilde{S}_z$ we get
	\begin{align*}
		\widetilde{S}_z \left( \xi_1 \cdots \xi_n \right)
		& =
		\widetilde{S}_z \left( 
		\frac{1}{n!}
			\sum\limits_{\sigma \in S_n}
			\xi_{\sigma(1)} 
			\star_z \cdots \star_z 
			\xi_{\sigma(n)}
		\right)
		\\
		& =
		\frac{1}{n!}
		\sum\limits_{\sigma \in S_n}
		\widetilde{S}_z \left( 
			\xi_{\sigma(1)} 
			\star_z \cdots \star_z 
			\xi_{\sigma(n)}
		\right)
		\\
		& =
		\frac{1}{n!}
		\sum\limits_{\sigma \in S_n}
		(-1)^n
		\xi_{\sigma(n)} 
		\star_z \cdots \star_z 
		\xi_{\sigma(1)}
		\\
		& =
		(-1)^n
		\xi_1 \cdots \xi_n.
	\end{align*}
	For the coproduct, we have well-known formula with shuffle permutations:
	\begin{equation*}
		\ocoproduct_z \left(
			\xi_1 \odot \cdots \odot \xi_n
		\right)
		=
		\sum\limits_{k=0}^n
		\sum\limits_{\substack{
			1 \leq i_1
			< \ldots <
			i_k \leq n
			\\
			I = \{i_1, \ldots, i_k\}
		}}
		\xi_{i_1} \odot \cdots \odot \xi_{i_k}
		\tensor
		\xi_1 
		\odot \cdots 
		\widehat{\xi_I}
		\cdots \odot
		\xi_n.
	\end{equation*}
	We can derive it from the fact that $\ocoproduct_z$ has the following form on 
	Lie algebra elements:
	\begin{equation*}
		\ocoproduct_z (\xi)
		=
		\xi \tensor \Unit
		+
		\Unit \tensor \xi.
	\end{equation*}
	The coproduct extends to $\algebra{U}(\lie{g}_z)$ by algebra 
	homomorphism. This yields
	\begin{align*}
		\ocoproduct_z \left(
			\xi_1 \odot \cdots \odot \xi_n
		\right)
		& =
		\ocoproduct_z \left( \xi_1 \right)
		\odot \cdots \odot
		\ocoproduct_z \left( \xi_n \right)
		\\
		&=
		\left( \xi_1 \tensor \Unit + \Unit \tensor \xi_1 \right)
		\odot \cdots \odot
		\left( \xi_n \tensor \Unit + \Unit \tensor \xi_n \right)
		\\
		&=
		\sum\limits_{k=0}^n
		\sum\limits_{\substack{
			1 \leq i_1
			< \ldots <
			i_k \leq n
			\\
			I = \{i_1, \ldots, i_k\}
		}}
		\xi_{i_1} \odot \cdots \odot \xi_{i_k}
		\tensor
		\xi_1 
		\odot \cdots 
		\widehat{\xi_I}
		\cdots \odot
		\xi_n.
	\end{align*}
	Now we can pull this back to $\Sym^{\bullet}(\lie{g}_z)$. We get a $\star_z$ 
	for every $\odot$. For symmetric tensors we have by linearity
	\begin{align*}
		&
		\widetilde{\ocoproduct}_z 
		\left( \xi_1 \cdots \xi_n \right)
		\\
		& =
		\widetilde{\ocoproduct}_z \left( 
			\frac{1}{n!}
			\sum\limits_{\sigma \in S_n}
			\xi_{\sigma(1)} 
			\star_z \cdots \star_z 
			\xi_{\sigma(n)}
		\right)
		\\
		& =
		\frac{1}{n!}
		\sum\limits_{\sigma \in S_n}
		\widetilde{\ocoproduct}_z \left( 
			\xi_{\sigma(1)} 
			\star_z \cdots \star_z 
			\xi_{\sigma(n)}
		\right)
		\\
		& =
		\frac{1}{n!}
		\sum\limits_{\sigma \in S_n}
		\sum\limits_{k=0}^n
		\sum\limits_{\substack{
			1 \leq i_1
			< \ldots <
			i_k \leq n
			\\
			I = \{i_1, \ldots, i_k\}
		}}
		\xi_{i_{\sigma(1)}} 
		\star_z \cdots \star_z
		\xi_{i_{\sigma(k)}}
		\tensor
		\xi_{\sigma(1)}
		\star_z \cdots
		\widehat{\xi_{\sigma(I)}}
		\cdots \star_z 
		\xi_{\sigma(n)}
		\\
		& =
		\frac{1}{n!}
		\sum\limits_{k=0}^n
		\sum\limits_{\sigma \in S_k}
		\sum\limits_{\tau \in S_{n - k}}
		\sum\limits_{
			\{i_1, \ldots, i_k\}
			\subseteq
			\{1, \ldots, n\}
		}
		\frac{n!}{k! (n - k)!}
		\cdot
		\xi_{i_{\sigma(1)}} 
		\star_z \cdots \star_z 
		\xi_{i_{\sigma(k)}}
		\tensor
		\xi_{\tau(1)}
		\star_z \cdots 
		\widehat{\xi_I}
		\cdots \star_z 
		\xi_{\tau(n)}
		\\
		& =
		\sum\limits_{k=0}^n
		\sum\limits_{
			\{i_1, \ldots, i_k\}
			\subseteq
			\{1, \ldots, n\}
		}
		\xi_{i_1} \cdots \xi_{i_k}
		\tensor
		\xi_1 \cdots \widehat{\xi_I} \cdots \xi_n.
	\end{align*}
\end{proof}
This lemma yields immediately the following result.
\begin{corollary}
	If we deform the symmetric tensor algebra of an AE-Lie algebra 
	$\Sym^{\bullet}(\lie{g})$ with the Gutt star product, the coproduct 
	and the antipode will remain undeformed.
\end{corollary}
\begin{remark}[Hopf structures on the tensor algebra]
	\mbox{}
	\begin{remarklist}
		\item
		On the first sight, this result looks astonishing, but actually it is 
		not. The Hopf algebra structure we found here is just the Hopf algebra 
		structure which comes from the tensor algebra of $\lie{g}$. To be more 
		precise: it is the Hopf algebra structure using the tensor product and 
		the shuffle coproduct. There is also a second Hopf algebra structure: the 
		one using the deconcatenation coproduct and the shuffle product. We find 
		these structures on every tensor algebra of a vector space, it does not 
		need to be a tensor algebra over a Lie algebra. Since the coalgebra 
		structure on $\algebra{U}(\lie{g}_z)$ is inherited from 
		$\Tensor^{\bullet}(\lie{g})$ and does not make use of the Lie bracket, 
		there is no reason why a rescaling of the Lie bracket should change it.
		
		\item
		What we have seen is hence just \emph{one} possibility to deform the 
		symmetric algebra. Another way to do so would be to deform the 
		costructure and leave the product untouched. Such a deformation of a Hopf 
		algebra would lead to the notion of \emph{quantum groups}.
	\end{remarklist}
\end{remark}

% AE-Lie Algebras
%


\section{Continuity of the Hopf Structure}
We need a topology on the tensor product in \eqref{Hopf:CoproductFormula},
since we want to prove the continuity of this map. As we have always used the 
projective tensor product for our construction so far, it seems just logic to do 
so again. The continuity of the two maps is then very easy to show.
\begin{proposition}
    \label{Prop:Hopf:CoproductContinuity}%
    Let $\lie{g}$ be an AE-Lie algebra and $R \geq 0$. For every continuous 
    seminorm $p$ and all $x \in \widehat{\Tensor}_R^\bullet(\lie{g})$
    the following estimates hold:
    \begin{equation}
        \label{Hopf:AntipodeContinuity}
        p_R \left( \widetilde{S}_z(x) \right)
        \leq
        p_R (x)
    \end{equation}
    and
    \begin{equation}
        \label{Hopf:CoproductContinuity}
        (p_R \tensor p_R)
        \left( \widetilde{\ocoproduct}_z(x) \right)
        \leq
        (2 p)_R (x).
    \end{equation}
\end{proposition}
\begin{proof}
	We just need to show both estimates on factorizing tensors and extend them 
	with the infimum argument. Inequality \eqref{Hopf:AntipodeContinuity} is 
	clear, since we only get a sign.
	To get Equation~\eqref{Hopf:CoproductContinuity}, we compute:
	\begin{align*}
		(p_R \tensor p_R)
        \left( \widetilde{\ocoproduct}_z
        	\left(
        		\xi_1 \tensor \cdots \tensor \xi_n
        	\right)
        \right)
        & =
		(p_R \tensor p_R)
        \left( 
			\sum\limits_{
        		I \subseteq
        		\{1, \ldots, n\}
        	}
        	\xi_I
        	\tensor
        	\xi_1 \cdots
        	\widehat{\xi_I}
        	\cdots \xi_n
        \right)
        \\
        & \leq
        \sum\limits_{
        	I \subseteq
        	\{1, \ldots, n\}
        }
        |I|!^R (n - |I|)!^R
        p^{|I|} \left( \xi_I \right)
        p^{n - |I|}
        \left( 
        	\xi_1 \cdots \widehat{\xi_I} \cdots \xi_n 
        \right)
        \\
        & \leq
        \sum\limits_{
        	I \subseteq
        	\{1, \ldots, n\}
        }
        |I|!^R (n - |I|)!^R
        p(\xi_1) \cdots p(\xi_n)
        \\
        & \leq
        \sum\limits_{
        	I \subseteq
        	\{1, \ldots, n\}
        }
        n!^R
        p(\xi_1) \cdots p(\xi_n)
        \\
        & =
        2^n n!^R
        p(\xi_1) \cdots p(\xi_n)
        \\
        & =
        (2p)_R \left(
        	\xi_1 \tensor \cdots \tensor \xi_n
        \right).
	\end{align*}
	This shows the statement.
\end{proof}
\begin{remark}
	We see that we get no dependence on the parameter $R$. Also this is not 
	surprising: the Hopf structure of the symmetric or the universal enveloping 
	algebra over a vector space is cocommutative. The symmetric tensor product is 
	commutative and its continuity estimate does not depend on the parameter $R$ 
	either. Note that for an abelian Lie algebra, also the product structure  
	remains undeformed and all Hopf algebra maps are continuous for $R \geq 0$. In 
	this sense, the independence of $R$ fits into the picture.
\end{remark}


The only maps which are left to consider are the unit and the counit. Since their 
continuity is clear by the definition of the $\Tensor_R$-topology, we get the 
a final result.
\begin{theorem}
    \label{Prop:Hopf:ContinuousHopf}%
    Let $\lie{g}$ be an AE-Lie algebra and $z \in \mathbb{K}$. Then, 
    if $R \geq 1$, $\widehat{\Sym}_R^\bullet (\lie{g})$ will be a locally convex 
    Hopf algebra. The same will hold for $\widehat{\Sym}_{1^-}^{\bullet}
    (\lie{g})$, if $\lie{g}$ is a nilpotent locally convex Lie algebra with 
    continuous Lie bracket and for $\widehat{\Sym}_{1^{--}}^{\bullet}(\lie{g})$, 
    if $\lie{g}$ is a uniformly topologically nilpotent Banach-Lie algebra.
\end{theorem}
