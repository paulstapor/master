
%
% Chapter 7 of my master thesis:
% The Hopf algebra structure
%

\chapter{The Hopf algebra structure}

\section{The Hopf algebra maps}
\label{sec:chap7_Coproduct}

\label{sec:Hopf}

Since we found an interesting locally convex topologies on the universal
enveloping algebra considered as an associative algebra, we want to know if the 
coalgebra structure on $\algebra{U}_R(\lie{g}_z)$ is continuous. 
Indeed, this is the case. To prove it, we first need an explicit formula for 
the coproduct
\begin{equation}
    \label{eq:CoProduct}
    \coproduct_z \colon
    \algebra{U}(\lie{g}_z)
    \longrightarrow
    \algebra{U}(\lie{g}_z)
    \tensor
    \algebra{U}(\lie{g}_z)
\end{equation}
of the universal enveloping algebra.  We pull back the coproduct
on the symmetric algebra and want to give an estimate there. It is
therefore helpful to extend the coproduct on the whole tensor algebra
again, just as we did with the star product. We define
\begin{equation}
    \label{eq:CoProductOnTensor}
    \ocoproduct_z \colon
    \Tensor^\bullet(\lie{g})[[z]]
    \longrightarrow
    \Sym^\bullet(\lie{g})[[z]]
    \tensor
    \Sym^\bullet(\lie{g})[[z]]
    , \quad
    \ocoproduct_z
    =
    (\mathfrak{q}_z^{-1} \tensor \mathfrak{q}_z^{-1})
    \circ
    \coproduct_z
    \circ
    \mathfrak{q}_z
    \circ
    \Symmetrizer.
\end{equation}
Now we can state an explicit formula for the coproduct.
\begin{lemma}
    \label{Thm:Hopf:CoproductFormula}%
    For $\xi_1, \ldots, \xi_n \in \lie{g}$ we have
    \begin{equation}
        \label{Hopf:CoproductFormula}
        \ocoproduct_z(\xi_1 \tensor \cdots \tensor \xi_n)
        =
        \frac{1}{n!}
        \sum\limits_{\sigma \in S_n}
        \sum\limits_{k=0}^n
        \sum\limits_{\substack{
            1 \leq i_1 < \cdots < i_k \leq n \\
            I = \{i_1, \ldots, i_k\} }}
        ( \xi_{\sigma(i_1)}
        \star_z \cdots \star_z
        \xi_{\sigma(i_k)} )
        \tensor
        ( \xi_{\sigma(1)} \star_z
        \cdots \widehat{\xi_{\sigma(I)}} \cdots
        \star_z \xi_{\sigma(n)} ),
    \end{equation}
    where $\widehat{\xi_{\sigma(I)}}$ means that the $\xi_i$ with $i
    \in \sigma(I)$ are left out.
\end{lemma}
\begin{proof}
    This formula comes from the well-known one for the shuffle coproduct
    on $\algebra{U}(\lie{g})$. The multiplication in $\algebra{U}
    (\lie{g})$ is translated into star products and is symmetrized
    before. One gets this formula by taking the coproduct on the Lie
    algebra and then extending it as an homomorphism of associative
    algebras.
\end{proof}

% AE-Lie Algebras
%


First, we need a topology on the tensor product in
\eqref{Hopf:CoproductFormula}, for which we take again the projective
tensor product. An easy way to prove continuity would be assuming that 
$\lie{g}$ is locally m-convex, $|z|< 2\pi$ and we could conclude by 
Lemma~\ref{Lemma:LCAna:PreContinuity2}. But we can also do it via 
Lemma~\ref{LCAna:PreContinuityIntermediateN} for AE-Lie algebras as
well.
\begin{proposition}
    \label{Prop:Hopf:CoproductContinuity}%
    Let $\lie{g}$ be an AE-Lie algebra, $R \geq 1$ and $z \in
    \mathbb{C}$. Then we have for all $x \in 
    \widehat{\Tensor}_R^\bullet(\lie{g})$, 
    $p$ a continuous seminorm with an asymptotic estimate $q$
    \begin{equation}
        \label{Hopf:CoproductContinuity}
        (p_R \tensor p_R)
        \left(\ocoproduct_z(x)\right)
        \leq
        (c q)_R (x),
    \end{equation}
    with $c = 16 \E (|z| + 1)$ being locally uniform in $z$.
\end{proposition}
\begin{proof}
    We give the estimate on factorizing tensors: Let $\xi_1, \ldots, \xi_n \in 
    \lie{g}$. We use Equation \eqref{Hopf:CoproductFormula} in ($a$), 
	Lemma~\ref{LCAna:PreContinuityIntermediateN} in ($b$) and get
    \begin{align*}
        &
        (p_R \tensor p_R) \left(
            \ocoproduct_z
            \left(
                \xi_1 \tensor \cdots \tensor \xi_n
            \right)
        \right)
        \\
        &\quad\ot{(a)}{\leq}
        \frac{1}{n!}
        \sum\limits_{\sigma \in S_n}
        \sum\limits_{k=0}^n
        \sum\limits_{\substack{
            1 \leq i_1 < \cdots < i_k \leq n \\
            I = \{i_1, \ldots, i_k\} }}
        p_R( \xi_{\sigma(i_1)}
        \star_z \cdots \star_z
        \xi_{\sigma(i_k)} )
        p_R( \xi_{\sigma(1)} \star_z
        \cdots
        \widehat{\xi_{\sigma(I)}}
        \cdots
        \star_z \xi_{\sigma(n)} )
        \\
        &\quad\ot{(b)}{\leq}
        \frac{1}{n!}
        \sum\limits_{\sigma \in S_n}
        \sum\limits_{k=0}^n
        \sum\limits_{\substack{
            1 \leq i_1 < \cdots < i_k \leq n \\
            I = \{i_1, \ldots, i_k\} }}
        c^k k!^R
        q(\xi_{\sigma(i_1)})
        \cdots
        q(\xi_{\sigma(i_k)})
        c^{n-k} (n-k)!^R
        q(\xi_{\sigma(1)})
        \cdots
        \widehat{q(\xi_{\sigma(I)})}
        \cdots
        q(\xi_{\sigma(n)})
    \end{align*}
    with $c = 8 \E (|z| + 1)$. Then
    \begin{align*}
    	(p_R \tensor p_R) \left(
            \ocoproduct_z
            \left(
                \xi_1 \tensor \cdots \tensor \xi_n
            \right)
        \right)
        & =
        \sum\limits_{k=0}^n
        \sum\limits_{\substack{
            1 \leq i_1 < \cdots < i_k \leq n \\
            I = \{i_1, \ldots, i_k\} }}
        c^n (n-k)!^R k!^R
        q(\xi_1) \cdots q(\xi_n)
        \\
        & \ot{(c)}{\leq}
        \sum\limits_{k=0}^n
        \binom{n}{k}
        c^n n!^R
        q(\xi_1) \cdots q(\xi_n)
        \\
        & =
        2^n c^n n!^R
        q(\xi_1) \cdots q(\xi_n)
        \\
        & =
        (2c q)_R \left(
            \xi_1 \tensor \cdots \tensor \xi_n
        \right).
    \end{align*}
    In ($c$) we just counted the number of the index sets $\{i_1,
    \ldots, i_k\}$ which is precisely $\binom{n}{k}$. We hence get the result 
    on all tensors and on the completion via the infimum argument and 
    continuous extension.
\end{proof}

% The Case of Nilpotent Lie Algebras
%

The continuity of the coproduct also holds if the Lie algebra is nilpotent 
and we take the projective limit $R \longrightarrow 1^-$. Therefore, we use 
Lemma~\ref{Nilpot:LemmaPreContinuityN} and conclude analogously.
\begin{proposition}
    \label{Thm:Hopf:CoproductContinuity2}%
    Let $\lie{g}$ be a locally convex nilpotent Lie algebra such that $N + 1$ 
    Lie brackets vanish, $0 \leq R < 1$ and $z \in \mathbb{C}$. Then for $c = 
    32 \E^2$ (locally uniform in $z$) and 
    $\epsilon = \frac{N - 1}{N}(1 - R)$ we have for all $x \in 
    \widehat{\Tensor}_R^\bullet(\lie{g})$ and $p$ a continuous seminorm with 
    asymptotic estimate $q$
    \begin{equation}\label{Hopf:CoproductContinuity2}
        (p_R \tensor p_R) \left(
        \ocoproduct_z (x) \right)
        \leq
        (c q)_{R + \epsilon} (x).
    \end{equation}
\end{proposition}

% The antipode
%


For a continuous Hopf structure, only the unit, the counit and the Antipode are 
left. Their continuity is given by the next proposition.
\begin{proposition}
    \label{Prop:Hopf:ContinuousHopf}%
    Let $\lie{g}$ be an AE-Lie algebra. Then, if $R \geq 1$,
    $\widehat{\Sym}_R^\bullet (\lie{g})$ for all $z \in \mathbb{C}$ 
    the unit $\eta$, the counit $\varepsilon$ and the antipode $S_z$ are 
    continuous. The same holds $\widehat{\Sym}_{1^-}^\bullet(\lie{g})$ if 
    $\lie{g}$ is a nilpotent, locally convex Lie algebra. with continuous Lie 
    bracket.
\end{proposition}
\begin{proof}
    The continuity of the unit and the counit is immediate
    from the definition of our topology. The antipode $S_z$
    is a multiplication with $-1$ on Lie algebra level which extends to
    $\algebra{U}_R(\lie{g}_z)$ as an algebra anti-homomorphism. Translated
    back to $\Sym_R^\bullet(\lie{g})$ via $\mathfrak{q}_z^{-1}$, this means 
    that for $\xi_1, \ldots, \xi_n \in \lie{g}$
    \begin{equation*}
    	S_z\left(
    		\xi_1 \ldots \xi_n
    	\right)
    	=
    	(-1)^n \xi_n \star_z \ldots \star_z \xi_1.
    \end{equation*}
    In order to get an estimate, we extend the antipode to the whole
    tensor algebra and use either Lemma \ref{Lemma:LCAna:LemmaPreContinuityN}
    (in the AE case) or Lemma \ref{Lemma:Nilpot:LemmaPreContinuityN} (in the
    nilpotent case). Then we conclude for arbitrary tensors by the
    infimum argument and extend the estimate to the completion.
\end{proof}


\subsection{A formula for the co-product}

\subsection{Continuity for the co-product}



\section{The whole Hopf algebra structure}
\label{sec:chap7_HopfAlgebra}