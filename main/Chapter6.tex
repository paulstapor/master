
%
% Chapter 6 of my master thesis:
% The nilpotent case
%

\chapter{Nilpotent Lie algebras}

At the end of the last chapter, we have seen that the Baker-Campbell-Hausdorff 
series and its convergence plays an important role for a topology on the 
universal enveloping algebra. It is thus natural to ask whether things will 
change, if we look at Lie algebras from which have a globally convergent BCH 
series. To make things not too complicated from the beginning, we focus on 
locally convex and truly nilpotent Lie algebras. Recall that a Lie algebra 
$\lie{g}$ is nilpotent, if there exists a $N \in \mathbb{N}$, such that for 
all $n \geq N$ and all $\xi_1, \ldots, \xi_n \in \lie{g}$ we have
\begin{equation}
	\label{Nilpot:StrongNilpotency}
	\ad_{\xi_1} \circ \ldots \circ \ad_{\xi_n}
	=
	0.
\end{equation}
In the infinite-dimensional case, this is \emph{not} the same as 
\begin{equation*}
	\label{Nilpot:WeakNilpotency}
	\left( \ad_{\xi} \right)^n
	=
	0
\end{equation*}
for all $\xi \in \lie{g}$ and and $n \geq N$, but (typically strictly) 
stronger. In the case of finite-dimensional Lie algebras, the notions 
\eqref{Nilpot:StrongNilpotency} and \eqref{Nilpot:WeakNilpotency} coincide due 
to the so-called theorem of Engel, which makes use of the existence of a 
finite series of nilpotent ideals in the Lie algebra. Such a terminating 
series doesn't need to exist in infinite dimensions and there are known 
counter-examples to it.

Before we look at this case more closely, let's first make a list of things, 
that we expect to change or not when we go to this more particular setting.
\begin{enumerate}
	\item
	In Example~\ref{LCAna:Ex:HeisenbergAlgebra} we have seen that we can not 
	expect to get a continuous algebra structure for $R < 1$, even for very 
	simple nilpotent, but non-abelian Lie algebras. Therefore, we should not 
	expect to get much larger completions now.
	
	\item
	In \cite{waldmann:2014a}, Waldmann showed that the Weyl-Moyal star product 
	converges in the $R$-topology for $R \geq \frac{1}{2}$. This so-called 
	Weyl algebra is, however, nothing but a quotient of the Heisenberg 
	algebra. It would be interesting to understand this a bit better, since we 
	know that we need $R \geq 1$ for the latter. The quotient procedure must 
	therefore have some strong influence on this construction. Can we 
	reproduce the value $R \geq \frac{1}{2}$ somehow by dividing out an ideal?
	
	\item
	The argument we used in Propositipon~\ref{LCAna:Prop:NoBetterTopology},
	namely the non-global convergence of BCH, is not given any more. Now, we 
	don't have a reason any longer to expect that exponentials won't be part 
	of the completion. In this sense, it would be at least nice to have 
	something more than ''just'' $R = 1$. Can we do that?
	
	\item
	As already mentioned, there are generalizations or weaker forms of 
	nilpotency in infinite-dimensions, especially for Banach-Lie algebras, 
	which are equivalent to the usual notion of nilpotency in finite 
	dimensions. If we get a stronger result for nilpotent Lie algebras, can we 
	extend it to some of these generalizations?
\end{enumerate}
The very fascinating and highly interesting answer to the three questions we 
just posed is: yes, we can. The first section of this chapter will be devoted 
to the question from point $(iii)$: we get a bigger completion by using a 
projective limit. We will also see how to get again the nice functorial 
properties we had before. In second section, we will reproduce the result by 
Waldmann, at least for the finite-dimensional case. The third part will take 
care of some generalizations of nilpotentcy for Banach-Lie algebras and will 
extend the result of the projective limit to a particular subcase there.



 
\section{The projective limit} 
\label{sec:chap6_ProjLim} 

\subsection{Continuity of the Product}

As already mentioned, it is possible to extend the continuity result. 
Therefore, we take a locally convex, nilpotent Lie algebra $\lie{g}$ and look 
at
\begin{equation*}
	\Sym_{1^-}^{\bullet}(\lie{g})
	=
	\projlim_{\epsilon \longrightarrow 0}
	\Sym_{1 - \epsilon}^{\bullet}(\lie{g}).
\end{equation*}
A tensor is in the completion $\widehat{Sym}_{1^-}^{\bullet}(\lie{g})$, 
when it lies for every $\epsilon > 0$ in the completion or 
$\Sym_{1 - \epsilon}^{\bullet}(\lie{g})$. Otherwise stated: Let $\algebra{P}$ 
be the set of all continuous seminorms of the Lie algebra $\lie{g}$, then 
\begin{equation*}
	f \in \widehat{Sym}_{1^-}^{\bullet}(\lie{g})
	\quad \Longleftrightarrow \quad
	p_{1 - \epsilon} (x) 
	< 
	\infty
	\quad
	\forall_{p \in \algebra{P}}
	\forall_{\epsilon > 0}.
\end{equation*}
So, if we want to show, that the Gutt star product is continuous in 
$\widehat{Sym}_{1^-}^{\bullet}(\lie{g})$, we need to show that for every 
$p \in \algebra{P}$ and  $R < 1$, there exists a $q \in \algebra{P}$ and a 
$R' < 1$, such that for all $x,y \in \Sym^{\bullet}(\lie{g})$ we have
\begin{equation*}
	p_R \left(
		x \star_z y
	\right)
	\leq
	q_{R'}(x)
	q_{R'}(y).
\end{equation*}
Before we prove the next proposition, we want to remind that locally convex, 
nilpotent Lie algebras are always AE Lie algebras. So the results we have 
found so far, are valid in this case.
\begin{theorem}
    \label{Nilpot:Thm:ProjLimit}%
    Let $\lie{g}$ be a nilpotent locally convex Lie algebra with
    continuous Lie bracket and $N \in \mathbb{N}$ such that $N + 1$
    Lie brackets vanish.
    \begin{theoremlist}
	    	\item \label{item:Nilpot:CnOperators}
	    	If $0 \leq R < 1$, the $C_n$-operators are continuous and fulfil the 
	    	estimate
	    	\begin{equation}
	    		\label{eq:Nilpot:CnOperators}
	    		p_R \left(
	    			C_n (x, y)
	    		\right)
	    		\leq
	    		\frac{1}{2 \cdot 8^n}
	    		(32 \E q)_{R + \epsilon}(x)
	    		(32 \E q)_{R + \epsilon}(y),
	    	\end{equation}
	    	for all $x, y \in \Sym_R^{\bullet}(\lie{g})$, where $p$ is a 
	    	continuous seminorm, $q$ an asymptotic estimate and 
	    	$\epsilon = \frac{N - 1}{N}(1 - R)$.
	    	
	    	\item \label{item_Nilpot:SEinsMinus}
	    	The Gutt star product $\star_z$ is continuous for the locally convex 
	    	projective limit $\Sym_{1^-}^\bullet(\lie{g})$ and we have
	    	\begin{equation}
	    		\label{eq:Nilpot:Continuity}
	    		p_R \left(
	    			x \star_z y
	    		\right)
	    		\leq
	    		(c q)_{R + \epsilon}(x)
	    		(c q)_{R + \epsilon}(y)
	    	\end{equation}
	    	with $c = 32 \E (|z| +1)$ and the $\epsilon$ from the first part.
	    	The Gutt star product extends continuously to
	    	$\widehat{\Sym}_R^{\bullet}(\lie{g})$, where it converges absolutely 
	    	and coincides with the formal series.
    \end{theoremlist}
\end{theorem}
\begin{proof}
    Again we use $\star_z$ on the whole tensor algebra and compute the 
    estimate for $\xi^{\tensor k}$ and $\eta^{\tensor \ell}$. The important 
    point is that now, we get restrictions for the values of $n$.
    Recall that $k + \ell - n$ is the symmetric degree of $C_n \left( 
    \xi^{\tensor k}, \eta^{\tensor \ell} \right)$ and that we must have
    \begin{equation*}
	    	(k + \ell - n) N
    		\geq
    		k + \ell
    		\quad 
    		\Longleftrightarrow 
    		\quad
    		n 
    		\leq
    		(k + \ell)
    		\frac{N - 1}{N}
    \end{equation*}
    for $C_n \left( \xi^{\tensor k}, \eta^{\tensor \ell} \right) \neq 0$. 
    This makes it possible to estimate $n!^{1-R}$ in 
    \eqref{LCAna:CnOperators}: set $\delta = \frac{N - 1}{N}$ and also denote 
    a factorial where we have non-integers, meaning the gamma function. We get
    \begin{align*}
        n!^{1-R}
        & \leq
        (\delta (k + \ell)!)^{1 - R}
        \\
        & \leq
        (\delta (k + \ell))^{(1 - R) \delta (k + \ell)}
        \\
        & \leq
        (k + \ell)^{(1 - R) \delta (k + \ell)}
        \\
        & =
        \left(
            (k + \ell)^{(k + \ell)}
        \right)^{(1 - R) \delta}
        \\
        & \leq
        \left(
            \E^{k + \ell} 2^{k + \ell} k! \ell!
        \right)^{(1-R) \delta}
        \\
        & =
        \left( (2 \E)^{\delta (1-R)} \right)^{k + \ell}
        k!^{\epsilon} \ell!^{\epsilon},
    \end{align*}
    using $\epsilon = \delta (1 - R)$. Hence
    \begin{align*}
        p_R \left(
        	C_n \left(
        		\xi^{\tensor k}, \eta^{\tensor \ell}
        	\right)
        \right)
        & \leq
        \frac{
        	\left(
        		(2 \E)^{\delta (1 - R)}
        	\right)^{k + \ell}
        	k!^{\epsilon}
        	\ell!^{\epsilon}
        }{2 \cdot 8^n}
        (16 q)_R \left( \xi^{\tensor k} \right)
        (16 q)_R \left( \eta^{\tensor \ell} \right)
        \\
        & \leq
        \frac{1}{2 \cdot 8^n}
        (c q)_{R + \epsilon} \left( \xi^{\tensor k} \right)
        (c q)_{R + \epsilon} \left( \eta^{\tensor \ell} \right)
    \end{align*}
    with $c = 16 (2 \E)^{\delta (1 - R)} \leq 32 \E$.
    We then get the estimate on all tensors by the infimum argument and extend 
    it to the completion. Note, that for every $R < 1$ we also have 
    $R + \epsilon < 1$ with the $\epsilon = \delta(1-R)$ from above. Iterating 
    this continuity estimate, we get closer and closer to $1$ and it is not 
    possible to repeat this process an arbitrary number of times and stop at 
    some value strictly less than $1$. For the second part, we can conclude 
    analogously to the second part of 
    Proposition~\ref{LCAna:Prop:Continuity1}.
\end{proof}
Again, we can do an easier proof by assuming submultiplicativity of the 
seminorms, since we get an alternative version of 
Lemma~\ref{LCAna:Lemma:PreContinuity2}:
\begin{lemma}
	\label{Nilpot:Lemma:PreContinuity2}
	Let $\lie{g}$ be a locally m-convex, nilpotent Lie algebra such that more 
	$N$ nested Lie brackets vanish. Let $p$ be a continuous seminorm, 
	$z \in \mathbb{K}$ and $R \geq 0$. Then, for every tensor 
	$x \in \Sym_R^{\bullet}(\lie{g})$ of degree at most $k \in \mathbb{N}$ and 
	$\eta \in \lie{g}$, 	we have the estimate
	\begin{equation}
		\label{Nilpot:PreContinuity}
		p_R \left( x \star_z \eta \right)
		\leq
		(k + 1)^R k^{N (1-R)} c
		p_R (x) p(\eta)
	\end{equation}
	with the constant $c = \sum_{n = 0}^N \frac{|B_n^*|}{n!} |z|^n$.
\end{lemma}
\begin{proof}
	Again, we do the estimate on factorizing tensors and apply the infimum 
	argument later. So let $\xi, \eta \in \lie{g}$, $k \in \mathbb{N}$, $p$ a 
	continuous seminorm on $\lie{g}$ and $z \in \mathbb{K}$. Then, we have for 
	$R \geq 0$
	\begin{align*}
		p_R \left(
			\xi^{ \tensor k } \star_z \eta
		\right)
		& =
		\sum\limits_{n = 0}^k
		(k + 1 - n)!^R \binom{k}{n}
		|B_n^*| |z|^n
		p^{k + 1 - n} \left(
			\xi^{k-n}
			\left( \ad_{\xi} \right)^n (\eta)
		\right)
		\\
		& \leq
		(k + 1)^R
		\sum\limits_{n = 0}^N
		\frac{ k! (k-n)!^R }{ (k-n)! n! }
		|B_n^*| |z|^n
		p(\xi)^k p(\eta)
		\\
		& =
		(k + 1)^R
		p_R \left( \xi^{\tensor k} \right)
		p(\eta)
		\sum\limits_{n = 0}^N
		\left( \frac{k!}{(k-n)!} \right)^{1-R}
		\frac{|B_n^*| |z|^n}{n!}
		\\
		& \leq
		(k + 1)^R
		k^{N (1-R)}
		p_R \left( \xi^{\tensor k} \right)
		p(\eta)
		\sum\limits_{n = 0}^N
		\frac{|B_n^*| |z|^n}{n!}.
	\end{align*}
\end{proof}
Now, we can iterate Lemma~\ref{Nilpot:Lemma:PreContinuity2} in the same way, 
we did it in Chapter 5:
\begin{proof}[Alternative Proof of Theorem~\ref{Nilpot:Thm:ProjLimitConti}]
	Again, we do the calculation only on factorizing tensors. We need to 
	transform the $k^{N(1-R)}$ into a very small factorial somehow. This is 
	possible, since for given $N \in \mathbb{N}$ and $0 \leq R < 1$, the 
	sequence
	\begin{equation*}
		\left( \frac{k^N}{\sqrt{k!}} \right)^{1-R}
	\end{equation*}
	converges to $0$ for $k \longrightarrow \infty$ and is therefore bounded 
	by some $\kappa_N > 0$. Hence we get
	\begin{equation*}
		k^{ N (1-R) } 
		\leq
		\kappa_N \sqrt{k!}^{1-R},
	\end{equation*}
	and together with Lemma~\ref{Nilpot:Lemma:PreContinuity2} we find
	\begin{equation*}
		p_R \left( x \star_z \eta \right)
		\leq
		(k + 1)^R k!^{\frac{1-R}{2}} c \kappa_N
		p_R (x) p(\eta)
	\end{equation*}
	for any tensor $x$ of degree at most $k$. Now, we can iterate this result
	for $\xi, \eta \in \lie{g}$, $R \geq 0$, $k, \ell \in \mathbb{N}$:
	\begin{align*}
		p_R \left(
			\xi^{\tensor k} star_z \eta^{\tensor \ell}
		\right)
		& =
		p_R \left(
			\xi^{\tensor k} star_z 
			\eta^{\star_z \ell}
		\right)
		\\
		& \leq
		(k + \ell)^R
		(k + \ell - 1)!^{\frac{1-R}{2}}
		c \kappa_N
		p_R \left(
			\xi^{\tensor k} star_z 
			\eta^{\star_z (\ell - 1)}
		\right)
		p(\eta)
		\\
		& \leq
		\quad \vdots
		\\
		& \leq
		\left(
			\frac{(k + \ell)!}{k!}
		\right)^R
		(k + \ell - 1)!^{\frac{1-R}{2}}
		\ldots
		k!^{\frac{1-R}{2^N}}
		(c \kappa_N)^{\ell}
		p_R \left( \xi^{\tensor k} \right)
		p(\eta)^{\ell}
		\\
		& \leq
		\binom{k + \ell}{k}^R 
		\ell!^R
		(k + \ell)!^{\frac{(2^N - 1) (1-R)}{2^N} }
		(c \kappa_N)^{\ell}
		p_R \left( \xi^{\tensor k} \right)
		p(\eta)^{\ell}
		\\
		& \leq
		2^{(k + \ell) R}
		\ell!^R
		k!^{\frac{(2^N - 1) (1-R)}{2^N} }
		\ell!^{\frac{(2^N - 1) (1-R)}{2^N} }
		2^{ (k + \ell) \frac{(2^N - 1) (1-R)}{2^N} }
		(c \kappa_N)^{\ell}
		p_R \left( \xi^{\tensor k} \right)
		p(\eta)^{\ell}
		\\
		& \leq
		(2 p)_{R + \epsilon} 
		\left( \xi^{\tensor k} \right)
		(2 c \kappa_N p)_{R + \epsilon} 
		\left( \eta^{\tensor \ell} \right),
	\end{align*}
	where we have set $\epsilon = \frac{(2^N - 1)(1 - R)}{2^N}$. From this,
	we have clearly $R + \epsilon < 1$, and we get the wanted result for the 
	projective limit.
\end{proof}
Of course, the projective limit case gives us a bigger completion. We
immediately end up with the following result:
\begin{corollary}
    \label{corollary:NilpotentCase}%
    Let $\lie{g}$ be a nilpotent, locally convex Lie algebra.
    \begin{corollarylist}
    \item \label{item:NilpotentHasExp} 
    		Let $\exp(\xi)$ be the
        exponential series for $\xi \in \lie{g}$, the we have $\exp(t
        \xi) \in \widehat{\Sym}_{1^-}^\bullet(\lie{g})$ for all $t
        \in \mathbb{K}$.
    \item \label{item:NilpotentExpGivesBCH} 
    		For $\xi, \eta \in
        \lie{g}$ and $z \neq 0$ we have $\exp(\xi) \ostar_z
        \exp(\eta) = \exp \left(\frac 1 z \bch{z \xi}{z \eta}
        \right)$.
    \item \label{item:NipotentOneParameterGroups}
    		For $s,t \in
        \mathbb{K}$ and $\xi \in \lie{g}$ we have $\exp(t \xi)
        \ostar_z \exp(s \xi) = \exp ((t + s) \xi)$.
    \end{corollarylist}
\end{corollary}
\begin{proof}
    For the first part, recall that the completion of the projective
    limit $1^-$ will contain all those series $(a_n)_{n \in
      \mathbb{N}_0}$ such that
    \begin{equation*}
        \sum\limits_{n=0}^{\infty}
        a_n n!^{1 - \epsilon} c^n
        <
        \infty
    \end{equation*}
    for all $c > 0$.  This is the case for the exponential series of
    $t \xi$ for $t \in \mathbb{K}$ and $\xi \in \lie{g}$. The second
    part follows from the fact that all the projections $\pi_n$ onto
    the homogeneous subspaces $\Sym_{\pi}^n$ are continuous. The third
    part is then a direct consequence of the second.
\end{proof}


%
% A bit Functioriality also in this case
%

\subsection{Representations and Functoriality}
\label{subsec:NilpotentFunctorialityRepresentations}

In the general AE case, we had some useful results concerning representations 
of Lie algebras and the functorialty of our construction. These results can be 
extended to the projective limit $\Sym_{1^-}^{\bullet}(\lie{g})$.
\begin{proposition}[universal property]
	\label{Nilpot:Prop:UnivProperty}
	Let $\lie{g}$ be a locally convex nilpotent Lie algebra, $\algebra{A}$ an 
	associative AE algebra and $\phi \colon: \lie{g} \longrightarrow 
	\algebra{A}$ is a continuous homomorphism of Lie algebras. Then, the 
	lifted homomorphisms from $\Sym_{1^-}^{\bullet}(\lie{g})$ and 
	$\algebra{U}(\lie{g}_z)$ to $\algebra{A}$ are continuous.
\end{proposition}
\begin{proof}
	The proof is exactly the same as in the general AE case, since there, 
	$R \geq 0$ was enough.
\end{proof}
Again, this construction will be not a universal in the categorial sense, 
since $\Sym_{1^-}^{\bullet}(\lie{g})$ fails to be AE. But also here, we get 
the case of continuous representations into a Banach space (and in particular 
into a finite-dimensional space) as a corollary.
\begin{corollary}[Continuous Representations]
    \label{Nilpot:Coro:ContinuousRepresentations}%
    Let $\mathcal{U}_R(\lie{g})$ the universal enveloping algebra of locally 
    convex nilpotent Lie algebra $\lie{g}$, then for every continuous 
    representation $\phi$ of $\lie{g}$ into the bounded linear operators 
    $\Bounded(V)$ on a Banach space $V$, the induced homomorphism of 
    associative algebras $\Phi \colon \mathcal{U}(\lie{g}) \longrightarrow 
    \Bounded(V)$ is continuous.
\end{corollary}


We can also extend the functoriality statement to the projective limit, but we 
need to get another version of Lemma~\ref{Lemma:LCAna:LemmaPreContinuityN} for 
nilpotent Lie algebras, since thisis the corner stone of the functoriality 
proof.
\begin{lemma}
    \label{Lemma:Nilpot:LemmaPreContinuityN}%
    Let $\lie{g}$ be locally convex nilpotent Lie algebra and $N \in 
    \mathbb{N}$ such that $N + 1$ Lie brackets vanish, $0 \leq R < 1$ and 
    $z \in \mathbb{C}$. Then for $p$ a continuous seminorm, $q$ an
    asymptotic estimate, $n \in \mathbb{N}$ and all $\xi_1, \ldots,
    \xi_n \in \lie{g}$ the following estimate
    \begin{equation}
        \label{Nilpot:LemmaPreContinuityN}
        p_R \left(
            \xi_1 \star_z \cdots \star_z \xi_n
        \right)
        \leq
        c^n n!^{R + \epsilon}
        q^n(\xi_1 \tensor \cdots \tensor \xi_n)
    \end{equation}
    holds with $c = 16 \E^2 (|z| + 1)$ and $\epsilon = \frac{N-1}{N}(1 - R)$
    and the estimate is locally uniform in $z$.
\end{lemma}
\begin{proof}
    We take $R < 1$ and go directly into the proof of
    Lemma~\ref{Lemma:LCAna:LemmaPreContinuityN} at
    \eqref{LCAna:PreContinuityIntermediateN}.  We know that, since we
    may have at most $N$ brackets, also the values for $\ell$ are
    restricted to
    \begin{equation*}
        \ell
        \leq
        \frac{N-1}{N} n
        =
        \delta n
    \end{equation*}
    Using that in the proof of Lemma~\ref{Lemma:LCAna:LemmaPreContinuityN} 
    leads to
    \begin{align*}
        &p_R \left(
            \xi_1 \star_z \cdots \star_z \xi_n
        \right)
        \\
        &\quad\leq
        \sum\limits_{\ell = 0}^{\delta n}
        (n - \ell)!^R
        \sum\limits_{\substack{
			1 \leq j \leq n-1 \\
			i_j \in \{0, \ldots, j\} \\
			\sum_{j = 1}^{n - 1} i_j = \ell
		}}
        |z|^{\ell}
        \frac{1!  (2 - i_1)! \cdots (n-1 - i_1 - \cdots - i_{n-2})!}
        {(1 - i_1)! \cdots (n-1 - i_1 - \cdots - i_{n-1})!}
        q(\xi_1) \cdots q(\xi_n)
        \\
        &\quad\leq
        \sum\limits_{\ell = 0}^{\delta n}
        (n - \ell)!^R
        \sum\limits_{\substack{
			1 \leq j \leq n-1 \\
			i_j \in \{0, \ldots, j\} \\
			\sum_{j = 1}^{n - 1} i_j = \ell
		}}
        |z|^{\ell} (2 \E)^n \ell!
        q(\xi_1) \cdots q(\xi_n)
        \\
        &\quad\leq
        (2 \E)^n (|z| + 1)^n
        q(\xi_1) \cdots q(\xi_n)
        \sum\limits_{\ell = 0}^{\delta n}
        (n - \ell)!^R \ell!
        \binom{n + \ell - 2}{\ell - 1}
    \end{align*}
    We have
    \begin{equation*}
        \ell!
        =
        \ell!^R
        \ell!^{1-R}
        \leq
        \ell!^R
        \left(
            (\delta n)^{\delta n}
        \right)^{1-R}
        \leq
        \ell!^R
        n^{\delta n (1-R)}
        \leq
        \ell!^R
        n!^{\delta (1 - R)}
        \E^{\delta n (1 - R)}.
    \end{equation*}
    Together with $\ell!^R (n - \ell)!^R \leq n!^R$ this gives
    \begin{align*}
        p_R \left(
            \xi_1 \star_z \cdots \star_z \xi_n
        \right)
        & \leq
        (2 \E)^n (|z| + 1)^n
        n!^R n!^{\delta (1 - R)}
        q(\xi_1) \cdots q(\xi_n)
        \sum\limits_{\ell = 0}^{\delta n}
        \binom{n + \ell - 2}{\ell - 1}
        e^{\delta n (1 - R)}
        \\
        & \leq
        (2 \E)^n (|z| + 1)^n
        \left(\E^{(1-R) \delta}\right)^n
        4^n n!^{R + \epsilon}
        q(\xi_1) \cdots q(\xi_n),
    \end{align*}
    with $\epsilon = \delta (1-R)$. It is clear that for all $R < 1$ we have
    $R + \epsilon < 1$. Set 
    \begin{equation*}
    	c 
    	= 
    	8 \E (|z|+1) \E^{(1-R)\delta}
    	\leq
    	16 \E^2 (|z|+1)
	\end{equation*}
	and note that the estimate is locally uniform in $z$, even
    though it will not be uniform in $z$.
\end{proof}
\begin{proposition}
	\label{Nilpot:Prop:Functoriality}
	Let $R \geq 1$, $\lie{g}, \lie{h}$ be two locally convex nilpotent Lie 
	algebras and $\phi \colon \lie{g} \longrightarrow \lie{h}$ a continuous 
	homomorphism between them. Then it lifts to a continuous unital 
	homomorphism of locally convex algebras $\Phi_z \colon 
	\algebra{U}_R(\lie{g}_z) \longrightarrow \algebra{U}_R(\lie{h}_z)$.
\end{proposition}
\begin{proof}
	The proof is mostly analogous to the one of Proposition 
	\ref{LCAna:Prop:Functoriality}.
\end{proof}



\section{Module structures}
\label{sec:chap6_Modules}

The projective limit $1^-$ is not the only additional structure we will get, if 
our Lie algebra $\lie{g}$ is nilpotent. Lemma~\ref{Nilpot:Lemma:PreContinuity2} 
allows the existence of certain module structures, if the seminorms on 
$\lie{g}$ are in addition submultiplicative. For every $R \in \mathbb{R}$,the 
symmetric tensor algebra $\Sym_R^{\bullet}(\lie{g})$ is a locally convex vector 
space. For $R \geq 0$, the (symmetric) tensor product is continuous, which is 
very important for many estimates, and for $R \geq 1^-$, we have an algebra 
structure. In between however, we have more than ''only'' vector spaces: The 
spaces $\Sym_R^{\bullet}(\lie{g})$ form locally convex modules over the 
$\Sym_{R'}^{\bullet}(\lie{g})$ for certain values of $R'$. The next proposition 
will makes this more exact.
\begin{proposition}[Bimodules in $\Sym_R^{\bullet}(\lie{g})$]
	\label{Nilpot:Prop:Bimodules}
	Let $\lie{g}$ be a nilpotent, locally m-convex Lie algebra such that 
	$N + 1$ 	Lie brackets vanish, $z \in 	\mathbb{C}$ and $0 \leq R < 1$. 
	Then, for all $x, y \in \Sym^{\bullet}(\lie{g})$ and every continuous 
	seminorm $p$, we have the estimates
	\begin{align}
		\label{Nilpot:BimoduleEstimate1}
		p_R \left(
			x \star_z y
		\right)
		& \leq
		\left(2^{N + 1} p\right)_R(x) 
		\left(2^{N + 1} c p\right)_{R + N(1-R)}(y)
		\\
	\intertext{and}
		\label{Nilpot:BimoduleEstimate2}
		p_R \left(
			x \star_z y
		\right)
		& \leq
		\left(2^{N + 1} c p\right)_{R + N(1-R)}(x)
		\left(2^{N + 1} p\right)_R(y) 
	\end{align}
	with $c = \sum_{n = 0}^N \frac{|B_n^*| |z|^n}{n!}$.
	Hence, the vector space $\widehat{\Sym}_R^{\bullet}
	(\lie{g})$ forms a bimodule over the algebra $\widehat{\Sym}_{R + N(1-
	R)}^{\bullet}(\lie{g})$. In particular, if $\lie{g}$ is 2-step nilpotent, 
	the vector space $\widehat{\Sym}_0^{\bullet}(\lie{g})$ is a 
	$\widehat{\Sym}_1^{\bullet}(\lie{g})$-bimodule.
\end{proposition}
\begin{proof}
	Again, we do the calculation on factorizing tensors: Let 
	$\xi, \eta \in \lie{g}$, $R \geq 0$, $k, \ell \in \mathbb{N}$.
	Using Lemma~\ref{Nilpot:Lemma:PreContinuity2}, we get
	\begin{align*}
		p_R \left(
			\xi^{\tensor k} star_z \eta^{\tensor \ell}
		\right)
		& =
		p_R \left(
			\xi^{\tensor k} star_z 
			\eta^{\star_z \ell}
		\right)
		\\
		& \leq
		(k + \ell)^R
		(k + \ell - 1)^{N (1-R)}
		p_R \left(
			\xi^{\tensor k} star_z 
			\eta^{\star_z (\ell - 1)}
		\right)
		p(\eta)
		\\
		& \leq
		\quad \vdots
		\\
		& \leq
		\left(
			\frac{(k + \ell)!}{k!}
		\right)^R
		\left(
			\frac{(k + \ell - 1)!}{(k - 1)!}
		\right)^{N (1-R)}
		c^{\ell}
		p_R \left( \xi^{\tensor k} \right)
		p(\eta)^{\ell}
		\\
		& \leq
		2^{k + l} 2^{N (k + \ell)}
		\ell!^{N (1-R)}
		c^{\ell}
		p_R \left( \xi^{\tensor k} \right)
		p(\eta)^{\ell}
		\\
		& =
		\left(2^{N + 1} p\right)_R 
		\left( \xi^{\tensor k} \right)
		\left(2^{N + 1} c p\right)_{R + N(1-R)}
		\left( \eta^{\tensor \ell} \right).
	\end{align*}
	The proof of the second estimate is analogous.
\end{proof}
\begin{remark}[Possible extensions]
	This result immediately poses new questions, like the dependence on the 
	formal parameter in this case, possible generalizations to ''weaker forms'' 
	of nilpotency and so on. They may be issues of some future work, but can't
	be addressed here, since we rather want to present something like a part of 
	the ''big picture'' which is opened by the $R$-topology, instead of getting 
	lost in its details too much. There are, without any doubt, questions that 
	are more significant than extending those estimates to very special cases 
	and 	finding sharp bounds there, although this is interesting and important, 
	too.
\end{remark}
Although it seems clear from the construction, that these bimodules cannot be 
there for general Lie algebras, we can give a concrete counter-example, which 
shows that there are Lie algebras, which don't allow them.
\begin{example}
	\label{Nilpot:Ex:NoModulesInGeneral}
	Choose $R < 1$ and take $\lie{g} = \mathbbm{R}^3$ with the basis $e_1, e_2, 
	e_3$ and the vector product as Lie bracket:
	\begin{equation*}
		[e_1, e_2] 
		= 
		e_3 
		\qquad 
		[e_2, e_3] 
		= 
		e_1 
		\qquad 
		[e_3, e_1] 
		= 
		e_2
	\end{equation*}
	Again, we take a $\ell^1$-norm $n$ such that $n(e_1) = n(e_2) = n(e_3) 
	= 1$. It has the nice property that for $k, \ell, m \in \mathbb{N}$ we get
	on the projective tensor product
	\begin{equation*}
		n^{k + \ell + m} \left(
			e_1^k e_2^{\ell} e_3^m
		\right)
		=
		1.
	\end{equation*}
	Now we define the sequence $(a_k)_{k \in \mathbb{N}}$
	\begin{equation*}
		a_k 
		= 
		\frac{1}{k!^R} e_1^k,
	\end{equation*}
	for which we get $n_R(a_k) = 1$. Now, we want to show that 
	$a_k \star_z e_2$ grows faster than exponentially:
	\begin{align*}
		n_R \left( a_k \star_z e_2 \right) 
		& = 
		n_R \left( 
			\sum\limits_{j = 0}^k 
			\binom{k}{j} B_j^* 
			\frac{1}{k!^R} 
			e_1^{n-j} 
			\left( 
				\operatorname{ad}_{e_1} 
			\right)^j(e_2) 
		\right)
		\\
		& =
		\sum\limits_{j = 0}^k 
		\binom{k}{j} 
		|B_j^*| 
		\frac{1}{k!^R} 
		(k-j+1)!^R 
		\underbrace{
			n^{k-j} \left( e_1 (e_2 \wedge e_3) \right)
		}_{ = 1}
		\\
		& = 
		\sum\limits_{j = 0}^k 
		(k-j+1)^R 
		\binom{k}{j}
		\frac{|B_j^*|}{j!} 
		\frac{(k-j)!^R j^R}{k!^R} 
		j!^{1-R}
		\\
		& = 
		\sum\limits_{j=0}^k 
		(k-j+1)^R 
		\binom{k}{j}^{1-R} 
		\frac{|B_j^*|}{j!} 
		j!^{1-R}
		\\
		& \geq 
		\sum\limits_{j=0}^k 
		\frac{|B_j^*|}{j!} 
		j!^{1-R}
		\\
		& \geq 
		\frac{|B_k^*|}{k!^R}.
	\end{align*}
	We know, that for $R < 1$ and any $c > 0$
	\begin{equation*}
		\limsup_{n \longrightarrow \infty}
		\frac{|B_n^*|}{c^n n!^R}
		=
		\infty,
	\end{equation*}
	and hence the limes superior of $n_R \left( a_k \star_z e_2 \right) $ grows 
	faster than any exponential function.
\end{example}



\section{The Heisenberg and the Weyl algebra}
\label{sec:chap6_HeisenbergWeyl}

Now we want to see how we get the link to the Weyl algebra from
\cite{waldmann:2014a}, since we have something like a discrepancy for the
parameter $R$ concerning the continuity of the product in the Weyl and the 
Heisenberg algebra. In the following, we will show that this gap actually makes 
a lot of sense. For simplicity, we consider the easiest case of
the Weyl/Heisenberg algebra with two generators $Q$ and $P$, but the 
generalization to the Heisenberg [Weyl] algebra in $2n + 1$ [$2n$] dimensions 
is immediate and goes without problems. 
Recall that the Weyl algebra is a quotient of the enveloping algebra of the 
Heisenberg algebra $\lie{h}$ which one gets from dividing out its center. So 
let $\mathsf{h} \in \mathbb{C}$ and we have a projection
\begin{equation}
    \label{Nilpot:WeylProjection}
    \pi \colon
    \widehat{\Sym}_R^\bullet(\lie{h})
    \longrightarrow
    \widehat{\mathcal{W}}_R(\lie{h})
    =
     \left(
    	\frac{\Sym_R^\bullet(\lie{h})}
    	{\langle E - \mathsf{h} \Unit \rangle}
    \right)^{\widehat{} }
\end{equation}
Of course we want to know if this projection is continuous.
\begin{proposition}
    \label{proposition:ProjectionWeylContinuous}%
    The projection $\pi$ is continuous for $R \geq 0$.
\end{proposition}
\begin{proof}
    We extend $\pi$ to the whole tensor algebra by symmetrizing
    beforehand. Let then $p$ be a continuous seminorm on $\lie{h}$, $k,
    \ell, m \in \mathbb{N}_0$. We have
    \begin{align*}
        p_R(\pi (
        	Q^{\tensor k} \tensor
        	P^{\tensor \ell} \tensor
        	E^{\tensor m}
        ) )
        & =
        p_R( Q^k P^{\ell} \mathsf{h}^m )
        \\
        & =
        |\mathsf{h}|^m (k + \ell)!^R
        p^{k + \ell}(Q^k P^{\ell})
        \\
        & \leq
        (|\mathsf{h}| + 1)^{k + \ell + m}
        (k + \ell + m)!^R
        p(Q)^k p(P)^{\ell} p(E)^m
        \\
        & =
        ((|\mathsf{h}| + 1) p)_R
        (Q^{\tensor k} \tensor
        P^{\tensor \ell} \tensor
        E^{\tensor m}).
    \end{align*}
    Then we do the usual infimum argument and have the result on
    arbitrary tensors again.
\end{proof}


To establish the link to the continuity results of the Weyl algebra,
we need more: $\pi \circ \ostar_z$ should to be continuous for $R \geq \frac 
1 2$.
\begin{proposition}
    \label{proposition:ContinuousProductInWeyl}%
    Let $R \geq \frac{1}{2}$ and $\pi$ the projection from
    \eqref{Nilpot:WeylProjection}. Then the map $\pi \circ
    \ostar_z$ is continuous.
\end{proposition}
\begin{proof}
    We need to get the estimate on factorizing tensors: Let $p$ be a
    continuous seminorm, $q$ an asymptotic estimate and $k, k',
    \ell, \ell', m, m' \in \mathbb{N}_0$. Then we have to get an
    estimate for
    \begin{align*}
        p_R \left(
            \pi\left(
                Q^k P^{\ell} E^m \ostar_z Q^{k'} P^{\ell'} E^{m'}
            \right)
        \right).
    \end{align*}
    If we calculate the star product explicitly, we see, that we only
    get Lie brackets where we have $P$'s and $Q$'s. Let $r = k + \ell
    + m$ and $s = k' + \ell' + m'$, then we can actually simplify the
    calculations by
    \begin{align*}
        p_R \left(
        \pi(Q^k P^{\ell} E^m
        \ostar_z    Q^{k'} P^{\ell'} E^{m'}
        ) \right)
        & =
        (p_R \circ \pi) \left(
        \sum\limits_{n=0}^{r + s - 1}
        z^n C_n(Q^k P^{\ell} E^m,
        Q^{k'} P^{\ell'} E^{m'})
        \right)
        \\
        & \leq
        \sum\limits_{n=0}^{r + s - 1}
        |z|^n
        (p_R \circ \pi) \left(
        C_n(Q^k P^{\ell} E^m,
        Q^{k'} P^{\ell'} E^{m'})
        \right)
        \\
        & \leq
        \sum\limits_{n=0}^{r + s - 1}
        |z|^n
        (p_R \circ \pi) \left(
        C_n(Q^r, P^s)
        \right)
        \\
        & =
        \sum\limits_{n=0}^{r + s - 1}
        |z|^n
        \frac{r! s!}{(r-n)! (s-n)! n!}
        (p_R \circ \pi) \left(
        Q^{r-n} P^{s-n} E^n
        \right)
        \\
        & =
        \sum\limits_{n=0}^{r + s - 1}
        |z|^n |\mathsf{h}|^n
        \frac{r! s!}{(r-n)! (s-n)! n!}
        p_R \left(
        Q^{r-n} P^{s-n}
        \right)
        \\
        & \leq
        \sum\limits_{n=0}^{r + s - 1}
        |z|^n |\mathsf{h}|^n
        \frac{r! s!}{(r-n)! (s-n)! n!}
        \frac{(r + s - 2n)!^R}{r!^R s!^R}
        p_R \left(Q^{\tensor r} \right)
        p_R \left(P^{\tensor s} \right)
        \\
        & \leq
        \sum\limits_{n=0}^{r + s - 1}
        |z|^n |\mathsf{h}|^n
        \binom{r}{n} \binom{s}{n}
        \frac{(r + s - 2n)!^R n!}{r!^R s!^R}
        p_R \left(Q^{\tensor r} \right)
        p_R \left(P^{\tensor s} \right)
        \\
        & \leq
        \sum\limits_{n=0}^{r + s - 1}
        |z|^n |\mathsf{h}|^n
        \binom{r}{n} \binom{s}{n}
        \frac{(r + s - 2n)!^R n!}{r!^R s!^R}
        p_R \left(Q^{\tensor r} \right)
        p_R \left(P^{\tensor s} \right)
        \\
        & \ot{(a)}{\leq}
        \sum\limits_{n=0}^{r + s - 1}
        |z|^n |\mathsf{h}|^n
        \binom{r}{n} \binom{s}{n}
        \binom{r + s}{s}^R
        \binom{r + s}{2n}^{-R}
        p_R \left(Q^{\tensor r} \right)
        p_R \left(P^{\tensor s} \right)
        \\
        & \leq
        \sum\limits_{n=0}^{r + s - 1}
        (|z| + 1)^n (|\mathsf{h}| + 1)^n
        4^{r + s}
        p_R \left(Q^{\tensor r} \right)
        p_R \left(P^{\tensor s} \right)
        \\
        & \leq
        \underbrace{
        (8 (|z| + 1) (|c| + 1))^{r + s}
        }_{ = c^{r + s}}
        p_R \left(Q^{\tensor r} \right)
        p_R \left(P^{\tensor s} \right)
        \\
        & =
        (c p)_R \left(Q^{\tensor r} \right)
        (c p)_R \left(P^{\tensor s} \right)
        \\
        & \ot{(b)}{ = }
        (c p)_R \left(
        Q^{\tensor k} \tensor
        P^{\tensor \ell} \tensor
        E^{\tensor m} \right)
        (c p)_R \left(
        Q^{\tensor k'} \tensor
        P^{\tensor \ell'} \tensor
        E^{\tensor m'} \right),
    \end{align*}
    where in (a) we expanded the fraction with $(r + s)!^R$ to get the
    two binomial coefficients. It is clear, that this step just works
    for $R \geq \frac{1}{2}$.  In (b) we used the fact that we can ask
    for $p(Q) = p(P) = p(E)$. Now we just need to use
    \begin{equation*}
        \left(
        	Q^{\tensor k} \tensor
        	P^{\tensor \ell} \tensor
        	E^{\tensor m}
        \right)
        \ostar_z
        \left(
        	Q^{\tensor k'} \tensor
        	P^{\tensor \ell'} \tensor
        	E^{\tensor m'}
        \right)
        =
        Q^k P^{\ell} E^m
        \ostar_z
        Q^{k'} P^{\ell'} E^{m'}
    \end{equation*}
    in the first line and we are done, since we can use again the
    infimum argument to expand this estimate to all tensors.
\end{proof}
The previous proposition can be seen as something like the ''finite-dimensional 
version'' of Lemma 3.10 in \cite{waldmann:2014a}, just that we took a large 
detour for proving it. One could, most probably, redo some more results of this 
paper using finite-dimensional versions the Heisenberg algebra and the 
projection onto the Weyl algebra, but this would yield, also most probably, 
nothing new. It is good to know that this connections exists, but it is not 
something which is very helpful to pursue, since an evident generalization to 
infinite dimensions doesn't seem be quite easy.




\section{Banach-Lie algebras}
\label{sec:chap6_TheEProperty}

Now we want to focus a bit on weaker notions than true nilpotency. Since there 
are many of them, we want to focus on the easier case of Banach-Lie algebras, 
where a certain classification and quite a lot of results already exist.


\subsection{Generalizations of nilpotency}

In \cite{mueller:XXXXa}, M\"uller gives a list of weaker forms of nilpotency in 
associative Banach algebras. We can mostly copy the ideas and use it for 
Banach-Lie algebras, too
\begin{definition}
	Let $\lie{g}$ be a Banach-Lie algebra in which the Lie bracket fulfils the 
	estimate
	\begin{equation*}
		\norm{ [\xi, \eta] }
		\leq
		\norm{\xi}
		\norm{\eta}.
	\end{equation*}
	Denote by $\mathbb{B}_1(0)$ all elements $\xi \in \lie{g}$ with 
	$\norm{\xi} = 1$. We say that
	\begin{definitionlist}
		\item
		$\lie{g}$ is topologically nil (or radical, or quasi-nilpotent), if
		every $\xi \in \lie{g}$ is quasi-nilpotent, i.e.
		\begin{equation*}
			\lim_{n \longrightarrow \infty}
			\norm{\ad_{\xi}^n}^{\frac{1}{n}}
			=
			0.
		\end{equation*}
		
		\item
		$\lie{g}$ is uniformly topologically nil, if
		\begin{equation*}
			\lim_{n \longrightarrow \infty}
			\mathcal{N}_1(n)
			=
			0.
		\end{equation*}
		for
		\begin{equation}
			\mathcal{N}_1(n)
			=
			\sup \left\{ 
			\left.
				\norm{ \ad_{\xi}^n}^{\frac{1}{n}} 
			\right|
				\xi \in \mathbb{B}_1(0)
			\right\}.
		\end{equation}
		
		\item
		$\lie{g}$ is topologically nilpotent, if for every sequence
		$(\xi_n)_{n \in \mathbb{N}} \subset \mathbb{B}_1(0)$ we have
		\begin{equation*}
			\lim_{n \longrightarrow \infty}
			\norm{ 
				\ad_{\xi_1} \circ \ldots \circ \ad_{\xi_n}
			}^{\frac{1}{n}}
			=
			0.
		\end{equation*}
		
		\item
		$\lie{g}$ is uniformly topologically nilpotent, if
		\begin{equation*}
			\lim_{n \longrightarrow \infty}
			\mathcal{N}(n)
			=
			0.
		\end{equation*}
		for
		\begin{equation}
			\mathcal{N}(n)
			=
			\sup \left\{ 
			\left.
				\norm{ 
					\ad_{\xi_1} \circ \ldots \circ \ad_{\xi_n}
				}^{\frac{1}{n}} 
			\right|
				\xi_1, \ldots, \xi_n \in \mathbb{B}_1(0)
			\right\}.
		\end{equation}
	\end{definitionlist}
\end{definition}
It is clear that $(ii) \Rightarrow (i)$ and $(iv) \Rightarrow (iii)$. In the 
associative case, we have $(iii) \Leftrightarrow (iv)$ and hence $(iii) 
\Rightarrow (ii)$. Of course, it is a good question, if this remains true for 
Banach-Lie algebras. We have already encountered notion $(i)$: Wojty\'nski 
proved it to be equivalent to the fact that the BCH series converges globally 
in \cite{wojtynski:2000a}. In the following, we will make use of notion $(iv)$: 
we will show, that it is possible to generalize the result of Theorem 
\ref{Nilpot:Thm:ProjLimit} to this case.


\subsection{A new projective Limit}

Let $\lie{g}$ be a uniformly topologically nilpotent Banach-Lie algebra. 
It is clear that we get a monotonously decreasing series 
$(\alpha_n)_{n \in \mathbb{N}}$ by defining
\begin{equation}
	\alpha_n
	=
	\sup_{m \geq n} \mathcal{N}(m).	
\end{equation}
We also have
\begin{equation*}
	\lim_{n \longrightarrow \infty}
	\frac{\log( \alpha_n )}{n}
	=
	\infty.
\end{equation*}

	 - statement for the finite-dimensional case