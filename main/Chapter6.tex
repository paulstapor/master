
%
% Chapter 6 of my master thesis:
% The nilpotent case
%

\chapter{Nilpotent Lie algebras}

At the end of the last chapter, we have seen that the Baker-Campbell-Hausdorff 
series and its convergence play an important role for a topology on the 
universal enveloping algebra. It is thus natural to ask whether things will 
change, if we look at Lie algebras with globally convergent BCH 
series. To make things not too complicated from the beginning, we focus on 
locally convex and truly nilpotent Lie algebras. Recall that a Lie algebra 
$\lie{g}$ is nilpotent, if there exists a $N \in \mathbb{N}$, such that for 
all $n > N$ and all $\xi_1, \ldots, \xi_n \in \lie{g}$ we have
\begin{equation}
	\label{Nilpot:StrongNilpotency}
	\ad_{\xi_1} \circ \cdots \circ \ad_{\xi_n}
	=
	0.
\end{equation}
In the infinite-dimensional case, this is a priori \emph{not} the same as 
\begin{equation}
	\label{Nilpot:WeakNilpotency}
	\left( \ad_{\xi} \right)^n
	=
	0
\end{equation}
for all $\xi \in \lie{g}$ and and $n > N$, but something
stronger. In the case of finite-dimensional Lie algebras, the notions 
\eqref{Nilpot:StrongNilpotency} and \eqref{Nilpot:WeakNilpotency} coincide due 
to the Engel theorem, which makes use of the existence of a 
finite descending series of nilpotent ideals in the Lie algebra. Such a 
terminating series does not need to exist in infinite dimensions. At least as 
soon, as we are in the setting of Banach Lie algebras, those two notions will 
coincide again, but note that one can give different forms of 
\emph{quasi-nilpotency}, which weaken he statements from 
\eqref{Nilpot:StrongNilpotency} and \eqref{Nilpot:WeakNilpotency}, respectively. 
Those generalized notions will not coincide any more. So we have to be careful.


Before we look at this case more closely, we want make a list of things first, 
that we expect to change or not when we go to this more particular setting.
\begin{enumerate}
	\item
	In Example~\ref{LCAna:Ex:HeisenbergAlgebra} we have seen that we can not 
	expect to get a continuous algebra structure for $R < 1$, even for very 
	simple nilpotent, but non-abelian Lie algebras. Therefore, we should not 
	expect to get much larger completions now.
	
	\item
	In \cite{waldmann:2014a}, Waldmann showed that for example the Weyl-Moyal 
	star product converges in the $\Tensor_R$-topology for $R \geq \frac{1}{2}$. 
	This so-called 	Weyl algebra is, however, nothing but a quotient of the 
	Heisenberg 	algebra. It would be interesting to understand this a bit better, 
	since we know that we need $R \geq 1$ for the Heisenberg algebra. The quotient 
	procedure must 
	therefore have some strong influence on this construction. Can we 
	reproduce the value $R \geq \frac{1}{2}$ somehow by dividing out an ideal?
	
	\item
	The argument we used in Propositipon~\ref{LCAna:Prop:NoBetterTopology},
	namely the non-global convergence of BCH, is not given any more. Now, there is 
	no reason to expect that exponentials will not be part 
	of the completion any longer. In this sense, it would be at least nice to have 
	something more than ''just'' $R = 1$. Can we do that?
	
	\item
	As already mentioned, there are generalizations or weaker forms of 
	nilpotency in infinite-dimensions, especially for Banach-Lie algebras, 
	which are equivalent to the usual notion of nilpotency in finite 
	dimensions. If we get a stronger result for nilpotent Lie algebras, will it be 
	possible to extend it to some of these generalizations?
\end{enumerate}
The very fascinating and highly interesting answer to the three questions from 
$(ii) - (iv)$ is: yes, we can. The first section of this chapter will be devoted 
to the question from point $(iii)$: we get a bigger completion by using a 
projective limit. We will also see how to get again the good functorial 
properties we had before. The next section treats another phenomenon, which was 
not there before: we can observe bimodule-structures within $\Sym_R^{\bullet}
(\lie{g})$. In third section, we will reproduce one of the results of 
Waldmann's, at least for the finite-dimensional case. The forth part will take 
care of some generalizations of nilpotentcy for Banach-Lie algebras and will 
extend the result of the projective limit to a particular subcase there.



 
\section{The projective limit} 
\label{sec:chap6_ProjLim} 

\subsection{Continuity of the Product}

As already mentioned, it is possible to extend the continuity result. 
Therefore, we take a locally convex, nilpotent Lie algebra $\lie{g}$ and look 
at
\begin{equation*}
	\Sym_{1^-}^{\bullet}(\lie{g})
	=
	\projlim_{\epsilon \longrightarrow 0}
	\Sym_{1 - \epsilon}^{\bullet}(\lie{g}).
\end{equation*}
A tensor will be in the completed vector space $\widehat{\Sym}_{1^-}^{\bullet}(\lie{g})$, if and only if it lies for every $\epsilon > 0$ in the completion of 
$\Sym_{1 - \epsilon}^{\bullet}(\lie{g})$. Otherwise stated: Let $\algebra{P}$ 
be the set of all continuous seminorms of the Lie algebra $\lie{g}$, then 
\begin{equation*}
	f \in \widehat{\Sym}_{1^-}^{\bullet}(\lie{g})
	\quad \Longleftrightarrow \quad
	p_{1 - \epsilon} (f) 
	< 
	\infty
	\quad
	\forall_{p \in \algebra{P}}
	\forall_{\epsilon > 0}.
\end{equation*}
So, if we want to show, that the Gutt star product is continuous on 
$\widehat{\Sym}_{1^-}^{\bullet}(\lie{g})$, we need to show that for every 
$p \in \algebra{P}$ and  $R < 1$, there exists a $q \in \algebra{P}$ and a 
$R'$ with $R \leq R' < 1$, such that we have for all $x,y \in \Sym^{\bullet}
(\lie{g})$
\begin{equation*}
	p_R \left(
		x \star_z y
	\right)
	\leq
	q_{R'}(x)
	q_{R'}(y).
\end{equation*}
Before we prove the next Theorem, we want to remind that locally convex, 
nilpotent Lie algebras are always AE Lie algebras. So the results we have 
found so far are valid in this case, too.
\begin{theorem}
    \label{Nilpot:Thm:ProjLimit}%
    Let $\lie{g}$ be a nilpotent locally convex Lie algebra with
    continuous Lie bracket and $N \in \mathbb{N}$ such that $N + 1$
    Lie brackets vanish.
    \begin{theoremlist}
	    	\item \label{item:Nilpot:CnOperators}
	    	If $0 \leq R < 1$, the $C_n$-operators are continuous and fulfil the 
	    	estimate
	    	\begin{equation}
	    		\label{eq:Nilpot:CnOperators}
	    		p_R \left(
	    			C_n (x, y)
	    		\right)
	    		\leq
	    		\frac{1}{2 \cdot 8^n}
	    		(32 \E q)_{R + \epsilon}(x)
	    		(32 \E q)_{R + \epsilon}(y),
	    	\end{equation}
	    	for all $x, y \in \Sym_R^{\bullet}(\lie{g})$, where $p$ is a 
	    	continuous seminorm, $q$ an asymptotic estimate and 
	    	$\epsilon = \frac{N - 1}{N}(1 - R)$.
	    	
	    	\item \label{item_Nilpot:SEinsMinus}
	    	The Gutt star product $\star_z$ is continuous for the locally convex 
	    	projective limit $\Sym_{1^-}^\bullet(\lie{g})$ and we have
	    	\begin{equation}
	    		\label{eq:Nilpot:Continuity}
	    		p_R \left(
	    			x \star_z y
	    		\right)
	    		\leq
	    		(c q)_{R + \epsilon}(x)
	    		(c q)_{R + \epsilon}(y)
	    	\end{equation}
	    	with $c = 32 \E (|z| +1)$ and the $\epsilon$ from the first part.
	    	The Gutt star product extends continuously to
	    	$\widehat{\Sym}_R^{\bullet}(\lie{g})$, where it converges absolutely 
	    	and coincides with the formal series.
    \end{theoremlist}
\end{theorem}
\begin{proof}
    We use again $\star_z$ on the whole tensor algebra and compute the 
    estimate for $\xi_1 \tensor \cdots \tensor \xi_k$ and $\eta_1 \tensor \cdots 
    \tensor \eta_\ell$. The important point is that now, we get restrictions for 
    the values of $n$. Recall that $k + \ell - n$ is the symmetric degree of 
    $C_n \left( \xi_1 \tensor \cdots \tensor \xi_k, \eta_1 \tensor \cdots 
    \tensor \eta_\ell \right)$ and that we can have at most $N$ letters in one
    symmetric factor. This means
    \begin{equation*}
	    	(k + \ell - n) N
    		\geq
    		k + \ell
    		\quad 
    		\Longleftrightarrow 
    		\quad
    		n 
    		\leq
    		(k + \ell)
    		\frac{N - 1}{N}.
    \end{equation*}
    Hence we can estimate $n!^{1-R}$ in Equation \eqref{LCAna:CnOperators}: set 
    $\delta = \frac{N - 1}{N}$ and also denote a factorial where we have 
    non-integers, meaning the gamma function. We get
    \begin{align*}
        n!^{1-R}
        & \leq
        (\delta (k + \ell)!)^{1 - R}
        \\
        & \leq
        (\delta (k + \ell))^{(1 - R) \delta (k + \ell)}
        \\
        & \leq
        (k + \ell)^{(1 - R) \delta (k + \ell)}
        \\
        & =
        \left(
            (k + \ell)^{(k + \ell)}
        \right)^{(1 - R) \delta}
        \\
        & \leq
        \left(
            \E^{k + \ell} 2^{k + \ell} k! \ell!
        \right)^{(1-R) \delta}
        \\
        & =
        \left( (2 \E)^{\delta (1-R)} \right)^{k + \ell}
        k!^{\epsilon} \ell!^{\epsilon},
    \end{align*}
    using $\epsilon = \delta (1 - R)$. Hence
    \begin{align*}
        p_R \big(
        	C_n \big(
        		\xi_1 \tensor \cdots \tensor \xi_k, 
        &
        		\eta_1 \tensor \cdots \tensor \eta_\ell
        	\big)
        \big)
        \\
        & \leq
        \frac{
        	\left(
        		(2 \E)^{\delta (1 - R)}
        	\right)^{k + \ell}
        	k!^{\epsilon}
        	\ell!^{\epsilon}
        }{2 \cdot 8^n}
        (16 q)_R \left( \xi_1 \tensor \cdots \tensor \xi_k \right)
        (16 q)_R \left( \eta_1 \tensor \cdots \tensor \eta_\ell \right)
        \\
        & \leq
        \frac{1}{2 \cdot 8^n}
        (c q)_{R + \epsilon} 
        \left( \xi_1 \tensor \cdots \tensor \xi_k \right)
        (c q)_{R + \epsilon} 
        \left( \eta_1 \tensor \cdots \tensor \eta_\ell \right)
    \end{align*}
    with $c = 16 (2 \E)^{\delta (1 - R)} \leq 32 \E$.
    We then get the estimate on all tensors by the infimum argument and extend 
    it to the completion. Note, that for every $R < 1$ we also have 
    $R + \epsilon < 1$ with the $\epsilon = \delta(1-R)$ from above. Iterating 
    this continuity estimate, we get closer and closer to $1$ and it is not 
    possible to repeat this process an arbitrary number of times and stop at 
    some value strictly less than $1$. For the second part, we can conclude 
    analogously to the second part of Theorem~\ref{Thm:LCAna:Continuity1}.
\end{proof}
We have proven one of the four statements. This projective limit is interesting, 
because it has a bigger completion than just $R = 1$. For example, we get the 
following result.
\begin{corollary}
    \label{corollary:NilpotentCase}%
    Let $\lie{g}$ be a nilpotent, locally convex Lie algebra.
    \begin{corollarylist}
    \item \label{item:NilpotentHasExp} 
    	Let $\exp(\xi)$ be the
        exponential series for $\xi \in \lie{g}$, then we have $\exp(t
        \xi) \in \widehat{\Sym}_{1^-}^\bullet(\lie{g})$ for all $t
        \in \mathbb{K}$.
    \item \label{item:NilpotentExpGivesBCH} 
    	For $\xi, \eta \in
        \lie{g}$ and $z \neq 0$ we have $\exp(\xi) \ostar_z
        \exp(\eta) = \exp \left(\frac 1 z \bch{z \xi}{z \eta}
        \right)$.
    \item \label{item:NipotentOneParameterGroups}
    	For $s,t \in
        \mathbb{K}$ and $\xi \in \lie{g}$ we have $\exp(t \xi)
        \ostar_z \exp(s \xi) = \exp ((t + s) \xi)$.
    \end{corollarylist}
\end{corollary}
\begin{proof}
    For the first part, recall that the completion of the projective
    limit $1^-$ will contain all those series $(a_n)_{n \in
      \mathbb{N}_0}$ such that
    \begin{equation*}
        \sum\limits_{n=0}^{\infty}
        a_n n!^{1 - \epsilon} c^n
        <
        \infty
    \end{equation*}
    for all $c > 0$.  This is the case for the exponential series of
    $t \xi$ for $t \in \mathbb{K}$ and $\xi \in \lie{g}$. The second
    part follows from the fact that all the projections $\pi_n$ onto
    the homogeneous subspaces $\Sym_{\pi}^n$ are continuous. The third
    part is then a direct consequence of the second.
\end{proof}
So the exponential series is what we want it to be, somehow. It generates a one 
parameter group. Recall that we can just exponentiate \emph{vectors}. If we wanted 
to exponentiate a quadratic tensor, this would yield something like a Gaussian, 
which is again \emph{not} part of the completion.


In the general case, we could find easier proof by assuming submultiplicativity of 
the seminorms. This is again the case for nilpotent Lie algebras. We get something 
like an alternative version of Lemma~\ref{LCAna:Lemma:PreContinuity2}.
\begin{lemma}
	\label{Nilpot:Lemma:PreContinuity2}
	Let $\lie{g}$ be a locally m-convex, nilpotent Lie algebra such that more than 
	$N$ nested Lie brackets vanish. Let $p$ be a continuous seminorm, 
	$z \in \mathbb{K}$ and $R \geq 0$. Then, for every tensor 
	$x \in \Sym_R^{\bullet}(\lie{g})$ of degree at most $k \in \mathbb{N}$ and 
	$\eta \in \lie{g}$, we have the estimate
	\begin{equation}
		\label{Nilpot:PreContinuity}
		p_R \left( x \star_z \eta \right)
		\leq
		c (k + 1)^R k^{N (1-R)}
		p_R (x) p(\eta)
	\end{equation}
	with the constant $c = \sum_{n = 0}^N \frac{|B_n^*|}{n!} |z|^n$.
\end{lemma}
\begin{proof}
	Again, we do the estimate on factorizing tensors and apply the infimum 
	argument later. So let $\xi_1, \ldots, \xi_k, \eta \in \lie{g}$, $k \in 
	\mathbb{N}$, $p$ a continuous seminorm on $\lie{g}$ and $z \in \mathbb{K}$. 
	Then, we have for $R \geq 0$
	\begin{align*}
		&
		p_R 
		\big(
			\xi_1 \tensor \cdots \tensor \xi_k \star_z \eta
		\big)
		\\
		& =
		\sum\limits_{n = 0}^k
		(k + 1 - n)!^R \binom{k}{n}
		|B_n^*| |z|^n
		p^{k + 1 - n} \left(
			\frac{1}{k!}
			\sum\limits_{\sigma \in S_k}
			\xi_{\sigma(1)} \cdots \xi_{\sigma_{k-n}}
			\left( 
				\ad_{\xi_{\sigma(k-n+1)}} 
				\circ \cdots \circ 
				\ad_{\xi_{\sigma(k)}} 
			\right) (\eta)
		\right)
		\\
		& \leq
		(k + 1)^R
		\sum\limits_{n = 0}^N
		\frac{ k! (k-n)!^R }{ (k-n)! n! }
		|B_n^*| |z|^n
		p\left( \xi_1 \right) \cdots p\left( \xi_k \right) 
		p(\eta)
		\\
		& =
		(k + 1)^R
		p_R \left(
			\xi_1 \tensor \cdots \tensor \xi_k
		\right)
		p(\eta)
		\sum\limits_{n = 0}^N
		\left( \frac{k!}{(k-n)!} \right)^{1-R}
		\frac{|B_n^*| |z|^n}{n!}
		\\
		& \leq
		(k + 1)^R
		k^{N (1-R)}
		p_R \left(
			\xi_1 \tensor \cdots \tensor \xi_k
		\right)
		p(\eta)
		\sum\limits_{n = 0}^N
		\frac{|B_n^*| |z|^n}{n!}.
	\end{align*}
\end{proof}
Now, we can iterate Lemma~\ref{Nilpot:Lemma:PreContinuity2} in the same way 
we did it in Chapter 5:
\begin{proof}[Alternative Proof of Theorem~\ref{Nilpot:Thm:ProjLimit}]
	Again, we do the calculation only on factorizing tensors. We need to 
	transform the $k^{N(1-R)}$ into a very small factorial somehow. This is 
	possible, since for given $N \in \mathbb{N}$ and $0 \leq R < 1$, the 
	sequence
	\begin{equation*}
		\left( \frac{k^N}{\sqrt{k!}} \right)^{1-R}
	\end{equation*}
	converges to $0$ for $k \longrightarrow \infty$ and is therefore bounded 
	by some $\kappa_N > 0$. Hence we get
	\begin{equation*}
		k^{ N (1-R) } 
		\leq
		\kappa_N \sqrt{k!}^{1-R},
	\end{equation*}
	and together with Lemma~\ref{Nilpot:Lemma:PreContinuity2} we find
	\begin{equation*}
		p_R \left( x \star_z \eta \right)
		\leq
		c \kappa_N
		(k + 1)^R k!^{\frac{1-R}{2}} 
		p_R (x) p(\eta)
	\end{equation*}
	for any tensor $x$ of degree at most $k$. Now, we can iterate this result
	for $\xi_1, \ldots, \xi_k, \eta_1, \ldots, \eta_\ell 
	\in \lie{g}$, $R \geq 0$:
	\begin{align*}
		&
		p_R \big(
			\xi_1 \tensor \cdots \tensor \xi_k 
			\star_z 
			\eta_1 \cdots \eta_\ell
		\big)
		\\
		& =
		p_R \left(
			\frac{1}{\ell!}
			\sum\limits_{\tau \in S_\ell}
			\xi_1 \tensor \cdots \tensor  \xi_k 
			\star_z
			\eta_{\tau(1)} \star_z \cdots \star_z \eta_{\tau(\ell)}
		\right)
		\\
		& \leq
		\frac{1}{\ell!}
		\sum\limits_{\tau \in S_\ell}
		c \kappa_N
		(k + \ell)^R
		(k + \ell - 1)!^{\frac{1-R}{2}}
		p_R \left(
			\frac{1}{\ell!}
			\sum\limits_{\tau \in S_\ell}
			\xi_1 \tensor \cdots \tensor \xi_k 
			\star_z
			\eta_{\tau(1)} \star_z \cdots \star_z \eta_{\tau(\ell-1)}
		\right)
		p\left( \eta_{\tau(\ell)} \right)
		\\
		& \leq
		\quad \vdots
		\\
		& \leq
		(c \kappa_N)^{\ell}
		\left(
			\frac{(k + \ell)!}{k!}
		\right)^R
		(k + \ell - 1)!^{\frac{1-R}{2}}
		\ldots
		k!^{\frac{1-R}{2^N}}
		p_R \left( \xi_1 \tensor \cdots \tensor \xi_k \right)
		p\left( \eta_{\tau(1)} \right)
		\cdots
		p\left( \eta_{\tau(\ell)} \right)
		\\
		& \leq
		(c \kappa_N)^{\ell}
		\binom{k + \ell}{k}^R 
		\ell!^R
		(k + \ell)!^{\frac{(2^N - 1) (1-R)}{2^N} }
		p_R \left( \xi_1 \tensor \cdots \tensor \xi_k \right)
		p\left( \eta_{\tau(1)} \right)
		\cdots
		p\left( \eta_{\tau(\ell)} \right)
		\\
		& \leq
		(c \kappa_N)^{\ell}
		2^{(k + \ell) R}
		\ell!^R
		k!^{\frac{(2^N - 1) (1-R)}{2^N} }
		\ell!^{\frac{(2^N - 1) (1-R)}{2^N} }
		2^{ (k + \ell) \frac{(2^N - 1) (1-R)}{2^N} }
		p_R \left( \xi_1 \tensor \cdots \tensor \xi_k \right)
		p\left( \eta_{\tau(1)} \right)
		\cdots
		p\left( \eta_{\tau(\ell)} \right)
		\\
		& \leq
		(2 p)_{R + \epsilon} 
		\left( \xi_1 \tensor \cdots \tensor \xi_k  \right)
		(2 c \kappa_N p)_{R + \epsilon} 
		\left( \eta_1 \tensor \cdots \tensor \eta_\ell \right),
	\end{align*}
	where we have set $\epsilon = \frac{(2^N - 1)(1 - R)}{2^N}$. From this,
	we have $R + \epsilon < 1$ and get the wanted result for the 
	projective limit.
\end{proof}
Again, just like in the case of AE-Lie algebras for $R \geq 1$, we can show that 
the star product depends analytically on the formal parameter.
\begin{proposition}[Dependence on $z$]
    \label{Nilpot:corollary:HolomorphicDependence}%
    Let $\lie{g}$ be a nilpotent locally convex Lie algebra, $0 \leq R < 1$ and $z 
    \in \mathbb{K}$, then for all $x, y \in \widehat{\Sym}_{1^-}^\bullet(\lie{g})$ 
    the map
    \begin{equation}
        \label{Nilpot:Holomorphicity}
        \mathbb{K} \ni z
        \longmapsto
        x \star_z y \in
        \widehat{\Sym}_{1^-}^\bullet(\lie{g})
    \end{equation}
    is analytic with (absolutely convergent) Taylor expansion at $z = 0$ 
    given by Equation~\eqref{Formulas:2MonomialsFormula1}. For 
    $\mathbb{K} = \mathbb{C}$, the collection of algebras $\left\{ \left( 
    \widehat{\Sym}_{1^-}^\bullet(\lie{g}), \star_z \right) \right\}_{z \in 
    \mathbb{C}}$ is an entire holomorphic deformation of the completed 
    symmetric tensor algebra $\left( \widehat{\Sym}_{1^-}^\bullet(\lie{g}), \vee 
    \right)$.
\end{proposition}
\begin{proof}
	The proof is completely analogue to the case of AE Lie algebras when $R = 1$.
\end{proof}


%
% A bit Functioriality also in this case
%

\subsection{Representations and Functoriality}
\label{subsec:NilpotentFunctorialityRepresentations}

In the general AE case, we had some useful results concerning representations 
of Lie algebras and the functorialty of our construction. These results can be 
extended to the projective limit $\Sym_{1^-}^{\bullet}(\lie{g})$.
\begin{proposition}[Universal Property]
	\label{Nilpot:Prop:UnivProperty}
	Let $\lie{g}$ be a locally convex nilpotent Lie algebra, $\algebra{A}$ an 
	associative AE algebra and $\phi \colon \lie{g} \longrightarrow 
	\algebra{A}$ is a continuous homomorphism of Lie algebras. Then, the 
	lifted homomorphisms from $\Sym_{1^-}^{\bullet}(\lie{g})$ and 
	$\algebra{U}(\lie{g}_z)$ to $\algebra{A}$ are continuous.
\end{proposition}
\begin{proof}
	The proof is exactly the same as in the general AE case, since there, 
	$R \geq 0$ was enough.
\end{proof}
Again, this construction will be not a universal in the categorial sense, 
since $\Sym_{1^-}^{\bullet}(\lie{g})$ fails to be AE. But also here, we get 
the case of continuous representations into a Banach space (and in particular 
into a finite-dimensional space) as a corollary.
\begin{corollary}[Continuous Representations]
    \label{Nilpot:Coro:ContinuousRepresentations}%
    Let $\algebra{U}_R(\lie{g}_z)$ the universal enveloping algebra of locally 
    convex nilpotent Lie algebra $\lie{g}$ with bracket scaled by $z \in 
    \mathbb{C}$, then for every continuous 
    representation $\phi$ of $\lie{g}$ into the bounded linear operators 
    $\Bounded(V)$ on a Banach space $V$, the induced homomorphism of 
    associative algebras $\Phi \colon \algebra{U}_R(\lie{g}_z) \longrightarrow 
    \Bounded(V)$ is continuous.
\end{corollary}
We can also extend the functoriality statement to the projective limit, but we 
need to get another version of Lemma~\ref{LCAna:Lemma:LemmaPreContinuityN} for 
nilpotent Lie algebras, since this is the corner stone of the functoriality 
proof.
\begin{lemma}
    \label{Lemma:Nilpot:LemmaPreContinuityN}%
    Let $\lie{g}$ be locally convex nilpotent Lie algebra and $N \in 
    \mathbb{N}$ such that $N + 1$ Lie brackets vanish, $0 \leq R < 1$ and 
    $z \in \mathbb{C}$. Then for $p$ a continuous seminorm, $q$ an
    asymptotic estimate, $n \in \mathbb{N}$ and all $\xi_1, \ldots,
    \xi_n \in \lie{g}$ the estimate
    \begin{equation}
        \label{Nilpot:LemmaPreContinuityN}
        p_R \left(
            \xi_1 \star_z \cdots \star_z \xi_n
        \right)
        \leq
        c^n n!^{R + \epsilon}
        q^n(\xi_1 \tensor \cdots \tensor \xi_n)
    \end{equation}
    holds with $c = 16 \E^2 (|z| + 1)$ and $\epsilon = \frac{N-1}{N}(1 - R)$
    and the estimate is locally uniform in $z$.
\end{lemma}
\begin{proof}
    We take $R < 1$ and go directly into the proof of
    Lemma~\ref{LCAna:Lemma:LemmaPreContinuityN} at
    \eqref{LCAna:PreContinuityIntermediateN}. We know that, since we
    may have at most $N$ brackets, also the values for $\ell$ are
    restricted to
    \begin{equation*}
        \ell
        \leq
        \frac{N-1}{N} n
        =
        \delta n
    \end{equation*}
    Using that in the proof of Lemma~\ref{LCAna:Lemma:LemmaPreContinuityN} 
    leads to
    \begin{align*}
        &p_R \left(
            \xi_1 \star_z \cdots \star_z \xi_n
        \right)
        \\
        &\quad\leq
        \sum\limits_{\ell = 0}^{\delta n}
        (n - \ell)!^R
        \sum\limits_{\substack{
			1 \leq j \leq n-1 \\
			i_j \in \{0, \ldots, j\} \\
			\sum_{j = 1}^{n - 1} i_j = \ell
		}}
        |z|^{\ell}
        \frac{1!  (2 - i_1)! \cdots (n-1 - i_1 - \cdots - i_{n-2})!}
        {(1 - i_1)! \cdots (n-1 - i_1 - \cdots - i_{n-1})!}
        q(\xi_1) \cdots q(\xi_n)
        \\
        &\quad\leq
        \sum\limits_{\ell = 0}^{\delta n}
        (n - \ell)!^R
        \sum\limits_{\substack{
			1 \leq j \leq n-1 \\
			i_j \in \{0, \ldots, j\} \\
			\sum_{j = 1}^{n - 1} i_j = \ell
		}}
        |z|^{\ell} (2 \E)^n \ell!
        q(\xi_1) \cdots q(\xi_n)
        \\
        &\quad\leq
        (2 \E)^n (|z| + 1)^n
        q(\xi_1) \cdots q(\xi_n)
        \sum\limits_{\ell = 0}^{\delta n}
        (n - \ell)!^R \ell!
        \binom{n + \ell - 2}{\ell - 1}
    \end{align*}
    We have
    \begin{equation*}
        \ell!
        =
        \ell!^R
        \ell!^{1-R}
        \leq
        \ell!^R
        \left(
            (\delta n)^{\delta n}
        \right)^{1-R}
        \leq
        \ell!^R
        n^{\delta n (1-R)}
        \leq
        \ell!^R
        n!^{\delta (1 - R)}
        \E^{\delta n (1 - R)}.
    \end{equation*}
    Together with $\ell!^R (n - \ell)!^R \leq n!^R$ this gives
    \begin{align*}
        p_R \left(
            \xi_1 \star_z \cdots \star_z \xi_n
        \right)
        & \leq
        (2 \E)^n (|z| + 1)^n
        n!^R n!^{\delta (1 - R)}
        q(\xi_1) \cdots q(\xi_n)
        \sum\limits_{\ell = 0}^{\delta n}
        \binom{n + \ell - 2}{\ell - 1}
        e^{\delta n (1 - R)}
        \\
        & \leq
        (2 \E)^n (|z| + 1)^n
        \left(\E^{(1-R) \delta}\right)^n
        4^n n!^{R + \epsilon}
        q(\xi_1) \cdots q(\xi_n),
    \end{align*}
    with $\epsilon = \delta (1-R)$. It is clear that for all $R < 1$ we have
    $R + \epsilon < 1$. Set 
    \begin{equation*}
    	c 
    	= 
    	8 \E (|z|+1) \E^{(1-R)\delta}
    	\leq
    	16 \E^2 (|z|+1)
	\end{equation*}
	and note that the estimate is locally uniform in $z$.
\end{proof}
\begin{proposition}
	\label{Nilpot:Prop:Functoriality}
	Let $R \geq 1$, $\lie{g}, \lie{h}$ be two locally convex nilpotent Lie 
	algebras and $\phi \colon \lie{g} \longrightarrow \lie{h}$ a continuous 
	homomorphism between them. Then it lifts to a continuous unital 
	homomorphism of locally convex algebras $\Phi_z \colon 
	\algebra{U}_R(\lie{g}_z) \longrightarrow \algebra{U}_R(\lie{h}_z)$.
\end{proposition}
\begin{proof}
	The proof is analogous to the one of Proposition 
	\ref{LCAna:Prop:Functoriality}.
\end{proof}



\section{Module structures}
\label{sec:chap6_Modules}

The projective limit $1^-$ is not the only additional structure we will get, if 
our Lie algebra $\lie{g}$ is nilpotent. For every $R \in \mathbb{R}$, the 
symmetric tensor algebra $\Sym_R^{\bullet}(\lie{g})$ is a locally convex vector 
space. For $R \geq 0$, the (symmetric) tensor product is continuous, which is 
very important for many estimates, and for $R \geq 1^-$, we have an algebra 
structure. In between however, we have more than ''only'' vector spaces: The 
spaces $\Sym_R^{\bullet}(\lie{g})$ form locally convex bimodules over the 
$\Sym_{R'}^{\bullet}(\lie{g})$ for certain values of $R'$. The next proposition 
will makes this more exact.
\begin{proposition}[Bimodules in $\Sym_R^{\bullet}(\lie{g})$]
	\label{Nilpot:Prop:Bimodules}
	Let $\lie{g}$ be a nilpotent, locally convex Lie algebra, $N \in 
	\mathbb{N}$ such that $N + 1$ Lie brackets vanish, $z \in \mathbb{K}$ and 
	$0 \leq R < 1$. Then, for all $x, y \in \Sym^{\bullet}(\lie{g})$ and every 
	continuous seminorm $p$, we have a continuous seminorm $q$, such that the 
	estimates
	\begin{align}
		\label{Nilpot:BimoduleEstimate1}
		p_R \left(
			x \star_z y
		\right)
		& \leq
		\left(16 q\right)_R(x) 
		\left(16 c q\right)_{R + N(1-R)}(y)
		\\
	\intertext{and}
		\label{Nilpot:BimoduleEstimate2}
		p_R \left(
			x \star_z y
		\right)
		& \leq
		\left(16 c q\right)_{R + N(1-R)}(x)
		\left(16 q\right)_R(y) 
	\end{align}
	hold with $c = (N \E)^{N (1- R)}$.
	Hence, the vector space $\widehat{\Sym}_R^{\bullet}(\lie{g})$ forms a 
	bimodule over the algebra $\widehat{\Sym}_{R + N(1-R)}^{\bullet}(\lie{g})$. 
	In particular, if $\lie{g}$ is 2-step nilpotent, 
	the vector space $\widehat{\Sym}_0^{\bullet}(\lie{g})$ is a 
	$\widehat{\Sym}_1^{\bullet}(\lie{g})$-bimodule.
\end{proposition}
\begin{proof}
	Note that for every degree $n$ we loose, we get a bracket of $\xi$'s and 
	$\eta$'s. Since we can not have too highly nested brackets, we get the 
	following bounds:
	\begin{equation*}
		n
		\leq
		N k
		\quad \text{ and } \quad
		n 
		\leq 
		N \ell.
	\end{equation*}
	We prove the statement only on factorizing tensors again. We 
	want to show Estiamte \eqref{Nilpot:BimoduleEstimate1} and take
	$x = \xi_1 \tensor \cdots \tensor \xi_k$ and $y = \eta_1 \tensor \cdots 
	\tensor \eta_\ell$. So
	\begin{align*}
		(\ell N)!^{1-R}
		& \leq
		(\ell N)^{(\ell N (1-R))}
		\\
		& \leq
		(N \E)^{\ell N (1- R)}
		\ell!^{N(1-R)}.
	\end{align*}
	This allows us again to go back to the proof of Theorem 
	\ref{Nilpot:Thm:ProjLimit} and we find
	\begin{align*}
		p_R \big(
			C_n \big(
				\xi_1 \tensor \cdots \tensor \xi_k,
		&
				\eta_1 \tensor \cdots \tensor \eta_\ell
			\big)
		\big)
		\\
		& \leq 
		\frac{ (N \E)^{\ell N (1- R)} \ell!^{N(1-R)} }
		{2 \cdot 8^n}
		(16 q)_R
		\left(
			\xi_1 \tensor \cdots \tensor \xi_k
		\right)
		(16 q)_R
		\left(
			\eta_1 \tensor \cdots \tensor \eta_\ell
		\right)
		\\
		& \leq
		\frac{1}{2 \cdot 8^n}
		(16 q)_R
		\left(
			\xi_1 \tensor \cdots \tensor \xi_k
		\right)
		(c q)_{R - N(1-R)}
		\left(
			\eta_1 \tensor \cdots \tensor \eta_\ell
		\right)
	\end{align*}
	with $c = 16 (N \E)^{N (1- R)}$. The rest of the proof is analogue to the 
	proofs of the Theorems \ref{Nilpot:Thm:ProjLimit} or
	 \ref{Thm:LCAna:Continuity1}.
\end{proof}
Once again, assuming submultiplicativity of the seminorms, it is possible to give 
an easier proof for Proposition~\ref{Nilpot:Prop:Bimodules} which relies on 
Lemma~\ref{Nilpot:Lemma:PreContinuity2}.
\begin{proof}[Alternative proof for Proposition~\ref{Nilpot:Prop:Bimodules}]
	Again, we do the calculation on factorizing tensors: Let 
	$\xi_1, \ldots, \xi_k, \eta_1, \ldots, \eta_\ell \in \lie{g}$, $R \geq 0$, $k, 
	\ell \in \mathbb{N}$. Using Lemma~\ref{Nilpot:Lemma:PreContinuity2}, we get
	\begin{align*}
		p_R \big(
			\xi_1 \tensor
		&			
			\cdots \tensor \xi_k
			\star_z
			\eta_1 \tensor \cdots \tensor \eta_\ell
		\big)
		\\
		&=
		p_R \left(
			\frac{1}{\ell!}
			\sum\limits_{\tau \in S_\ell}
			\xi_1 \tensor \cdots \tensor \xi_k
			\star
			\eta_{\tau(1)} \star_z \cdots \star_z \eta_{\tau(\ell)}
		\right)
		\\
		& \leq
		c (k + \ell)^R
		(k + \ell - 1)^{N (1-R)}
		\frac{1}{\ell!}
		\sum\limits_{\tau \in S_\ell}
		p_R \left(
			\xi_1 \tensor \cdots \tensor \xi_k
			\star
			\eta_{\tau(1)} \star_z \cdots \star_z \eta_{\tau(\ell-1)}
		\right)
		p\left( \eta_{\tau(\ell)} \right)
		\\
		& \leq
		\quad \vdots
		\\
		& \leq
		c^{\ell}
		\left(
			\frac{(k + \ell)!}{k!}
		\right)^R
		\left(
			\frac{(k + \ell - 1)!}{(k - 1)!}
		\right)^{N (1-R)}
		p_R \left(
			\xi_1 \tensor \cdots \tensor \xi_k
		\right)
		p\left( \eta_{\tau(1} \right)
		\cdots
		p\left( \eta_{\tau(\ell)} \right)
		\\
		& \leq
		c^{\ell}
		2^{k + l} 2^{N (k + \ell)}
		\ell!^{N (1-R)}
		p_R \left(
			\xi_1 \tensor \cdots \tensor \xi_k
		\right)
		p\left( \eta_{\tau(1} \right)
		\cdots
		p\left( \eta_{\tau(\ell)} \right)
		\\
		& =
		\left(2^{N + 1} p\right)_R 
		\left(
			\xi_1 \tensor \cdots \tensor \xi_k
		\right)
		\left(2^{N + 1} c p\right)_{R + N(1-R)}
		\left(
			\eta_1 \tensor \cdots \tensor \eta_\ell
		\right).
	\end{align*}
	The proof of the second estimate is analogous.
\end{proof}
\begin{remark}[Possible extensions]
	This result immediately raises new questions, like the one about possible 
	generalizations to ''weaker forms'' of nilpotency, for example. They may be 
	issues of some future work, but can not
	be addressed here, since we rather want to present something like  
	the ''big picture'', instead of getting 
	lost in its details too much. There are, without any doubt, questions that 
	are more significant than extending those estimates to very special cases 
	and 	finding sharp bounds there, although this is interesting and 
	important, 
	too.
\end{remark}
Though it seems clear from the construction that these bimodules can not be 
there for general Lie algebras, we can give a concrete counter-example, which 
shows that there are Lie algebras, which don not allow them.
\begin{example}
	\label{Nilpot:Ex:NoModulesInGeneral}
	Choose $R < 1$ and take $\lie{g} = \mathbbm{R}^3$ with the basis $e_1, e_2, 
	e_3$ and the vector product as Lie bracket:
	\begin{equation*}
		[e_1, e_2] 
		= 
		e_3 
		\qquad 
		[e_2, e_3] 
		= 
		e_1 
		\qquad 
		[e_3, e_1] 
		= 
		e_2
	\end{equation*}
	Again, we take a $\ell^1$-norm $n$ such that $n(e_1) = n(e_2) = n(e_3) 
	= 1$. It has the nice property that for $k, \ell, m \in \mathbb{N}$ we get
	on the projective tensor product
	\begin{equation*}
		n^{k + \ell + m} \left(
			e_1^k e_2^{\ell} e_3^m
		\right)
		=
		1.
	\end{equation*}
	Now choose an $\epsilon > 0$ such that $R + \epsilon < 1$ and we define the 
	sequence $(a_k)_{k \in \mathbb{N}}$
	\begin{equation*}
		a_k 
		= 
		\frac{1}{k!^R} e_1^k,
	\end{equation*}
	for which we get $\lim_{k \longrightarrow \infty} n_R(a_k) = 0$. Now, we want 
	to show that $a_k \star_z e_2$ grows faster than exponentially:
	\begin{align*}
		n_R \left( a_k \star_z e_2 \right) 
		& = 
		n_R \left( 
			\sum\limits_{j = 0}^k 
			\binom{k}{j} B_j^* 
			\frac{1}{k!^{R + \epsilon}} 
			e_1^{n-j} 
			\left( 
				\operatorname{ad}_{e_1} 
			\right)^j(e_2) 
		\right)
		\\
		& =
		\sum\limits_{j = 0}^k 
		\binom{k}{j} 
		|B_j^*| 
		\frac{1}{k!^{R + \epsilon}} 
		(k-j+1)!^R 
		\underbrace{
			n^{k-j} \left( e_1 (e_2 \wedge e_3) \right)
		}_{ = 1}
		\\
		& = 
		\sum\limits_{j = 0}^k 
		(k-j+1)^R 
		\binom{k}{j}
		\frac{|B_j^*|}{j!} 
		\frac{(k-j)!^R j^R}{k!^{R + \epsilon}} 
		j!^{1-R}
		\\
		& = 
		\sum\limits_{j=0}^k 
		(k-j+1)^R 
		\binom{k}{j}^{1-R} 
		\frac{|B_j^*|}{j!} 
		\frac{j!^{1-R}}{k!^\epsilon}
		\\
		& \geq 
		\sum\limits_{j=0}^k 
		\frac{|B_j^*|}{j!} 
		\frac{j!^{1-R}}{k!^\epsilon}
		\\
		& \geq 
		\frac{|B_k^*|}{k!^{R + \epsilon}}.
	\end{align*}
	We know that for $R +\epsilon < 1$ and any $c > 0$
	\begin{equation*}
		\limsup_{n \longrightarrow \infty}
		\frac{|B_n^*|}{c^n n!^{R + \epsilon}}
		=
		\infty.
	\end{equation*}
	Hence the limes superior of $n_R \left( a_k \star_z e_2 \right) $ grows 
	faster than any exponential function and can not be absorbed into the seminorm 
	of $e_2$. So the multiplication in the module can not be continuous.
\end{example}



\section{The Heisenberg and the Weyl algebra}
\label{sec:chap6_HeisenbergWeyl}

Now we want to see how we get the link to the Weyl algebra from
\cite{waldmann:2014a}, since we have something like a discrepancy for the
parameter $R$ concerning the continuity of the product in the Weyl and the 
Heisenberg algebra. In the following, we will show that this gap actually makes 
a lot of sense. For simplicity, we consider the easiest case of
the Weyl/Heisenberg algebra with two generators $Q$ and $P$, but the 
calculation for the Heisenberg [Weyl] algebra in $2n + 1$ [$2n$] dimensions 
is done exactly in the same way.
Recall that the Weyl algebra is a quotient of the enveloping algebra of the 
Heisenberg algebra $\lie{h}$ which one gets from dividing out its center. So 
let $h \in \mathbb{K}$ and we have a projection
\begin{equation}
    \label{Nilpot:WeylProjection}
    \pi \colon
    \Sym_R^\bullet(\lie{h})
    \longrightarrow
    \mathcal{W}_R(\lie{h})
    =
    \frac{\Sym_R^\bullet(\lie{h})}
    {\langle E - h \Unit \rangle}
\end{equation}
Of course we want to know if this projection is continuous.
\begin{proposition}
    \label{proposition:ProjectionWeylContinuous}%
    The projection $\pi$ is continuous for $R \geq 0$.
\end{proposition}
\begin{proof}
    We extend $\pi$ to the whole tensor algebra by symmetrizing
    beforehand. Let then $p$ be a continuous seminorm on $\lie{h}$, $k,
    \ell, m \in \mathbb{N}_0$. We have
    \begin{align*}
        p_R(\pi (
        	Q^{\tensor k} \tensor
        	P^{\tensor \ell} \tensor
        	E^{\tensor m}
        ) )
        & =
        p_R( Q^k P^{\ell} h^m )
        \\
        & =
        |h|^m (k + \ell)!^R
        p^{k + \ell}(Q^k P^{\ell})
        \\
        & \leq
        (|h| + 1)^{k + \ell + m}
        (k + \ell + m)!^R
        p(Q)^k p(P)^{\ell} p(E)^m
        \\
        & =
        ((|h| + 1) p)_R
        (Q^{\tensor k} \tensor
        P^{\tensor \ell} \tensor
        E^{\tensor m}).
    \end{align*}
    Then we do the usual infimum argument and have the result on
    arbitrary tensors again.
\end{proof}


To establish the link to the continuity results of the Weyl algebra,
we need more: $\pi \circ \ostar_z$ should to be continuous for $R \geq \frac 
1 2$.
\begin{proposition}
    \label{proposition:ContinuousProductInWeyl}%
    Let $R \geq \frac{1}{2}$ and $\pi$ the projection from
    \eqref{Nilpot:WeylProjection}. Then the map $\pi \circ
    \ostar_z$ is continuous.
\end{proposition}
\begin{proof}
    Since we are in finite dimensions, we can choose a
    submultiplicative norm $p$ with $p(Q) = p(P) = p(E)$ without
    restriction. Moreover, let $k, k', \ell, \ell', m, m' \in
    \mathbb{N}_0$. Then we have to get an estimate for $p_R \left(
        \pi\left( Q^k P^{\ell} E^m \ostar_z Q^{k'} P^{\ell'} E^{m'}
        \right) \right)$.  If we calculate the star product
    explicitly, we see, that we only get Lie brackets where we have
    $P$'s and $Q$'s. Let $r = k + \ell + m$ and $s = k' + \ell' + m'$,
    then we can actually simplify the calculations by
    \begin{align*}
        p_R \left(
        \pi(Q^k P^{\ell} E^m
        \ostar_z    Q^{k'} P^{\ell'} E^{m'}
        ) \right)
        & =
        (p_R \circ \pi) \left(
        \sum\limits_{n=0}^{r + s - 1}
        z^n C_n(Q^k P^{\ell} E^m,
        Q^{k'} P^{\ell'} E^{m'})
        \right)
        \\
        & \leq
        \sum\limits_{n=0}^{r + s - 1}
        |z|^n
        (p_R \circ \pi) \left(
        C_n(Q^k P^{\ell} E^m,
        Q^{k'} P^{\ell'} E^{m'})
        \right)
        \\
        & \leq
        \sum\limits_{n=0}^{r + s - 1}
        |z|^n
        (p_R \circ \pi) \left(
        C_n(Q^r, P^s)
        \right)
        \\
        & =
        \sum\limits_{n=0}^{r + s - 1}
        |z|^n
        \frac{r! s!}{(r-n)! (s-n)! n!}
        (p_R \circ \pi) \left(
        Q^{r-n} P^{s-n} E^n
        \right)
        \\
        & =
        \sum\limits_{n=0}^{r + s - 1}
        |z|^n |h|^n
        \frac{r! s!}{(r-n)! (s-n)! n!}
        p_R \left(
        Q^{r-n} P^{s-n}
        \right)
        \\
        & \leq
        \sum\limits_{n=0}^{r + s - 1}
        |z|^n |h|^n
        \frac{r! s!}{(r-n)! (s-n)! n!}
        \frac{(r + s - 2n)!^R}{r!^R s!^R}
        p_R \left(Q^{\tensor r} \right)
        p_R \left(P^{\tensor s} \right)
        \\
        & \leq
        \sum\limits_{n=0}^{r + s - 1}
        |z|^n |h|^n
        \binom{r}{n} \binom{s}{n}
        \frac{(r + s - 2n)!^R n!}{r!^R s!^R}
        p_R \left(Q^{\tensor r} \right)
        p_R \left(P^{\tensor s} \right)
        \\
        & \leq
        \sum\limits_{n=0}^{r + s - 1}
        |z|^n |h|^n
        \binom{r}{n} \binom{s}{n}
        \frac{(r + s - 2n)!^R n!}{r!^R s!^R}
        p_R \left(Q^{\tensor r} \right)
        p_R \left(P^{\tensor s} \right)
        \\
        & \ot{(a)}{\leq}
        \sum\limits_{n=0}^{r + s - 1}
        |z|^n |h|^n
        \binom{r}{n} \binom{s}{n}
        \binom{r + s}{s}^R
        \binom{r + s}{2n}^{-R}
        p_R \left(Q^{\tensor r} \right)
        p_R \left(P^{\tensor s} \right)
        \\
        & \leq
        \sum\limits_{n=0}^{r + s - 1}
        (|z| + 1)^n (|h| + 1)^n
        4^{r + s}
        p_R \left(Q^{\tensor r} \right)
        p_R \left(P^{\tensor s} \right)
        \\
        & \ot{(b)}{ \leq }
        \underbrace{
        (8 (|z| + 1) (|h| + 1))^{r + s}
        }_{ = \tilde{c}^{r + s}}
        p_R \left(Q^{\tensor r} \right)
        p_R \left(P^{\tensor s} \right)
        \\
        & =
        (\tilde{c} p)_R \left(Q^{\tensor r} \right)
        (\tilde{c} p)_R \left(P^{\tensor s} \right)
        \\
        & \ot{(c)}{ = }
        (\tilde{c} p)_R \left(
        Q^{\tensor k} \tensor
        P^{\tensor \ell} \tensor
        E^{\tensor m} \right)
        (\tilde{c} p)_R \left(
        Q^{\tensor k'} \tensor
        P^{\tensor \ell'} \tensor
        E^{\tensor m'} \right),
    \end{align*}
    where we have set $c = 8 (|z| + 1) (|h| + 1)$. We rearranged the factorials in 
    ($a$) and used $R \geq \frac{1}{2}$. The estimates ($b$) are the standard 
    binomial coefficient estimates. In ($c$) we used $p(Q) = p(P) = p(E)$. Now
    we just use
    \begin{equation*}
        \left(
        	Q^{\tensor k} \tensor
        	P^{\tensor \ell} \tensor
        	E^{\tensor m}
        \right)
        \ostar_z
        \left(
        	Q^{\tensor k'} \tensor
        	P^{\tensor \ell'} \tensor
        	E^{\tensor m'}
        \right)
        =
        Q^k P^{\ell} E^m
        \ostar_z
        Q^{k'} P^{\ell'} E^{m'}
    \end{equation*}
    and the infimum argument to expand this estimate to all tensors.
    This concludes the proof.
\end{proof}
The previous proposition can be seen as something like the ''finite-dimensional 
version'' of Lemma 3.10 in \cite{waldmann:2014a}, just that we took a large 
detour for proving it. One could, most probably, redo some more results of this 
paper using finite-dimensional versions the Heisenberg algebra and the 
projection onto the Weyl algebra, but this would yield, also most probably, 
nothing new. It is good to know that this connections exists, but it is not 
something which is very helpful to pursue, since an evident generalization to 
infinite dimensions does not seem be obvious.




\section{Banach-Lie algebras}
\label{sec:chap6_TheEProperty}

Now we want to focus a bit on weaker notions than true nilpotency. Since there 
are many of them, we want to restrict to the easier case of Banach-Lie algebras, 
where a somewhat developed theory already exists.


\subsection{Generalizations of nilpotency}

In \cite{muller:1994a}, M\"uller gives a list of weaker forms of nilpotency for 
associative Banach algebras. We can mostly copy the ideas and use them for 
Banach-Lie algebras, too
\begin{definition}
	Let $\lie{g}$ be a Banach-Lie algebra in which the Lie bracket fulfils the 
	estimate
	\begin{equation*}
		\norm{ [\xi, \eta] }
		\leq
		\norm{\xi}
		\norm{\eta}.
	\end{equation*}
	Denote by $\mathbb{B}_1(0)$ all elements $\xi \in \lie{g}$ with 
	$\norm{\xi} = 1$. We say that
	\begin{definitionlist}
		\item
		$\lie{g}$ is topologically nil (or radical, or quasi-nilpotent), if
		every $\xi \in \lie{g}$ is quasi-nilpotent, i.e.
		\begin{equation*}
			\lim_{n \longrightarrow \infty}
			\norm{\ad_{\xi}^n}^{\frac{1}{n}}
			=
			0.
		\end{equation*}
		
		\item
		$\lie{g}$ is uniformly topologically nil, if
		\begin{equation*}
			\lim_{n \longrightarrow \infty}
			\mathcal{N}_1(n)
			=
			0.
		\end{equation*}
		for
		\begin{equation}
			\mathcal{N}_1(n)
			=
			\sup \left\{ 
			\left.
				\norm{ \ad_{\xi}^n}^{\frac{1}{n}} 
			\right|
				\xi \in \mathbb{B}_1(0)
			\right\}.
		\end{equation}
		
		\item
		$\lie{g}$ is topologically nilpotent, if for every sequence
		$(\xi_n)_{n \in \mathbb{N}} \subset \mathbb{B}_1(0)$ we have
		\begin{equation*}
			\lim_{n \longrightarrow \infty}
			\norm{ 
				\ad_{\xi_1} \circ \ldots \circ \ad_{\xi_n}
			}^{\frac{1}{n}}
			=
			0.
		\end{equation*}
		
		\item
		$\lie{g}$ is uniformly topologically nilpotent, if
		\begin{equation*}
			\lim_{n \longrightarrow \infty}
			\mathcal{N}(n)
			=
			0.
		\end{equation*}
		for
		\begin{equation}
			\mathcal{N}(n)
			=
			\sup \left\{ 
			\left.
				\norm{ 
					\ad_{\xi_1} \circ \ldots \circ \ad_{\xi_n}
				}^{\frac{1}{n}} 
			\right|
				\xi_1, \ldots, \xi_n \in \mathbb{B}_1(0)
			\right\}.
		\end{equation}
	\end{definitionlist}
\end{definition}
It is clear that $(ii) \Rightarrow (i)$ and $(iv) \Rightarrow (iii)$. In the 
associative case, we have $(iii) \Leftrightarrow (iv)$ and hence $(iii) 
\Rightarrow (ii)$. Of course, it is a good question, if this remains true for 
Banach-Lie algebras. We have already encountered notion $(i)$: in 
\cite{wojtynski:1998a} Wojty\'nski gave a proof that it is equivalent to the 
global convergence of the BCH series. In the following, we will make use of notion 
$(iv)$: we will show, that it is possible to generalize the result of Theorem 
\ref{Nilpot:Thm:ProjLimit} to this case.



\subsection{An adapted $\Tensor_R$-topology}

The idea will be to change the $\Tensor_R$-topology a bit: Instead of taking 
$n!^R$ as weights with $0 \leq R < 1$, we take sequence $(\alpha_n)_{n \in 
\mathbb{N}}$ with a certain asymptotic 
behaviour for $n \longrightarrow \infty$ and use $\frac{n!}{\alpha_n}$ as 
weights. This will generalize the idea of $n!^R$ and will be the starting point 
for estimates.

First, we observe that every uniformly topologically nilpotent Banach-Lie 
algebra $\lie{g}$ comes with a characteristic, monotonously decreasing sequence
\begin{equation}
	\label{Nilpot:CharDownSequence}
	\omega_n
	=
	\sup_{m \geq n} \mathcal{N}(m).	
\end{equation}
If there exists a $N \in \mathbb{N}$, such that $\omega_n = 0$ for all $n \geq 
N$, then $\lie{g}$ is actually nilpotent and we can use the results of the 
first section in this chapter. We may hence restrict to those Banach-Lie 
algebras, where we have $\omega_n > 0$ for all $n \in \mathbb{N}$. This allows 
the next definition.
\begin{definition}[Rapidly increasing sequences]
	Let $\lie{g}$ be a uniformly topologically nilpotent Banach-Lie algebra and 
	$(\omega_n)_{n \in \mathbb{n}}$ the sequence defined in 
	\eqref{Nilpot:CharDownSequence}. Then we 
	define the characteristic sequence $(\chi_n^{\lie{g}})_{n \in \mathbb{N}}$ 
	of $\lie{g}$ by
	\begin{equation}
		\label{Nilpot:CharSequence}
		\chi_n^{\lie{g}}
		=
		\max
		\left\{ 
			\frac{1}{\omega_n}
			,
			2
		\right\}.
	\end{equation}
	We moreover say, that a sequence $(\alpha_n)_{n \in \mathbb{N}}$ in 
	$(1, \infty)$ is $\lie{g}$-rapidly increasing, if it fulfils the following 
	properties:
	\begin{definitionlist}
		\item
		It grows fast than exponentially, i.e.
		\begin{equation*}
			\lim_{n \longrightarrow \infty}
			\frac{\log \left(\alpha_n \right)}{n}
			=
			\infty.
		\end{equation*}
		
		\item
		There exists a constant $c > 0$, such that
		\begin{equation*}
			\alpha_n 
			\leq 
			c^n  \chi_n^{\lie{g}}.
		\end{equation*}
	\end{definitionlist}
	We denote by $\mathfrak{I}_{\lie{g}}$ the set of all $\lie{g}$-rapidly 
	increasing sequences.
\end{definition}
Clearly, $(\chi_n^{\lie{g}})_{n \in \mathbb{N}}$ is a $\lie{g}$-rapidly 
increasing sequence itself. Note that we get with this definition for every 
sequence $(\xi_n)_{n \in \mathbb{N}} \subset \mathbb{B}_1(0)$
\begin{equation}
	\label{Nilpot:ChiEstimate}
	\norm{
		[ \ldots [[\xi_1, \xi_2], \xi_3], \ldots \xi_n]
	}
	\leq
	\frac{2^n}{\chi^{\lie{g}}_n}.
\end{equation}
\begin{remark}
	The number $2$ in \eqref{Nilpot:CharSequence} may look a bit confusing at 
	the first sight, since it is somehow arbitrary, but for technical reasons, 
	we will need $\chi^{\lie{g}}_n > 1$ later. 
	So actually every real number $c > 1$ 
	could have been used there. In this sense, the previous definition is rather 
	a technical tool than a ''general idea''.
\end{remark}
Each $(\alpha_n)_{n \in \mathbb{N}} \in \algebra{I}_{\lie{g}}$ gives a continuous 
seminorm.
\begin{definition}[Adapted seminorms]
	Let $(\alpha_n)_{n \in \mathbb{N}} \in \mathfrak{I}_{\lie{g}}$. Then
	\begin{equation*}
		p_{\alpha}
		=
		\sum\limits_{n = 0}^{\infty}
		\frac{n!}{\alpha_n}
		\norm{\cdot}^{\tensor[\pi] n}
	\end{equation*}
	defines a seminorm on the tensor algebra with the projective tensor product 
	$\Tensor_{\pi}^{\bullet}(\lie{g})$. We denote the set of all continuous 
	seminorms with respect to those	coming from such sequences by $\algebra{P}$.
\end{definition}
It will be important to see that every rapidly increasing sequence 
$(\alpha_n)_{n \in \mathbb{N}}$ yields a continuous function $f_{\alpha}$ by
\begin{equation}
	\label{Nilpot:IncreasingFunction}
	f_{\alpha}
	\colon
	\mathbb{R}_0^+
	\longrightarrow
	\mathbb{R}^+
	, \quad
	f_{\alpha}(0)
	=
	2
	,\
	f_{\alpha}(n)
	=
	\log \left( \alpha_n \right)
	,\
	\forall_{n \in \mathbb{N}},
\end{equation}
and linear interpolation between the values at the integers. The idea behind is 
that this will allow us to use a technical lemma, which we now introduce. This 
lemma shows that there are always ''many'' rapidly increasing functions in a 
certain sense. It is taken from a work \cite{mitiagin.rolewicz.zelazko:1962a} by 
Mitiagin, Rolewicz and \.{Z}elazko, where it is stated in Lemma~2.1 and Lemma~2.2.
\begin{lemma}
	\label{Nilpot:Lemma:MRZGrothLemma}
	Let $f\colon \mathbb{R}_0^+ \longrightarrow \mathbb{R}^+$ be a continuous 
	functions, such that
	\begin{equation}
		\label{Nilpot:GrowthProperty}
		\lim_{n \longrightarrow \infty}
		\frac{f(x)}{x}
		=
		\infty.
	\end{equation}
	Then there exists a convex, continuous function $g \colon \mathbb{R}_0^+ 
	\longrightarrow \mathbb{R}^+$, fulfilling \eqref{Nilpot:GrowthProperty} and
	\begin{equation}
		\label{Nilpot:SplittingProperty}
		g \left(
			t_1 + \cdots + t_n
		\right)
		\leq
		8 \left(
			g \left( t_1 \right)
			+ \cdots +
			g \left( t_n \right)
		\right)
		+
		f(n)
		, \quad
		\forall_{n \in \mathbb{N}}
		\text{ and all }
		t_i \in \mathbb{R}_0^+.
	\end{equation}
\end{lemma}
\begin{proof}
	We refer the reader to the paper \cite{mitiagin.rolewicz.zelazko:1962a},
	since we just want to use this lemma and do not want to go too much into
	details here.
\end{proof}
Note that if we have $(\alpha_n)_{n \in \mathbb{N}} \in \mathfrak{I}_{\lie{g}}$, 
then we can apply Lemma~\ref{Nilpot:Lemma:MRZGrothLemma} to 
the function $f_{\alpha}$, which is defined according to Equation 
\ref{Nilpot:IncreasingFunction}.



\subsection{A new continuity result}

Now, we have finally prepared our toolbox well enough to prove a new result.
\begin{proposition}
	\label{Nilpot:Prop:TopNilBanachLie}
	Let $\lie{g}$ be a uniformly topologically nilpotent Banach-Lie algebra,
	$(\alpha_n)_{n \in \mathbb{N}} \in \mathfrak{I}_{\lie{g}}$, $p_{\alpha}$ the 
	corresponding seminorm according to \eqref{Nilpot:IncreasingFunction} and 
	$z \in \mathbb{K}$. Then, there exists a series $(\beta_n)_{n \in \mathbb{N}} 
	\in \mathfrak{I}_{\lie{g}}$, such that for all $x,y \in \Tensor^{\bullet}
	(\lie{g})$ we have the estimate
	\begin{equation}
		\label{Nilpot:TopNilBanachLie}
		p_{\alpha} \left(
			x \star_z y
		\right)
		\leq
		(c p)_{\beta} (x)
		(c p)_{\beta} (y)
	\end{equation}
	with a $c > 0$, which only depends on $\alpha, z$ and the Lie algebra 
	$\lie{g}$.
\end{proposition}
\begin{proof}
	We compute the estimate on factorizing tensors and extend it with the 
	infimum argument later. Let hence $k, \ell \in \mathbb{N}$ and 
	$\xi_1, \ldots, \xi_k, \eta_1, \ldots, \eta_\ell \in \lie{g}$. 
	We need to estimate the $C_n$-operators for $n = 0, 1, \ldots, k + \ell - 1$. 
	Therefore we note $r = k + \ell - n$ and use the short-hand notation for the 
	sums, which appear in the Gutt star product, again.
	We take $(\alpha_n)_{n \in \mathbb{N}} \in \mathfrak{I}_{\lie{g}}$ and get 
	for $p_{\alpha}$
	\begin{align*}
		p_{\alpha} \big(
			C_n \big( 
			\xi_1 \tensor \cdots \tensor \xi_k,
		&
			\eta_1 \tensor \cdots \tensor \eta_\ell 
			\big)
		\big)
		\\
		& =
		p_{\alpha}
		\bigg(
         	\frac{1}{r!}
			\sum\limits_{\sigma, \tau}
			\sum\limits_{a_i, b_j}
			\bchtilde{a_1}{b_1}{\xi_{\sigma(i)}}{\eta_{\tau(j)}}
			\cdots
			\bchtilde{a_r}{b_r}{\xi_{\sigma(i)}}{\eta_{\tau(j)}}
        \bigg)
        \\
        & \leq 
    		\frac{1}{r!}
    		\frac{r!}{\alpha_r}
    		\sum\limits_{\sigma, \tau}
			\sum\limits_{a_i, b_j}
       	\norm{
    			\bchtilde{a_1}{b_1}{\xi_{\sigma(i)}}{\eta_{\tau(j)}}
    		}
    		\cdots
    		\norm{
    			\bchtilde{a_r}{b_r}{\xi_{\sigma(i)}}{\eta_{\tau(j)}}
    		}
        \\
        & \leq
    	\frac{k! \ell!}{\alpha_r}
    	\sum\limits_{a_i, b_j}
       	\frac{2}{\chi_{a_1 + b_1}^{\lie{g}}}
       	\ldots
       	\frac{2}{\chi_{a_r + b_r}^{\lie{g}}}
       	\norm{\xi_1} \cdots \norm{\xi_k}
       	\norm{\eta_1} \cdots \norm{\eta_\ell},
    \end{align*}
	where we have used the estimate from Lemma \ref{LCAna:Lemma:BCHTermsEstiamte}
	 $(\ref{Item:BCHEstimate})$ and Estimate \eqref{Nilpot:ChiEstimate}
	in the last step. Rearranging this, we have
	\begin{equation*}
		p_{\alpha} \left(
			C_n\left( \xi^{\tensor k}, \eta^{\tensor \ell} \right)
		\right)
		\leq
    	k! \ell! 2^r
       	\norm{\xi_1} \cdots \norm{\xi_k}
       	\norm{\eta_1} \cdots \norm{\eta_\ell}
    		\sum\limits_{\substack{a_1, b_1, \ldots, a_r, b_r \geq 0 \\
        		a_i + b_i \geq 1 \\
        		a_1 + \ldots + a_r = k \\
        		b_1 + \ldots + b_r = \ell
       	}}
       	\frac{1}
       	{
       		\alpha_r
       		\cdot
       		\chi^{\lie{g}}_{a_1 + b_1}
       		\cdots
       		\chi^{\lie{g}}_{a_r + b_r}
       	},
	\end{equation*}
	and we would like to find a $(\beta_n)_{n \in \mathbb{N}} \in 
	\mathfrak{I}_{\lie{g}}$ such that
	\begin{equation}
		\label{Nilpot:SavingSequence}
		\sup
		\left\{
		\left.
			\frac{\beta_k \cdot \beta_{\ell}}
	       	{
	       		\alpha_r
	       		\cdot
	       		\chi^{\lie{g}}_{a_1 + b_1}
	       		\ldots
	       		\chi^{\lie{g}}_{a_r + b_r}
	       	}
	    \right|
	    		k, \ell \in \mathbb{N},\
	    		a_i + b_i \geq 1,\
        		\sum_i a_i = k,\
        		\sum_j b_j = \ell
		\right\}
		\leq
		\kappa^{k + \ell}
	\end{equation}
	for some $\kappa > 0$, just depending on $(\alpha)$. Then we would have
	\begin{align*}
		p_{\alpha} \big(
			C_n \big( 
			\xi_1 \tensor \cdots \tensor \xi_k,
		&
			\eta_1 \tensor \cdots \tensor \eta_\ell 
			\big)
		\big)
		\\
		& \leq
    	\frac{k! \ell!}{\beta_k \beta_{\ell}} 
    	2^r
       	\norm{\xi_1} \cdots \norm{\xi_k}
       	\norm{\eta_1} \cdots \norm{\eta_\ell}
    		\sum\limits_{\substack{a_1, b_1, \ldots, a_r, b_r \geq 0 \\
        		a_i + b_i \geq 1 \\
        		a_1 + \ldots + a_r = k \\
        		b_1 + \ldots + b_r = \ell
       	}}
       	\kappa^{k + \ell}
       	\\
       	& =
       	2^{-n}
       	(2 \kappa)^{k + \ell}
       	p_{\beta}
       	\left( \xi_1 \tensor \cdots \tensor \xi_k \right)
       	p_{\beta}
       	\left( \eta_1 \tensor \cdots \tensor \eta_\ell \right)
       	\sum\limits_{\substack{a_1, b_1, \ldots, a_r, b_r \geq 0 \\
        		a_i + b_i \geq 1 \\
        		a_1 + \ldots + a_r = k \\
        		b_1 + \ldots + b_r = \ell
       	}}
       	1
       	\\
       	& \leq
       	\frac{1}{2 \cdot 8^n} 
       	(16 \kappa)^{k + \ell}
       	p_{\beta}
       	\left( \xi_1 \tensor \cdots \tensor \xi_k \right)
       	p_{\beta}
       	\left( \eta_1 \tensor \cdots \tensor \eta_\ell  \right)
       	\\ 
       	& =
       	\frac{1}{2 \cdot 8^n} 
       	(16 \kappa p)_{\beta}
       	\left( \xi_1 \tensor \cdots \tensor \xi_k \right)
       	(16 \kappa p)_{\beta}
       	\left( \eta_1 \tensor \cdots \tensor \eta_\ell  \right).
    \end{align*}
    From this we could conclude analogously to the procedure in the proof of 
    Theorem~\ref{Thm:LCAna:Continuity1} and the statement would follow.
    However, we need to show the existence of a $(\beta_n)_{n \in \mathbb{N}} 
    \in \algebra{I}_{\lie{g}}$ and a $\kappa > 0$, such that 
    \eqref{Nilpot:SavingSequence} holds.
    \begin{lemma}
    		For $f_{\alpha}$ defined as in \eqref{Nilpot:IncreasingFunction}, we 
    		take the function $g$ we get from 
    		Lemma~\ref{Nilpot:Lemma:MRZGrothLemma}. Then the sequence 
    		$((\beta_n)_{n \in \mathbb{N}}$ defined by
    		\begin{equation}
    			\beta_n
    			=
    			\exp \left(
    				\frac{g(n)}{8}
    			\right)
    		\end{equation}
    		and $\kappa = c \E^{8 g(1)}$ fulfil \eqref{Nilpot:SavingSequence}, 
    		where $c > 0$ is a constant such that $\alpha_n \leq c^n 
    		\chi^{\lie{g}}_n$.
    \end{lemma}
    \begin{subproof}
    		First, note that there is a fixed $c \geq 1$ such that
    		\begin{equation*}
    			\alpha_n
    			\leq
    			c^n \chi^{\lie{g}}_n
    			\quad \Longleftrightarrow \quad
    			\frac{1}{\chi(\lie{g})_n}
    			\leq
    			\frac{c'^n}{\alpha_n}
    		\end{equation*}
    		by the definition of a rapidly increasing sequence.
    		Denote $a_i + b_i = n_i$. Then we have
    		\begin{equation*}
    			\frac{\beta_k \cdot \beta_{\ell}}
	       	{
	       		\alpha_r
	       		\cdot
	       		\chi^{\lie{g}}_{n_1}
	       		\cdots
	       		\chi^{\lie{g}}_{n_r}
	       	}
	       	\leq
    			\frac{c^{k + \ell} \beta_k \cdot \beta_{\ell}}
	       	{
	       		\alpha_r
	       		\cdot
	       		\alpha_{n_1}
	       		\cdots
	       		\alpha_{n_r}
	       	}
    		\end{equation*}
    		and hence
    		\begin{align*}
    		\log 
    		&
    		\left(
    			\frac{\beta_k \cdot \beta_{\ell}}
		       	{
		       		\alpha_r
		       		\cdot
	  	     		\chi^{\lie{g}}_{n_1}
	 	      		\cdots
	  	     		\chi^{\lie{g}}_{n_r}
		       	}
	       	\right)
	       	\\
	       	& \leq
    		\log \left(
    			\frac{c^{k + \ell} \beta_k \cdot \beta_{\ell}}
		       	{
		       		\alpha_r
		       		\cdot
		       		\alpha_{n_1}
		       		\ldots
		       		\alpha_{n_r}
		       	}
	       	\right)
	       	\\
	       	& =
	       	(k + \ell) \log(c)
	       	+
	       	\frac{g(k)}{8}
	       	+
	       	\frac{g(\ell)}{8}
	       	-
	       	f_{\alpha}(r)
	       	-
	       	f_{\alpha}(n_1)
	       	- \cdots -
	       	f_{\alpha}(n_r)
	       	\\
	       	& \ot{(a)}{\leq}
	       	(k + \ell) \log(c)
	       	+
	       	\frac{g(k + \ell)}{8}
			-
	       	f_{\alpha}(r)
	       	-
	       	f_{\alpha}(n_1)
	       	- \cdots -
	       	f_{\alpha}(n_r)
	       	\\
	       	& \ot{(b)}{\leq}
			(k + \ell) \log(c)
	       	+
	       	g(n_1) + \cdots + g(n_r)
	       	+
	       	\frac{f_{\alpha}(r)}{8}
	       	\\
	       	& \qquad
	       	-
	       	f_{\alpha}(r)
	       	-
	       	f_{\alpha}(n_1)
	       	- \cdots -
	       	f_{\alpha}(n_r)
	       	\\
	       	& \ot{(c)}{\leq}
	       	(k + \ell) \log(c)
	       	+
	       	\left( g(n_1) - f_{\alpha}(n_1) \right)
	       	+ \cdots + 
	       	\left( g(n_r) - f_{\alpha}(n_r) \right)
	       	\\
	       	& \ot{(d)}{\leq}
	       	(k + \ell) \log(c')
	       	+
	       	8 g(1) n_1
	       	+ \cdots + 
	       	8 g(1) n_r
	       	\\
	       	& =
	       	(k + \ell)
	       	(\log(c') + 8 g(1)).
    		\end{align*}
    	In (a), we have used the convexity of $g$ and 
    	Lemma~\ref{Nilpot:Lemma:MRZGrothLemma} in (b). 
    	Then, (c) is just $\frac{f_\alpha(n)}{8} \leq f_\alpha(n)$.
    	In (d) we used Lemma~\ref{Nilpot:Lemma:MRZGrothLemma} again 
    	by estimating
    	\begin{equation*}
    		g(n)
    		=
    		g(1 + \ldots + 1)
    		\leq
    		8 n g(1) + f_{\alpha}(n)
    		\quad \Longleftrightarrow \quad
    		g(n) - f_\alpha(n)
    		\leq8 n g(1).
    	\end{equation*}
    	Now we need to exponentiate the inequality we just found and get
    	\begin{equation*}
   			\frac{\beta_k \cdot \beta_{\ell}}
	       	{
	       		\alpha_r
	       		\cdot
  	     		\chi^{\lie{g}}_{n_1}
 	      		\cdots
  	     		\chi^{\lie{g}}_{n_r}
	       	}
	       	\leq
	       	\E^{k + \ell)(\log(c') + 8 g(1))}
	       	=
	       	\left( c \E^{8 g(1)} \right)^{k + \ell}.
    	\end{equation*}
    	Since the $n_1, \ldots, n_r$ have been arbitrary, 
    	$(\beta_n)_{n \in \mathbb{N}}$ and $\kappa$ fulfil 
    	\eqref{Nilpot:SavingSequence} and the Lemma is proven.
    \end{subproof}
    Now, we only need to take the sum of the $C_n$. 
    We know this is possible from Chapter 5, and 
    we finally get a continuity estimate which is locally uniform in $z$.
\end{proof}


At the end of Chapter 5, we showed that a finite-dimensional Lie algebra 
$\lie{g}$ is nilpotent if and only if its universal enveloping algebra 
$\algebra{U}(\lie{g})$ admitted a locally convex topology, such that the 
following three things are fulfilled.
\begin{enumerate}
	\item
	The product in $\algebra{U}(\lie{g})$ is continuous.

	\item
	For every $\xi \in \lie{g}$ the series $\exp(\xi)$ converges 
	absolutely in the completion of $\algebra{U}(\lie{g})$.

	\item
	Pulling back the topology to the symmetric tensor algebra, the projection 
	and inclusion maps with respect to the graded structure
	\begin{equation*}
		\Sym^{\bullet}(\lie{g})
    		\ot{$\pi_n$}{\longrightarrow}
   	    		\Sym^n(\lie{g})
    	    	\ot{$\iota_n$}{\longrightarrow}
    		\Sym^{\bullet}(\lie{g})
	\end{equation*}
	are continuous for all $n \in \mathbb{N}$.
\end{enumerate}
For Banach-Lie algebras, we came quite close to a similar statement: we know 
from Proposition~\ref{LCAna:Prop:NoBetterTopology} and the result of 
Wojty\`nski, that a Banach-Lie algebra must at least be \emph{topologically 
nil} to satisfy the three upper points, so being topologically nil is 
\emph{necessary}. We also know, that a uniformly topologically nilpotent 
Banach-Lie algebra $\lie{g}$ allows us to construct such a locally convex topology 
on $\algebra{U}(\lie{g})$ explicitly, hence uniform topological nilpotency is 
\emph{sufficient}. Maybe it is possible to find a notion of generalized 
nilpotency, which is equivalent to those three points.
